\section{Function Types}

Given types $A$ and $B$, we can construct the function type $A \to B$,
representing functions from $A$ to $B$.
Functions are a primitive concept in type theory,
and we provide a brief introduction here.

We can \textbf{apply} a function $f : A \to B$ to an element $a : A$ to obtain an element of $B$, denoted $f(a)$. In type theory, it is common to omit the parentheses and write the application simply as $f\, a$.

There are two equivalent ways to construct function types: either by direct definition or by using $\lambda$-abstraction. Introducing a function by definition means that we introduce a function by giving it a name (let's say, $f$) and saying we define $f : A \to B$ by giving an equation
\begin{equation} \label{lambda_definition}
  f(x) \coloneqq \Phi
\end{equation}
where $x$ is a variable and $\Phi$ is an expression which may use $x$. In order for this to be valid, we have to check that $\Phi : B$ assuming $x : A$.

Now we can compute $f(a)$ by replacing the variable $x$ in $\Phi$ with $a$. As an example, consider the function $f : \mathbb{N} \to \mathbb{N}$ which is defined by $f(x) \coloneqq x + x$. Then $f(2)$ is \emph{judgmentally equal} to $2 + 2$.

If we don't want to introduce a name for the function, we can use \textbf{$\lambda$-abstraction}.
Given an expression $\Phi$ of type $B$ which may use $x : A$, as above, we write $\lambda(x : A). \Phi$ to
indicate the same function defined by (\ref{lambda_definition}). Thus, we have
$$ (\lambda(x : A). \Phi) : A \to B. $$

\begin{example}
  The previously defined function  has the typing judgment
  $$ (\lambda(x : \mathbb{N}). x + x) : \mathbb{N} \to \mathbb{N}. $$

  As another example, for any types $A$ and $B$ and any element $y : B$, we have a \textbf{constant function}
  $$ (\lambda(x : A). y) : A \to B. $$

  The \textbf{identity function} on any type $A$ is given by
  $$ (\lambda(x : A). x) : A \to A. $$
\end{example}

By convention, the ``scope'' of the variable binding ``$\lambda x.$'' is the entire rest of the expression, unless delimited with parentheses. Thus, for instance, $\lambda x. x + x$ should be parsed as $\lambda x.(x + x)$, not as $(\lambda x. x) + x$ .


Now a $\lambda$-abstraction is a function, so we can apply it to an argument $a : A$. We then have the following computation rule ($\beta$-reduction), which is a \textbf{definitional equality}:
$$ (\lambda x. \Phi)(a) \equiv \Phi' $$
where $\Phi'$ is the expression $\Phi$ in which all occurrences of $x$ have been replaced by $a$.
\begin{example}
  Continuing the above example, we have
  $$ (\lambda x. x + x)(2) \equiv 2 + 2. $$
\end{example}

Note that from any function $f : A \to B$, we can construct a lambda abstraction function $\lambda x. f(x)$.
Since this is by definition ``the function that applies $f$ to its argument'' we consider it to be definitionally
equal to $f$ (\textbf{$\eta$-conversion}):
$$ f \equiv (\lambda x. f(x)). $$

This equality is the uniqueness principle for function types, because it shows that $f$ is uniquely determined by its values.

The introduction of functions by definitions with explicit parameters can be reduced to simple definitions by using $\lambda$-abstraction: i.e., we can read a definition of $f : A \to B$ by
$$ f(x) \coloneqq \Phi
$$
as
$$ f \coloneqq \lambda x. \Phi. $$

% When doing calculations involving variables, we have to be careful when replacing a variable with an expression that also involves variables, because we want to preserve the binding structure of expressions. As an example, consider $f : \mathbb{N} \to (\mathbb{N} \to \mathbb{N})$ defined as
% $$  f(x) \coloneqq \lambda y. x + y. $$

% Now if we have assumed somewhere that $y : \mathbb{N}$, then what is $f(y)$? It would be wrong to just naively replace $x$ by $y$ everywhere in the expression ``$\lambda y. x + y$'' defining $f(x)$, obtaining $\lambda y. y + y$, because this means that $y$ gets captured. Previously, the substituted $y$ was referring to our assumption, but now it is referring to the argument of the $\lambda$-abstraction. Hence, this naive substitution would destroy the binding structure, allowing us to perform calculations which are semantically unsound.

% But what is $f(y)$ in this example? Note that bound variables such as $y$ in the expression $\lambda y. x + y$ have only a local meaning, and can be consistently replaced by any other variable, preserving the binding structure. Indeed, $\lambda y. x + y$ is declared to be judgmentally equal (\textbf{\alpha-conversion}) to $\lambda z. x + z$. It follows that $f(y)$ is judgmentally equal to $\lambda z. y + z$, and that answers our question.

% Of course, this should all be familiar to any mathematician: it is the same phenomenon as the fact that if $f(x) \coloneqq \int_1^2 \frac{dt}{x-t}$, then $f(t)$ is not $\int_1^2 \frac{dt}{t-t}$ but rather $\int_1^2 \frac{ds}{t-s}$. A $\lambda$-abstraction binds a dummy variable in exactly the same way that an integral does.

% We have seen how to define functions in one variable. One way to define functions in several variables is called currying (after the mathematician Haskell Curry).

% The idea of currying is to represent a function of two inputs $a : A$ and $b : B$ as a function which takes one input $a : A$ and returns another function, which then takes a second input $b : B$ and returns the result. That is, we consider two-variable functions to belong to an iterated function type, $f : A \to (B \to C)$. We may also write this without the parentheses, as $f : A \to B \to C$, with associativity to the right as the default convention. Then given $a : A$ and $b : B$, we can apply $f$ to $a$ and then apply the result to $b$, obtaining $f(a)(b) : C$. To avoid the proliferation of parentheses, we allow ourselves to write $f(a)(b)$ as $f(a, b)$ even though there are no products involved. When omitting parentheses around function arguments entirely, we write $f \, a \, b$ for $(f \, a) \, b$, with the default associativity now being to the left so that $f$ is applied to its arguments in the correct order.

% Our notation for definitions with explicit parameters extends to this situation: we can define a named function $f : A \to B \to C$ by giving an equation
% $$  f(x, y) \coloneqq \Phi $$
% where $\Phi : C$ assuming $x : A$ and $y : B$. Using $\lambda$-abstraction this corresponds to
% $$ f \coloneqq \lambda x. \lambda y. \Phi, $$
% which may also be written as
% $$ f \coloneqq x \mapsto y \mapsto \Phi. $$


When performing calculations involving variables, we must carefully preserve the \textbf{binding structure} of expressions during substitution. Consider the function $f : \mathbb{N} \to (\mathbb{N} \to \mathbb{N})$ defined as:
$$ f(x) \coloneqq \lambda y. x + y $$

Suppose we have assumed $y : \mathbb{N}$ somewhere in our context. What is $f(y)$?

A naive approach would replace $x$ with $y$ directly in the expression $\lambda y. x + y$, yielding $\lambda y. y + y$. However, this substitution is \textbf{semantically incorrect} because it causes \textbf{variable capture}: the free variable $y$ (referring to our assumption) becomes bound by the $\lambda$-abstraction, fundamentally altering the expression's meaning.

The correct approach uses \textbf{$\alpha$-conversion} (variable renaming).
Since bound variables have only local scope, we can consistently rename them while preserving binding structure.
The expression $\lambda y. x + y$ is judgmentally equal to $\lambda z. x + z$ for any fresh variable $z$. Therefore:
$$ f(y) \equiv \lambda z. y + z $$

This phenomenon parallels familiar mathematical practice: if $f(x) \coloneqq \int_1^2 \frac{dt}{x-t}$,
then $f(t)$ equals $\int_1^2 \frac{ds}{t-s}$, not the ill-defined $\int_1^2 \frac{dt}{t-t}$.
Lambda abstractions bind dummy variables exactly as integrals do.

For functions of multiple variables, we employ \textbf{currying} (named after mathematician Haskell Curry). Instead of using product types, we represent a two-argument function as a function returning another function.

A function taking inputs $a : A$ and $b : B$ to produce output in $C$ has type:
$$ f : A \to (B \to C) \equiv A \to B \to C $$
where the arrow associates to the right by convention.

Given $a : A$ and $b : B$, we apply $f$ sequentially: first to $a$, then the result to $b$, obtaining $f(a)(b) : C$.

To simplify notation and avoid excessive parentheses, we adopt several conventions. We write $f(a)(b)$ as $f(a, b)$ for abbreviated application. Without parentheses entirely, $f \, a \, b$ means $(f \, a) \, b$ following left-associative application. For multi-parameter definitions, we write $f(x, y) \coloneqq \Phi$ where $\Phi : C$ under assumptions $x : A$ and $y : B$.

Using $\lambda$-abstraction, such definitions correspond to:
$$ f \coloneqq \lambda x. \lambda y. \Phi $$

Alternative notation using map symbols:
$$ f \coloneqq x \mapsto y \mapsto \Phi $$

This currying approach extends naturally to functions of three or more arguments, allowing us to represent any multi-argument function as a sequence of single-argument functions.
\begin{example}[Functions in Lean]
  \begin{lstlisting}[language=Lean]
-- A simple function
def double : Nat -> Nat := x + x

#eval double 3     -- Output: 6

\end{lstlisting}
  in steps
  \begin{align}
    double 3                                      \\
     & \equiv (x + x) [x:= 3] \text{substitution} \\
     & \equiv 3 + 3                               \\
     & \equiv 6
  \end{align}
\end{example}
\newpage
\begin{example}[Functions in Lean using lmabda ]
  \begin{lstlisting}[language=Lean]
-- Lambda abstraction form
def double' : Nat -> Nat := fun x => x + x
-- or Equivalently
def double' := (\lambda x : Nat => x + x)

#eval double' 3     -- Output: 6

\end{lstlisting}
  in steps
  \begin{align}
    double 3                                                    \\
     & \equiv (\lambda x : Nat, x + x)(3) \text{Beta reduction} \\
     & \equiv (x + x) [x:= 3] \text{substitution}               \\
     & \equiv 3 + 3                                             \\
     & \equiv 6
  \end{align}
\end{example}
\begin{example}[Currying in Lean]
  \begin{lstlisting}[language=Lean]
def add : Nat -> (Nat -> Nat) := fun x => (fun y => x + y)
-- or equivalently (rightassociative notation)
def add : Nat -> Nat -> Nat := fun x => fun y => x + y
#eval add 3 4   -- Output: 7

\end{lstlisting}
  in steps
  \begin{align}
    add 3 4                                                     \\
     & \equiv (add 3) 4                                         \\
     & \equiv (\lambda x : Nat, x + x)(3) \text{Beta reduction} \\
     & \equiv (x + x) [x:= 3] \text{substitution}               \\
     & \equiv 3 + 3                                             \\
     & \equiv 6
  \end{align}
\end{example}
\begin{example}[Higher order Function in Lean]
  \begin{lstlisting}[language=Lean]
def HAdd : (Nat -> Nat) -> Nat := h => h 3 + h 4

#eval HAdd add -- Output: 14

\end{lstlisting}
  in steps
  \begin{align}
    add 3 4                                                     \\
     & \equiv (add 3) 4                                         \\
     & \equiv (\lambda x : Nat, x + x)(3) \text{Beta reduction} \\
     & \equiv (x + x) [x:= 3] \text{substitution}               \\
     & \equiv 3 + 3                                             \\
     & \equiv 6
  \end{align}
\end{example}
\begin{example}
  [Copmposing Functions in Lean]
  \begin{lstlisting}[language=Lean]
def compose : (B -> C) -> (A -> B) -> (A -> C) :=
  fun f g => fun x => g (f x) --  left associative 
#eval compose add 3 -- Output: 12
\end{lstlisting}
\end{example}
\newpage
Y combinator Not strongly normalizing
\begin{example}
  [Combinators in Lean]
  \begin{lstlisting}[language=Lean]
def K : A -> (B -> A) := fun x => fun _ => x  
def S : (A -> B -> C) -> (A -> B) -> (A -> C) :=
  fun f g => fun x => f x (g x)

def Y : (A -> A) -> A := 
  fun f => (fix Y x => f (Y f x))
#eval K 5 true -- Output: 5 
#eval S add 3 4 -- Output: 7
\end{lstlisting}
\end{example}
We now examine the inference rules for function types.
\paragraph{Formation Rule}
\[
  \frac{A : \mathsf{Type} \quad B : \mathsf{Type}}{A \to B : \mathsf{Type}}
\]

\paragraph{Introduction Rule (lambda abstraction)}
\[
  \frac{x : A \vdash \Phi : B}{\lambda x.\Phi : A \to B}
\]

\paragraph{Elimination Rule (application)}
\[
  \frac{f : A \to B \quad a : A}{f(a) : B}
\]

\paragraph{Computation Rule (beta reduction)}
\[
  (\lambda x. \Phi)(a) \equiv \Phi[x := a]
\]

\paragraph{Uniqueness Principle (eta conversion) or equality}
\[
  f \equiv \lambda x. f(x)
\]
\begin{notation}
  $[x := a]$ stands for substitution.
  % or as explained before $\alpha$-conversion.
\end{notation}
Strong normalization Church Rossen theorem states that every term in the simply typed lambda calculus
reduces to a normal form, meaning that it cannot be reduced further.
This is a crucial property of the type system, ensuring that every well-typed term has a unique normal form.

POLYMORPHISM HERE?
\begin{example}[Polmorhic Function in Lean]
  \begin{lstlisting}[language=Lean] 
def id : {A : Type} -> A -> A := fun {A} x => x 
#eval id 5         -- Output: 5
#eval id true      -- Output: true  
#eval id "Hello"   -- Output: "Hello"
\end{lstlisting}
\end{example}
