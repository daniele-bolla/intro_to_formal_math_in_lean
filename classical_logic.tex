
\section{Logic and Proposition as Types}

Logic is the study of reasoning, branching into various systems.
We refer to \textbf{classical logic} as the one that underpins much 
of traditional mathematics. 
It's the logic of the ancient Greeks (not fair) and truthtables, and it remains 
used nowadays for pedagogical reasons.
We first introduce \textbf{propositional logic}, which is the simplest 
form of classical logic. 
Later we will extend this to \textbf{predicate (or first-order) logic}, which includes 
quantifiers and predicates.
In this setting, a \textbf{proposition} is a statement that is either true or false, 
and a \textbf{proof} is a logical argument that establishes the truth of a 
proposition.
Propositions are constructed via \textbf{formulas} built from 
\textbf{propositional variables} 
(also called atomic propositions) combined with logical \textbf{connectives} such as ``and'' ($\wedge$), 
``or'' ($\vee$), ``not'' ($\neg$), ``implies'' ($\Rightarrow$), and ``if and only if'' ($\Leftrightarrow$). 
These connectives allow the creation of complex or compound propositions.
\begin{definition}[Propositional Formula](\cite{thompson1999types})
A \textbf{propositional formula} is either:
\begin{itemize}
    \item A \textbf{propositional variable}: $X_0, X_1, X_2, \ldots$, or
    \item A \textbf{compound formula} formed by combining formulas using connectives:
    \[
        (A \wedge B), \quad (A \Rightarrow B), \quad (A \vee B), \quad \bot, \quad (A \Leftrightarrow B), \quad (\neg A)
    \]
    where $A$ and $B$ are formulas themselves.
\end{itemize}
\end{definition}

We are going to describe classical logic though a formal framework 
called \textbf{natural deduction system}  developed by
Gentzen in the 1930s (\cite{wadler2015propositions}). It specifies 
rules for deriving 
\textbf{conclusions} from \textbf{premises} (assumptions from other propositions), 
called \textbf{inference rules}.

\begin{example}[Deductive style rule]
Here is an hypothetical example of inference rule.
\begin{prooftree}
  \AxiomC{$P_1$}
  \AxiomC{$P_2$}
  \AxiomC{$\cdots$}
  \AxiomC{$P_n$}
  \QuaternaryInfC{$C$}
\end{prooftree}
Where the $P_1, P_2, \ldots, P_n$, above the line, are hypothetical premises and, the hypothetical conclusion $C$ is below the line.
\end{example}
The inference rules needed are:
\begin{itemize}
    \item \textbf{Introduction rules} specify how to form compound propositions from simpler ones, and
    \item \textbf{Elimination rules} specify how to use compound propositions to derive information about their components.
\end{itemize}
Let's look at how we can define some connectives.
\paragraph{Conjunction ($\land$)}
\begin{itemize}
    \item Introduction
    \begin{prooftree}
        \AxiomC{$A$}
        \AxiomC{$B$}
        \RightLabel{$\land$-Intro}
        \BinaryInfC{$A \land B$}
    \end{prooftree}
    \item Elimination
    \noindent
      \begin{minipage}[t]{0.5\textwidth}
        \begin{prooftree}
          \AxiomC{$A \land B$}
          \RightLabel{$\land$-Elim$_1$}
          \UnaryInfC{$A$}
        \end{prooftree}
      \end{minipage}\hfill
      \begin{minipage}[t]{0.5\textwidth}
        \begin{prooftree}
          \AxiomC{$A \land B$}
          \RightLabel{$\land$-Elim$_2$}
          \UnaryInfC{$B$}
        \end{prooftree}
      \end{minipage}
\end{itemize}
\paragraph{Disjunction ($\lor$)}
\begin{itemize}
    \item Introduction
          \begin{minipage}[t]{0.5\textwidth}
    \begin{prooftree}
        \AxiomC{$A$}
        \RightLabel{$\lor$-Intro$_1$}
        \UnaryInfC{$A \lor B$}
    \end{prooftree}
      \end{minipage}\hfill
      \begin{minipage}[t]{0.5\textwidth}
    \begin{prooftree}
        \AxiomC{$B$}
        \RightLabel{$\lor$-Intro$_2$}
        \UnaryInfC{$A \lor B$}
    \end{prooftree}
      \end{minipage}
    \item Elimination (Proof by cases)
    \begin{prooftree}
        \AxiomC{$A \lor B$}
        \AxiomC{$[A] \vdash C$}
        \AxiomC{$[B] \vdash C$}
        \RightLabel{$\lor$-Elim}
        \TrinaryInfC{$C$}
    \end{prooftree}
\end{itemize}
\paragraph{Implication ($\to$)}
\begin{itemize}
    \item Introduction
    \begin{prooftree}
        \AxiomC{$[A] \vdash B$}
        \RightLabel{$\to$-Intro}
        \UnaryInfC{$A \to B$}
    \end{prooftree}
    \item Elimination (Modus Ponens)
    \begin{prooftree}
        \AxiomC{$A \to B$}
        \AxiomC{$A$}
        \RightLabel{$\to$-Elim}
        \BinaryInfC{$B$}
    \end{prooftree}
\end{itemize}

\begin{notation}
We use $A \vdash B$ (called turnstile) to designate a deduction of $B$ from $A$. 
It is employed in Gentzen’s \textbf{sequent calculus} (\cite{girard1989proofs}), 
whereas in natural deduction the corresponding symbol is\[
\begin{array}{c}
A \\
\vdots \\
B
\end{array}
\]
There are some minor differences, in fact, which I don't fully understand.
The square brackets around a premise $[A]$ mean that the premise $A$ is meant to 
be \textbf{discharged} at the conclusion. The classical example is the 
introduction rule for the implication connective.
To prove an implication $A \to B$, we assume $A$ 
(shown as $[A]$), derive $B$ under this assumption, and then discharge the 
assumption $A$ to conclude that $A \to B$ holds without the assumption. 
The turnstile is predominantly used in judgments and type theory with 
the meaning of ``entails that''. 
\end{notation}

Lean has its own syntax for connectives and their relative inference rules.
For instance $A \land B$ can be presented as \lstinline[language=lean]|And(A, B)| or \lstinline[language=lean]|A ∧ B|,
. Its untroduction rule is constructed by
 \lstinline[language=lean]|And.intro _ _| or shortly
 \lstinline[language=lean]|⟨_, _⟩| (underscore are placeholder for assumptions or "propositional functions"). 
 The pair $A \land B$ can be then consumed using elimination 
rules \lstinline[language=lean]|And.left| and \lstinline[language=lean]|And.right|.
\begin{example}\label{ex:conj_intro}
    Let's look at our first Lean example
  \begin{lstlisting}[language=lean]
    example (H_A : A) (H_B : B) : (A ∧ B) := And.intro H_A H_B
  \end{lstlisting}
Lean aims to resemble the language used in mathematics. 
For instance, when defining a function or expression, one can use keywords such as 
\lstinline[language=lean]|theorem| or \lstinline[language=lean]|def|.
Here, I used \lstinline[language=lean]|example|, which is handy for defining anonymous expressions 
for demonstration purposes. 
After that comes the statement to be proved:
\begin{lstlisting}[language=lean]
  (H_A : A) (H_B : B) : (A ∧ B) 
\end{lstlisting}
Meaning given a proof of $A$ and a proof of $B$ we can form a proof of $(A \land B)$.
The operator \lstinline[language=lean]|:=| assigns a value (or return an expression) for the statement which
 "has to be a proof of it".
\lstinline[language=lean]|And.intro| is implemented as:
\begin{lstlisting}[language=lean]
  And.intro: p → q → (p ∧ q).
\end{lstlisting}
It says: if you give me a proof of $p$ and a proof of $q$, 
then i return a proof of $p \land q$.
We therefore conclude the proof by directly giving 
\lstinline[language=lean]|And.intro H_A H_B|.
Here another way of writing the same statment.
\begin{lstlisting}[language=lean]
  example (H_p : p) (H_B : B) : And(A, B) := ⟨H_p, H_B⟩
\end{lstlisting}
\end{example}

This system of inference rules allows us to construct proofs in an 
algorithmic and systematical way, organized in what is called a \textbf{proof tree}. 
To reduce complexity, we follow a 
\textbf{top-down} approach (see \cite{thompson1999types} and 
\cite{nordstrom1990programming}).
This methodology forms the basis of \textbf{proof assistants} like Lean, 
Coq, and Agda, which help 
verify the correctness of mathematical proofs by checking each step 
against these rules.
We will see later that Lean, in fact, provides an info view of the proof tree 
which helps us 
understand and visualize 
the proof structure.
Let's examine a concrete example of a proof.
\begin{example}[Associativity of Conjunction]
We prove that $(A \land B) \land C$ implies $A \land (B \land C)$.
First, from the assumption $(A \land B) \land C$, we can derive $A$:
\begin{prooftree}
  \AxiomC{$(A \land B) \land C$}
  \RightLabel{$\land E_1$}
  \UnaryInfC{$A \land B$}
  \RightLabel{$\land E_1$}
  \UnaryInfC{$A$}
\end{prooftree}
Second, we can derive $B \land C$:
\begin{prooftree}
  \AxiomC{$(A \land B) \land C$}
  \RightLabel{$\land E_1$}
  \UnaryInfC{$A \land B$}
  \RightLabel{$\land E_2$}
  \UnaryInfC{$B$}
  \AxiomC{$(A \land B) \land C$}
  \RightLabel{$\land E_2$}
  \UnaryInfC{$C$}
  \RightLabel{$\land I$}
  \BinaryInfC{$B \land C$}
\end{prooftree}
Finally, combining these derivations we obtain $A \land (B \land C)$:
\begin{prooftree}
  \AxiomC{$(A \land B) \land C \vdash A$}
  \AxiomC{$(A \land B) \land C \vdash B \land C$}
  \RightLabel{$\land I$}
  \BinaryInfC{$A \land (B \land C)$}
\end{prooftree}
\end{example}
\begin{example}[Lean Implementation]
Let us now implement the same proof in Lean.
\begin{lstlisting}[language=Lean, caption=Associativity of Conjunction in Lean]
theorem and_associative : (A ∧ B) ∧ C → A ∧ (B ∧ C) :=
  fun h : (A ∧ B) ∧ C =>
    -- First, from the assumption (A ∧ B) ∧ C, we can derive A:
    have ab : A ∧ B := h.left -- extracts (derive) a proof of (A ∧ B) from the assumption
    have a : A := ab.left -- extracts A from (A ∧ B)
    -- Second, we can derive B ∧ C (here we only extract b and c and combine them in the next step) 
    have c : C := h.right
    have b : B := ab.right
    -- Finally, combining these derivations we obtain A ∧ (B ∧ C)
    show A ∧ (B ∧ C) from ⟨a, ⟨b, c⟩⟩
\end{lstlisting}
We introduce the \lstinline[language=lean]|theorem| with the name 
\lstinline[language=lean]|and_associative|, 
which can be referenced in subsequent proofs. 
The type signature \lstinline[language=lean]|(A ∧ B) ∧ C → A ∧ (B ∧ C)| 
represents our logical implication.
The \lstinline[language=lean]|:=| operator introduces the 
proof term that establishes the theorem's validity. 
In the previous code example this proof was directly given.
Here, we construct it using a function with the \lstinline[language=lean]|fun| keyword. 
Why a function? We have already encountered the Curry-Howard correspondence in Lean 
previously, though without explicitly stating it. 
According to this correspondence, a proof of an implication can be 
understood as a function that takes a hypothesis as input and produces 
the desired conclusion as output. We will revisit this concept in more 
detail later.
The \lstinline[language=lean]|have| keyword introduces local 
lemmas within our proof scope, allowing us to break down complex 
reasoning into manageable intermediate steps, mirroring our natural deduction proof of before.
Finally, the \lstinline[language=lean]|show| keyword presents our final result. 
\end{example}
To capture more complex mathematical ideas, we extend our system from 
propositional logic to \textbf{predicate logic}.  
A \textbf{predicate} is a statement or proposition that depends on a variable.
In propositional logic we represent a proposition simply by $P$.  
In predicate logic, this is generalized: a predicate is written as $P(a)$, 
where $a$ is a variable. Notice that a predicate is just a function.
This extension allows us to introduce \textbf{quantifiers}:  
$\forall$ (``for all'') and $\exists$ (``there exists'').  
These quantifiers express that a given formula holds either for every object 
or for at least one object, respectively.
In Lean if \lstinline[language=lean]|α| is any type, we can represent a 
predicate \lstinline[language=lean]|P| on \lstinline[language=lean]|α| as 
an object of type \lstinline[language=lean]|α → Prop|.
\lstinline[language=lean]|Prop| stands for proposition, and it is an 
essential component of Lean’s type system.
For now, we can think of it as a special type whose 
inhabitants are proofs; somewhat 
paradoxically, a type of types. 
\lstinline|Prop| stands for \emph{proposition}, and it is an essential component of Lean’s type system.
Thus given an \lstinline[language=lean]|x : α| (an element
with type \lstinline[language=lean]|α| ) 
\lstinline[language=lean]|P(x) : Prop| would be representative of a proposition holding for \lstinline[language=lean]|x|.

When introducing variables into a formal language we must keep in mind that the specific choice 
of a variable name can be substituted without 
changing the meaning of the predicate or statement. This should feel familiar from mathematics, 
where the meaning of an expression does not depend on the names we assign to variables.
Some variables are \textbf{bound} (constrained), while others remain \textbf{free} 
(arbitrary, in programming often called "dummy" variables). 
When substituting variables, it is important to ensure that this distinction is preserved.
This phenomenon, called \textbf{variable capture}, parallels familiar mathematical practice: 
if $f(x) \coloneqq \int_1^2 \frac{dt}{x-t}$, then $f(t)$ equals $\int_1^2 \frac{ds}{t-s}$, 
not the ill-defined $\int_1^2 \frac{dt}{t-t}$. The same principle applies to predicate logic. For example, consider
\[
\exists y.\,(y > x).
\]
This states that for a given $x$ there exists a $y$ such that $y > x$. 
If we naively substitute $y+1$ for $x$, we would obtain
\[
\exists y.\,(y > y+1),
\]
where the $y$ in $y+1$ has been \textbf{captured} by the quantifier $\exists y$. 
This transforms the original statement from "there exists some $y$ greater than the free variable $x$" into the always-false statement 
"there exists some $y$ greater than itself plus one."
To avoid the probelm, in the above example, we would first rename the bound variable to something fresh 
say $z$, obtaining $\exists z.\,(z > x)$, and then safely substitute to get $\exists z.\,(z > y+1)$.
\begin{notation}
  We use the notation $\phi[t/x]$ for \textbf{substitution}, meaning all occurrences of the free 
  variable $x$ in formula (or expression) $\phi$ are replaced by term $t$.
\end{notation}
We can now present the inference rules for quantifiers.
\paragraph{Universal Quantification ($\forall$)}
\begin{itemize}
    \item Introduction
    \begin{prooftree}
    \AxiomC{$A$}
    \RightLabel{$\forall I$}
    \UnaryInfC{$\forall x.\,A$}
    \end{prooftree}
    The variable $x$ must be arbitrary in the derivation of $A$. 
    This rule captures statements like 
    $\forall x \in \mathbb{N}$, $x$ has a successor, 
    but would not apply to $\forall x \in \mathbb{N}$, $x$ is prime 
    (since we cannot derive this for an arbitrary natural number).
    \item Elimination
    \begin{prooftree}
    \AxiomC{$\forall x.\,A$}
    \RightLabel{$\forall E$}
    \UnaryInfC{$A[t/x]$}
    \end{prooftree}
    The conclusion $A[t/x]$ represents the substitution of term $t$ for variable $x$ in formula $A$. 
    From a proof of $\forall x.\,A(x)$ we can infer $A(t)$ for any term $t$.
\end{itemize}
\paragraph{Existential Quantification ($\exists$)}
\begin{itemize}
    \item Introduction
    \begin{prooftree}
    \AxiomC{$A[t/x]$}
    \RightLabel{$\exists I$}
    \UnaryInfC{$\exists x.\,A$}
    \end{prooftree}
    The substitution premise means that if we can find a specific term $t$ for which $A(t)$ holds, 
    then we can introduce the existential quantifier. 
    The introduction rule requires a witness $t$ for which the predicate holds.
    \item Elimination
    \begin{prooftree}
    \AxiomC{$\exists x.\,A$}
    \AxiomC{$[A] \vdash B$}
    \RightLabel{$\exists E$}
    \BinaryInfC{$B$}
    \end{prooftree}
    To eliminate an existential quantifier, we assume $A$ holds for some witness 
    and derive $B$ without making any assumptions about the specific witness.
\end{itemize}
We can give an informal reading of the quantifiers as infinite logical operations:
\begin{align*}
\forall x.\,A(x) &\equiv A(a) \land A(b) \land A(c) \land \ldots \\
\exists x.\,A(x) &\equiv A(a) \lor A(b) \lor A(c) \lor \ldots
\end{align*}
The expression $\forall x.\, P(x)$ can be understood as  generalized form of implication. 
If $P$ is any proposition, then $\forall x.\, P$ expresses that $P$ holds 
regardless of the choice of $x$. When $P$ is a predicate, depending on $x$, this captures the 
idea that we can derive $P$ from any assumption about $x$.
Morover, there is a duality between universal and existential quantification.
We shall develop all this dicussions further after 
exploring their computational (type theoretical) meaning.
\begin{example}
  Lean espresses quantifiers as follow.
  \begin{lstlisting}[language=Lean, caption=For All]
  ∀ (x : X), P x
  forall (x : X), P x -- another notation
  \end{lstlisting}
  \begin{lstlisting}[language=Lean, caption=Exists]
  ∃ (x : X), P x
  exist (x : X), P x -- another notation
  \end{lstlisting}
  Where \lstinline[language=lean]|x| is a varible with a type \lstinline[language=lean]|X|,
  and \lstinline[language=lean]|P x| is a proposition, or predicate, holding for \lstinline[language=lean]|x|.
\end{example}

\begin{example}[Existential introduction in Lean]
  When introducing an \textbf{existential} proof, 
  we need a \textbf{pair} consisting 
  of a witness and a proof that this witness 
  satisfies the statement.
  \begin{lstlisting}[language=lean]
example (x : Nat) (h : x > 0) : ∃ y, y < x :=
  Exists.intro 0 h -- or shortly ⟨0, h⟩
\end{lstlisting}
\end{example}
The \textbf{existential elimination rule} 
( \lstinline[language=lean]|Exists.elim|) performs the opposite operation. 
It allows us to prove a proposition $Q$ 
from $\exists x, P(x)$ by showing 
that $Q$  follows from $P(w)$  for an \textbf{arbitrary} 
value $w$.
\begin{example}[Existential elimination in Lean]
  The existential rules can be interpreted as an infinite 
  disjunction, 
  so that existential elimination naturally corresponds to a \textbf{proof by cases} (with only one single case). 
  In Lean, this reasoning is carried out using \textbf{pattern matching}, 
  a known mechanism in functional programming for dealing with cases,  
  with \lstinline[language=lean]|let| or \lstinline[language=lean]|match|, as well as by using \lstinline[language=lean]|cases| or 
  \lstinline[language=lean]|rcases| construct. 
  \begin{lstlisting}[language=lean]
  example (h : ∃ n : Nat, n > 0) : ∃ n : Nat, n > 0 :=
    match h with
    | ⟨witness, proof⟩ => ⟨witness, proof⟩
  \end{lstlisting}
\end{example}
\begin{example}
  The \textbf{universal quantifier} may be regarded as a generalized function.
Accordingly, In Lean, universal elimination is simply function application.
\begin{lstlisting}[language=lean]
example : ∀ n : Nat, n ≥ 0 :=
  fun n => Nat.zero_le n
\end{lstlisting}
\end{example}
Functions are primitive objects in type theory.
For example, it is interesting to note that a relation can be expressed as a function:
\lstinline[language=lean]|R : α → α → Prop|.
Similarly, when defining a predicate (\lstinline[language=lean]|P : α → Prop|) we must first declare 
\lstinline[language=lean]|α : Type| to be some arbitrary type. 
This is what is called \textbf{polymorphism}.
A canonical example is the identity function, written as 
\lstinline[language=lean]|α → α|, where 
\lstinline[language=lean]|α| is a type variable. 
It has the same type for 
both its domain and codomain, this means it can be 
applied to booleans (returning a boolean), numbers (returning a number), 
functions (returning a function), and so on.
In the same spirit, we can define a transitivity property of a relation as follows:
\begin{lstlisting}[language=lean]
def Transitive (α : Type) (R : α → α → Prop) : Prop :=
  ∀ x y z, R x y → R y z → R x z
\end{lstlisting}
To use \lstinline[language=lean]|Transitive|, we must provide both the type 
\lstinline[language=lean]|α| and the relation itself. 
For example, here is a proof of transitivity for the less-than relation on
 $\mathbb{N}$ ( in Lean \lstinline[language=lean]|Nat| or \lstinline[language=lean]|ℕ|):
\begin{lstlisting}[language=lean]
theorem le_trans_proof : Transitive Nat (· ≤ · : Nat → Nat → Prop) :=
  fun x y z h1 h2 => Nat.le_trans h1 h2 -- this lemma is provided by Lean 
\end{lstlisting}
Looking at this code, we immediately notice that explicitly 
passing the type argument \lstinline[language=lean]|Nat| is somewhat repetitive. 
Lean allows us to omit it by letting the type inference mechanism fill it in automatically.
 This is achieved by using \textbf{implicit arguments} with curly brackets:
\begin{lstlisting}[language=lean]
def Transitive {α : Type} (R : α → α → Prop) : Prop :=
  ∀ x y z, R x y → R y z → R x z
theorem le_trans_proof : Transitive (· ≤ · : Nat → Nat → Prop) :=
  fun x y z h1 h2 => Nat.le_trans h1 h2 
\end{lstlisting}
Lean's type inference system is quite powerful: in many cases, types can be completely 
inferred without explicit annotations. For instance, in earlier examples, Lean automatically 
inferred that the types of $A$, $B$, and $C$ were \lstinline[language=lean]|Prop|.
Let us now revisit the transitivity proof, but this time for the less-than-equal relation on 
the rational numbers (\lstinline[language=lean]|Rat| or \lstinline[language=lean]|ℚ|) instead.
\begin{lstlisting}[language=lean]
import Mathlib

theorem rat_le_trans : Transitive (· ≤ · :   Rat → Rat → Prop) :=
  fun _ _ _ h1 h2 => Rat.le_trans h1 h2
\end{lstlisting}
Here, \lstinline[language=lean]|Rat| denotes the rational numbers in Lean, 
and \lstinline[language=lean]|Rat.le_trans| is the transitivity lemma 
for \lstinline[language=lean]|≤| on rational numbers, provided by Mathlib.
We import Mathlib to access \lstinline[language=lean]|Rat| 
and \lstinline[language=lean]|le_trans|. 
Mathlib is the community‑driven mathematical 
library for Lean, containing a large body of formalized mathematics 
and ongoing development.
It is the de
facto standard library for both programming and proving
in Lean \cite{mathlib2020}, we will dig into it as we go along.
Notice that we used a function to discharge the universal 
quantifiers required by transitivity. The underscores indicate 
unnamed variables that we do not use later. If we had named 
them, say \lstinline|x y z|, then:
\lstinline[language=lean]|h1| would be a proof of \lstinline[language=lean]|x ≤ y|,
\lstinline[language=lean]|h2| would be a proof of \lstinline[language=lean]|y ≤ z|,
and \lstinline[language=lean]|Rat.le_trans h1 h2| produces a proof of \lstinline[language=lean]|x ≤ z|.
The \lstinline[language=lean]|Transitive| definition is imported from Mathlib and similarly defined as before. 
\begin{example}
  The code can be made more readable using tactic mode. 
  This mode comes with an info view showing the goal to solve and proof structure.
  In this mode, you use tactics, commands provided by Lean or defined by users,
  to carry out proof steps succinctly, avoid code repetition, 
  and automate common patterns. 
  This often yields shorter, clearer proofs than writing the full term by hand.
  \begin{lstlisting}[language=lean]
  import Mathlib

  theorem rat_le_trans : Transitive (· ≤ · : Rat → Rat → Prop) := by
    intro x y z hxy hyz
    exact le_trans hxy hyz
  \end{lstlisting}
  This proof performs the same steps but is much easier to read. 
  Using \lstinline[language=lean]|by| we enter Lean’s tactic mode, 
  which (with the info view) 
  shows the current goal and context. 
  Move your cursosr just before \lstinline[language=lean]|by|
  and observe the info view change. 
  The goal is shwon displayed \lstinline[language=lean]|⊢ Transitive fun x1 x2 ↦ x1 ≤ x2| at first.
  The tactic \lstinline[language=lean]|intro| introduces 
  the variables and hypotheses corresponding to the universal quantifiers 
  and assumptions.
  Now position your cursor just before \lstinline[language=lean]|exact|
  and observe the info view again. 
  The goal is now \lstinline[language=lean]|⊢ x ≤ z|, with the context 
  showing the variables ans hypothesis introduced by the previous tactic.
  \lstinline[language=lean]|exact| closes the goal 
  by supplying the term \lstinline[language=lean]|Rat.le_trans hxy hyz| that matches with the goal
  (the specification of \lstinline[language=lean]|Trqansitive|  ).
  You can over each tactic to see its definition and documentation.
\end{example}
In these examples we havec used predeined lemmas such as 
\lstinline[language=lean]|Nat.le_trans| and
\lstinline[language=lean]|Rat.le_trans|, just to simplify the presentation.
We can now dig in to the implementation of these lemmas.
Let’s look at the source code of \lstinline[language=lean]|Rat.le_trans|.
The Mathlib 4 documentation website is at:
\url{https://leanprover-community.github.io/mathlib4_docs}, and
The documentation for 
\lstinline[language=lean]|Rat.le_trans| is at:
\url{https://leanprover-community.github.io/mathlib4_docs/Mathlib/Algebra/Order/Ring/Unbundled/Rat.html#Rat.le_trans}
Click the “source” link there to jump to the implementation in the Mathlib repository. In editors like 
VS Code you can also jump directly to the definition (Ctrl-click; Cmd-click on macOS).
Lean has built-in types \texttt{Nat} (natural numbers), \texttt{Int} (integers), 
and \texttt{Rat} (rational numbers). While Lean provides basic arithmetic and 
order relations for these types, many advanced properties and theorems live in Mathlib.
% \begin{lstlisting}[language=lean]
% variable (a b c : Rat)
% protected lemma le_trans (hab : a ≤ b) (hbc : b ≤ c) : a ≤ c := by
%   rw [Rat.le_iff_sub_nonneg] at hab hbc
%   have := Rat.add_nonneg hab hbc
%   simp_rw [sub_eq_add_neg, add_left_comm (b + -a) c (-b), add_comm (b + -a) (-b), add_left_comm (-b) b (-a), add_comm (-b) (-a), add_neg_cancel_comm_assoc, ← sub_eq_add_neg] at this
%   rwa [Rat.le_iff_sub_nonneg]
% \end{lstlisting}
The proof uses several tactics and lemmas from Mathlib. You can follow 
the proof step by step using the info view in tactic mode, by piositioning the cursor
on each line and observing the changes in the goal and context.
The \lstinline[language=lean]|rw| or \lstinline[language=lean]|rewrite| tactic  
is very common and sintactically similar to
the mathematical practice of rewriting an expression using an equality.
In this case, with \lstinline[language=lean]|at|, we use it to rewrite the 
hypotheses \lstinline[language=lean]|hab| 
and \lstinline[language=lean]|hbc|
using the another Mathlib's lemma \lstinline[language=lean]|Rat.le_iff_sub_nonneg|, 
which states that for any two rational numbers \lstinline[language=lean]|x| and
\lstinline[language=lean]|y|, \lstinline[language=lean]|x ≤ y| 
is equivalent to \lstinline[language=lean]|0 ≤ y - x|.
Thus we now have the hypotheses tranformerd to :  
\begin{lstlisting}[language=lean]
  hab : 0 ≤ b - a
  hbc : 0 ≤ c - b
\end{lstlisting}
The \lstinline[language=lean]|have| tactic introduces an intermediate result. If you omit a name, Lean assigns it the default name \lstinline[language=lean]|this|. In our situation, from \lstinline[language=lean]|hab : a ≤ b| and \lstinline[language=lean]|hbc : b ≤ c| we can derive that \lstinline[language=lean]|b - a| and \lstinline[language=lean]|c - b| are nonnegative, hence their sum is nonnegative:
\begin{lstlisting}[language=lean]
this : 0 ≤ b - a + (c - b)
\end{lstlisting}
The most involved step uses \lstinline[language=lean]|simp_rw| to 
simplify the expression via a sequence of other existing Mathlib's lemmas. 
The tactic \lstinline[language=lean]|simp_rw| is a variant of \lstinline[language=lean]|simp|: it performs rewriting using the simp set (and any lemmas you provide), applying the rules in order and in the given direction. Lemmas that \lstinline[language=lean]|simp| can use are typically marked with the \lstinline[language=lean]|@[simp]| attribute. This is particularly useful for simplifying algebraic expressions and equations.
After these simplifications we obtain:
\begin{lstlisting}[language=lean]
this : 0 ≤ c - a
\end{lstlisting}
Clearly, the proof relies mostly on \lstinline[language=lean]|Rat.add_nonneg|. 
Its source code is fairly involved and uses advanced features 
that are beyond our current scope. Nevertheless, it highlights 
an important aspect of formal mathematics in Mathlib.
Mathlib defines \lstinline[language=lean]|Rat| as an instance of 
a linear ordered field, implemented via a normalized fraction 
representation: a pair of integers (numerator and denominator) 
with positive denominator and coprime numerator and denominator \cite{mathlibdoc}. 
To achieve this, it uses a \textbf{structure}. In Lean, a structure is a dependent record type 
used to group together related fields or properties as a single data type.
Unlike ordinary records, the type of later fields may depend on the values of earlier ones.
Defining a structure automatically introduces a constructor (usually mk) and projection 
functions that retrieve (deconstruct) the values of its fields.
Structures may also include proofs expressing properties that the fields must satisfy.

% \begin{lstlisting}[language=lean]
%   structure Rat where
%   /-- Constructs a rational number from components.
%   We rename the constructor to `mk'` to avoid a clash with the smart constructor. -/
%   mk' ::
%   /-- The numerator of the rational number is an integer. -/
%   num : Int
%   /-- The denominator of the rational number is a natural number. -/
%   den : Nat := 1
%   /-- The denominator is nonzero. -/
%   den_nz : den ≠ 0 := by decide
%   /-- The numerator and denominator are coprime: it is in "reduced form". -/
%   reduced : num.natAbs.Coprime den := by decide
%   ...
% \end{lstlisting}
In order to work with rational numbers in Mathlib, we use the
\lstinline[language=lean]|Rat.mk'| constructor to create a rational number from
its numerator and denominator, if omitted the default would be \lstinline[language=lean]|Rat.mk|.
The fields \lstinline[language=lean]|den_nz| and \lstinline[language=lean]|reduced| are proofs that 
the denominator is nonzero and that the numerator and denominator are coprime, respectively. 
These proofs are automatically generated by Lean's \lstinline[language=lean]|decide| tactic, which can 
solve certain decidable propositions (to be discussed in the next section).
\begin{example} 
  Here is how we can define and manipulate rational numbers in Lean.
  % \begin{lstlisting}[language=lean]
  % def half : Rat := Rat.mk' 1 2
  % def third : Rat := Rat.mk' 1 3
  % -- #eval evaluate the expression and print the result
  % #eval half.den -- outputs 2
  % #eval half + third -- outputs 5/6
  % -- #check prints the type of an expression
  % #check half.den -- outputs : Nat
  % #check half -- outputs : Rat
  % #check half + third -- outputs : Rat
  % \end{lstlisting}
\end{example}
When working with rational numbers, or more generally with structures, we must provide the 
required proofs as arguments to the constructor (or Lean must be able to ensure them).
For instance \lstinline[language=lean]|Rat.mk' 1 0| or \lstinline[language=lean]|Rat.mk' 2 6| would be rejected.
In the case of rationals, Mathlib unfolds the definition through
\lstinline[language=lean]|Rat.numDenCasesOn|. This principle states that, to prove a property of an 
arbitrary rational number, it suffices to consider numbers of the form \lstinline[language=lean]|n /. d| 
in canonical (normalized) form, with \lstinline[language=lean]|d > 0| and \lstinline[language=lean]|gcd n d = 1|.
This reduction allows mathlib to transform proofs about \lstinline[language=lean]|ℚ| 
into proofs about \lstinline[language=lean]|ℤ| and \lstinline[language=lean]|ℕ|, 
and then lift the result back to rationals.
\begin{example}
  I will present a simplified version of this implementation.
%   \begin{lstlisting}[language=lean]
% import Mathlib
% open Rat 

% lemma add_nonneg_simplified : 0 ≤ a → 0 ≤ b → 0 ≤ a + b := by
%   intro ha hb
%   -- Convert hypotheses to numerator conditions
%   rw [← num_nonneg] at ha hb
%   -- Express rationals in divInt form and apply addition formula
%   rw [← num_divInt_den a, ← num_divInt_den b, divInt_add_divInt]
%   -- Use divInt_nonneg_iff_of_pos_right to reduce to integer arithmetic
%   · rw [divInt_nonneg_iff_of_pos_right]
%     · -- Prove numerator ≥ 0
%       exact Int.add_nonneg (Int.mul_nonneg ha (Int.natCast_nonneg _))
%                            (Int.mul_nonneg hb (Int.natCast_nonneg _))
%     · -- Prove denominator > 0
%       norm_cast
%       exact Nat.mul_pos (Nat.pos_of_ne_zero a.den_nz) (Nat.pos_of_ne_zero b.den_nz)
%   · norm_cast; exact a.den_nz
%   norm_cast; exact b.den_nz

% \end{lstlisting}
In this version, we open the 
\lstinline[language=lean]|Rat| \textbf{namespace} to access its definitions and lemmas directly 
(Notice that i use \lstinline[language=lean]|num_nonneg| instead of 
\lstinline[language=lean]|Rat.num_nonneg| in the next line).
The proof begins by introducing the hypotheses \lstinline[language=lean]|ha| and \lstinline[language=lean]|hb|
that \lstinline[language=lean]|a| and \lstinline[language=lean]|b| are nonnegative.
The \lstinline[language=lean]|rw ... at| tactic rewrites these hypotheses using the lemma
\lstinline[language=lean]|num_nonneg|, which states that a rational number
is nonnegative if and only if its numerator is nonnegative. 
We use \lstinline[language=lean]|←| to indicate the direction of rewriting (from right to left).
Next, we express the rational numbers in terms of their numerator and denominator using
\lstinline[language=lean]|num_divInt_den|, and apply the addition formula
for rational numbers represented as \lstinline[language=lean]|divInt_add_divInt|.
The goal is then to prove that the resulting numerator is nonnegative.
\lstinline[language=lean]|num_divInt_den| transofms a rational number \lstinline[language=lean]|r| 
into the form \lstinline[language=lean]|r.num /. ↑r.den|. The \lstinline[language=lean]|↑| symbol denotes
the coercion from natural numbers to integers 
(remember in our definition of Rat, the numerator is an integer and the denominator a natural number, 
but here we need to translate everything to integers). 
We now have 3 goals.
We use \lstinline[language=lean]|divInt_nonneg_iff_of_pos_right| to reduce this to proving that the 
numerator is nonnegative, given that the denominator is positive.
This requires two subgoals: proving the numerator is nonnegative and the denominator is positive.
For the numerator, we use \lstinline[language=lean]|Int.add_nonneg|
to show that the sum of two nonnegative integers is nonnegative.
For the denominator, we first translate the problem from integers to natural numbers, using \lstinline[language=lean]|norm_cast|.
and use \lstinline[language=lean]|Nat.mul_pos| to show that the product of two positive natural numbers is positive.
Finally, we use \lstinline[language=lean]|norm_cast| to handle the necessary type casts between integers 
and natural numbers automatically and close the remaining goals with the 
nonzero denominator conditions given from the Rat structure (\lstinline[language=lean]|den_nz|).
Lean encourages to separate subgoals with \lstinline[language=lean]|·| and proper indentation, 
making the corresponding proof more readable.
\end{example}
We made extensive use of type casting and coercions in this proof, handled by 
the \lstinline[language=lean]|norm_cast| tactic,wich requires some explanation (\cite{lewis_madelaine_simplifying_casts_coercions_2020}).
Lean type system lack of subtypes means that types like \lstinline[language=lean]|Nat|, 
\lstinline[language=lean]|Int|, and \lstinline[language=lean]|Rat|
are distinct and do not have a subtype relationship.
In order to translate between these types, we need to use explicit type casts or coercions.
For example, natural numbers (\lstinline[language=lean]|Nat|) can be coerced to integers (\lstinline[language=lean]|Int|) and integers can be coerced
to rational numbers (\lstinline[language=lean]|Rat|).
The \lstinline[language=lean]|norm_cast| tactic simplifies expressions involving such coercions
by normalizing them, making it easier to reason about mixed-type expressions.
It  will be otherwise a long and tedious process to manually insert and manage 
these coercions throughout the proof.
\lstinline[language=lean]|norm_cast| is another example of a tactic that leverages 
Lean's metaprogramming capabilities to automate common proof patterns. 
(I CAN DISCUSS THIS FURTHER IF NEEDED).

The theorem previously used with natural numbers, \lstinline[language=lean]|Nat.le_trans|, 
is part of Lean’s internal library at /lean/Init/Prelude.lean. 
Mathlib is built on top of this base library.
More generally, the transitivity property holds not only for naturals but also for integers, 
reals, and, in fact, for any partially ordered set. 
Mathlib provides a general lemma \lstinline[language=lean]|le_trans| for any type 
\lstinline[language=lean]|α| endowed with partial ordering.
This is achieved through type classes, Lean’s mechanism for defining and working with 
abstract algebraic structures in an ad hoc polymorphic manner.
Type classes provide a powerful and flexible way to specify properties and 
operations that can be shared across different types, thereby enabling 
polymorphism and code reuse.
Ad hoc polymorphism arises when a function is defined over several distinct types, 
with behavior that varies depending on the type. A standard example (\cite{wadler_blott_ad_hoc_polymorphism_1988}) is overloaded 
multiplication: the same symbol denotes multiplication of integers 
(e.g. \lstinline[language=lean]|3 * 3|) and of floating-point numbers 
(e.g. \lstinline[language=lean]|3.14 * 3.14|).
By contrast, parametric polymorphism occurs when a function is defined over a 
range of types but acts uniformly on each of them. For instance, the length 
function applies in the same way to a list of integers and to a list of 
floatingpoints.

Under the hood, a type class is a structure. An important aspect of structures,
and hence type classes, is that they are powered by hierarchy and composition.
For example, a monoid is a semigroup with an identity element, and a group is a monoid with inverses. In Lean, we can express this
by defining a \lstinline[language=lean]|Monoid| structure that extends 
the \lstinline[language=lean]|Semigroup| structure, 
and a \lstinline[language=lean]|Group| structure 
that extends the \lstinline[language=lean]|Monoid| structure
using the \lstinline[language=lean]|extends| keyword.

% \begin{lstlisting}[language=lean]
% -- A semigroup has an associative binary operation
% structure Semigroup (α : Type*) where
%   mul : α → α → α
%   mul_assoc : ∀ a b c : α, mul (mul a b) c = mul a (mul b c)
% -- A monoid extends semigroup with an identity element  
% structure Monoid (α : Type*) extends Semigroup α where
%   one : α
%   one_mul : ∀ a : α, mul one a = a
%   mul_one : ∀ a : α, mul a one = a  
% -- A group extends monoid with inverses
% structure Group (α : Type*) extends Monoid α where
%   inv : α → α
%   mul_left_inv : ∀ a : α, mul (inv a) a = one     
% \end{lstlisting}

The symbol \lstinline[language=lean]|*| on  \lstinline[language=lean]|(α : Type*)| 
indicates a universe variable (we will discuss universes later). Sometimes, 
in order to avoid
inconsistencies between types (like Girard's paradox), universes must be specified explicitly. 
This is an example
of universe polymorphism, thus we have seen all the polymorphism flavors in Lean.
On the other hand, Type classes are defined using the \lstinline[language=lean]|class| keyword, 
which is syntactic sugar for defining a structure.
Thus the previuous example can be rewritten similarly, using type classes:
% \begin{lstlisting}[language=lean]
% -- A semigroup has an associative binary operation
% class Semigroup (α : Type*) where
%   mul : α → α → α
%   mul_assoc : ∀ a b c : α, mul (mul a b) c = mul a (mul b c)
% -- A monoid extends semigroup with an identity element  
% class Monoid (α : Type*) extends Semigroup α where
%   one : α
%   one_mul : ∀ a : α, mul one a = a
%   mul_one : ∀ a : α, mul a one = a  
% -- A group extends monoid with inverses
% class Group (α : Type*) extends Monoid α where
%   inv : α → α
%   mul_left_inv : ∀ a : α, mul (inv a) a = one     
% \end{lstlisting}
The main diffrence is that type classes support \textbf{instance resolution}, using the keyword 
\lstinline[language=lean]|instance| to declare that a particular type is an instance of a type class, 
wich inherits 
the properties and operations defined in the type class.

Let's look at a more concrete example, say we would like to define a a generic \lstinline[language=lean]|Transitive| 
structure with field \lstinline[language=lean]|le_trans|:
\begin{lstlisting}[language=lean]
structure TransitiveProperty (α : Type) where
  le_trans : α → α → α
\end{lstlisting}
Now in overload differnt typeds such as Rat, INtn and NAt we would define:
% \begin{lstlisting}[language=lean]
%   def polymorphic_le_trans (s : TransitiveProperty α) (x : α) : α :=
%   TransitiveProperty.le_trans x x

%   #eval double { add := Nat.le_trans } 10
%   20 30 -- true

%   #eval double { add := Int.mul } -10
%   20 -- false

%   #eval double { add := Int.add } 10
%   20
% \end{lstlisting}

This idea is extensively used in Mathlib to define and work with algebraic structures.
% Morover a class can extend other classes, allowing for the composition of properties and operations.
% For example, we can define a type class for a preorder, which is a set equipped with
% a reflexive and transitive relation derived from the less-than-or-equal and less-than type classes.

Instances of a type class can be automatically inferred by Lean's type inference system,
allowing for concise and expressive code.
This mechanism is particularly useful for defining and working with algebraic structures,
such as groups, rings, and fields, as well as order structures like preorders and partial orders.
Mathematically, a partially ordered set consists of a set \lstinline[language=lean]|P| 
and a binary relation 
\lstinline[language=lean]|≤| 
on P that is transitive and reflexive (\cite{mathinlean} Structures)

% \begin{lstlisting}[language=lean, caption=Preorder Type Class in Lean]
% -- A preorder is a reflexive, transitive relation `≤` with `a ≤ b` defined in the obvious way.
% class Preorder (α : Type*) extends LE α, LT α where
%   le_refl : ∀ a : α, a ≤ a
%   le_trans : ∀ a b c : α, a ≤ b → b ≤ c → a ≤ c
%   lt := fun a b => a ≤ b ∧ ¬ b ≤ a
%   lt_iff_le_not_ge : ∀ a b : α, a < b ↔ a ≤ b ∧ ¬ b ≤ a := by intros; rfl

% instance [Preorder α] : Lean.Grind.Preorder α where
%   le_refl := Preorder.le_refl
%   le_trans := Preorder.le_trans _ _ _
%   lt_iff_le_not_le := Preorder.lt_iff_le_not_ge _ _
% \end{lstlisting}

  The \texttt{class Preorder} declares a type class over a type 
 \(\alpha\), bundling the \(\leq\) and \(<\) relations 
 (inherited via \texttt{extends LE alpha, LT alpha}) with the
  preorder axioms: reflexivity (\texttt{le\_refl}) and 
  transitivity (\texttt{le\_trans}).
  The theorem \texttt{lt\_iff\_le\_not\_ge} provides a characterization of the strict order, 
  proved automatically (\texttt{by intros; rfl}).
  The \texttt{instance} declaration connects the \texttt{Preorder} class to Lean’s \texttt{Grind} 
  tactic automation, which allows automatic reasoning with preorder properties.

This design pattern is the foundation of Lean’s powerful mathematical library, 
allowing complex abstract algebraic and order structures to be expressed succinctly and compositionally.

% \section*{References}

% \begin{enumerate}
%  \item \textit{Lean Reference Manual: Type Classes}. \url{https://lean-lang.org/doc/reference/latest/Type-Classes/} [Accessed Sep 2025] \label{ref:lean-type-classes}
%  \item Floris Van Doorn, \textit{Functional Programming in Lean: Type Classes and Polymorphism}. \url{https://leanprover.github.io/functional_programming_in_lean/type-classes/polymorphism.html} [Accessed Sep 2025] \label{ref:fp-lean-typeclasses}
%  \item Mathlib Documentation, \textit{Mathlib.Order.Defs}, Lean community. \url{https://leanprover-community.github.io/mathlib4_docs/Mathlib/Order/Defs/PartialOrder.html} [Accessed Sep 2025] \label{ref:mathlib-order-defs}
%  \item \textit{Lean 4.23.0 Release Notes}. \url{https://lean-lang.org/doc/reference/4.23.0-rc2/releases/v4.23.0/} [Accessed Sep 2025] \label{ref:lean-release}
%  \item M. Carneiro et al., \textit{Use and abuse of instance parameters in the Lean Mathematical Library}, 2022. \url{https://arxiv.org/pdf/2202.01629.pdf} [Accessed Sep 2025] \label{ref:instance-abuse}
%  \item Lean.Meta.Tactic.Grind Documentation. \url{https://leanprover-community.github.io/mathlib4_docs/Init/Grind/Tactics.html} [Accessed Sep 2025] \label{ref:grind-tactic}
% \end{enumerate}