
\section{Introduction to Logic}

Logic is the study of reasoning, branching into various systems.
We refer to \textbf{classical logic} as the one that underpins much 
of traditional mathematics. 
It's the logic of the ancient Greeks (not fair) and truthtables, and it remains 
used nowadays for pedagogical reasons.
We first introduce \textbf{propositional logic}, which is the simplest 
form of classical logic. 
Later we will extend this to \textbf{first-order logic}, which includes 
quantifiers and predicates.
In this setting, a \textbf{proposition} is a statement that is either true or false, 
and a \textbf{proof} is a logical argument that establishes the truth of a 
proposition.
Propositions are constructed via \textbf{formulas} built from 
\textbf{propositional variables} 
(also called atomic propositions) combined with logical \textbf{connectives} such as ``and'' ($\wedge$), 
``or'' ($\vee$), ``not'' ($\neg$), ``implies'' ($\Rightarrow$), and ``if and only if'' ($\Leftrightarrow$). 
These connectives allow the creation of complex or compound propositions.
\begin{definition}[Propositional Formula](\cite{thompson1999types})
A \textbf{propositional formula} is either:
\begin{itemize}
    \item A \textbf{propositional variable}: $X_0, X_1, X_2, \ldots$, or
    \item A \textbf{compound formula} formed by combining formulas using connectives:
    \[
        (A \wedge B), \quad (A \Rightarrow B), \quad (A \vee B), \quad \bot, \quad (A \Leftrightarrow B), \quad (\neg A)
    \]
    where $A$ and $B$ are formulas themselves.
\end{itemize}
\end{definition}

We are going to describe classical logic though a formal framework 
called \textbf{natural deduction system}  developed by
Gentzen in the 1930s (\cite{wadler2015propositions}). It that specifies 
rules for deriving 
\textbf{conclusions} from \textbf{premises} (assumptions from other propositions), 
called \textbf{inference rules}.

\begin{example}[Deductive style rule]
Here is an hypothetical example.
\begin{prooftree}
  \AxiomC{$P_1$}
  \AxiomC{$P_2$}
  \AxiomC{$\cdots$}
  \AxiomC{$P_n$}
  \QuaternaryInfC{$C$}
\end{prooftree}
where the premises $P_1, P_2, \ldots, P_n$ and the conclusion $C$.
\end{example}
The inference rules needed are:
\begin{itemize}
    \item \textbf{Introduction rules} specify how to form compound propositions from simpler ones, and
    \item \textbf{Elimination rules} specify how to use compound propositions to derive information about their components.
\end{itemize}
Let's look at how we can define some connectives.
\paragraph{Conjunction ($\land$)}
\begin{itemize}
    \item Introduction
    \begin{prooftree}
        \AxiomC{$A$}
        \AxiomC{$B$}
        \RightLabel{$\land$-Intro}
        \BinaryInfC{$A \land B$}
    \end{prooftree}
    \item Elimination
    \begin{prooftree}
        \AxiomC{$A \land B$}
        \RightLabel{$\land$-Elim$_1$}
        \UnaryInfC{$A$}
    \end{prooftree}
    \qquad
    \begin{prooftree}
        \AxiomC{$A \land B$}
        \RightLabel{$\land$-Elim$_2$}
        \UnaryInfC{$B$}
    \end{prooftree}
\end{itemize}
\paragraph{Disjunction ($\lor$)}
\begin{itemize}
    \item Introduction
    \begin{prooftree}
        \AxiomC{$A$}
        \RightLabel{$\lor$-Intro$_1$}
        \UnaryInfC{$A \lor B$}
    \end{prooftree}
    \qquad
    \begin{prooftree}
        \AxiomC{$B$}
        \RightLabel{$\lor$-Intro$_2$}
        \UnaryInfC{$A \lor B$}
    \end{prooftree}
    \item Elimination (Proof by cases)
    \begin{prooftree}
        \AxiomC{$A \lor B$}
        \AxiomC{$[A] \vdash C$}
        \AxiomC{$[B] \vdash C$}
        \RightLabel{$\lor$-Elim}
        \TrinaryInfC{$C$}
    \end{prooftree}
\end{itemize}
\paragraph{Implication ($\to$)}
\begin{itemize}
    \item Introduction
    \begin{prooftree}
        \AxiomC{$[A] \vdash B$}
        \RightLabel{$\to$-Intro}
        \UnaryInfC{$A \to B$}
    \end{prooftree}
    \item Elimination (Modus Ponens)
    \begin{prooftree}
        \AxiomC{$A \to B$}
        \AxiomC{$A$}
        \RightLabel{$\to$-Elim}
        \BinaryInfC{$B$}
    \end{prooftree}
\end{itemize}

\begin{notation}
We use $A \vdash B$ (called turnstile) to designate a deduction of $B$ from $A$. 
It is employed in Gentzen’s \emph{sequent calculus} (\cite{girard1989proofs}), 
whereas in natural deduction the corresponding symbol is\[
\begin{array}{c}
A \\
\vdots \\
B
\end{array}
\]
There are some minor differences, in fact, which I don't fully understand.
The square brackets around a premise $[A]$ mean that the premise $A$ is meant to 
be \textbf{discharged} at the conclusion. The classical example is the 
introduction rule for the implication connective.
To prove an implication $A \to B$, we assume $A$ 
(shown as $[A]$), derive $B$ under this assumption, and then discharge the 
assumption $A$ to conclude that $A \to B$ holds without the assumption. 
The turnstile is predominantly used in judgments and type theory with 
the meaning of ``entails that''. 
\end{notation}

Lean has its own syntax for connectives and their relative inference rules.
For instance $A \land B$ can be presented as \lstinline[language=lean]|And(A, B)| or \lstinline[language=lean]|A ∧ B|,
. Its untroduction rule is constructed by
 \lstinline[language=lean]|And.intro _ _| or shortly
 \lstinline[language=lean]|⟨_, _⟩| (underscore are placeholder for assumptions "functions"). 
 The pair $A \land B$ can be then consumed using elimination 
rules \lstinline[language=lean]|And.left| and \lstinline[language=lean]|And.right|.
\begin{example}
    Let's look at our first Lean example
  \begin{lstlisting}[language=lean]
    example (H_A : A) (H_B : B) : (A ∧ B) := And.intro H_A H_B
  \end{lstlisting}
Lean aims to resemble the language used in mathematics. 
For instance, when defining a function or expression, one can use keywords such as 
\lstinline[language=lean]|theorem| or \lstinline[language=lean]|def|.
Here, I used \lstinline[language=lean]|example|, which is handy for defining anonymous expressions 
for demonstration purposes. 
After that comes the statement to be proved:
\begin{lstlisting}[language=lean]
  (H_A : A) (H_B : B) : (A ∧ B) 
\end{lstlisting}
Meaning given a proof of $A$ and a proof of $B$ we can form a proof of $(A \land B)$.
The operator \lstinline[language=lean]|:=| assigns a value (or return an expression) for the statement which
 "has to be a proof of it".
\lstinline[language=lean]|And.intro| is implemented as:
\begin{lstlisting}[language=lean]
  And.intro: p → q → (p ∧ q).
\end{lstlisting}
It says: if you give me a proof of $p$ and a proof of $q$, 
then i return a proof of $p \land q$.
We therefore conclude the proof by directly giving 
\lstinline[language=lean]|And.intro H_A H_B|.
Here another way of writing the same statment.
\begin{lstlisting}[language=lean]
  example (H_p : p) (H_B : B) : And(A, B) := ⟨H_p, H_B⟩
\end{lstlisting}
\end{example}

This system of inference rules allows us to construct proofs in an 
algorithmic and systematical way, organized in what is called a \textbf{proof tree}. 
To reduce complexity, we follow a 
\textbf{top-down} approach (see \cite{thompson1999types} and 
\cite{nordstrom1990programming}).
This methodology forms the basis of \textbf{proof assistants} like Lean, 
Coq, and Agda, which help 
verify the correctness of mathematical proofs by checking each step 
against these rules.
We will see later that Lean, in fact, provides an info view of the proof tree 
which helps us 
understand and visualize 
the proof structure.
Let's examine a concrete example of a proof.
\begin{example}[Associativity of Conjunction]
We prove that $(A \land B) \land C$ implies $A \land (B \land C)$.
First, from the assumption $(A \land B) \land C$, we can derive $A$:
\begin{prooftree}
  \AxiomC{$(A \land B) \land C$}
  \RightLabel{$\land E_1$}
  \UnaryInfC{$A \land B$}
  \RightLabel{$\land E_1$}
  \UnaryInfC{$A$}
\end{prooftree}
Second, we can derive $B \land C$:
\begin{prooftree}
  \AxiomC{$(A \land B) \land C$}
  \RightLabel{$\land E_1$}
  \UnaryInfC{$A \land B$}
  \RightLabel{$\land E_2$}
  \UnaryInfC{$B$}
  \AxiomC{$(A \land B) \land C$}
  \RightLabel{$\land E_2$}
  \UnaryInfC{$C$}
  \RightLabel{$\land I$}
  \BinaryInfC{$B \land C$}
\end{prooftree}
Finally, combining these derivations we obtain $A \land (B \land C)$:
\begin{prooftree}
  \AxiomC{$(A \land B) \land C \vdash A$}
  \AxiomC{$(A \land B) \land C \vdash B \land C$}
  \RightLabel{$\land I$}
  \BinaryInfC{$A \land (B \land C)$}
\end{prooftree}
\end{example}
\begin{example}[Lean Implementation]
Let us now implement the same proof in Lean.
\begin{lstlisting}[language=Lean, caption=Associativity of Conjunction in Lean]
theorem and_associative : (A ∧ B) ∧ C → A ∧ (B ∧ C) :=
  fun h : (A ∧ B) ∧ C =>
    -- First, from the assumption (A ∧ B) ∧ C, we can derive A:
    have ab : A ∧ B := h.left -- extracts (derive) a proof of (A ∧ B) from the assumption
    have a : A := ab.left -- extracts A from (A ∧ B)
    -- Second, we can derive B ∧ C (here we only extract b and c and combine them in the next step) 
    have c : C := h.right
    have b : B := ab.right
    -- Finally, combining these derivations we obtain A ∧ (B ∧ C)
    show A ∧ (B ∧ C) from ⟨a, ⟨b, c⟩⟩
\end{lstlisting}
We introduce the \lstinline[language=lean]|theorem| with the name 
\lstinline[language=lean]|and_associative|, 
which can be referenced in subsequent proofs. 
The type signature \lstinline[language=lean]|(A ∧ B) ∧ C → A ∧ (B ∧ C)| 
represents our logical implication.
The \lstinline[language=lean]|:=| operator introduces the 
proof term that establishes the theorem's validity. 
In the previous code example this proof was directly given.
Here, we construct it using the a function with the \lstinline[language=lean]|fun| keyword. 
Why a function? We have already encountered the Curry-Howard correspondence in Lean 
previously, though without explicitly stating it. 
According to this correspondence, a proof of an implication can be 
understood as a function that takes a hypothesis as input and produces 
the desired conclusion as output. We will revisit this concept in more 
detail later.
The \lstinline[language=lean]|have| keyword introduces local 
lemmas within our proof scope, allowing us to break down complex 
reasoning into manageable intermediate steps, mirroring our natural deduction proof.
Finally, the \lstinline[language=lean]|show| keyword presents our final result. 
\end{example}

To capture more complex mathematical ideas, we extend our system from 
propositional logic to \textbf{predicate logic}.  
A \textbf{predicate} is a statement or proposition that depends on a variable.
In propositional logic we represent a proposition simply by $P$.  
In predicate logic, this is generalized: a predicate is written as $P(a)$, 
where $a$ is a variable. Notice that a predicate is just a fuicntion.
This extension allows us to introduce \textbf{quantifiers}:  
$\forall$ (``for all'') and $\exists$ (``there exists'').  
These quantifiers express that a given formula holds either for every object 
or for at least one object, respectively.

In Lean if \lstinline[language=lean]|α| is any type, we can represent a 
unary predicate \lstinline[language=lean]|P| on \lstinline[language=lean]|α|  as 
an object of type \lstinline[language=lean]|α → Prop|.
\lstinline[language=lean]|Prop| stands for \emph{proposition}, and it is an 
essential component of Lean’s type system.
For now, we can think of it as a special type; somewhat 
paradoxically, a type of types. ( add a discussion on decidability?)
Thus given an \lstinline[language=lean]|x : α| (an element with type signature ) should  

When introducing variables into a formal language we must keep in mind that the specific choice 
of a variable name can be substituted without 
changing the meaning of the predicate or statement. This should feel familiar from mathematics, 
where the meaning of an expression does not depend on the names we assign to variables.
Some variables are \textbf{bound} (constrained), while others remain \textbf{free} 
(arbitrary, in programming often called "dummy" variables). 
When substituting variables, it is important to ensure that this distinction is preserved.
This phenomenon, called \textbf{variable capture}, parallels familiar mathematical practice: 
if $f(x) \coloneqq \int_1^2 \frac{dt}{x-t}$, then $f(t)$ equals $\int_1^2 \frac{ds}{t-s}$, 
not the ill-defined $\int_1^2 \frac{dt}{t-t}$. The same principle applies to logic.
Variable capture occurs when a free variable becomes bound by a quantifier during substitution, 
fundamentally changing the meaning of the expression. For example, consider
\[
\exists y.\,(y > x).
\]
This states that for a given $x$ there exists a $y$ such that $y > x$. 
If we naively substitute $y+1$ for $x$, we would obtain
\[
\exists y.\,(y > y+1),
\]
where the $y$ in $y+1$ has been \textbf{captured} by the quantifier $\exists y$. 
This transforms the original statement from "there exists some $y$ greater th
an the free variable $x$" into the always-false statement 
"there exists some $y$ greater than itself plus one."

To avoid capture, we must use \textbf{capture-avoiding substitution}. 
In the above example, we would first rename the bound variable to something fresh 
(say $z$), obtaining $\exists z.\,(z > x)$, and then safely substitute to get $\exists z.\,(z > y+1)$.
In programming, this concept is also familiar through block-structured code and scope. 

% In Lean, for instance, when you are inside a block such as a function, theorem, or section, 
% all the variables inside this context are bound to this specific scope (block) and cannot 
% be referenced outside. 
We use the notation $\phi[t/x]$ for \textbf{substitution}, meaning all occurrences of the free 
variable $x$ in formula $\phi$ are replaced by term $t$.

We can now present the inference rules for quantifiers.
\paragraph{For all ($\forall$)}
\begin{itemize}
    \item Introduction
    \begin{prooftree}
    \AxiomC{$A$}
    \RightLabel{$\forall I$}
    \UnaryInfC{$\forall x.\,A$}
    \end{prooftree}
    \item Elimination
    \begin{prooftree}
    \AxiomC{$\forall x.\,A$}
    \RightLabel{$\forall E$}
    \UnaryInfC{$A[t/x]$}
    \end{prooftree}
\end{itemize}
\paragraph{Exixst ($\exists$)}
  \begin{itemize}
    \item Introduction
    \begin{prooftree}
    \AxiomC{$A[t/x]$}
    \RightLabel{$\exists I$}
    \UnaryInfC{$\exists x.\,A$}
    \end{prooftree}
    \item Elimination
    \begin{prooftree}
    \AxiomC{$[A] \vdash B$}
    \RightLabel{$\forall E$}
    \UnaryInfC{$B$}
    \end{prooftree}
\end{itemize}
Here i need to explain in words the mewianing odft thje subsitution iin the inference rules, 
and the froall is a kind of 
chain of and and exist a chain of or that neds a witness and a predicxatg proving that wirtnteess.


\begin{example}
  Lean espresses quantifiers with dependent types.
  \begin{lstlisting}[language=Lean, caption=For All]
  ∀ (x : X), P x
  \end{lstlisting}

  \begin{lstlisting}[language=Lean, caption=Exists]
  ∃ (x : X), P x
  \end{lstlisting}
\end{example}

Under the CH correspondence, a we can get a proof of $\forall x. P(x)$ by an arbitrary $P(a)$.
highlithing tha fact that a proof is just a function


Good point for introducing polymorhpism in Lean.
It could be nice to give a proof of transitivity and what is a relation in type theory. ( a function)
% \textbf{Three syntactic forms:}
% \begin{itemize}
%     \item Explicit: \lstinline[language=lean]|∀ (x : X), P x|
%     \item Type inference: \lstinline[language=lean]|∀ x, P x| (when Lean can infer the type)
%     \item Implicit binding: \lstinline[language=lean]|∀ \{x : X\}, P x| (Lean infers the value automatically)
% \end{itemize}

% \textbf{Proof techniques:}
% \begin{itemize}
%     \item \textbf{Introduction}: Use the \lstinline[language=lean]|intro| tactic to introduce an arbitrary element
%     \item \textbf{Elimination}: Apply the hypothesis to a specific value using function application
%     \item \textbf{Specialization}: Use the \lstinline[language=lean]|specialize| tactic to instantiate a general hypothesis
% \end{itemize}

% \textbf{Example:}
% \begin{verbatim}
% theorem universal_example : ∀ (n : ℕ), n + 0 = n := by
%   intro n
%   rfl
% \end{verbatim}

% \subsection{Existential Quantifier (\(\exists\))}

% The existential quantifier \(\exists\) states that there exists at least one element of a type for which a property holds. In Lean, it is typed as \texttt{\\exists} with the syntax:

% \begin{verbatim}
% ∃ (x : X), P x
% \end{verbatim}

% This reads as "there exists some \texttt{x} of type \texttt{X} such that \texttt{P x} holds."

% \textbf{Important note:} Unlike the universal quantifier, the existential quantifier does not support implicit binding. 
% Writing \texttt{∃ \{x : X\}, P x} results in an error.

% \textbf{Internal representation:} The existential quantifier is defined as an inductive type:
% \begin{verbatim}
% inductive Exists.{u} : {α : Sort u} → (α → Prop) → Prop
% constructors:
% Exists.intro : ∀ {α : Sort u} {p : α → Prop} (w : α), p w → Exists p
% \end{verbatim}

% \textbf{Proof techniques:}
% \begin{itemize}
%     \item \textbf{Introduction}: Use \texttt{Exists.intro} with a witness and proof: \texttt{⟨witness, proof⟩}
%     \item \textbf{Elimination}: Use \texttt{cases} or \texttt{match} to extract the witness and proof
% \end{itemize}

% \textbf{Example:}
% \begin{verbatim}
% theorem existential_example : ∃ (n : ℕ), n > 0 :=
%   ⟨1, Nat.zero_lt_succ 0⟩
% \end{verbatim}
