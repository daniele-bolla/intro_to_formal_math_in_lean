\documentclass[12pt,a4paper]{book}
\usepackage{thesis}
\usepackage{hyperref}
\usepackage{stmaryrd} 
\usepackage{listings}
\usepackage{amssymb}
\usepackage{amsmath}
\usepackage{mathtools}
\usepackage{bussproofs}
\usepackage{color}
\usepackage[backend=biber,style=alphabetic]{biblatex}
\definecolor{codegreen}{rgb}{0,0.6,0}
\definecolor{codegray}{rgb}{0.5,0.5,0.5}
\definecolor{codepurple}{rgb}{0.58,0,0.82}
\definecolor{backcolour}{rgb}{0.95,0.95,0.92}


\addbibresource{bibliography.bib}

\definecolor{keywordcolor}{rgb}{0.7, 0.1, 0.1}   % red
\definecolor{tacticcolor}{rgb}{0.0, 0.1, 0.6}    % blue
\definecolor{commentcolor}{rgb}{0.4, 0.4, 0.4}   % grey
\definecolor{symbolcolor}{rgb}{0.0, 0.1, 0.6}    % blue
\definecolor{sortcolor}{rgb}{0.1, 0.5, 0.1}      % green
\definecolor{attributecolor}{rgb}{0.7, 0.1, 0.1} % red
\def\lstlanguagefiles{lstlean.tex}
% set default language
\lstset{language=lean}

% Optional packages for your specific needs
% \usepackage{tikz}           % For diagrams
% \usepackage{pgfplots}       % For plots
% \usepackage{listings}       % For code listings
% \usepackage{algorithm2e}    % For algorithms
% \usepackage{hyperref}       % For hyperlinks (load last)

% Thesis metadata - CUSTOMIZE THESE
\title{Introduction to Formal Mathematics in Lean  with an example in Topology}
\author{Daniele Bolla}
\matrikelnr{21-694-187}
\supervisor{Prof. David Loeffler}
\submitdate{December 31, 2025}
% \thesistype{Bachelor Thesis}  % Optional: defaults to "Bachelor Thesis"
% \faculty{Faculty of Mathematics and Computer Science}  % Optional: already set
% \universitylogo{FernUni}      % Optional: path to logo file

% Theorem environments
% \newtheorem{thm}{Theorem}[chapter]
% \newtheorem{cor}[thm]{Corollary}
% \newtheorem{prop}[thm]{Proposition}
% \newtheorem{lem}[thm]{Lemma}
% \theoremstyle{definition}
% \newtheorem{defn}[thm]{Definition}
% \newtheorem{axiom}[thm]{Axiom}
% \theoremstyle{remark}
% \newtheorem{rmk}[thm]{Remark}
% \newtheorem{exmp}[thm]{Example}
% \newtheorem{exer}[thm]{Exercise}

% Custom commands for mathematics
\newcommand{\R}{\mathbb{R}}
\newcommand{\N}{\mathbb{N}}
\newcommand{\Z}{\mathbb{Z}}
\newcommand{\Q}{\mathbb{Q}}
\newcommand{\C}{\mathbb{C}}
\newcommand{\set}[1]{\{#1\}}
\newcommand{\abs}[1]{\lvert#1\rvert}
\newcommand{\norm}[1]{\lVert#1\rVert}
\newtheorem{theorem}{Theorem}[section]
\newtheorem{proposition}[theorem]{Proposition}
\newtheorem{definition}[theorem]{Definition}  
\newtheorem{remark}[theorem]{Remark}
\newtheorem{example}[theorem]{Example}
\newtheorem{notation}[theorem]{Notation}
\newtheorem{note}[theorem]{Note}
\numberwithin{figure}{theorem}
\begin{document}

% Title page
\makethesistitle

% Declaration
\begin{declaration}
    I hereby declare that this thesis is my own work, and that no other sources have been used except those indicated in the bibliography. I confirm that generative AI tools were not used in writing any part of this thesis, though they may have been used for preliminary research purposes only.

    \vspace{3cm}

    \noindent
    Place, Date \hfill Signature

    \vspace{1cm}

    \noindent
    \rule{4cm}{0.5pt} \hfill \rule{6cm}{0.5pt}
\end{declaration}

% Abstract
\begin{thesisabstract}
    % This bachelor thesis explores applications of fixed point theory
    % in mathematical analysis, with particular focus on
    % the Banach fixed point theorem and its generalizations.
    % The work investigates both theoretical foundations and practical
    % applications of fixed point theorems in solving equations and
    % optimization problems.

    % The main contributions of this thesis include a comprehensive
    % survey of classical fixed point theorems, detailed proofs of key
    % results, and several novel applications to problems in numerical
    % analysis and differential equations. We demonstrate how fixed
    % point theory provides a unifying framework for understanding
    % convergence properties of iterative methods and existence theorems
    % for solutions of functional equations.

    % The thesis concludes with a discussion of recent developments in the
    % field and potential directions for future research, including extensions
    % to non-metric spaces and applications to machine learning algorithms.

    This thesis serves as an introduction to formal mathematics using the
    Lean proof assistant. After a brief overview of the Curry--Howard
    correspondence, we explore how mathematical structures and
    properties are defined and utilized in Lean and Mathlib.
    Finally, we present a formalization of the topologist's sine curve,
    which has been merged into the Mathlib library,
    contributing to the broader Lean community.

    During the work on this thesis, I have explored new concepts
    in both mathematics and computer science, starting from logic and
    type theory within the Curry--Howard correspondence.
    The practice of formal and constructive mathematics has influenced
    my approach to mathematical reasoning. In constructing the rational
    numbers in Lean, I explored the algebraic construction using quotients,
    which is reflected in Lean through quotient types. I further examined
    how algebraic structures such as groups and rings are defined using
    structures and type classes. For more technical depth, I studied
    filters to generalize the notion of convergence in topology.
    The formalization of the topologist's sine curve has deepened
    my understanding of connectedness, closures, and continuity in topology.


\end{thesisabstract}

% Acknowledgments
\begin{acknowledgments}
    I would like to express my sincere gratitude to Prof. David Loeffler for
    the consistent support.
    His mentorship has been particualary beneficial in enlightening my path through
    formal mathematics, shaping the thesis and giving me a chance to
    actually contribute to Lean community.

    Moreover, I deeply thank FernUni for the opportunity to pursue my
    studies. I feel honored to have been able to study
    mathematics at this stage of my life through a flexible and
    high-quality program.

    The expertise of all professors have
    been exceptional, and I hope to continue my studies in the future.
\end{acknowledgments}

% Table of contents
\tableofcontents

\section{Introduction}
This serves as a brief starting point for understanding how the Curry-Howard correspondence appears in Lean, 
as well as being an introduction to the language itslef. 
Lean is both a \textbf{functional programming language} and a \textbf{theorem prover}.
We'll focus primarily on its role as a theorem prover. 
But what does this mean, and how can that be achieved?

A programming language defines a \textbf{set of rules, semantics, and syntax} for writing programs. 
To achieve a goal, a programmer must write a program that meets given specifications. 
There are two primary approaches: \textbf{program derivation} and \textbf{program verification} 
( \cite{nordstrom1990programming} Section 1.1).
In \textbf{program verification}, the programmer first writes a program and then proves it meets 
the specifications. This approach checks for errors at \textbf{runtime} when the code executes.
In \textbf{program derivation}, the programmer writes a proof that a program with certain properties exists, 
then extracts a program from that proof. This approach enables specification 
checking during \textbf{compilation}, catching errors before execution.
This distinction corresponds to \textbf{dynamic} versus \textbf{static} \textbf{type systems}. 
Most programming languages employ both approaches. For example, C checks operations on \texttt{int} 
at compile time but requires the programmer to ensure correctness with \texttt{void*} at run time.
Lean emphasizes program derivation.
Its type system is highly advanced and flexible, allowing the expression and verification 
of a wide range of mathematical statements, and this is what makes Lean a powerful \textbf{theorem prover}. 
Lean's type system is based on \textbf{Type Theory}, a branch of marthematics and logic tahta aims to provide a 
foundation for all mathematics and wich is a programming language itself.

It's important to note that Type Theory is not a single, unified theory, but rather a family of 
related theories with various extensions and ongoing developments and rich hiostorical ramifications. 
Creating a language like Lean require
careful consideration of which rules and features to include.
We shall give a brief overview of the historical development of type theory, and an a introduction on 
what comes next.

(From \cite{carneiro2019typetheorylean})
Type theory emerged as a fundamental response to Bertrand Russell's paradox. 
Considers the set $S = \{x \mid x \notin x\}$ 
(the set of all sets that do not contain themselves). This is a paradoxical construction, 
leading to the contradiction $S \in S \iff S \notin S$. 
Ernst Zermelo and Fraenkel addressed the contradiction by introducing Zermelo-Fraenkel set theory (ZFC), 
which became the standard axiomatization in modern mathematics. 
ZFC provides an untyped but stratified view of the mathematical universe, 
maintaining classical logical principles while avoiding paradoxes through careful axiomatization.
Russell chose a fundamentally different path. He recognized expressions 
like $A(A)$ or $x \in x$ as ill-typed, introducing his theory of types.
Hs's first systematic response was \textbf{Ramified Type Theory}, wich turned out to be problematic.
In the 1930s, Alonzo Church developed \textbf{Lambda Calculus} as a foundation for mathematics, 
initially pursuing a type-free approach. However, Church's original untyped system suffered from 
inconsistencies. To address these issues, 
Church introduced the \textbf{Simply Typed Lambda Calculus} in 1940 (\cite{church1940formulation}).  
This system is a version of \textbf{Simple Type Theory}, a framework able to replace 
set-theory and propositional logic.
Lambda calculus influenced the development of many programming languages as being a foundation for
functional programming.
Per Martin-L\"{o}f revolutionized type theory in the 1970s by introducing 
\textbf{dependent types} that can depend on values of other types.
Think for instance of a of vectors of length $n$ or a sequnce of $n$ elements.
\textbf{Dependent Type Theory} extends the expressive power of type systems 
by allowing the representation of quantifiers, 
providing a framework capable of replacing set theory and predicate logic.
Dependent Type Theory is a derivation of \textbf{Martin-L\"{o}f Type Theory} 
(also known as \textbf{Intuitionistic Type Theory}).
Martin-L\"{o}f's system embraced constructive principles, requiring that the existence of mathematical 
objects be demonstrated through explicit construction rather than classical proof by contradiction. 
Martin-L\"{o}f Type Theory also introduced \textbf{identity types} to represent equality.
In the 1980s, Thierry Coquand and G\'{e}rard Huet introduced the \textbf{Calculus of Constructions} (CoC), 
synthesizing insights from Martin-L\"{o}f's dependent type theory with higher-order \textbf{polymorphism}. 
The Calculus of Constructions served as the theoretical foundation for the Coq proof assistant, 
one of the most influential interactive theorem provers.
The original CoC was later extended with \textbf{inductive types} to form the 
\textbf{Calculus of Inductive Constructions} (CIC). Inductive types allow for the definition 
of data structures like natural numbers, lists, and trees. 
The Lean theorem prover, developed by Leonardo de Moura and others, is also based on CIC 
but incorporates several important refinements and differences from Coq's implementation. 

A central insight in type theory is the \textbf{Curry-Howard correspondence}, 
which establishes a profound connection between logic and computation. 
Also known as the \textbf{propositions-as-types} principle, this correspondence 
represents one of the most elegant discoveries in the foundations 
of mathematics and computer science.
It serves also well as a good introduction to type theory, and will be used in this discussion.
Nontheless it continuelsy shows new applications and interpretations in modern type theories.
The Curry-Howard correspondence was independently discovered by multiple researchers.
\textbf{Haskell Curry} (1934) first observed the connection between combinatory logic and 
Hilbert-style proof systems.
\textbf{William Alvin Howard} (1969) significantly extended the correspondence to natural deduction 
and the simply typed lambda calculus in his seminal work ``The Formulae-as-Types Notion of Construction.'' 
The correspondence was further developed through \textbf{N.G. de Bruijn's AUTOMATH system} (1967), 
which was the first working proof checker and demonstrated the practical viability of mechanical 
proof verification. Amongst its technical innovations are a discussion of the
irrelevance of proofs when working in a classical context, which is one
of the reasons advanced by de Bruijn for the separation between the notions of 
type and prop in the system \cite{thompson1999types}. Lean also adopts this separation .
\textbf{Per Martin-L\"{o}f's type theory} extended the correspondence to
dependent types, allowing for the representation of quantifiers and identity types.
Modern proof assistants like Coq, Lean, Agda, and Isabelle/HOL all leverage 
variants of the Curry-Howard correspondence to enable formal verification of mathematical theorems 
and software correctness properties.

% Throughout this discussion, we will introduce how to read Type Theory notation and explore key concepts 
% including:
% \begin{itemize}
% \item Impredicativity
% \item Decidability
% \item Computability
% \item Derivability
% \item Soundness
% \item Completeness
% \item Intyensionality 
% \item Extensionality
% \end{itemize}


\chapter{Logic and Proposition as Types}

\section{First Order Logic}
Logic is the study of reasoning, branching into various systems.
We refer to \textbf{classical logic} as the one that underpins much
of traditional mathematics.
It's the logic of truthtables.
We first introduce \textbf{propositional logic}, which is the simplest
form of classical logic.
Later we will extend this to \textbf{predicate (or first-order) logic}, which includes
\textbf{predicates} and \textbf{quantifiers}.
In this setting, a \textbf{proposition} is a statement that is either true or false,
and a \textbf{proof} is a logical argument that establishes the truth of a
proposition.
Propositions can be combined with logical \textbf{connectives} such as ``and'' ($\wedge$),
``or'' ($\vee$), ``not'' ($\neg$),``false'' ($\bot$), ,``true'' ($\top$)
``implies'' ($\Rightarrow$),  and ``if and only if'' ($\Leftrightarrow$).
These connectives allow the creation of complex or compound propositions.
Here how connectives are defined in Lean:
\begin{example}[LogicaL connectives in Lean]\mbox{}
  \begin{lstlisting}[language=lean]
    #check And (a b : Prop) : Prop
    #check Or (a b : Prop) : Prop
  \end{lstlisting}
  \lstinline[language=lean]|Prop| is the proposition type mentioned before.
\end{example}
Logic is often formalized through a framework known as the \textbf{natural deduction system},
developed by Gentzen in the 1930s (\cite{wadler2015propositions}).
This approach brings logic closer to a computable, algorithmic system.
It specifies rules for deriving
\textbf{conclusions} from \textbf{premises} (assumptions from other propositions),
called \textbf{inference rules}.
\begin{example}[Deductive style rule]
  Here is an hypothetical example of inference rule (\cite{nordstrom1990programming}, page 35).
  \begin{prooftree}
    \AxiomC{$P_1$}
    \AxiomC{$P_2$}
    \AxiomC{$\cdots$}
    \AxiomC{$P_n$}
    \QuaternaryInfC{$C$}
  \end{prooftree}
  Where the $P_1, P_2, \ldots, P_n$, above the line, are hypothetical
  premises and, the hypothetical conclusion $C$ is below the line.
\end{example}
The inference rules needed are, \textbf{introduction rules} wich specify
how to form compound propositions from simpler ones, and
\textbf{elimination rules} needed to derive information about them.
Let's look at how we can define some connectives, first using
natural deduction (from \cite{thompson1999types}, Section 1.1).
\paragraph{Conjunction ($\land$)}
\paragraph{Introduction Rule}
\begin{prooftree}
  \AxiomC{$A$}
  \AxiomC{$B$}
  \RightLabel{$\land$-Intro}
  \BinaryInfC{$A \land B$}
\end{prooftree}
\paragraph{Elimination Rule}
\mbox{}\\[0.5em]
\noindent
\begin{minipage}[t]{0.48\textwidth}
  \vspace{0pt}
  \begin{prooftree}
    \AxiomC{$A \land B$}
    \RightLabel{$\land$-Elim$_1$}
    \UnaryInfC{$A$}
  \end{prooftree}
\end{minipage}\hfill
\begin{minipage}[t]{0.48\textwidth}
  \vspace{0pt}
  \begin{prooftree}
    \AxiomC{$A \land B$}
    \RightLabel{$\land$-Elim$_2$}
    \UnaryInfC{$B$}
  \end{prooftree}
\end{minipage}
\paragraph{Disjunction ($\lor$)}
\paragraph{Introduction Rule}
\mbox{}\\[0.5em]
\noindent
\begin{minipage}[t]{0.48\textwidth}
  \vspace{0pt}
  \begin{prooftree}
    \AxiomC{$A$}
    \RightLabel{$\lor$-Intro$_1$}
    \UnaryInfC{$A \lor B$}
  \end{prooftree}
\end{minipage}\hfill
\begin{minipage}[t]{0.48\textwidth}
  \vspace{0pt}
  \begin{prooftree}
    \AxiomC{$B$}
    \RightLabel{$\lor$-Intro$_2$}
    \UnaryInfC{$A \lor B$}
  \end{prooftree}
\end{minipage}
\paragraph{ Elimination (Proof by cases)}
\begin{prooftree}
  \AxiomC{$A \lor B$}
  \AxiomC{$[A] \vdash C$}
  \AxiomC{$[B] \vdash C$}
  \RightLabel{$\lor$-Elim}
  \TrinaryInfC{$C$}
\end{prooftree}
\paragraph{Implication ($\to$)}
\paragraph{Introduction Rule}
\begin{prooftree}
  \AxiomC{$[A] \vdash B$}
  \RightLabel{$\to$-Intro}
  \UnaryInfC{$A \to B$}
\end{prooftree}
\paragraph{ Elimination (Modus Ponens)}
\begin{prooftree}
  \AxiomC{$A \to B$}
  \AxiomC{$A$}
  \RightLabel{$\to$-Elim}
  \BinaryInfC{$B$}
\end{prooftree}
\begin{notation}
  We use $A \vdash B$ (called turnstile) to designate a
  deduction of $B$ from $A$.
  It is used in judgments and type theory with
  the meaning of ``entails that''.
  The square brackets around a premise $[A]$ mean that the premise $A$ is meant to
  be \textbf{discharged} at the conclusion. The classical example is the
  introduction rule for the implication connective.
  To prove an implication $A \to B$, we assume $A$
  (shown as $[A]$), derive $B$ under this assumption, and then discharge the
  assumption $A$ to conclude that $A \to B$ holds without the assumption.
\end{notation}
\section{Primitive Types}
Type theory employs this porocedure too,
by referring to deduction
rules as \textbf{judments}.
A type judgment has the form $\Gamma \vdash t : T$,
meaning: under \textbf{context} $\Gamma$ (a list of typed variables),
the term $t$ has type $T$.
Using formal inference rules in the type judgment
system, such as \textbf{introduction} and \textbf{elimination} rules,
we can construct new compound types from existing ones.
\begin{example}[Judgment style rule]
  \mbox{}
  \begin{prooftree}
    \AxiomC{$\Gamma \vdash$}
    \AxiomC{$p_1:P_1$}
    \AxiomC{$p_2:P_2$}
    \AxiomC{$\cdots$}
    \AxiomC{$P_n$}
    \QuinaryInfC{$C$}
  \end{prooftree}
\end{example}
Technically, there are two more inference rules that we will not consider in this setting:
\textbf{formation rules}, used to declare that a type is well-defined, and
\textbf{computation rules}, which specify how a term will be evaluated.
Moreover, without going too deep into the jargon,
one specific judgment is
$\Gamma \vdash A \equiv B\ \text{type}$, which means ``types $A$ and $B$ are
\textbf{judgmentally (or definitionally) equal} in context $\Gamma$.''
Similarly for terms, $\Gamma \vdash t_1 \equiv t_2 : A$ means ``terms $t_1$ and $t_2$ are
judgmentally equal of type $A$ in context $\Gamma$.''
\paragraph{Brief explanation of equality in type theory}

In Lean, the operator \lstinline[language=lean]|:=|
stands for definitional equality and is used by the kernel to verify proof equality.
A mathematician proving a theorem applies a series of \textbf{reduction rules} to
simplify the proof.
Similarly, one can think of this
computational reduction process in formal verification.
However, a computer cannot simply employ the same informal
approach to equality that a mathematician might use intuitively.
A rigorous explanation of definitional equality
goes beyond the scope of this thesis.
To state it simply: \textbf{two terms are definitionally equal when they
  reduce to the same normal for}. A \textbf{normal for} represents the most
reduced state of a term, obtained by systematically applying a
sequence of reduction rules until no further reductions are possible.
In contrast, \textbf{propositional equality} requires additional logical
bridges and propositions to establish equivalence.
Because it is grounded in logical propositions rather
than pure computation, propositional equality is not
directly computable by the type checker and must be explicitly proved.

Let's now construct new types from given types $A$ and $B$.
\paragraph{Product Type}
As a fundamental example, $A \times B$
denotes the type of pairs $(a, b)$ where $a : A$ and $b : B$,
called the \textbf{product type}.
\paragraph{Introduction Rule (pairing)}
\begin{prooftree}
  \AxiomC{$a : A$}
  \AxiomC{$b : B$}
  \BinaryInfC{$(a, b) : A \times B$}
\end{prooftree}
In Lean:
\begin{lstlisting}[language=lean]
Prod.mk a b : Prod A B   -- or A × B
(a, b) : A × B           
⟨a, b⟩ : A × B           
\end{lstlisting}
\paragraph{Elimination Rules (projections)}\mbox{}\\[0.5em]
\noindent
\begin{minipage}[t]{0.48\textwidth}
  \vspace{0pt}
  \begin{prooftree}
    \AxiomC{$p : A \times B$}
    \UnaryInfC{$\mathsf{fst}(p) : A$}
  \end{prooftree}
\end{minipage}\hfill
\begin{minipage}[t]{0.48\textwidth}
  \vspace{0pt}
  \begin{prooftree}
    \AxiomC{$p : A \times B$}
    \UnaryInfC{$\mathsf{snd}(p) : B$}
  \end{prooftree}
\end{minipage}

\noindent In Lean:
\begin{lstlisting}[language=lean]
  p.1 : A       -- or Prod.fst p
  p.2 : B       -- or Prod.snd p
\end{lstlisting}
\paragraph{Sum Type}
The \textbf{sum type} $A + B$ (also called a coproduct or disjoint union) consists of values that are
either of type $A$ (tagged with $\mathsf{inl}$) or
of type $B$ (tagged with $\mathsf{inr}$).
\paragraph{Introduction Rules (injections)}
\mbox{}\\[0.5em]
\noindent
\begin{minipage}[t]{0.48\textwidth}
  \vspace{0pt}
  \begin{prooftree}
    \AxiomC{$a : A$}
    \UnaryInfC{$\mathsf{inl}(a) : A + B$}
  \end{prooftree}
\end{minipage}\hfill
\begin{minipage}[t]{0.48\textwidth}
  \vspace{0pt}
  \begin{prooftree}
    \AxiomC{$b : B$}
    \UnaryInfC{$\mathsf{inr}(b) : A + B$}
  \end{prooftree}
\end{minipage}

\noindent In Lean:
\begin{lstlisting}[language=lean]
Sum.inl a : Sum A B   -- or A ⊕ B
Sum.inr b : Sum A B
\end{lstlisting}
\paragraph{Elimination Rule (case analysis)}
\begin{prooftree}
  \AxiomC{$p : A + B$}
  \AxiomC{$\begin{array}{c}  f : (A \implies C) \end{array}$}
  \AxiomC{$\begin{array}{c}  g : (B \implies C) \end{array}$}
  \TrinaryInfC{$\mathsf{cases}(p, f, g) : C$}
\end{prooftree}
\newpage
In Lean:
\begin{lstlisting}[language=lean]
example (p : Sum A B) (f : A → C) (g : B → C) : C := by
  cases p with
  | inl x => f x
  | inr y => g y
\end{lstlisting}
\paragraph{Function Types}
The type of the form $A \to B$, used in the sum elimination rule
represents functions from $A$ to $B$.
\paragraph{Introduction Rule (function application or lambda abstraction)}
\begin{prooftree}
  \AxiomC{$\begin{array}{c} x : A  \vdash  \Phi : B \end{array}$}
  % \UnaryInfC{$\lambda x.\Phi : A \to B$}
  \UnaryInfC{$f  : A \to B$}
\end{prooftree}
Where $f$ is a function that maps any element $x : A$ to an element $\Phi : B$.
In Lean, lambda abstraction is written using \lstinline[language=lean]|fun| or \lstinline[language=lean]|λ|:
\begin{lstlisting}[language=lean]
def identityFun (A : Type) : A → A := fun x => x
\end{lstlisting}
\paragraph{Elimination Rule (application)}
\begin{prooftree}
  \AxiomC{$f : A \to B$}
  \AxiomC{$a : A$}
  \BinaryInfC{$f(a) : B$}
\end{prooftree}
In Lean, function application is written using juxtaposition:
\begin{lstlisting}[language=lean]
example (f : A → B) (a : A) : B := f a
\end{lstlisting}
Functions are a primitive concept in type theory. We can \textbf{apply} a function
$f : A \to B$ to an element $a : A$ to obtain an element of $B$, denoted $f(a)$.
In type theory, it is common to omit the parentheses and write the application
simply as $f \, a$.

\section{The Curry Howard Isomorphism}
We have been preparing for this argument, and the reader will have surely
noticed a strong similarity when defining logical connectives
using deduction rules; they are remarkably similar to types
constructed using type judgments. For instance, function
types can be seen as implications.
This is not a coincidence, but rather a fundamental theorem
first proven by Haskell Curry and William Howard.
It forms the core of type theory and establishes
a deep connection between logic, computation, and mathematics.
\textbf{Implication} ($P \Rightarrow Q$) corresponds to the \textbf{function type} ($P \to Q$).
A proof of an implication is a function that transforms any proof
of the premise into a proof of the conclusion.
\noindent\textbf{Conjunction} ($P \land Q$) corresponds
to the \textbf{product type} ($P \times Q$).
A proof of a conjunction consists of a pair containing proofs of both conjuncts.
\noindent\textbf{Disjunction} ($P \lor Q$) corresponds
to the \textbf{sum type} ($P + Q$).
A proof of a disjunction is either a proof of the
first disjunct or a proof of the second disjunct.
Same goes for the rest of the connectives.
Lean uses inference rules and type
judgments as well as computing connectives using each related type.
For instance, $A \land B$ can be represented as \lstinline[language=lean]|And(A, B)| or \lstinline[language=lean]|A ∧ B|.
Its introduction rule is constructed by
\lstinline[language=lean]|And.intro _ _| or simply
\lstinline[language=lean]|⟨_, _⟩| (underscores are placeholders).
The pair $A \land B$ can then be consumed using elimination
rules \lstinline[language=lean]|And.left| and \lstinline[language=lean]|And.right|.

\begin{example}\label{ex:conj_intro_2}
  Let's look at a simple Lean example:
  \begin{lstlisting}[language=lean]
    example {a b : Prop} (ha : a) (hb : b) : (a ∧ b) := And.intro ha hb
  \end{lstlisting}
  Using brackets \lstinline[language=lean]|{ }|, we let Lean infer
  that $a$ and $b$ are propositions (\lstinline[language=lean]|Prop|).
  The example means that given a proof of $a$ (\lstinline[language=lean]|ha|)
  and a proof of $b$  (\lstinline[language=lean]|hb|) ,
  we can form a proof of $(a \land b)$.
  \lstinline[language=lean]|And.intro| is implemented as:
  \begin{lstlisting}[language=lean]
    And.intro : p -> q -> (p ∧ q)
  \end{lstlisting}
  It says: if you give me a proof of $p$ and a proof of $q$,
  then I return a proof of $p \land q$.
  We therefore conclude the proof by directly giving
  \lstinline[language=lean]|And.intro ha hb|.
  Here is another way of writing the same statement:
  \begin{lstlisting}[language=lean]
    example (ha : a) (hb : b) : And(a, b) := ⟨ha, hb⟩
  \end{lstlisting}
\end{example}
\noindent
For a more concrete example, let's look at how
proof normalization using a system of inference rules
corresponds to computation in Lean.
To reduce complexity of a \textbf{proof tree} in natural deduction,
one, tipically, follows a
\textbf{top-down} approach,
unfolding each component to be proved step by step.
\begin{example}[Associativity of Conjunction]
  We prove that $(A \land B) \land C$ implies $A \land (B \land C)$.
  First, from the assumption $(A \land B) \land C$, we can derive $A$:
  \begin{prooftree}
    \AxiomC{$(A \land B) \land C$}
    \RightLabel{$\land E_1$}
    \UnaryInfC{$A \land B$}
    \RightLabel{$\land E_1$}
    \UnaryInfC{$A$}
  \end{prooftree}
  Second, we can derive $B \land C$:
  \begin{prooftree}
    \AxiomC{$(A \land B) \land C$}
    \RightLabel{$\land E_1$}
    \UnaryInfC{$A \land B$}
    \RightLabel{$\land E_2$}
    \UnaryInfC{$B$}
    \AxiomC{$(A \land B) \land C$}
    \RightLabel{$\land E_2$}
    \UnaryInfC{$C$}
    \RightLabel{$\land I$}
    \BinaryInfC{$B \land C$}
  \end{prooftree}
  Finally, combining these derivations we obtain $A \land (B \land C)$:
  \begin{prooftree}
    \AxiomC{$(A \land B) \land C \vdash A$}
    \AxiomC{$(A \land B) \land C \vdash B \land C$}
    \RightLabel{$\land I$}
    \BinaryInfC{$A \land (B \land C)$}
  \end{prooftree}
\end{example}
\newpage
\begin{example}[Lean Implementation]
  Let us now implement the same proof in Lean.
  \begin{lstlisting}[language=lean]
theorem and_associative {a b c : Prop} : (a ∧ b) ∧ c → a ∧ (b ∧ c) :=
  fun h : (a ∧ b) ∧ c →
  -- First, from the assumption (a ∧ b) ∧ c, we can derive a:
  have hab : a ∧ b := h.left
  have ha : a := hab.left 
  -- Second, we can derive b ∧ c (here we only extract b and c and combine them in the next step)
  have hc : c := h.right
  have hb : b := hab.right
  -- Finally, combining these derivations we obtain a ∧ (b ∧ c)
  show a ∧ (b ∧ c) from ⟨ha, ⟨hb, hc⟩⟩
\end{lstlisting}
  We introduce the \lstinline[language=lean]|theorem| with the name
  \lstinline[language=lean]|and_associative|.
  The type signature \lstinline[language=lean]|(a ∧ b) ∧ c → a ∧ (b ∧ c)|
  represents our logical implication.
  Here, we construct the implication proof using a
  function (following the Curry Howard isomorphism) with the \lstinline[language=lean]|fun| keyword.
  The \lstinline[language=lean]|have| keyword introduces local
  lemmas within our proof scope, allowing us to break down complex
  reasoning into manageable intermediate steps,
  mirroring our natural deduction proof from before.
  Just before the keyword \lstinline[language=lean]|show|,
  the info view displays the following
  context and goal:
  \begin{lstlisting}[language=lean]
  a b c : Prop
  h : (a ∧ b) ∧ c
  hab : a ∧ b
  ha : a
  hc : c
  hb : b
  ⊢ a ∧ b ∧ c
\end{lstlisting}
  Resembling type judgments, the goal is juxtaposed after the turnstile ($\vdash$).
  What comes before it is the current context.
  Finally, \lstinline[language=lean]|show a ∧ (b ∧ c) from ⟨ha, ⟨hb, hc⟩⟩|
  asserts that we are constructing a proof of \lstinline[language=lean]|a ∧ (b ∧ c)|
  using the term \lstinline[language=lean]|⟨ha, ⟨hb, hc⟩⟩|.
  The \lstinline[language=lean]|show| keyword makes the proof
  more readable
  and ensures that the provided
  proof term matches the stated
  goal up to definitional equality.
  As mentioned already, two types (or terms) are definitionally equal in Lean when they
  are identical after computation
  and unfolding of definitions; in other words, when Lean's type checker
  can mechanically verify they are the same without requiring additional proof steps.
  Here, the goal \lstinline[language=lean]|⊢ a ∧ b ∧ c| is definitionally
  equal to \lstinline[language=lean]|a ∧ (b ∧ c)| due to how conjunction
  associates, so \lstinline[language=lean]|show| accepts this statement.
  If we had tried to use \lstinline[language=lean]|show| with a type that
  was only propositionally equal
  but not definitionally equal, Lean would reject it.
\end{example}
\section{Predicate Logic and Dependency}
To capture more complex mathematical ideas, we extend our system from
propositional logic to \textbf{predicate logic}.
A \textbf{predicate} is a statement or proposition that depends on a variable.
In propositional logic we represent a proposition simply by $P$.
In predicate logic, this is generalized.
A predicate is written as $P(a)$,
where $a$ is a variable. Notice that a predicate is just a function.
This extension allows us to introduce \textbf{quantifiers}:
$\forall$ (``for all'') and $\exists$ (``there exists'').
These quantifiers express that a given formula holds either for every object
or for at least one object, respectively.
In Lean if \lstinline[language=lean]|α| is any type, we can represent a
predicate \lstinline[language=lean]|P| on \lstinline[language=lean]|α| as
an object of type \lstinline[language=lean]|α → Prop|.
Thus given an \lstinline[language=lean]|x : α| (an element
with type \lstinline[language=lean]|α| )
\lstinline[language=lean]|P(x) : Prop| would be representative of a proposition
holding for \lstinline[language=lean]|x|.
We can give an informal reading of the quantifiers as infinite logical operations:
\begin{align*}
  \forall x.\,P(x) & \equiv P(a) \land P(b) \land P(c) \land \ldots \\
  \exists x.\,P(x) & \equiv P(a) \lor P(b) \lor P(c) \lor \ldots
\end{align*}
The dot symbol following the quantifier, as in $\forall x.,$, binds
every occurrence of the variable $x$ in the expression $P(x)$.
The expression $\forall x.\, P(x)$ can be understood as a generalized form of conjunction.
It expresses that $P$ holds for all possible values of $x$.
Similarly, $\exists x.\, P(x)$ is a generalized disjunction, expressing that $P$ holds
for at least one value of $x$.
Under the Curry-Howard isomorphism, universal quantifiers correspond to
\textbf{dependent function types} (also called Pi types, written $\Pi$),
while existential quantifiers correspond to
\textbf{dependent pair types} (also called Sigma types, written $\Sigma$).
These are constructs from dependent type theory, which provides a way to interpret
predicates or, more generally, types depending on some data or variable.
Technically the correspondence is not that immediate and actually Lean implements,
\lstinline[language=lean]|Exists| and \lstinline[language=lean]|Forall|
using as inductive types (this follows also for the previously defined connectives).
This time we are not going to involve deduction rules or type judgments.
Instead, we will extend the isomorphism
to quantifiers directly
by presenting the Lean syntax.
\begin{example}[Quantifiers in Lean]
  Lean expresses quantifiers as follows:
  \begin{lstlisting}[language=lean]
variable (X : Type) (P : X → Prop)
 (∀ (x : X), P x) -- ∀ corresponds to Pi type Π
 (∃ (x : X), P x) -- ∃ corresponds to Sigma type Σ
  \end{lstlisting}
\end{example}
\newpage
\begin{example}[Universal introduction in Lean]
  The \textbf{universal introduction rule} allows us to prove $\forall x, P(x)$
  by proving $P(x)$ for an \textbf{arbitrary} $x$.
  In Lean, this corresponds to constructing a function:
  \begin{lstlisting}[language=lean]
  example : ∀ n : Nat, n ≥ 0 :=
    fun n => Nat.zero_le n 
  \end{lstlisting}
  From the \lstinline[language=lean]|Nat| module in Lean, we use
  \lstinline[language=lean]|zero_le|, a built-in theorem that already
  proves the statement.
\end{example}
\begin{example}[Universal elimination in Lean]
  The \textbf{universal elimination rule} allows us to instantiate
  a universally quantified statement with a specific value.
  In Lean, this is simply function application:
  \begin{lstlisting}[language=lean]
  example (h : ∀ n : Nat, n ≥ 0) : 5 ≥ 0 :=
    h 5
  \end{lstlisting}
\end{example}
\begin{example}[Existential introduction in Lean]
  When introducing an \textbf{existential} proof,
  we need a \textbf{pair} consisting
  of a witness and a proof that this witness
  satisfies the statement.
  \begin{lstlisting}[language=lean]
  example (x : Nat) (h : x > 0) : ∃ y, y < x :=
    ⟨0, h⟩
  \end{lstlisting}
  Notice that \lstinline[language=lean]|⟨0, h⟩| is a product type holding
  data (the witness) and a proof that it satisfies the property.
\end{example}
\begin{example}[Existential elimination in Lean]
  The \textbf{existential elimination rule}
  (\lstinline[language=lean]|Exists.elim|) allows us to prove a proposition $Q$
  from $\exists x, P(x)$ by showing that $Q$ follows from $P(w)$
  for an \textbf{arbitrary} value $w$.
  The existential quantifier can be interpreted as an infinite disjunction,
  so existential elimination naturally corresponds to a \textbf{proof by cases}
  (with a single case).
  In Lean, this is done using \textbf{pattern matching}
  with \lstinline[language=lean]|cases|:
  \begin{lstlisting}[language=lean]
  example (h : ∃ n : Nat, n > 0) : ∃ n : Nat, n > 0 := by
    cases h with
    | intro witness proof => ⟨witness, proof⟩
  \end{lstlisting}
\end{example}

\section{Constructive Mathematics}

Mathematicians have traditionally worked within \textbf{classical logic},
using \textbf{sets} as the primary means of structuring mathematical objects.
In contrast, \textbf{type theory} does not take sets as its primitive notion,
nor is it built by first applying logic and then adding structure.
Instead, logic is internal to type theory and is based on \textbf{constructive}
(or \textbf{intuitionistic}) logic, introduced by Brouwer and formalized by
Heyting (see, e.g., \cite{girard1989proofs}, Ch 1, page 6).
A major point of departure from classical logic is that, in constructive logic,
statements cannot simply be classified as true or false;
their truth depends on whether a proof exists.
There are many conjectures, such as the Riemann Hypothesis,
for which we do not yet know whether a proof or disproof exists,
so we cannot say whether they are true or false.
Consequently, constructive logic does not universally accept principles such
as the \textbf{axiom of choice} or the \textbf{law of excluded middle}
(every proposition is either true or false) as axioms.
As a consequence, proof by contradiction does not work in this setting
without additional justification.
Constructive logic emphasizes that a statement is only
considered true if we can explicitly provide a \textbf{witness} for it.
This is what makes constructive mathematics inherently \textbf{computable}.
% We already touched on this concept in the previous section.
% In particular, we presented the logical connectives via the
% Brouwer--Heyting--Kolmogorov (BHK) interpretation.
We also emphasized that, constructively,
a proof of existence consists of a pair:
a witness together with a proof that the stated property holds for that witness.
\begin{example}[Constructive existence proof]
  We give a \textbf{constructive proof} in Lean that there exist natural numbers
  $a$ and $b$ such that $a + b = 7$:
  \begin{lstlisting}[language=lean]
example : ∃ a b : Nat, a + b = 7 := ⟨3, 4, rfl⟩
\end{lstlisting}
  To prove an existential statement, we provide \textbf{witnesses}
  (concrete values $a = 3$ and $b = 4$) and a \textbf{proof}
  that the predicate holds ($3 + 4 = 7$).
\end{example}
In classical mathematics, one might attempt a proof by contradiction.
However, this approach is not directly accepted in constructive mathematics,
as it doesn't provide explicit witnesses for the claimed objects.
Nonetheless, while constructive at its core, Lean allows users to
invoke classical principles, such as contraposition or proof by contradiction,
through \textbf{tactics} ((to be explained later))
like \lstinline[language=lean]|exfalso|.
\begin{example}[Reasoning from false]
  Here is an example of deriving any proposition from a contradiction:
  \begin{lstlisting}[language=lean]
  example (p : Prop) (h : False) : p := by
    exfalso
    exact h
  \end{lstlisting}
  This example takes a proposition $p$ to prove and a false hypothesis $h$.
  The \lstinline[language=lean]|exfalso| tactic transforms the goal into
  $\vdash \mathsf{False}$, meaning we now need to derive a contradiction.
  Since we already have a false hypothesis $h$,
  we can provide it using the \lstinline[language=lean]|exact| tactic.
\end{example}
\chapter{Describe and use properties}

It is interesting to note that a relation can be expressed as a function:
\lstinline[language=lean]|R : α → α → Prop|.
Similarly, when defining a predicate (\lstinline[language=lean]|P : α → Prop|) we must first declare
\lstinline[language=lean]|α : Type| to be some arbitrary type.
This is what is called \textbf{polymorphism}, more specifically \textbf{parametrical polymorphism}.
A canonical example is the identity function, written as
\lstinline[language=lean]|α → α|, where
\lstinline[language=lean]|α| is a type variable.
It has the same type for
both its domain and codomain, this means it can be
applied to booleans (returning a boolean), numbers (returning a number),
functions (returning a function), and so on.
In the same spirit, we can define a transitivity property of a relation as follows:
\begin{lstlisting}[language=lean]
def Transitive (α : Type) (R : α → α → Prop) : Prop :=
  ∀ x y z, R x y → R y z → R x z
\end{lstlisting}
To use \lstinline[language=lean]|Transitive|, we must provide both the type
\lstinline[language=lean]|α| and the relation itself.
For example, here is a proof of transitivity for the less-than relation on
$\mathbb{N}$ ( in Lean \lstinline[language=lean]|Nat| or \lstinline[language=lean]|ℕ|):
\begin{lstlisting}[language=lean]
theorem le_trans_proof : Transitive Nat (· ≤ · : Nat → Nat → Prop) :=
  fun x y z h1 h2 => Nat.le_trans h1 h2 -- this lemma is provided by Lean 
\end{lstlisting}
Looking at this code, we immediately notice that explicitly
passing the type argument \lstinline[language=lean]|Nat| is somewhat repetitive.
Lean allows us to omit it by letting the type inference mechanism fill it in automatically.
This is achieved by using \textbf{implicit arguments} with curly brackets:
\begin{lstlisting}[language=lean]
def Transitive {α : Type} (R : α → α → Prop) : Prop :=
  ∀ x y z, R x y → R y z → R x z
theorem le_trans_proof : Transitive (· ≤ · : Nat → Nat → Prop) :=
  fun x y z h1 h2 => Nat.le_trans h1 h2 
\end{lstlisting}
Lean's type inference system is quite powerful: in many cases, types can be completely
inferred without explicit annotations.
\newpage
\begin{example}[Type Inference in Lean]\mbox{}
  \begin{lstlisting}[language=lean]
  def double (n : Nat) := n + n
  -- Lean infers return type is Nat because n : Nat and + : Nat → Nat → Nat
  def id {α : Type} (x : α) : α := x
  #check id 5        -- Lean infers α = Nat
  #check id "hello"  -- Lean infers α = String
  \end{lstlisting}
\end{example}
Let us now revisit the transitivity proof, but this time for the less-than-equal relation on
the rational numbers (\lstinline[language=lean]|Rat| or \lstinline[language=lean]|ℚ|) instead.
\newpage
\begin{lstlisting}[language=lean]
  import Mathlib

  theorem rat_le_trans : Transitive (· ≤ · :   Rat → Rat → Prop) :=
    fun _ _ _ h1 h2 => Rat.le_trans h1 h2
\end{lstlisting}
Here, \lstinline[language=lean]|Rat.le_trans| is the transitivity lemma
for \lstinline[language=lean]|≤| on rational numbers, provided by Mathlib.
We import Mathlib to access \lstinline[language=lean]|Rat|
and \lstinline[language=lean]|le_trans|.
Mathlib is the community‑driven mathematical
library for Lean, containing a large body of formalized mathematics
and ongoing development.
It is the defacto standard library for both programming and proving
in Lean \cite{mathlib2020}, we will dig into it as we go along.
Notice that we used a function to discharge the universal
quantifiers required by transitivity. The underscores indicate
unnamed variables that we do not use later. If we had named
them, say \lstinline|x y z|, then:
\lstinline[language=lean]|h1| would be a proof of \lstinline[language=lean]|x ≤ y|,
\lstinline[language=lean]|h2| would be a proof of \lstinline[language=lean]|y ≤ z|,
and \lstinline[language=lean]|Rat.le_trans h1 h2| produces a proof of \lstinline[language=lean]|x ≤ z|.
The \lstinline[language=lean]|Transitive| definition is imported from Mathlib and similarly
defined as before.
\begin{example}
  The code can be made more readable using \textbf{tactic mode}.
  In this mode, you use tactics,
  commands provided by Lean or defined by users, to
  carry out proof steps succinctly, avoid code repetition,
  and automate common patterns.
  This often yields shorter, clearer proofs than writing
  the full term by hand.
  \begin{lstlisting}[language=lean]
  import Mathlib

  theorem rat_le_trans : Transitive (· ≤ · : Rat → Rat → Prop) := by
    intro x y z hxy hyz
    exact Rat.le_trans hxy hyz
\end{lstlisting}
  This proof performs the same steps but is much easier to read.
  Using \lstinline[language=lean]|by| we enter Lean's tactic mode.
  Move your cursor just before \lstinline[language=lean]|by|.
  The goal is initially displayed as \lstinline[language=lean]|⊢ Transitive fun x1 x2 ↦ x1 ≤ x2|.
  The tactic \lstinline[language=lean]|intro| is mainly used to introduce
  variables and hypotheses corresponding to universal quantifiers
  and assumptions into the context (essentially deconstructing universal quantifiers and implications).
  Now position your cursor just before \lstinline[language=lean]|exact|
  and observe the info view again.
  The goal is now \lstinline[language=lean]|⊢ x ≤ z|, with the context
  showing the variables and hypotheses introduced by the previous tactic.
  The \lstinline[language=lean]|exact| tactic closes the goal
  by supplying the term \lstinline[language=lean]|Rat.le_trans hxy hyz| that exactly matches the goal
  (the specification of \lstinline[language=lean]|Transitive|).
  You can hover over each tactic to see its definition and documentation.
\end{example}
\section{Exploring Mathlib (The Rat structure)}
In these examples we cheated and have used predefined lemmas such as
\lstinline[language=lean]|Nat.le_trans| and
\lstinline[language=lean]|Rat.le_trans|, just to simplify the presentation.
We can now dig into the implementation of these lemmas.
Let's look at the source code of \lstinline[language=lean]|Rat.le_trans|.
The Mathlib 4 documentation website is at
\url{https://leanprover-community.github.io/mathlib4_docs}, and
the documentation for
\lstinline[language=lean]|Rat.le_trans| is at
\url{https://leanprover-community.github.io/mathlib4_docs/Mathlib/Algebra/Order/Ring/Unbundled/Rat.html#Rat.le_trans}.
Click the "source" link there to jump to the implementation in the Mathlib repository. In editors like
VS Code you can also jump directly to the definition (Ctrl+click; Cmd+click on macOS).
Another way to check source code is by using \lstinline[language=lean]|#print Rat.le_trans|.
\begin{lstlisting}[language=lean]
variable (a b c : Rat)
protected lemma le_trans (hab : a ≤ b) (hbc : b ≤ c) : a ≤ c := by
  rw [Rat.le_iff_sub_nonneg] at hab hbc
  have := Rat.add_nonneg hab hbc
  simp_rw [sub_eq_add_neg, add_left_comm (b + -a) c (-b), add_comm (b + -a) (-b), add_left_comm (-b) b (-a), add_comm (-b) (-a), add_neg_cancel_comm_assoc, ← sub_eq_add_neg] at this
  rwa [Rat.le_iff_sub_nonneg]
\end{lstlisting}
The proof uses several tactics and lemmas from Mathlib.
The \lstinline[language=lean]|rw| or \lstinline[language=lean]|rewrite| tactic
is very common and sintactically similar to
the mathematical practice of rewriting an expression using an equality.
In this case, with \lstinline[language=lean]|at|, we use it to rewrite the
hypotheses \lstinline[language=lean]|hab|
and \lstinline[language=lean]|hbc|
using another Mathlib's lemma \lstinline[language=lean]|Rat.le_iff_sub_nonneg|,
which states that for any two rational numbers \lstinline[language=lean]|x| and
\lstinline[language=lean]|y|, \lstinline[language=lean]|x ≤ y|
is equivalent to \lstinline[language=lean]|0 ≤ y - x|.
Thus we now have the hypotheses tranformerd to :
\begin{lstlisting}[language=lean]
  hab : 0 ≤ b - a
  hbc : 0 ≤ c - b
\end{lstlisting}
The \lstinline[language=lean]|have| tactic introduces an intermediate result.
If you omit a name, Lean assigns it the default name \lstinline[language=lean]|this|.
In our situation, from \lstinline[language=lean]|hab : a ≤ b| and \lstinline[language=lean]|hbc : b ≤ c|
we can derive that \lstinline[language=lean]|b - a| and \lstinline[language=lean]|c - b|
are nonnegative, hence their sum is nonnegative:
\begin{lstlisting}[language=lean]
  this : 0 ≤ b - a + (c - b)
\end{lstlisting}
The most involved step uses \lstinline[language=lean]|simp_rw| to
simplify the expression via a sequence of other existing Mathlib's lemmas.
The tactic \lstinline[language=lean]|simp_rw| (TO EXPLAIN similar to rw but can see inside bihnders to unfold better,
in contrast rw haas more option suggesting dfor simple forms and woers beteter in that sense).
This is particularly useful for simplifying algebraic expressions and equations.
After these simplifications we obtain:
\begin{lstlisting}[language=lean]
  this : 0 ≤ c - a
\end{lstlisting}
Clearly, the proof relies mostly on \lstinline[language=lean]|Rat.add_nonneg|.
Its source code is fairly involved and uses advanced features
that are beyond our current scope. Nevertheless, it highlights
an important aspect of formal mathematics in Mathlib.
Mathlib defines \lstinline[language=lean]|Rat| as an instance of
a linear ordered field, implemented via a normalized fraction
representation: a pair of integers (numerator and denominator)
with positive denominator and coprime numerator and denominator \cite{mathlibdoc}.
To achieve this, it uses a \textbf{structure}. In Lean, a structure is a dependent record
(or product type) type  used to group together related fields or properties as a single data type.
Unlike ordinary records, the type of later fields may depend on the values of earlier ones.
Defining a structure automatically introduces a constructor (usually mk) and projection
functions that retrieve (deconstruct) the values of its fields.
Structures may also include proofs expressing properties that the fields must satisfy.
\begin{lstlisting}[language=lean]
  structure Rat where
    /-- Constructs a rational number from components.
    We rename the constructor to `mk'` to avoid a clash with the smart constructor. -/
    mk' ::
    /-- The numerator of the rational number is an integer. -/
    num : Int
    /-- The denominator of the rational number is a natural number. -/
    den : Nat := 1
    /-- The denominator is nonzero. -/
    den_nz : den ≠ 0 := by decide
    /-- The numerator and denominator are coprime: it is in "reduced form". -/
    reduced : num.natAbs.Coprime den := by decide
\end{lstlisting}
In order to work with rational numbers in Mathlib, we use the
\lstinline[language=lean]|Rat.mk'| constructor to create a rational number from
its numerator and denominator, if omitted the default would be \lstinline[language=lean]|Rat.mk|.
The fields \lstinline[language=lean]|den_nz| and \lstinline[language=lean]|reduced| are proofs that
the denominator is nonzero and that the numerator and denominator are coprime, respectively.
These proofs are automatically generated by Lean's \lstinline[language=lean]|decide| tactic, which can
solve certain decidable propositions (to be discussed in the next section).
\begin{example}
  Here is how we can define and manipulate rational numbers in Lean.
  \begin{lstlisting}[language=lean]
    def half : Rat := Rat.mk' 1 2
    def third : Rat := Rat.mk' 1 3
  \end{lstlisting}
\end{example}
When working with rational numbers, or more generally with structures, we must provide the
required proofs as arguments to the constructor (or Lean must be able to ensure them).
For instance \lstinline[language=lean]|Rat.mk' 1 0| or \lstinline[language=lean]|Rat.mk' 2 6|
would be rejected.
In the case of rationals, Mathlib unfolds the definition through
\lstinline[language=lean]|Rat.numDenCasesOn|. This principle states that, to prove a property of an
arbitrary rational number, it suffices to consider numbers of the form \lstinline[language=lean]|n /. d|
in canonical (normalized) form, with \lstinline[language=lean]|d > 0| and \lstinline[language=lean]|gcd n d = 1|.
This reduction allows mathlib to transform proofs about \lstinline[language=lean]|ℚ|
into proofs about \lstinline[language=lean]|ℤ| and \lstinline[language=lean]|ℕ|,
and then lift the result back to rationals.

Let's return to \lstinline[language=lean]|Rat.add_nonneg|, which was the important
lemma used in the proof of \lstinline[language=lean]|Rat.le_trans|.
We are going to provide a simplified version by also constructing a different
implementation of rational numbers from Mathlib's approach.
However, the main approach for working with rational numbers remains the same as in Mathlib:
projecting operations to natural numbers and integers first.
Let's start by creating a structure for our rational numbers:
\newpage
\begin{lstlisting}[language=lean]
import Mathlib

structure myPreRat where
  num : Int
  den : Nat
  den_pos : 0 < den
\end{lstlisting}
Notice the similarity with Mathlib's definition. You might have observed that we are not including
the coprimality condition, the name \lstinline[language=lean]|myPreRat| will become clear later.
Our initial focus is to prove \lstinline[language=lean]|myPreRat.add_nonneg|.
We structure our code as follows:
\begin{lstlisting}[language=lean]
import Mathlib

structure myPreRat where
  num : Int
  den : Nat
  den_pos : 0 < den

namespace myPreRat

lemma add_nonneg (a b : myPreRat) : 0 ≤ a → 0 ≤ b → 0 ≤ a + b := by 
  sorry

end myPreRat
\end{lstlisting}
The \lstinline[language=lean]|namespace| keyword is used to define self-contained modules.
For instance, outside of its scope,
one can refer to \lstinline[language=lean]|add_nonneg| as
\lstinline[language=lean]|myPreRat.add_nonneg|.
At this stage, Lean will complain because we haven't yet defined the operations
\lstinline[language=lean]|≤| or \lstinline[language=lean]|+| for our type
\lstinline[language=lean]|myPreRat|. Let's address this next.

Operations such as addition or less-than-or-equal need to be defined for each type
(addition
for natural numbers, less-or-equal for integers, and so on).
This is achieved through \textbf{type classes}, Lean's mechanism for defining and working with
\textbf{algebraic structures}.
Type classes provide a powerful and flexible way to specify properties and operations that can be
shared across different types called \textbf{ad hoc polymorphism}.
A standard example for ad hoc polymorphism (\cite{wadler_blott_ad_hoc_polymorphism_1988})
is overloaded multiplication:
the same symbol \lstinline[language=lean]|*| denotes multiplication of integers
(e.g., \lstinline[language=lean]|3 * 3|) and of floating-point numbers
(e.g., \lstinline[language=lean]|3.14 * 3.14|).
By contrast, parametric polymorphism occurs when a function is defined over a range of types
but acts uniformly on each of them. For instance, the \lstinline[language=lean]|List.length|
function applies in the same way to a list of integers and to a list of floating-point numbers.
Lean exposes type classes for common operations like:
\newpage
\begin{lstlisting}[language=lean]
class Add (α : Type u) where
  add : α → α → α

class LE (α : Type u) where
  le : α → α → Prop
\end{lstlisting}
Type classes are, under the hood, just structures where you similarly describe fields
for each operation.
The important features of type classes are type inference and instances.
When we use these operations, the square brackets in their definitions indicate that the type class
argument is \textbf{instance implicit}; it should be synthesized automatically using
typeclass resolution.
This is Lean's analogue of Haskell's typeclass constraints (e.g., \texttt{add :: Add a => a -> a -> a}).
We can register instances for specific types:
\begin{lstlisting}[language=lean]
instance : Add Nat where
  add := Nat.add

instance : Add Int where
  add := Int.add
\end{lstlisting}
In our case, we define instances for \lstinline[language=lean]|myPreRat|:
\begin{lstlisting}[language=lean]
instance : LE myPreRat where
  le r₁ r₂ := r₁.num * ↑r₂.den ≤ r₂.num * ↑r₁.den

instance : Add myPreRat where
  add r₁ r₂ := {
    num := r₁.num * ↑r₂.den + r₂.num * ↑r₁.den,
    den := r₁.den * r₂.den,
    den_pos := Nat.mul_pos r₁.den_pos r₂.den_pos
  }
\end{lstlisting}
Once these instances are defined, Lean can automatically infer which operation to use
when we write \lstinline[language=lean]|a + b| or \lstinline[language=lean]|a ≤ b|
for values of type \lstinline[language=lean]|myPreRat|.
We also want to define zero within our definition of rational numbers:
\begin{lstlisting}[language=lean]
def zero : myPreRat := { num := 0, den := 1, den_pos := by decide }
instance : OfNat myPreRat 0 where
  ofNat := zero
\end{lstlisting}
With \lstinline[language=lean]|OfNat| typeclass we are telling Lean that,
in a context expecting
\lstinline[language=lean]|myPreRat|, the number \lstinline[language=lean]|0|
must be transfomer into our
\lstinline[language=lean]|zero| definition.

Let's finally address the proof:
\newpage
\begin{lstlisting}[language=lean]
lemma add_nonneg (a b : myPreRat) : 0 ≤ a → 0 ≤ b → 0 ≤ a + b := by
  simp only [nonneg_iff]
  intro ha hb
  apply Int.add_nonneg
  · exact Int.mul_nonneg ha (Int.natCast_nonneg b.den)
  · exact Int.mul_nonneg hb (Int.natCast_nonneg a.den)
\end{lstlisting}
Starting from \lstinline[language=lean]|add_nonneg|, we first simplify using
\lstinline[language=lean]|nonneg_iff| (to be defined), which states that a rational number is non-negative
if and only if its numerator is non-negative. This transforms the goal to
\lstinline[language=lean]|⊢ 0 ≤ a.num → 0 ≤ b.num → 0 ≤ (a + b).num|.
We then introduce the two hypotheses \lstinline[language=lean]|ha : 0 ≤ a.num| and
\lstinline[language=lean]|hb : 0 ≤ b.num| using \lstinline[language=lean]|intro|.
Now we only need to prove that the numerator of their sum is non-negative.
By our definition of addition for \lstinline[language=lean]|myPreRat|, the numerator of
\lstinline[language=lean]|a + b| is \lstinline[language=lean]|a.num * ↑b.den + b.num * ↑a.den|.
Since the numerator is an integer, we can use lemmas for integers defined in Mathlib.
We use \lstinline[language=lean]|apply| to match our goal with the lemma
\lstinline[language=lean]|Int.add_nonneg| (the relative lemma on integers), which states that the sum of two non-negative
integers is non-negative. The \lstinline[language=lean]|apply| tactic works backwards:
given a goal \lstinline[language=lean]|⊢ G| and a lemma
\lstinline[language=lean]|lemma : P → Q → G|, it replaces the goal with two new subgoals
\lstinline[language=lean]|⊢ P| and \lstinline[language=lean]|⊢ Q|.
In our case, \lstinline[language=lean]|Int.add_nonneg| requires proving that both summands
are non-negative.
We close the first goals with \lstinline[language=lean]|Int.mul_nonneg ha (Int.natCast_nonneg b.den)|,
where \lstinline[language=lean]|ha| provides the non-negativity of the numerator
\lstinline[language=lean]|a.num|, and \lstinline[language=lean]|Int.natCast_nonneg b.den|
provides the non-negativity of the denominator \lstinline[language=lean]|b.den|.
The \lstinline[language=lean]|Int.natCast_nonneg| is needed to
\textbf{casts} \lstinline[language=lean]|b.den| from \lstinline[language=lean]|Nat| to
\lstinline[language=lean]|Int| (i am going to discuss casting and coercion in later section).
The second goal follows symmetrically.

We only need to examine \lstinline[language=lean]|nonneg_iff|:
\begin{lstlisting}[language=lean]
lemma nonneg_iff (r : myPreRat) : 0 ≤ r ↔ 0 ≤ r.num := by
  constructor <;> intro h
  · change 0 * r.den ≤ r.num * 1 at h; simp at h; exact h
  · change 0 * r.den ≤ r.num * 1; simp; exact h
\end{lstlisting}
Since this is a biconditional statement, we use \lstinline[language=lean]|constructor|
to split the proof into two directions. The combinator \lstinline[language=lean]|<;>|
applies the following tactic to all goals generated
by the previous tactic, so \lstinline[language=lean]|<;> intro h| introduces
the hypothesis \lstinline[language=lean]|h| in both directions.
For the forward direction, we have \lstinline[language=lean]|h : 0 ≤ r| and need
to prove \lstinline[language=lean]|0 ≤ r.num|. The \lstinline[language=lean]|change|
tactic unfolds our definition of \lstinline[language=lean]|≤| for
\lstinline[language=lean]|myPreRat|. Recall that we defined
\lstinline[language=lean]|r₁ ≤ r₂| as \lstinline[language=lean]|r₁.num * ↑r₂.den ≤ r₂.num * ↑r₁.den|.
When applied to \lstinline[language=lean]|0 ≤ r|, this becomes
\lstinline[language=lean]|0 * r.den ≤ r.num * 1| (since \lstinline[language=lean]|0.den = 1|).
The command \lstinline[language=lean]|change 0 * r.den ≤ r.num * 1 at h| applies
this transformation to the hypothesis \lstinline[language=lean]|h|.
Then \lstinline[language=lean]|simp at h| simplifies the arithmetic, reducing
\lstinline[language=lean]|0 * r.den| to \lstinline[language=lean]|0| and
\lstinline[language=lean]|r.num * 1| to \lstinline[language=lean]|r.num|,
giving us \lstinline[language=lean]|h : 0 ≤ r.num|.
Finally, \lstinline[language=lean]|exact h| completes the proof.
The backward direction follows symmetrically.
Here the full proof: [\href{https://live.lean-lang.org/#codez=JYWwDg9gTgLgBAWQIYwBYBtgCMBQODOMUArgMYzFQCmcIAngArUBKKcA7qldTnHAHbEQcAFxwAkvxi84AEyr9RcAHIoZ8/gH1I+JQAY4AHjkK8/JCCr4wSUjXpMqraTmD9CSfnaUAZAKK0jCxsnNxUMug0UICBBHBQgEEEogC8cdEAdILCAFRwgImECWkacIAmRHHxGUJwOfnpGnhuHl40YgCCsrKBjs4cXDx8SO2pZclwAN4yfJkjMRXZeQVFANRls1XztQoANBMmiiIpM0U5C1s7GtoQuvsqKGkgxOgXuocKTyvnOjIAvnjyAGZwABe3AgSgcwXg11GAkq1z0m12IwAjAiPpcRlg6CZSMB5HAfg0YJ5vGIAPJ/VTwcFONgGUJ9OAQClsa7AqAQPCREAgJACCD8fhUADmmmAfwBAAooGCgjSYABKfQlOJwQAphHADKUoKtrpiZKR+YQSORoEYANwAPjgbiIoNQMgA7XBSKhPEKaAZjoUFMrtVMcki4GxUGa4PhQGAg/AQ3AqAAPWzRx3O138d0atbaopa1YB0Ph8Ch+OJuD2nBcnlB9qafj8wVCuAS3lYGVdFCKsSaoNwQBJhBnSi2+13ecsW7q6DIC5H+egsQBtWsC4Wi8UAXRkNvZpd5qFw/TAYFnEikaQGshrdeFyeL5GPMDuDwvS4brsbknv5hgAGEkIQn/W4Cwb1+Hla8E1vd8H0eRcAN3N8T0/H8/xg4Ug2A0CgA}{link to Lean live}]
% \begin{example}
%   We present a simplified implementation of addition
%   non-negativity for rationals (\lstinline[language=lean]|Rat.add_nonneg| ),
%   maintaining a similar approach: projecting everything to the natural numbers and
%   integers first. To illustrate the proof technique clearly, we avoid using existing lemmas
%   from the Rat module in Mathlib.
%   Mathlib is indeed organized into modules by mathematical
%   domain (e.g., Nat, Int, Rat).
%   % Lemmas are typically namespaced (e.g., Rataddnonneg) 
%   % and often marked protected to prevent namespace pollution. 

%   We start by defining helper lemmas needed in the main proof.
%   Given a natural number
%   (which in this case represents the denominator of a rational number) that is not
%   equal to zero, we prove it must be positive. This follows directly by
%   applying the Mathlib lemma \lstinline[language=lean]|Nat.pos_of_ne_zero|:
%   \newpage
%   \begin{lstlisting}[language=lean]
%   import Mathlib

%   lemma nat_ne_zero_pos (den : ℕ) (h_den_nz : den ≠ 0) : 0 < den :=
%     Nat.pos_of_ne_zero h_den_nz
% \end{lstlisting}
%   The naming convention follows Mathlib
%   best practices aiming to be descriptive by indicating
%   types and properties involved.

%   The following lemma is slightly more involved.
%   It states that if a rational number (num / den)
%   is non-negative, then its numerator must also be non-negative:
%   \begin{lstlisting}[language=lean]
% lemma rat_num_nonneg {num : ℤ} {den : ℕ} (hden_pos : 0 < den)
%     (h : (0 : ℚ) ≤ num / den) : 0 ≤ num := by
%   contrapose! h
%   have hden_pos_to_rat : (0 : ℚ) < den := Nat.cast_pos.mpr hden_pos
%   have hnum_neg_to_rat : num < (0 : ℚ) := Int.cast_lt.mpr h
%   exact div_neg_of_neg_of_pos hnum_neg_to_rat hden_pos_to_rat
% \end{lstlisting}
%   The lemma requires the denominator to be positive as well as the non-negativity of the rational number,
%   expressed as \lstinline[language=lean]|num / den| where the types of
%   \lstinline[language=lean]|num| and \lstinline[language=lean]|den| are inferred.
%   First, notice the type annotation \lstinline[language=lean]|(0 : ℚ)|.
%   This explicit type annotation on zero forces the entire equation to be casted
%   into rational numbers.
%   Without this annotation, Lean would infer \lstinline[language=lean]|0|
%   as a natural number by default. However, since the main theorem we are proving concerns rational numbers,
%   we must ensure all comparisons occur in \lstinline[language=lean]|ℚ|.
%   The tactic \lstinline[language=lean]|contrapose!| does what you might expect: it proves a statement by contraposition. According to the documentation:
%   \begin{itemize}
%     \item \lstinline[language=lean]|contrapose| turns a goal \lstinline[language=lean]|P → Q| into \lstinline[language=lean]|¬ Q → ¬ P|
%     \item \lstinline[language=lean]|contrapose!| turns a goal \lstinline[language=lean]|P → Q| into \lstinline[language=lean]|¬ Q → ¬ P| and pushes negations inside \lstinline[language=lean]|P| and \lstinline[language=lean]|Q| using \lstinline[language=lean]|push_neg|
%     \item \lstinline[language=lean]|contrapose h| first reverts the local assumption \lstinline[language=lean]|h|, then uses \lstinline[language=lean]|contrapose| and \lstinline[language=lean]|intro h|
%     \item \lstinline[language=lean]|contrapose! h| first reverts the local assumption \lstinline[language=lean]|h|, then uses \lstinline[language=lean]|contrapose!| and \lstinline[language=lean]|intro h|
%   \end{itemize}
%   In our case, \lstinline[language=lean]|contrapose! h| transforms the goal from
%   proving \lstinline[language=lean]|0 ≤ num| to assuming \lstinline[language=lean]|num < 0|
%   and proving \lstinline[language=lean]|num / den < 0|.
%   We then introduce two local hypotheses. The first, \lstinline[language=lean]|hden_pos_to_rat|,
%   proves that the denominator
%   is positive when cast to rationals, using \lstinline[language=lean]|Nat.cast_pos|.
%   The suffix \lstinline[language=lean]|.mpr| selects the ``modus ponens reverse''
%   direction of the biconditional (the \lstinline[language=lean]|←|
%   direction of the \lstinline[language=lean]|↔|).
%   Next, we introduce \lstinline[language=lean]|hnum_neg_to_rat|,
%   which expresses that the numerator is negative when cast to rationals, using \lstinline[language=lean]|Int.cast_lt| with \lstinline[language=lean]|.mpr| again.
%   Finally, we apply \lstinline[language=lean]|div_neg_of_neg_of_pos|,
%   which states that dividing a negative number by a positive number yields a
%   negative result, thus completing the proof by contraposition.
%   Note that we are allowing ourselves to use existing lemmas from Mathlib,
%   such as \lstinline[language=lean]|div_neg_of_neg_of_pos| from the \lstinline[language=lean]|Field| module,
%   but not from the \lstinline[language=lean]|Rat| module,
%   to keep the presentation clear and focused on the main proof techniques.
%   \newpage
%   Now we can prove the main result:
%   \begin{lstlisting}[language=lean]
% lemma rat_add_nonneg (a b : Rat) : 0 ≤ a → 0 ≤ b → 0 ≤ a + b := by

%   intro ha hb
%   cases a with | div a_num a_den a_den_nz a_cop =>
%   cases b with | div b_num b_den b_den_nz b_cop =>
%   -- Goal: ⊢ 0 ≤ ↑a_num / ↑a_den + ↑b_num / ↑b_den
%   rw[div_add_div] -- applies the addition formula requiring two new goals 
%   · sorry 
%   · sorry 
%   · sorry
% \end{lstlisting}
%   \newpage
%   We first introduce the two hypotheses \lstinline[language=lean]|ha| and
%   \lstinline[language=lean]|hb| into the context using \lstinline[language=lean]|intro|.
%   As mentioned earlier, a structure can be viewed as a product type or a record type with
%   a single constructor. The tactic \lstinline[language=lean]|cases a with|
%   exposes the fields of \lstinline[language=lean]|Rat|: the
%   numerator ()\lstinline[language=lean]|a_num|), denominator
%   (\lstinline[language=lean]|a_den|), the proof that the denominator is non-zero
%   (\lstinline[language=lean]|a_den_nz|), and the coprimality condition
%   (\lstinline[language=lean]|a_cop|). Notice how the goal transforms
%   the rationals \lstinline[language=lean]|a| and \lstinline[language=lean]|b| into:
%   \begin{lstlisting}[language=lean]
% ⊢ 0 ≤ ↑a_num / ↑a_den + ↑b_num / ↑b_den
% \end{lstlisting}
%   where \lstinline[language=lean]|↑| denotes type coercion from
%   \lstinline[language=lean]|ℤ| or \lstinline[language=lean]|ℕ| to
%   \lstinline[language=lean]|ℚ|.
%   Now we rewrite the goal using \lstinline[language=lean]|rw [div_add_div]|,
%   a theoprem from the \lstinline[language=lean]|Field| module,
%   which applies the addition formula for division.
%   Let us briefly examine the source code of this theorem:
%   \begin{lstlisting}[language=lean]
% variable [Semifield K] {a b d : K}

% theorem div_add_div (a : K) (c : K) (hb : b ≠ 0) (hd : d ≠ 0) :
%     a / b + c / d = (a * d + b * c) / (b * d) := ...
% \end{lstlisting}
%   The type \lstinline[language=lean]|K| here is assumed to be a
%   \lstinline[language=lean]|Semifield|. The \lstinline[language=lean]|variable|
%   keyword is a way to declare parameters that are potentially used across
%   multiple theorems or definitions. We will explore Lean's powerful algebraic
%   hierarchy and the meaning of the square brackets \lstinline[language=lean]|[ ]|
%   in a later section.
%   Using this rewrite is particularly time-saving, since otherwise one would have to
%   establish the well-definedness of rational addition in terms of the underlying
%   structure (a non-trivial task).
%   This theorem requires proofs \lstinline[language=lean]|(hb : b ≠ 0)| and
%   \lstinline[language=lean]|(hd : d ≠ 0)|, generating two additional side goals.
%   We handle each goal separately using the focusing bullet \lstinline[language=lean]|·|.
%   The first bullet addresses the main goal (proving the sum is non-negative),
%   while the subsequent bullets discharge the non-zero denominator conditions.
%   I have omitted the actual proofs, here, using \lstinline[language=lean]|sorry|,
%   which we haven't mentioned before. \lstinline[language=lean]|sorry| is a useful
%   feature of Lean that tells the system to accept an incomplete proof for the time being,
%   allowing you to continue development without proving every detail immediately.
%   We can now tackle the remaining goals:
%   \newpage
%   \begin{lstlisting}[language=lean]
% · -- Goal: ⊢ 0 ≤ (↑a_num * ↑b_den + ↑a_den * ↑b_num) / (↑a_den * ↑b_den)
%   have hnum_nonneg : (0 : ℚ) ≤ a_num * b_den + a_den * b_num := by
%     have ha_num_nonneg := by
%       have ha_den_pos := nat_ne_zero_pos a_den a_den_nz
%       exact rat_num_nonneg ha_den_pos ha
%     have hb_num_nonneg := by
%       have hb_den_pos := nat_ne_zero_pos b_den b_den_nz
%       exact rat_num_nonneg hb_den_pos hb
%     apply add_nonneg -- works for any OrderedAddCommMonoid
%     · apply mul_nonneg -- works for any OrderedSemiring
%       · exact Int.cast_nonneg.mpr ha_num_nonneg
%       · exact Nat.cast_nonneg b_den
%     · apply mul_nonneg
%       · exact Nat.cast_nonneg a_den
%       · exact Int.cast_nonneg.mpr hb_num_nonneg

%   have hden_nonneg : (0 : ℚ) ≤ a_den * b_den := by
%     rw [← Nat.cast_mul]
%     exact Nat.cast_nonneg (a_den * b_den)
%   exact div_nonneg hnum_nonneg hden_nonneg

% · exact Nat.cast_ne_zero.mpr a_den_nz -- Goal: ⊢ ↑a_den ≠ 0
% · exact Nat.cast_ne_zero.mpr b_den_nz -- Goal: ⊢ ↑b_den ≠ 0
% \end{lstlisting}

%   We introduce two key hypotheses, \lstinline[language=lean]|hnum_nonneg|
%   and \lstinline[language=lean]|hden_nonneg|, which will be required by
%   \lstinline[language=lean]|div_nonneg| from the
%   \lstinline[language=lean]|GroupWithZero| module.
%   This lemma provides us with a term that directly validates our statement.
%   Note that \lstinline[language=lean]|div_nonneg| is a generalized lemma that applies
%   not only to rational numbers but to all ordered groups with zero that are
%   also partially ordered.
%   The hypothesis \lstinline[language=lean]|hnum_nonneg| proves that the numerator
%   is non-negative by working with the coerced expressions in
%   \lstinline[language=lean]|ℚ|. It uses \lstinline[language=lean]|add_nonneg| and
%   \lstinline[language=lean]|mul_nonneg|, which are general theorems that work for
%   any ordered additive commutative monoid and ordered semiring, respectively.
%   The actual reasoning is done using integer-related theorems
%   (via \lstinline[language=lean]|Int.cast_nonneg|) for the numerators and natural
%   number theorems (via \lstinline[language=lean]|Nat.cast_nonneg|) for the denominators.
%   The hypothesis \lstinline[language=lean]|hden_nonneg| proves that the
%   denominator is non-negative by working entirely with natural numbers.
%   We use the rewrite \lstinline[language=lean]|rw [← Nat.cast_mul]|,
%   which moves the coercion (in this case from \lstinline[language=lean]|ℕ|
%   to \lstinline[language=lean]|ℚ|) inside the multiplication:
%   \lstinline[language=lean]|↑(m * n) = ↑m * ↑n|. The \lstinline[language=lean]|←|
%   symbol means that we want the transformation from right to
%   left (i.e., we apply the equality in reverse to move the cast inward).
%   Type casts and coercions require these kinds of rewrite rules, not only
%   for multiplication but also for addition and other operations, and
%   similarly for \lstinline[language=lean]|ℤ| or other numerical types.
%   These lemmas, such as \lstinline[language=lean]|Nat.cast_mul|,
%   \lstinline[language=lean]|Int.cast_add|, etc., ensure that algebraic operations
%   commute with type coercions.
% \end{example}
\section{Coercions and Type Casting}
We extensively used type casting and coercions in this proof, which requires some
explanation \cite{lewis_madelaine_simplifying_casts_coercions_2020}.
Lean's type system lacks subtyping, means that types
like \lstinline[language=lean]|ℕ|, \lstinline[language=lean]|ℤ|, and \lstinline[language=lean]|ℚ|
are distinct and do not have a subtype relationship. In order to translate between these types,
we need to use explicit type casts or rely on automatic coercions. For example,
natural numbers (\lstinline[language=lean]|ℕ|) can be coerced to integers (\lstinline[language=lean]|ℤ|),
and integers can be coerced to rational numbers (\lstinline[language=lean]|ℚ|).
Casting and coercion are related but distinct concepts:
\begin{itemize}
  \item \textbf{Casting} refers to the explicit conversion of a value from one type to another,
        typically using functions like \lstinline[language=lean]|Int.cast| or \lstinline[language=lean]|Nat.cast|.
        These functions have accompanying lemmas that preserve properties across type conversions,
        such as \lstinline[language=lean]|Int.cast_lt| and \lstinline[language=lean]|Nat.cast_pos|.
  \item \textbf{Coercion}, on the other hand, is a more general mechanism that allows
        Lean to automatically convert between types when needed.
        More generally, in expressions like \lstinline[language=lean]|x + y| where \lstinline[language=lean]|x|
        and \lstinline[language=lean]|y| are of different types,
        Lean will automatically coerce them to a common type. For example, if \lstinline[language=lean]|x : ℕ|
        and \lstinline[language=lean]|y : ℤ|, then \lstinline[language=lean]|x|
        will be coerced to \lstinline[language=lean]|ℤ|.
\end{itemize}
The notation \lstinline[language=lean]|↑| denotes an explicit coercion
(in between cast and coercion).
To illustrate the expected behavior of coercion simplification, consider
the expression \lstinline[language=lean]|↑m + ↑n < (10 : ℤ)|,
where \lstinline[language=lean]|m, n : ℕ| are cast to \lstinline[language=lean]|ℤ|.
The expected normal form is \lstinline[language=lean]|m + n < (10 : ℕ)|,
since \lstinline[language=lean]|+|,
<,
and the numeral \lstinline[language=lean]|10| are polymorphic
(i.e., they can work with any numerical type such as \lstinline[language=lean]|ℤ|
or \lstinline[language=lean]|ℕ|). The simplification should proceed as follows:
\begin{enumerate}
  \item Replace the numeral on the right with the cast of a natural number:
        \lstinline[language=lean]|↑m + ↑n < ↑(10 : ℕ)|
  \item Factor \lstinline[language=lean]|↑| to the outside on the left:
        \lstinline[language=lean]|↑(m + n) < ↑(10 : ℕ)|
  \item Eliminate both casts to obtain an inequality over \lstinline[language=lean]|ℕ|:
        \lstinline[language=lean]|m + n < (10 : ℕ)|
\end{enumerate}
Lean provides tactics like \lstinline[language=lean]|norm_cast|
to simplify expressions involving such coercions.
The \lstinline[language=lean]|norm_cast| tactic normalizes casts
by pushing them outward and eliminating redundant coercions, often simplifying
proofs significantly by reducing goals to their ``native'' types.
\section{Quotients}
Back to our rational numbers definition.
We have actually made a very bad construction for rational numbers.
Let's look at what gopes wrong:
\begin{lstlisting}[language=lean]
example : myPreRat.mk 2 4 (by decide) ≠ myPreRat.mk 1 2 (by decide) := by 
  simp
\end{lstlisting}
This example proves that \lstinline[language=lean]|myPreRat.mk 2 4| and
\lstinline[language=lean]|myPreRat.mk 1 2| are not equal, even though
mathematically $\frac{2}{4} = \frac{1}{2}$! Indeed, the name
\lstinline[language=lean]|myPreRat| was already alluding to
the need for further work.
In mathematics, we treat different representations of the same rational number
as equivalent through an \textbf{equivalence relation} (\cite{algebrology_rationals_2019}). We consider fractions
like $\frac{1}{2}$, $\frac{2}{4}$, and $\frac{3}{6}$ as belonging to the same
equivalence class. To be more precise, mathematics uses \textbf{quotients} to
group elements of a set by an equivalence relation.
For instance, the equivalence
class $[\frac{1}{2}] = \{ \ldots, -\frac{1}{2}, \frac{1}{2}, \frac{2}{4}, \frac{3}{6}, \ldots\}$
represents all fractions equivalent to $\frac{1}{2}$.
Thus the set of rational numbers is the set of \textbf{representatives}
$\mathbb{Q} = \{ \ldots, [\frac{0}{1}], [\frac{1}{1}], [\frac{1}{2}], [\frac{1}{3}], \ldots\}$.
Algebraically, each rational number can be represented as a pair of
integers $(a, b)$ where the second component is non-zero.
Moreover, this construction must be ``justified'' by an equivalence relation $\sim_\mathbb{Q}$:
$$\mathbb{Q} = (\mathbb{Z} \times \mathbb{Z}^*) / \negthickspace\sim_\mathbb{Q}$$
The equivalence relation for rational numbers (seen as pairs of integers) is defined as:
$$(a,b) \sim_\mathbb{Q} (c,d) \iff ad = bc \quad \text{for all } (a,b), (c,d) \in \mathbb{Z}\times\mathbb{Z}^*$$
The relation $\sim_\mathbb{Q}$ must satisfy reflexivity, symmetry, and transitivity.
Moreover, each operation defined on the quotient set must be \textbf{well-defined}; that is,
the result must not depend on our choice of representatives.
If we naively define addition as
$$(a,b) + (c,d) = (a+c,b+d) \quad \text{for all } (a,b), (c,d) \in \mathbb{Z}\times\mathbb{Z}^*,$$
we encounter a problem. Consider:
$$(1,2) + (1,3) = (1+1, 2+3) = (2,5) \quad \text{and} \quad (1,2) + (2,6) = (1+2, 2+6) = (3,8).$$
But $(1,3) \sim_\mathbb{Q} (2,6)$ since $1 \cdot 6 = 2 \cdot 3$, yet $(2,5) \not\sim_\mathbb{Q} (3,8)$
since $2 \cdot 8 = 16 \neq 15 = 5 \cdot 3$.
A well-defined definition for addition is instead:
$$(a,b) + (c,d) = (ad+bc,bd) \quad \text{for all } (a,b), (c,d) \in \mathbb{Z}\times\mathbb{Z}^*.$$
We will verify this.

Similarly, in type theory and Lean we have \textbf{quotient types},
which allow us to think mathematically and
construct new types by mean of equivalence relations.
We are going to define proper rational numbers using quotient types in Lean.
As mentioned earlier, this differs from Mathlib's approach,
which achieves the
same goal by mechanically reducing each rational number to a
canonical form
using coprimality
(ensuring the numerator and denominator have no common factors).
We need to start from the notion of equivalence relations.
We have already seen how to define relations in Lean. The structure
\lstinline[language=lean]|Equivalence| is precisely a relation with fields
\lstinline[language=lean]|refl|, \lstinline[language=lean]|symm|, and
\lstinline[language=lean]|trans| for defining an equivalence relation.
You can inspect its definition using \lstinline[language=lean]|#print Equivalence|.
Another important component in defining a quotient is the
\lstinline[language=lean]|Setoid| typeclass, which encapsulates an equivalence
relation on a given type. In particular, it requires that the relation given is
indeed an equivalence relation, which is verified through the
\lstinline[language=lean]|iseqv| field. Here is how we begin defining the
rational numbers using the \lstinline[language=lean]|Setoid| typeclass and
\lstinline[language=lean]|Quotient| type:
\begin{lstlisting}[language=lean]
instance myRel : Setoid myPreRat where
  r p q := p.num * q.den = q.num * p.den
  iseqv := by
    constructor
    · intro p; rfl
    · rintro ⟨p, p', hp'⟩ ⟨q, q', hq'⟩
      simp [Eq.comm]
    · rintro ⟨p, p', hp'⟩ ⟨q, q', hq'⟩ ⟨r, r', hr'⟩ hpq hqr
      simp_all
      apply mul_left_cancel₀ (mod_cast hq'.ne' : q' ≠ (0 : ℤ))
      grind

abbrev myRat : Type := Quotient myRel
\end{lstlisting}
In the \lstinline[language=lean]|iseqv| field, we prove reflexivity, symmetry,
and transitivity of our relation. The reflexivity case
(\lstinline[language=lean]|intro p; rfl|) is immediate.
The symmetry case uses \lstinline[language=lean]|Eq.comm|
to swap the sides of the equation.
For transitivity, the key step involves multiplying both sides of our equations by
\lstinline[language=lean]|q.den| and then canceling this common factor using
\lstinline[language=lean]|mul_left_cancel₀| (left multiplication cancellation for
integers), which states that if $a \cdot b = a \cdot c$ and $a \neq 0$, then $b = c$.
The expression \lstinline[language=lean]|mod_cast hq'.ne' : q' ≠ (0 : ℤ)| takes
\lstinline[language=lean]|hq'| (which states \lstinline[language=lean]|0 < q.den|
as a natural number) and casts it to a proof that \lstinline[language=lean]|q.den ≠ 0|
as an integer.
Finally, \lstinline[language=lean]|abbrev| is a Lean keyword that creates a
type abbreviation. Unlike \lstinline[language=lean]|def|, which creates a new
definition that must be unfolded explicitly, with \lstinline[language=lean]|abbrev|
Lean can automatically treats \lstinline[language=lean]|myRat|
as \lstinline[language=lean]|Quotient myRel| without requiring manual unfolding.
\newpage
We can now propoerly define myRat and adding properties and operations.
\begin{lstlisting}[language=lean]
namespace myRat

instance : LE myRat where
  le r₁ r₂ := Quotient.lift₂ (fun a b ↦ a ≤ b) myRel_respects_le r₁ r₂

instance : Add myRat where
  add r₁ r₂ := Quotient.lift₂ (fun a b ↦ ⟦a + b⟧)
    (fun a₁ b₁ a₂ b₂ ha hb ↦ Quotient.sound (myRel_respects_add a₁ b₁ a₂ b₂ ha hb))
    r₁ r₂

instance : OfNat myRat 0 where
  ofNat := ⟦myPreRat.zero⟧

lemma add_nonneg (a b : myRat) : 0 ≤ a → 0 ≤ b → 0 ≤ a + b := by
  induction a using Quotient.ind with | _ a =>
  induction b using Quotient.ind with | _ b =>
  intro ha hb
  exact myPreRat.add_nonneg a b ha hb

end myRat
\end{lstlisting}
Notice how we reuse the previous \lstinline[language=lean]|add_nonneg|
lemma for the quotient type using \lstinline[language=lean]|Quotient.ind|.
The syntax \lstinline[language=lean]|induction a using Quotient.ind with  _ a =>|
unwraps the quotient value \lstinline[language=lean]|a : myRat| to its underlying
representative \lstinline[language=lean]|a : myPreRat|.

As before, we used the instance mechanism to add operations such as
\lstinline[language=lean]|≤| and \lstinline[language=lean]|+| for
\lstinline[language=lean]|myPreRat|. However, when lifting these operations
to \lstinline[language=lean]|myRat| (the quotient type), we need to ensure
they are \textbf{well-defined}.
This is achieved using \lstinline[language=lean]|Quotient.lift₂|. To define
a function from a quotient type, such as $f : \text{Quotient } S \to \beta$
where $S$ is the setoid, it is necessary to provide an underlying function
$f' : \alpha \to \beta$ and prove that for all $x, y : \alpha$, if
$x \approx y$ (under the equivalence relation), then $f'(x) = f'(y)$.
Morover it helps to splitting the woprk. First we prove the underlying property on
\lstinline[language=lean]|myPreRat|, then lift it to the quotient type.
Here we extract the main proof into a separate theorem:
\newpage
\begin{lstlisting}[language=lean]
private theorem le_respects_equiv_forward
    (a₁ b₁ a₂ b₂ : myPreRat)
    (ha : a₁ ≈ a₂) (hb : b₁ ≈ b₂)
    (h : a₁ ≤ b₁) : a₂ ≤ b₂ := by
  have pos_prod : (0: Int) < (a₁.den * b₁.den) := 
    myPreRat.den_prod_pos a₁ b₁
  have pos_prod2 : 0 < (a₂.den * b₂.den : Int) := 
    myPreRat.den_prod_pos a₂ b₂
  apply Int.le_of_mul_le_mul_right _ pos_prod
  calc (a₂.num * b₂.den) * (a₁.den * b₁.den)
      = a₂.num * a₁.den * b₂.den * b₁.den := by ring
    _ = a₁.num * a₂.den * b₂.den * b₁.den := by rw [← ha]
    _ = a₁.num * b₁.den * (a₂.den * b₂.den) := by ring
    _ ≤ b₁.num * a₁.den * (a₂.den * b₂.den) :=
        Int.mul_le_mul_of_nonneg_right h (Int.le_of_lt pos_prod2)
    _ = b₁.num * b₂.den * a₁.den * a₂.den := by ring
    _ = b₂.num * b₁.den * a₁.den * a₂.den := by rw [← hb]
    _ = (b₂.num * a₂.den) * (a₁.den * b₁.den) := by ring

theorem myRel_respects_le (a₁ b₁ a₂ b₂ : myPreRat) :
    a₁ ≈ a₂ → b₁ ≈ b₂ → (a₁ ≤ b₁) = (a₂ ≤ b₂) := by
  intro ha hb
  simp only [eq_iff_iff]
  constructor
  · exact le_respects_equiv_forward a₁ b₁ a₂ b₂ ha hb
  · exact fun h => le_respects_equiv_forward a₂ b₂ a₁ b₁ ha.symm hb.symm h
\end{lstlisting}
In \lstinline[language=lean]|myRel_respects_le|, we transform
the equality of propositions into a biconditional using
\lstinline[language=lean]|eq_iff_iff|, then prove both directions with
\lstinline[language=lean]|constructor|. Since both goals are symmetrical,
we reuse \lstinline[language=lean]|le_respects_equiv_forward|.
The heart of the proof is in \lstinline[language=lean]|le_respects_equiv_forward|.
The \lstinline[language=lean]|private| keyword ensures that
\lstinline[language=lean]|le_respects_equiv_forward| is only accessible
within the current namespace, keeping our interface clean.
Given $h : a_1 \leq b_1$ (meaning $a_1.\text{num} \cdot b_1.\text{den} \leq b_1.\text{num} \cdot a_1.\text{den}$)
and equivalences $ha : a_1 \approx a_2$ and $hb : b_1 \approx b_2$
(meaning $a_1.\text{num} \cdot a_2.\text{den} = a_2.\text{num} \cdot a_1.\text{den}$
and $b_1.\text{num} \cdot b_2.\text{den} = b_2.\text{num} \cdot b_1.\text{den}$),
we need to prove $a_2 \leq b_2$ (i.e., $a_2.\text{num} \cdot b_2.\text{den} \leq b_2.\text{num} \cdot a_2.\text{den}$).
The strategy is to introduce a common positive factor and use the given information.
We apply \lstinline[language=lean]|Int.le_of_mul_le_mul_right|, which states
that to prove $X \leq Y$, it suffices to prove $X \cdot Z \leq Y \cdot Z$
for positive $Z$, then cancel $Z$. We choose $Z = a_1.\text{den} \cdot b_1.\text{den}$
(shown positive by \lstinline[language=lean]|pos_prod|).
The \lstinline[language=lean]|calc| block proves
$(a_2.\text{num} \cdot b_2.\text{den}) \cdot (a_1.\text{den} \cdot b_1.\text{den})
  \leq (b_2.\text{num} \cdot a_2.\text{den}) \cdot (a_1.\text{den} \cdot b_1.\text{den})$:
First, we rearrange the left side and substitute using $ha$:
\begin{align*}
  (a_2.\text{num} \cdot b_2.\text{den}) \cdot (a_1.\text{den} \cdot b_1.\text{den})
   & = a_2.\text{num} \cdot a_1.\text{den} \cdot b_2.\text{den} \cdot b_1.\text{den}   \\
   & = a_1.\text{num} \cdot a_2.\text{den} \cdot b_2.\text{den} \cdot b_1.\text{den}
  \quad \text{(by } ha\text{)}                                                         \\
   & = a_1.\text{num} \cdot b_1.\text{den} \cdot (a_2.\text{den} \cdot b_2.\text{den})
\end{align*}
Then we apply $h : a_1 \leq b_1$ (i.e., $a_1.\text{num} \cdot b_1.\text{den} \leq b_1.\text{num} \cdot a_1.\text{den}$),
multiplying both sides by the positive factor $(a_2.\text{den} \cdot b_2.\text{den})$:
$$a_1.\text{num} \cdot b_1.\text{den} \cdot (a_2.\text{den} \cdot b_2.\text{den})
  \leq b_1.\text{num} \cdot a_1.\text{den} \cdot (a_2.\text{den} \cdot b_2.\text{den})$$
Finally, we rearrange the right side and substitute using $hb$:
\begin{align*}
  b_1.\text{num} \cdot a_1.\text{den} \cdot (a_2.\text{den} \cdot b_2.\text{den})
   & = b_1.\text{num} \cdot b_2.\text{den} \cdot a_1.\text{den} \cdot a_2.\text{den}     \\
   & = b_2.\text{num} \cdot b_1.\text{den} \cdot a_1.\text{den} \cdot a_2.\text{den}
  \quad \text{(by } hb\text{)}                                                           \\
   & = (b_2.\text{num} \cdot a_2.\text{den}) \cdot (a_1.\text{den} \cdot b_1.\text{den})
\end{align*}
After canceling the common positive factor $(a_1.\text{den} \cdot b_1.\text{den})$,
we obtain $a_2.\text{num} \cdot b_2.\text{den} \leq b_2.\text{num} \cdot a_2.\text{den}$,
which is precisely $a_2 \leq b_2$.

We also need to prove that the addition operation is well-defined on the quotient:
\begin{lstlisting}[language=lean]
theorem myRel_respects_add (a₁ b₁ a₂ b₂ : myPreRat) :
    a₁ ≈ a₂ → b₁ ≈ b₂ → (a₁ + b₁) ≈ (a₂ + b₂) := by
  intro ha hb
  calc (a₁.num * b₁.den + b₁.num * a₁.den) * (a₂.den * b₂.den)
      = a₁.num * b₁.den * a₂.den * b₂.den + b₁.num * a₁.den * a₂.den * b₂.den 
        := by ring
    _ = a₂.num * a₁.den * b₁.den * b₂.den + b₂.num * b₁.den * a₁.den * a₂.den 
        := by rw [← ha, ← hb]; ring
    _ = (a₂.num * b₂.den + b₂.num * a₂.den) * (a₁.den * b₁.den) 
        := by ring
\end{lstlisting}
We need to show that if $a_1 \approx a_2$ and $b_1 \approx b_2$, then
$(a_1 + b_1) \approx (a_2 + b_2)$.
By unfolding the addition definition and the relation we end up proving:
\begin{align*}
  (a_1.\text{num} \cdot b_1.\text{den} + b_1.\text{num} \cdot a_1.\text{den}) \cdot (a_2.\text{den} \cdot b_2.\text{den})
  \\ = \\ (a_2.\text{num} \cdot b_2.\text{den} + b_2.\text{num} \cdot a_2.\text{den}) \cdot (a_1.\text{den}
  \cdot b_1.\text{den})
\end{align*}
The \lstinline[language=lean]|calc| proof proceeds in three steps. First, we distribute
the product over the sum on the left side using \lstinline[language=lean]|ring|.
Next, we apply the equivalences $ha$ and $hb$ using \lstinline[language=lean]|rw [← ha, ← hb]|,
which substitutes $a_1.\text{num} \cdot a_2.\text{den}$ with $a_2.\text{num} \cdot a_1.\text{den}$
and $b_1.\text{num} \cdot b_2.\text{den}$ with $b_2.\text{num} \cdot b_1.\text{den}$.
Finally, we factor the expression back into sum-times-product form using
\lstinline[language=lean]|ring|, obtaining the right side of the desired equality.

We finally have a minimal and well-defined solution for showing
\lstinline[language=lean]|myRat.add_nonneg|. However, our original discussion
was about proving transitivity of the less-or-equal operator. In our earlier work
with natural numbers, we used \lstinline[language=lean]|Nat.le_trans|,
a theorem specifically for natural numbers that is part of Lean's core library.
However, the transitivity property holds not only for naturals but also for integers,
reals, and any partially ordered set. Rather than duplicating this theorem for each type,
Mathlib provides a general lemma \lstinline[language=lean]|le_trans| that works
for any type \lstinline[language=lean]|α| endowed with a partial ordering.

Mathlib achieves this through type classes and a carefully constructed algebraic hierarchy.
We have already touched on this concept when we used type classes such as
\lstinline[language=lean]|Add| and \lstinline[language=lean]|LE| to define operations
on \lstinline[language=lean]|myRat|. Aware that rational numbers form a linearly ordered
set (and hence a partially ordered set), we now enhance \lstinline[language=lean]|myRat|
with the \lstinline[language=lean]|PartialOrder| type class:
\begin{lstlisting}[language=lean]
instance : PartialOrder myRat where
  le_refl p := by
    induction p using Quotient.ind with | _ a =>
    exact myPreRat.le_refl a

  le_trans p q r := by
    induction p using Quotient.ind with | _ a =>
    induction q using Quotient.ind with | _ b =>
    induction r using Quotient.ind with | _ c =>
    intro hab hbc
    exact myPreRat.le_trans a b c hab hbc

  le_antisymm p q := by
    induction p using Quotient.ind with | _ a =>
    induction q using Quotient.ind with | _ b =>
    intro hab hba
    exact Quotient.sound (myPreRat.le_antisymm a b hab hba)
\end{lstlisting}
The structure follows the same pattern we saw earlier: use \lstinline[language=lean]|Quotient.ind|
to unwrap quotient values to their representatives, then apply the corresponding proof
for \lstinline[language=lean]|myPreRat|. The antisymmetry case uses
\lstinline[language=lean]|Quotient.sound|, which states that if two representatives
are equivalent (i.e., $a \approx b$), then their quotient equivalence classes are equal
(i.e., $\llbracket a \rrbracket = \llbracket b \rrbracket$).
\newpage
The underlying proofs for \lstinline[language=lean]|myPreRat| are:
\begin{lstlisting}[language=lean]
theorem le_refl (a : myPreRat) : a ≤ a := by
  exact Int.le_refl _

theorem le_trans (a b c : myPreRat) : a ≤ b → b ≤ c → a ≤ c := by
  intro hab hbc
  apply Int.le_of_mul_le_mul_right _ b.den_pos_int
  calc (a.num * c.den) * b.den
      = (a.num * b.den) * c.den := by ring
    _ ≤ (b.num * a.den) * c.den :=
        Int.mul_le_mul_of_nonneg_right hab (Int.le_of_lt c.den_pos_int)
    _ = (b.num * c.den) * a.den := by ring
    _ ≤ (c.num * b.den) * a.den :=
        Int.mul_le_mul_of_nonneg_right hbc (Int.le_of_lt a.den_pos_int)
    _ = (c.num * a.den) * b.den := by ring

theorem le_antisymm (a b : myPreRat) : a ≤ b → b ≤ a → a ≈ b := by
  intro hab hba
  exact Int.le_antisymm hab hba
\end{lstlisting}
Reflexivity and antisymmetry are straightforward applications of the corresponding
integer properties. For transitivity, given $hab : a \leq b$ (meaning
$a.\text{num} \cdot b.\text{den} \leq b.\text{num} \cdot a.\text{den}$) and
$hbc : b \leq c$ (meaning $b.\text{num} \cdot c.\text{den} \leq c.\text{num} \cdot b.\text{den}$),
we need to prove $a \leq c$ (i.e., $a.\text{num} \cdot c.\text{den} \leq c.\text{num} \cdot a.\text{den}$).
The strategy is to introduce a common positive factor $b.\text{den}$, prove
$(a.\text{num} \cdot c.\text{den}) \cdot b.\text{den} \leq (c.\text{num} \cdot a.\text{den}) \cdot b.\text{den}$,
then cancel it using \lstinline[language=lean]|Int.le_of_mul_le_mul_right|.
The \lstinline[language=lean]|calc| chain proceeds as follows: we rearrange to
introduce $a.\text{num} \cdot b.\text{den}$, apply $hab$ (multiplying by the
positive factor $c.\text{den}$ to preserve the inequality), rearrange to
introduce $b.\text{num} \cdot c.\text{den}$, apply $hbc$ (multiplying by
$a.\text{den}$), and finally rearrange to match the required form. After
canceling $b.\text{den}$, we obtain $a \leq c$.
Here the full construction of myRat: [\href{https://live.lean-lang.org/#codez=JYWwDg9gTgLgBAWQIYwBYBtgCMBQODOMUArgMYzFQCmcIAngArUBKKcA7qldTnHAHbEQcAFxwAkvxi84AEyr9RcAHIoZ8/gH1I+JQAY4AHjkK8/JCCr4wSUjXpMqraTmD9CSfnaUAZAKK0jCxsnNxUMug0UICBBHBQgEEEogC8cdEAdILCAFRwgImECWkacIAmRHHxGUJwOfnpGnhuHl40YgCCsrKBjs4cXDx8SO2pZclwAN4yfJkjMRXZeQVFANRls1XztQoANBMmiiIpM0U5C1s7GtoQuvsqKGkgxOgXuocKTyvnOjIAvnjyAGZwABe3AgSgcwXg11GAkq1z0m12IwAjAiPpcRlg6CZSMB5HAfg0YJ5vGIAPJ/VTwcFONgGUJ9OAQClsa7AqAQPCREAgJC7J6aNzwAAUUDBQRpMAAlPojHARYUFEpJFKMXQZFB2HAANrKjIoADCSEITwAujIqAAPWzwKAKrSfHBcnl8sDs2RvEWxBJirooaViAzGT12tYnPYSKT+lKY81W8i0B5vF72y4CqTvV46NMuJ28/gQfj8KgAcwFfwBIp9EP9cAMpVFgBTCWslOKra4xvikAuEEjkaBGADcAD44IL2XBUDIAO1wUioTzFmgGY4h+urHJIuBsVADuD4UBgLfwHdwS3WifT2fz/iL5sroprqYb3f78C7s/xyeOqjc3kDd35oWJZyryWBVhKNZ1lucCAEmEzalGBcFQbyyxge2ap8K+h4FugWJaoBRalsA5ZmnwY6gvOE64P0YBgLhEYwGk/6aARJaXh+8C6vcjyscWE68kKurmDAhrGrxcBYHakrsXGnFSHcibiagYGCfJwmiTALEFoRW5SXgZ7gJE4HOHcADWcAAExwAALHKmLYriVDSoABkSdBCZlwJuVlCvZ8g4vIUYSViWF4GgVDQD+cCRJo1B/OgIHGX6Si8qUvLobG566tFsXxZooVcBFwjRUQni6EKoGzolKpiClEmwXVpSkPVtVNelZFSOO85gcppAyEgtH0VlVCaEymjcZo0XjVAwDFqg8CaBJdr8oKMikEg6BNeV66zlJaySXUfCHXAKRbU+i0aNKOSkCG6FxG4xY7AtpQ+dtSC7VdN1JDsR1cYmk2JqNvExTNc38SpQ0jX8E3wNdaL4Nm0lHQtx12dtsMKJdulFLd003o9LZCtdZ37RjaxvdjX1HT98njf9jyA9pJbA7Nx5YJtEOjeg8Dk5mqaCojh0LSdROVDkPP8JjJN7NGWK4w9OBhYVUXDZ4MDAPgdDcglaFuRByUtohDXQXBb3EyG0avZ9QUyORYNUUgGXxhDqvq5rwhdfb+n8B01LOPU7hEk0gTMFQ8ViAAylQMAQLiuvdPS4R8KKh4AI4jGA20p+bcBZ2dGcHaO+BUCnABuqo7F2Ae9tHUA7DOttgLuUBxXXd0daCgAX5GACJgAA5AiqB94Al+RwB3KcIin/cTpPQ/fXuB7an4WddtypGHTOuNEJ33dwH3A/D6P4851PqAz6PUAIlAJ9XyPg9p6ftdU/P4CaOt6Bz/1dFYrTVB/Jpa1NHQIAAII5QgAgO6NahBp69wyFQXuShJ5wFckKAwYhAAkhJKAWR1iy41kHgV0wAS4oBoIragRVhrUGsFQcg8Ni7EEIZoP40B2BICgHgo65VYhYFiEgRIWBEhiF9n6HYQpKI1ViIACSItzxGlKInW3C4BSP4VguUqBkqxAQtEGsvCDYCJljIecJcaBZldOApQKCxDKmlEGJAGxFA5G4SGQKQjGIfDdG8WxElogGKQEY3eqZTGyCsgGWU5VyhHAkuExUljIwjBcUtQJHi+HxD6gNLEHMoY/zGomaaLM4ALRMW6Va61Nq8O2vw96IE7F7TsSovgKQylnVsSGBxUT7FeKtvZOW+MGnpCaW0vaAyHHVJxpqLUgAEwn4mvPgQstx9NFh0iJYSWmRN2jje6+NNGvWqTkZZESKkXWSHPPgv1Hh00hlpICpZcmgzUapRi0VObwEKeAiyKjZmOOJkMuZKyykUyCm3B6SMUYVOJjsn5ES/nRJlnEMZkzlLTPySjHy5R+mVM4Ssxxhz1l4wVgVMhwdQ4xSsGAahMB4ZGU4V46RkSqr+h2J4qROi4IKKUYkOCVLNHShOjohCMjy7tS3vxKiMgsKMn4PRLUxcyxQ2In8E0cAAC0irGSik/vRUxpKLTSE7N2IgZAa4yXPNlElZLaEp3oSXRhzDWEdE8QonR/DhXKSNfGP4xBFBqKSCOE1VCaGaDoQwphGpbU0qdfa2I840gay1spaNbsLx4vCgS+gIdHiUNJf6/8VTqWOoEXHJKIgGWSJpSyktTqOWeNQloxRIFEioX5W1Uc7dnXUVnCUqp5TqnVu2eiqF7SDkYznr0rtvzvmDsUD2pp4L+2DM6bLDZwKGmooWc0/Z4KJ1wAbaOyFM6BlNqfhqbU8KkAInhVgE0TdF2C2RY0hZm7t1osObstdiphlrJhd0nASAsBYGoGXVNLI4AABU6CkpGAARWIBANWCgqR0DTWYCwJLbD2AQ2oVwAdiTNDgP4YOIReiJ2VkMb01woMweAHBtImA/6JCFO6xQFVABlhNBBC0pU1EozWaiaUQvQpMw40EkcA2g+3Q/ABOfVBgxGGGR6DsH5I0ZgHRhj0EwIscAGfkKEJKAHPyFR9GPVzNzck1tcAWPkfk4xfAEAPUdCFBx9Npqs2DAjWGxIlFlKYJ2NJhI/tBM4fJJSfD8A6SEZkEyQL1x1PxLZBAbTnIfzOmYuJcqdVBFicgi2XkSEDb1WQlu1L+j2qyANcAAs0FiD7hvHAczlH5JuA6OwYAaA4AAB8kW8m9Tbb2JWytgQq/darcnauMXqxwJrai2sLTAp1wVnVeQusdvB30jEkuMz4hVdzuABOByEwwVhat1qkjYdwILPQwgRAob/eKh4D3NuK+QUrihDz9aqzVqjo3GvNcm9BGbh0OIFoeZduKW48B8GKlAUqu8c5xAFYderPWntwBe3xN7dXvZja++146Q4djw4e2VtOyPBsUfe+jz7E2kXTZx0dPHasyuiiJ6jkbZPxutaRU1X7s2KI/qor1I6/34ng8hxVJqHseqg+Vq/KQrstap1h0VhHUPGdDdJw11n32OvU7h91/HihCeVZRyrtHavMdTex7jltYusAO357JYnFno3WfR3Z8UJloouxjcIDbPPlJIGkjgBQom/ZAA}{link to Lean live}]

With  the use of a type class suich as PartialOrder we thus nenahnced myRat to be
a member of an algebraic hierarchy. This allows us to use general theorems
about partial orders, such as transitivity of less-or-equal, without
redefining them specifically for myRat.
This apporoach is at the base of Mathlib's design, and needs some discussion.
\section{Type Classes and Algebraic Hierarchy}

Type classes provide a powerful and flexible
way to specify properties and operations that can be
shared across different types, thereby enabling polymorphism and code reuse.
Ad hoc polymorphism
arises when a function or operator is defined over several distinct types, with behavior that varies
depending on the type.
A standard example \cite{wadler_blott_ad_hoc_polymorphism_1988} is overloaded multiplication:
the same symbol \lstinline[language=lean]|*| denotes multiplication of integers
(e.g., \lstinline[language=lean]|3 * 3|) and of floating-point numbers
(e.g., \lstinline[language=lean]|3.14 * 3.14|).
By contrast, parametric polymorphism occurs when a function is defined over a range of types
but acts uniformly on each of them. For instance, the \lstinline[language=lean]|List.length|
function applies in the same way to a list of integers and to a list of floating-point numbers.

Under the hood, a type class is a structure.
An important aspect of structures, and hence type classes, is that they support hierarchy
and composition through inheritance. For example, mathematically, a monoid is a semigroup
with an identity element, and a group is a monoid with inverses. In Lean, we can express this
by defining a \lstinline[language=lean]|Monoid| structure that extends the
\lstinline[language=lean]|Semigroup| structure, and a
\lstinline[language=lean]|Group| structure that extends the
\lstinline[language=lean]|Monoid| structure using the
\lstinline[language=lean]|extends| keyword:
\newpage
\begin{lstlisting}[language=lean]
-- A semigroup has an associative binary operation
structure Semigroup (α : Type*) where
  mul : α → α → α
  mul_assoc : ∀ a b c : α, mul (mul a b) c = mul a (mul b c)

-- A monoid extends semigroup with an identity element  
structure Monoid (α : Type*) extends Semigroup α where
  one : α
  one_mul : ∀ a : α, mul one a = a
  mul_one : ∀ a : α, mul a one = a

-- A group extends monoid with inverses
structure Group (α : Type*) extends Monoid α where
  inv : α → α
  mul_left_inv : ∀ a : α, mul (inv a) a = one
\end{lstlisting}
The symbol \lstinline[language=lean]|*| in \lstinline[language=lean]|(α : Type*)|
indicates a universe variable (we will discuss universes later). Sometimes,
to avoid inconsistencies between types (such as Girard's paradox),
universes must be specified explicitly. This is an example of universe polymorphism.
Thus we have now seen all the polymorphism flavors in Lean: parametric, ad hoc, and universe polymorphism.

Type classes are defined using the \lstinline[language=lean]|class| keyword,
which is syntactic sugar for defining a structure. Thus, the previous example
can be rewritten using type classes:
\newpage
\begin{lstlisting}[language=lean]
-- A semigroup has an associative binary operation
class Semigroup (α : Type*) where
  mul : α → α → α
  mul_assoc : ∀ a b c : α, mul (mul a b) c = mul a (mul b c)

-- A monoid extends semigroup with an identity element  
class Monoid (α : Type*) extends Semigroup α where
  one : α
  one_mul : ∀ a : α, mul one a = a
  mul_one : ∀ a : α, mul a one = a

-- A group extends monoid with inverses
class Group (α : Type*) extends Monoid α where
  inv : α → α
  mul_left_inv : ∀ a : α, mul (inv a) a = one
\end{lstlisting}
The main difference is that type classes support \textbf{instance resolution}.
We use the keyword \lstinline[language=lean]|instance| to declare that a particular type is an
instance of a type class, which inherits the properties and operations defined in the type class.
Instances can be automatically inferred by Lean's type inference system,
allowing for concise and expressive code.
For example, we can declare that \lstinline[language=lean]|ℤ| is a group under addition:
\begin{lstlisting}[language=lean]
instance : Group ℤ where
  mul := Int.add
  one := 0
  inv := Int.neg
  mul_assoc := Int.add_assoc
  one_mul := Int.zero_add
  mul_one := Int.add_zero
  mul_left_inv := Int.neg_add_cancel
\end{lstlisting}
Now, any theorem proven for an arbitrary \lstinline[language=lean]|Group α|
automatically applies to \lstinline[language=lean]|ℤ| without any additional work.

\subsubsection{Analysis and the \lstinline[language=lean]|TopologicalSpace| Class}

We have roughly seen how Lean constructively builds the rational numbers
from naturals and integers.
However, when dealing with real numbers, the approach includes
the use of the axiom of choice, which, as discussed in our section on
constructive mathematics, is not constructively acceptable.
When constructive methods are insufficient, Lean provides classical
axioms through the \lstinline[language=lean]|Classical| module.
Real numbers in Mathlib are constructed as equivalence classes of Cauchy
sequences of rational numbers \cite{mathlibdoc}.
This construction, combined with the reliance on classical axioms, comes
at the cost of computability: most operations on real numbers must be
marked \lstinline[language=lean]|noncomputable|, indicating they cannot
be executed algorithmically.
For example, the sine function requires this marker:
\begin{lstlisting}[language=lean]
noncomputable def realSin (x : ℝ) : ℝ := Real.sin x
\end{lstlisting}
As a linearly ordered field with a metric, they instantiate multiple
structures: normed space, metric space, uniform space, and normal
topological space \cite{mathlib2020}.

In the next section, I will present an example of formalization that requires
working with real numbers and topological spaces, the latter providing
foundational tools for analysis concepts like continuity and convergence.
Notions such as convergence and continuity in $\mathbb{R}$ can be defined
using the metric space structure with the familiar $\varepsilon$-$\delta$
formulation:
\begin{lstlisting}[language=lean]
def ConvergesTo (s : ℕ → ℝ) (a : ℝ) :=
  ∀ ε > 0, ∃ N, ∀ n ≥ N, |s n - a| < ε
\end{lstlisting}
where \lstinline[language=lean]|s| is a sequence
and \lstinline[language=lean]|a| is the limit point.

More generally, these concepts are formalized in topology.
In standard textbooks, a topological space
is defined as a set equipped with a collection of open sets
satisfying certain axioms. This is reflected in
Mathlib's \lstinline[language=lean]|TopologicalSpace| type class:
\begin{lstlisting}[language=lean]
class TopologicalSpace (α : Type*) where
  IsOpen : Set α → Prop
  isOpen_univ : IsOpen univ
  isOpen_inter : ∀ s t, IsOpen s → IsOpen t → IsOpen (s ∩ t)
  isOpen_sUnion : ∀ s, (∀ t ∈ s, IsOpen t) → IsOpen (sUnion s)
\end{lstlisting}

Recall that Lean treats sets as predicates
(\lstinline[language=lean]|Set α := α → Prop|),
where set membership is function application.
Here, \lstinline[language=lean]|IsOpen| is a predicate on sets indicating whether
they are open. The three axioms correspond to the standard topological axioms:
\begin{itemize}
  \item The whole space is open (\lstinline[language=lean]|isOpen_univ|)
  \item The intersection of two open sets is open (\lstinline[language=lean]|isOpen_inter|)
  \item The union of any collection of open sets is open (\lstinline[language=lean]|isOpen_sUnion|)
\end{itemize}
Using this structure, global continuity can be defined in terms
of open sets, anmd indeed this is how it is defined in Mathlib:
\begin{lstlisting}[language=lean]
structure Continuous (f : X → Y) : Prop where
  isOpen_preimage : ∀ s, IsOpen s → IsOpen (f ⁻¹' s)
\end{lstlisting}
However, when proving results about limits, convergence, and continuity,
Mathlib uses an alternative way based on \textbf{filters}.
This approach may seem more abstract initially, but it provides
a more general and powerful framework that works uniformly
across all topological spaces and, by extension, metric spaces.

\subsubsection{Limits and Convergence with Filters}

The concept of a lmit is quiet extended, there are many
types of limits to consider.
For intance the limit of a function at a point,
limits at infinity (from above or below), one-sided limits (from the left or right)
and so on.
Defining each of these separately would require a
huge amount of work to include in Mathlib,
with significant duplication of theorems and proofs.
Moreover, many fundamental theorems
(like the characterization of continuity via limits)
would need to be reproved for each type of limit.
Fortunately, Bourbaki solved this issue by introducing the notion
of \textbf{filters} to unify
all concepts of limits, convergence, neighborhoods and terms like eventually or
frequently often into a single framework.
Intuitively, a filter represents a notion of "sufficiently large" subsets.
More fomrally, a filter $F$ on a type $X$ is a collection of subsets
of $X$ satisfying three axioms:
\begin{lstlisting}[language=lean]
structure Filter (α : Type*) where
  sets : Set (Set α)
  univ_sets : Set.univ ∈ sets
  sets_of_superset {x y} : x ∈ sets → x ⊆ y → y ∈ sets
  inter_sets {x y} : x ∈ sets → y ∈ sets → x ∩ y ∈ sets
\end{lstlisting}
The field sets suggest to thinof a filter as
a collection sets. The three axioms correspond to:
\begin{enumerate}
  \item  $X \in F$ (the whole space is in the filter)
  \item  If $U \in F$ and $U \subseteq V$, then $V \in F$
        (supersets of "large" sets are "large")
  \item  If $U, V \in F$, then $U \cap V \in F$
        (finite intersections of "large" sets are "large")
\end{enumerate}
Note that if $F$ is a filter that contains the empty set,
then it contains all subsets of $X$.
Filters are quiet categorical objects and we are going to
to intuivetly make use of them instead of descibing it fomrally.
In particular we are going to use some of the following concetps:
\begin{itemize}
  \item \textbf{Neighborhood filter} \lstinline[language=lean]|𝓝 x|: In a topological space,
        this filter contains all neighborhoods of the point $x$.
        A set is in \lstinline[language=lean]|𝓝 x| if it contains an
        open set containing $x$. This captures the idea of "near $x$."
  \item \textbf{At top filter} \lstinline[language=lean]|atTop : Filter ℕ|: Contains sets that
        include all sufficiently large natural numbers.
        Formally, $U \in$ \lstinline[language=lean]|atTop| if and only if there
        exists $N$ such that $\{n \mid n \geq N\} \subseteq U$.
        This captures the idea of "$n \to \infty$."
  \item \lstinline[language=lean]|∀ᶠ x in f, p x| (\lstinline[language=lean]|f.Eventually p|):
        "Eventually in filter $f$, property $p$ holds."
        This means there exists some set $U \in f$ such that $p$ holds for all $x \in U$.
        For example, \lstinline[language=lean]|∀ᶠ n in atTop, n > 100| means
        "for all sufficiently large $n$, we have $n > 100$."
  \item \lstinline[language=lean]|∃ᶠ x in f, p x| (\lstinline[language=lean]|f.Frequently p|):
        "Frequently in filter $f$, property $p$ holds." This means for every set $U \in f$, there exists
        some $x \in U$ where $p$ holds. This captures the idea that $p$ holds "infinitely often" or
        "arbitrarily close."
        For example, \lstinline[language=lean]|∃ᶠ n in atTop, Even n| means
        "there are arbitrarily large even numbers."
  \item \lstinline[language=lean]|Tendsto f l₁ l₂|: "Function $f$ tends from filter
        $l_1$ to filter $l_2$." This is used for convergence.
\end{itemize}
\begin{example}
  We can express convergence of a sequence $s_n$ to its limit $a$ using filters.
  \lstinline[language=lean]|Tendsto s atTop (𝓝 a)|; meaning $s_n \to a$ as $n \to \infty$
\end{example}
\begin{example}
  Another more insigthfull example is the definition of local continuity;
  at a point $x$ or restricted to a subset.
  As seen before, the structure \lstinline[language=lean]|Continuous|,
  for global continuity, is defined
  in terms of open sets. However, Mathlib
  also provides alternative definitions
  for local continuity: \lstinline[language=lean]|ContinuousAt|
  defines continuity at a single point,
  \lstinline[language=lean]|ContinuousWithinAt| defines continuity within a
  set at a point, and
  \lstinline[language=lean]|ContinuousOn| defines continuity on an entire set.
  All these local characterizations are defined in terms of filters.
  The connection between the global continuity definition and the filter-based
  local definitions is established by the following fundamental theorem:
  \begin{lstlisting}[language=lean]
theorem Continuous.tendsto (hf : Continuous f) (x) : 
    Tendsto f (𝓝 x) (𝓝 (f x)) :=
  ((nhds_basis_opens x).tendsto_iff <| nhds_basis_opens <| f x).2 fun t ⟨hxt, ht⟩ =>
    ⟨f ⁻¹' t, ⟨hxt, ht.preimage hf⟩, Subset.rfl⟩
\end{lstlisting}
  The key concept here is \lstinline[language=lean]|FilterBasis|.
  Similar to how a topological space can be defined via a basis of open sets,
  a filter can be defined via a basis of sets.
  A basis  $B$ for a filter $F$ is a nonemprty collection of sets wich
  preserve intersectiohns;
  for any two sets $U, V \in B$, there exists a set $W \in B$ such
  that $W \subseteq U \cap V$.
  \begin{lstlisting}[language=lean]
  structure FilterBasis (α : Type*) where
    sets : Set (Set α)
    nonempty : sets.Nonempty
    inter_sets {x y} : x ∈ sets → y ∈ sets → ∃ z ∈ sets, z ⊆ x ∩ y
  \end{lstlisting}
  Given a basis, we can not only generate a filter,
  but also help proving properties about it, restricitng to one basis only.
  To better understand the proof, we can expand it slightly:
  \newpage
  \begin{lstlisting}[language=lean]
theorem Continuous'.tendsto (hf : Continuous f) (x) :
    Tendsto f (𝓝 x) (𝓝 (f x)) := by
  rw [(nhds_basis_opens x).tendsto_iff (nhds_basis_opens (f x))]
  intro t ⟨hft_in, ht_open⟩
  use f ⁻¹' t
  constructor
  · exact ⟨hft_in, ht_open.preimage hf⟩
  · exact Subset.rfl
\end{lstlisting}

  The statemnt \lstinline[language=lean]|Tendsto f (𝓝 x) (𝓝 (f x))| is reformulation of
  continuity in terms of neighborhood filters \lstinline[language=lean]|𝓝 x|.
  Meaning that for every neighborhood of $f(x)$, there exists a neighborhood of $x$
  that maps into it under $f$.
  In particular the neighborhood filter \lstinline[language=lean]|𝓝 x| has a basis
  consisting of all open sets containing the point $x$,
  wich is stated by \lstinline[language=lean]|nhds_basis_opens x|.
  This recall the standard defintion of neighborhoods in topology;
  a set is a neighborhood of $x$
  if it contains an open set containing $x$.
  Thus the line
  \lstinline[language=lean]|rw [(nhds_basis_opens x).tendsto_iff (nhds_basis_opens (f x))]|
  restricts the goal in terms of these basis open sets.
  This converts our goal to: for every open neighborhood (set) $t$ of $f(x)$,
  there exists an open neighborhood $s$ of $x$ with $f^{-1} s \subseteq t$.
  Next, we introduce an arbitrary open neighborhood $t$ of $f(x)$
  using \lstinline[language=lean]|intro t ⟨hft_in, ht_open⟩|.
  The natural choice for the witness neighborhood $s$ is
  \lstinline[language=lean]|f ⁻¹' t|, which contains $x$ because $f(x) \in t$.
  We construct this using \lstinline[language=lean]|use f ⁻¹' t|.
  The first part of the goal is to show that $x \in f^{-1}(t)$
  and that $f^{-1}(t)$ is open.
  This is done using \lstinline[language=lean]|constructor; exact ⟨hft_in, ht_open.preimage hf⟩|,
  leveraging the global continuity of $f$: since $t$ is open and $f$ is continuous,
  the preimage $f^{-1}(t)$ is also open.
  The second part of the goal is to show that $\forall x \in f^{-1}(t), f(x) \in t$,
  which always holds by the definition of preimage; \lstinline[language=lean]|exact Subset.rfl|.
  Here the full example: [\href{https://live.lean-lang.org/#codez=JYWwDg9gTgLgBAWQIYwBYBtgCMBQOJgCmAdnACoEToQDmAnnAGLDoyFRwDKhMOAbkijAkWdITgBvABpwAmnABacAFzk6RAFQBffoOGjxAbQqRqNYAGMk6TmCQXxUgLpxjlM5eu3742U91CImJwABQAZipwMoBJhHIAlKEAHpFScXhohNCEIHAAwhDEMMDEAK4QJQDOAOQ1AHRsxAAmFTAQoagRqvmFxWWVcGEJIYkJyjhw5CTNrQOhgKwbgLs7cCPzS+HLcaMAvHBYdONwUADuriHEqM0A+lhIFcAVlwQkFRv1Uy0Ql8BhEWcXDzc7g8nsQXusRnF/BNijAoG14IAL8g6MC+xAANHBUCiQYBL8gOlXEEUA3gQATqqcF4EwsBRaUBKFlaUAOAHa4IREvZEcjURisY8iMRamAoIRQEgaOIOniJqz2ZyuCUsBUeLUoGF0EA}{link to Lean live}]

  Using this bridge from global to local continuity,
  we can understand the following connection:
  \begin{lstlisting}[language=lean]
def ContinuousAt (f : X → Y) (x : X) :=
  Tendsto f (𝓝 x) (𝓝 (f x))
theorem Continuous.continuousAt (h : Continuous f) : ContinuousAt f x :=
  h.tendsto x
\end{lstlisting}
  The remaining local continuity concepts are defined similarly:
  \begin{lstlisting}[language=lean]
def ContinuousWithinAt (f : X → Y) (s : Set X) (x : X) : Prop :=
  Tendsto f (𝓝[s] x) (𝓝 (f x))
def ContinuousOn (f : X → Y) (s : Set X) : Prop :=
  ∀ x ∈ s, ContinuousWithinAt f s x
\end{lstlisting}
  Note that \lstinline[language=lean]|𝓝[s] x| denotes the neighborhood filter
  of $x$ restricted
  to the set $s$, allowing us to study the behavior of functions on arbitrary subsets.
  Next we will use \lstinline[language=lean]|ContinuousOn| in our formalization.
\end{example}

\chapter{Formalizing the topologist's sine curve}

As part of my thesis work, with the help and revision from Prof David Loeffler,
I have formalized a well-known counterexample in topology:
the \textbf{topologist’s sine curve}.
This classic example illustrates a space that is \textbf{connected}
but not \textbf{path-connected}.
My original proof follows Conrad's paper (\cite{Conrad_connnotpathconn}),
with a few modifications
and some differences from the final
formalization \href{https://leanprover-community.github.io/mathlib4_docs/Counterexamples/TopologistsSineCurve.html}{\textbf{Counterexamples – Topologist's Sine Curve}}.
The topologist's sine curve is defined as the graph of $y = \sin(1/x)$
for $x \in (0, \infty)$,
together with the origin $(0, 0)$.
We define three sets in $\mathbb{R}^2$:
\begin{itemize}
  \item $S$: the oscillating curve $\{(x, \sin(1/x)) : x > 0\}$
  \item $Z$: the singleton set $\{(0, 0)\}$
  \item $T$: their union $S \cup Z$
\end{itemize}
In Lean, this is expressed as follows:
\begin{lstlisting}[language=lean]
  open Real Set
  def pos_real := Ioi (0 : ℝ)
  noncomputable def sine_curve := fun x ↦ (x, sin (x⁻¹))
  def S : Set (ℝ × ℝ) := sine_curve '' pos_real
  def Z : Set (ℝ × ℝ) := { (0, 0) }
  def T : Set (ℝ × ℝ) := S ∪ Z
\end{lstlisting}
We open the \lstinline[language=lean]|Real| and \lstinline[language=lean]|Set| namespaces
to avoid prefixing real number and set operations with \lstinline[language=lean]|Real.|
and \lstinline[language=lean]|Set.|, respectively.
We define the interval $(0, \infty)$ as \lstinline[language=lean]|pos_real|,
using the predefined notation \lstinline[language=lean]|Ioi 0|, from \lstinline[language=lean]|Set|.
The function \lstinline[language=lean]|sine_curve| maps a positive real number
to a point on the topologist's sine curve in $\mathbb{R}^2$.
Here, \lstinline[language=lean]|''| denotes the image of a set under a function.
It's noncomputable because it involves the sine function,
which is not computable in Lean's core logic.
The sets \lstinline[language=lean]|S|, \lstinline[language=lean]|Z|,
and \lstinline[language=lean]|T|
are defined using set operations,
and \lstinline[language=lean]|{ (0, 0) }| denotes the singleton
set containing the point $(0, 0)$.
The sets are subsets of the product space $\mathbb{R}^2$,
represented as \lstinline[language=lean]|ℝ × ℝ|.
The sin function \lstinline[language=lean]|sin| is defined in the
\lstinline[language=lean]|Real|.

The goal is to prove that $T$ is connected but not path-connected.
Let's start with connectedness.
\section{$T$ is connected}
First of all one can directlly see that  $S$ is connected, since it is the
image of the set ($(0, \infty)$) under the continuous map
$x \mapsto (x, \sin(1/x))$ and a interval in $\mathbb{R}$ is connected.
Moreover, the closure of $S$ is connected, and every set in between a connected
set and its closure are connected.
Since $T$ is contained in the closure of  $S$, $T$ is connected.
This is how a mathematician would argue informally, using known facts.
However, in a formal proof, one must justify each step.
For instance, justifying that $S$ is connected
requires proving that the map
$x \mapsto (x, \sin(1/x))$ is continuous on $(0, \infty)$
and that $(0, \infty)$ is connected.
As we have seen, even showing that a rational number is non-negative
requires several steps and the use of various lemmas from Mathlib.
Similarly, proving that a set is connected can involve multiple steps
for the
newer programmer.
We can use the structure \lstinline[language=lean]|IsConnected|,
to set up the statement and see if we can argue similarly in Lean.
\begin{lstlisting}[language=lean]
lemma S_is_conn : IsConnected S := by sorry
\end{lstlisting}
In the file where \lstinline[language=lean]|IsConnected| is defined,
\texttt{Topology/Connected/Basic.lean}, we see that it requires $S$ to be nonempty and preconnected.
One can verify this by unfolding \lstinline[language=lean]|IsConnected| in the goal.
\begin{lstlisting}[language=lean]
lemma S_is_conn : IsConnected S := by
  unfold IsConnected 
  ⊢ S.Nonempty ∧ IsPreconnected S
  sorry
\end{lstlisting}
Following the definition of \lstinline[language=lean]|IsPreconnected|, we see that it captures
the usual definition; $S$ cannot be
partitioned into two nonempty disjoint open sets.
This trivially requires
nonemptiness to make sense.
The \lstinline[language=lean]|unfold| tactic helps to expand definitions; one can use it to expand the definition of $S$ or
\lstinline[language=lean]|pos_real| defined before, as well as other Mathlib expressions.
Reflecting our argument, we can check if Mathlib includes the fact
that every interval
is connected and that connectedness is preserved
under continuous maps.
Indeed, in \texttt{Topology/Connected/Interval.lean}, we find the theorem
\lstinline[language=lean]|isConnected_Ioi.image|, stating that the image of an
interval of the form $(a, \infty)$
under a continuous map is connected.

\begin{lstlisting}[language=lean]
lemma S_is_conn : IsConnected S := by
  apply isConnected_Ioi.image
  -- ⊢ ContinuousOn sine_curve (Ioi 0) 
  sorry
\end{lstlisting}
The theorem \lstinline[language=lean]|isConnected_Ioi.image| requires proving the continuity of the map
on the interval $(0, \infty)$, which is expressed as
\lstinline[language=lean]|ContinuousOn sine_curve (Ioi 0)|.
The predicate \lstinline[language=lean]|ContinuousOn f S|
expresses that a function $f$ is continuous on a set $S$, which is what we need to prove now.
The function $x \mapsto (x, \sin(1/x))$ is continuous on $(0, \infty)$ as the
product of two functions continuous on the given domain; the identity map $x \mapsto x$
and the map $x \mapsto \sin(1/x)$.
Here is the full proof in Lean:
\begin{lstlisting}[language=lean]
lemma inv_is_continuous_on_pos_real : ContinuousOn (fun x : ℝ => x⁻¹) (pos_real) := by
  apply ContinuousOn.inv₀
  · exact continuous_id.continuousOn
  · intro x hx; exact ne_of_gt hx

lemma sin_comp_inv_is_continuous_on_pos_real : ContinuousOn
 (sine_curve) (pos_real) := by
  apply ContinuousOn.prodMk continuous_id.continuousOn
  apply continuous_sin.comp_continuousOn
  exact inv_is_continuous_on_pos_real
\end{lstlisting}
Starting from the bottom lemma, \lstinline[language=lean]|ContinuousOn.prodMk| states that the product of two functions continuous on a set is continuous on that set,
requiring a proof of the continuity of each component.
The first component is the identity map, which is continuous on any set. Mathlib provides
\lstinline[language=lean]|continuous_id.continuousOn| for this purpose.
The second component is the composition of the sine function with the inverse function.
The sine function is continuous everywhere, and for this we can use
\lstinline[language=lean]|continuous_sin|.
The method \lstinline[language=lean]|comp_continuousOn| is accessible from the
fact that \lstinline[language=lean]|continuous_sin| gives
an instance of a continuous map and is generalized
in the \lstinline[language=lean]|ContinuousOn| module.
The theorem \lstinline[language=lean]|Continuous.comp_continuousOn|
states that the composition of a continuous function with a function
that is continuous on a set is continuous on that set,
and requires proof of the continuity
on the set of the inner function.
We separate the proof that the inverse function is continuous on the positive reals
into the auxiliary lemma \lstinline[language=lean]|inv_is_continuous_on_pos_real|.
The theorem \lstinline[language=lean]|continuousOn_inv₀| states that, if a function
is continuous and non-zero on a set, then its inverse is continuous on that set.
The continuity of the identity map is proved as before.
The second argument requires proving that $x \neq 0$ for all $x$ in $(0, \infty)$.
\begin{lstlisting}[language=lean]
  · intro x hx
    exact ne_of_gt hx
\end{lstlisting}
The hypothesis \lstinline[language=lean]|hx| states that $x$ is in $(0, \infty)$,
which implies that $x > 0$.
The theorem \lstinline[language=lean]|ne_of_gt| states that if a
real number is greater than zero,
then it is non-zero, which completes the proof.
Thus the final proof goes as follows:
\begin{lstlisting}[language=lean]
lemma S_is_conn : IsConnected S := by
  apply isConnected_Ioi.image 
  · exact sin_comp_inv_is_continuous_on_pos_real
\end{lstlisting}

When writing a proof, one starts by working out the informal argument on paper.
Then one tries to translate it into Lean, step by step, looking for theorems in Mathlib.
Afterwards, one can try to optimize the proof by removing unnecessary steps or refactoring it.
Proving properties like continuity and connectedness is very common,
and there are obviously ways to achieve this with less work.
Let's showcase a refactoring of the entire proof.
First, the auxiliary lemmas
can be reduced to one-liners.
\begin{lstlisting}[language=lean]
lemma inv_is_continuous_on_pos_real : ContinuousOn (fun x : ℝ => x⁻¹) (pos_real) :=
  ContinuousOn.inv₀ (continuous_id.continuousOn) (fun _ hx =>  ne_of_gt hx)
  
lemma sin_comp_inv_is_continuous_on_pos_real : ContinuousOn
 (sine_curve) (pos_real) :=
 ContinuousOn.prodMk continuous_id.continuousOn <|
  Real.continuous_sin.comp_continuousOn <| (inv_is_continuous_on_pos_real)
\end{lstlisting}
We removed the \lstinline[language=lean]|by| keyword since we can provide a \textbf{term}
that directly proves the statement.
In \lstinline[language=lean]|inv_is_continuous_on_pos_real|, we directly apply
\lstinline[language=lean]|ContinuousOn.inv₀| with the two required arguments.
Notice that we can use a lambda function \lstinline[language=lean]|fun _ hx =>  ne_of_gt hx|
to prove that $x \neq 0$ for all $x$ in $(0, \infty)$
(recall the propositions-as-types correspondence).
In the next lemma, we use
the \lstinline[language=lean]|<\|| reverse application operator,
which allows us to avoid parentheses by changing the order of application.
This means that \lstinline[language=lean]|f <| g <| h| is
equivalent to \lstinline[language=lean]|f (g h)|.
% In our case,
% \lstinline[language=lean]|Real.continuous_sin.comp_continuousOn <\||
% \lstinline[language=lean]|(inv_is_continuous_on_pos_real)|
% is equivalent to
% \lstinline[language=lean]|Real.continuous_sin.comp_continuousOn (inv_is_continuous_on_pos_real)|.
We can inline these two lemmas into the main proof to get a final one-liner:
\begin{lstlisting}[language=lean]
lemma S_is_conn : IsConnected S :=
  isConnected_Ioi.image sine_curve <| continuous_id.continuousOn.prodMk <|
    continuous_sin.comp_continuousOn <|
    ContinuousOn.inv₀ continuous_id.continuousOn (fun _ hx => ne_of_gt hx)
\end{lstlisting}
Notice again the use of the \textbf{pipe} operator.
Reading from left to right, we are building up the proof by successive applications:
\begin{itemize}
  \item We start with \lstinline[language=lean]|isConnected_Ioi.image sine_curve|, which states that the image of $(0, \infty)$ under \lstinline[language=lean]|sine_curve| is connected if we can prove the function is continuous.
  \item We then apply \lstinline[language=lean]|continuous_id.continuousOn.prodMk|, which constructs the product of two continuous functions.
  \item Next, \lstinline[language=lean]|continuous_sin.comp_continuousOn| provides the continuity of the sine composition.
  \item Finally, \lstinline[language=lean]|ContinuousOn.inv₀ continuous_id.continuousOn (fun _ hx => ne_of_gt hx)| proves the continuity of the inverse function on positive reals.
\end{itemize}
The entire chain can be read as building the continuity proof from the innermost function (the inverse) outward to the complete sine curve function, which is then used to prove that $S$ is connected.
% The expression \lstinline[language=lean]|continuousOn_inv₀.mono fun _ hx ↦ hx.ne'|
% applies the theorem and provides the required arguments.
% The \lstinline[language=lean]|mono| method allows us to weaken the domain of
% continuity from \lstinline[language=lean]|x : x = 0 |
% to \lstinline[language=lean]|pos_real|,
% which is a subset of the former.
% The lambda function \lstinline[language=lean]|fun _ hx ↦ hx.ne'| proves that $x \neq 0$ for all $x$ in $(0, \infty)$.
% This is a common pattern in Lean, where we often need to prove that certain
% conditions hold for all elements of a set.


% \begin{lstlisting}[language=lean]
% lemma sine_curve_is_continuous_on_pos_real_one_liner : ContinuousOn (sine_curve) (pos_real) :=
%  continuous_id.continuousOn.prodMk <| Real.continuous_sin.comp_continuousOn
%    <| continuousOn_inv₀.mono fun _ hx ↦ hx.ne'
% \end{lstlisting}
Since the intersection of $Z$ and $S$ is empty, we cannot
directly conclude that $T$ is connected from the connectedness of its components alone.
However, we can use the fact that every subset between a connected set and its closure is connected.
\begin{theorem}
  Let $C$ be a connected topological space, and denote $\overline{C}$ as its closure.
  It follows that every subset $C \subseteq S \subseteq \overline{C}$ is connected.
\end{theorem}
In Mathlib, this theorem is available as
\lstinline[language=lean]|IsConnected.subset_closure|.
We can set up the statement and progress from there.
\begin{lstlisting}[language=lean]
theorem T_is_conn : IsConnected T := by
  apply IsConnected.subset_closure
  · exact S_is_conn -- ⊢ IsConnected ?s
  · tauto_set -- ⊢ S ⊆ T
  · sorry -- ⊢ T ⊆ closure S
\end{lstlisting}
The theorem requires three goals:
\begin{itemize}
  \item That $S$ is connected, which was already proved in \lstinline[language=lean]|S_is_conn|.
  \item That $S \subseteq T$, which is a trivial set operation.
        The tactic \lstinline[language=lean]|tauto_set| handles this kind of set tautologies.
  \item That $T \subseteq \overline{S}$ (the closure of $S$), which requires proof.
\end{itemize}
Let's continue with the final point.
\begin{lstlisting}[language=lean]
lemma T_sub_cls_S : T ⊆ closure S := by
  intro x hx
  cases hx with
  | inl hxS => exact subset_closure hxS
  | inr hxZ =>
      sorry
\end{lstlisting}
Proving that one set is contained in another can be done naively in a pointwise manner.
We introduce an element $x \in \mathbb{R}^2$ together with the proof that $x \in T$.
Since $T$ is a union, we use \lstinline[language=lean]|cases| to separate the two cases.
When $x \in S$, the goal is trivially solved by \lstinline[language=lean]|exact subset_closure hxS|.
The case where $x \in Z$, requires more work.

Now a trick. Looking for existing theorems using mathlib documentation is
quiet challenging, while you are still learning
the sintax and adapt to the naming convention.
One can use several ways to look for the exact theorems.
A useful tool is Loogle (similar to Haskell's Hoogle),
which helps you find theorems by their type signature or name patterns.
You can access it at \url{https://loogle.lean-lang.org/} or
use it directly in VS Code.
Depending on the previous work in the file,
Lean can already unify the goal with available
theorems and suggest the next step.
For some tactics, adding a question mark causes Lean to automatically search for
the next step involving the use of that tactic.
For instance, one can type
\lstinline[language=lean]|apply?| or
\lstinline[language=lean]|exact?|, which search
for applicable lemmas or definitions
to close the goal.
The tactics \lstinline[language=lean]|rw?|
and \lstinline[language=lean]|simp?|
work similarly but for rewriting and simplification.

At this point, \lstinline[language=lean]|apply?| suggests several ways to proceed,
some involving filters:
\begin{lstlisting}
Try this: refine Frequently.mem_closure ?_
Remaining subgoals:
  ⊢ ∃ᶠ (x : ℝ × ℝ) in 𝓝 x, x ∈ S
\end{lstlisting}
and others following the more familiar metric space approach:
\begin{lstlisting}
Try this: refine Metric.mem_closure_iff.mpr ?_
Remaining subgoals:
  ⊢ ∀ ε > 0, ∃ b ∈ S, dist x b < ε
\end{lstlisting}

The best approach, however, is to think first about how you would tackle the problem
on paper, as mentioned earlier. Since we are working with a metrizable topology on $\mathbb{R}$,
we know that the closure of a set contains all its limit points.
To show that the point $(0, 0)$ is contained in the closure of $S$,
we need to show that it is a limit point of $S$.
Thus, one can define a sequence in $S$ tending to $(0, 0)$,
and the result follows. Instead of using properties of a metric space,
we use filters, as explained before to work with limits.
We can prove that $T \subseteq \overline{S}$, by
showing that the origin is a limit point of $S$.
We construct a sequence $f : \mathbb{N} \to \mathbb{R}^2$ in $S$
converging to $(0,0)$ using
\lstinline[language=lean]|Tendsto|:
\newpage
\begin{lstlisting}[language=lean]
lemma T_sub_cls_S : T ⊆ closure S := by
  intro x hx
  cases hx with
  | inl hxS => exact subset_closure hxS
  | inr hxZ =>
      rw [hxZ]
      -- Define sequence: f(n) = (1/(nπ), 0)
      let f : ℕ → ℝ × ℝ := fun n => ((n * Real.pi)⁻¹, 0)
      -- Show f converges to (0, 0)
      have hf : Tendsto f atTop (𝓝 (0, 0)) := by
        refine Tendsto.prodMk_nhds ?_ tendsto_const_nhds
        exact tendsto_inv_atTop_zero.comp
          (Tendsto.atTop_mul_const' Real.pi_pos tendsto_natCast_atTop_atTop)
      -- Show f eventually takes values in S
      have hf' : ∀ᶠ n in atTop, f n ∈ S := by
        filter_upwards [eventually_gt_atTop 0] with n hn
        exact ⟨(n * Real.pi)⁻¹,
          inv_pos.mpr (mul_pos (Nat.cast_pos.mpr hn) Real.pi_pos),
          by simp [f, sine_curve, inv_inv, Real.sin_nat_mul_pi]⟩
      -- Apply sequential characterization of closure
      exact mem_closure_of_tendsto hf hf'
\end{lstlisting}
The proof is already reduced as much as possible.
Let's break down what's happening in without getting into details.
Using \lstinline[language=lean]|let|, we define
$f(n) = \left(\frac{1}{n\pi}, 0\right)$,
which we will show converges to $(0,0)$
and stays in $S$.

\begin{itemize}

  \item \textbf{Convergence proof} (\lstinline[language=lean]|hf|):
        We show \lstinline[language=lean]|Tendsto f atTop (𝓝 (0, 0))|.
        With \lstinline[language=lean]|Tendsto.prodMk_nhds|,
        we need to show that both coordinates tend to $0$, separately.
        For the first coordinate, we compose
        \lstinline[language=lean]|tendsto_inv_atTop_zero|
        (which states $\frac{1}{x} \to 0$ as $x \to \infty$) with the
        fact that $n\pi \to \infty$.
        The second constant coordinate is handled by
        \lstinline[language=lean]|tendsto_const_nhds|
  \item \textbf{Membership proof} (\lstinline[language=lean]|hf'|):
        We show \lstinline[language=lean]|∀ᶠ n in atTop, f n ∈ S|, meaning $f(n) \in S$
        for all sufficiently large $n$.
        We use \lstinline[language=lean]|filter_upwards|,
        which allows us to combine
        hypotheses about properties that hold eventually to prove another property holds eventually.
        Here, we combine it with \lstinline[language=lean]|eventually_gt_atTop 0|,
        which states that eventually $n > 0$.
        For such $n$, we show $f(n) = \left(\frac{1}{n\pi}, 0\right)$ is in $S$ by noting that
        the second term is:
        $$
          \sin\left(\frac{1}{\left(\frac{1}{n\pi}\right)}\right) = \sin(n\pi) = 0.
        $$

\end{itemize}
Finally, \lstinline[language=lean]|mem_closure_of_tendsto|
combines these facts:
if a sequence eventually stays in $S$ and converges to $x$,
then $x$ is in the closure of $S$.
% (ALTERNATIVE PROOF)

% \begin{proof}
%   To show that $S$ lies in the closure of $S^+$, we have to express each $p \in S$ as a limit of a
%   sequence of points in $S^+$. If $p \in S^+$, we use the constant sequence $\{p, p, \ldots\}$. If $p = (0, y)$ with
%   $|y| \leq 1$, we argue as follows. Certainly $y = \sin(\theta)$ for some $\theta \in [-\pi, \pi]$, whence $y = \sin(\theta + 2n\pi)$
%   for all positive integers $n$. Thus, for $x_n = 1/(\theta + 2n\pi) > 0$ we have $\sin(1/x_n) = y$ for all $n$. Since
%   $x_n \to 0$ as $n \to \infty$, we have $(x_n, \sin(1/x_n)) = (x_n, y) \to (0, y)$.
% \end{proof}
\subsubsection{Finalising the first part of the proof}
If you are a one-liner enthusiast like me, you don't mind trying to combine
bits and pieces to get a clean final result.
We can simplify the final theorem as follows initially:
\begin{lstlisting}[language=lean]
theorem T_is_conn : IsConnected T := 
  IsConnected.subset_closure S_is_conn (by tauto_set) T_sub_cls_S
\end{lstlisting}
The second argument is still in tactic mode with \lstinline[language=lean]|by tauto_set|,
but it looks clean and we can keep it as is.
With a bit of courage, we can also inline the proof of \lstinline[language=lean]|S_is_conn|
(while \lstinline[language=lean]|T_sub_cls_S| is way too long to inline)
to get a more self-contained one-liner:
\begin{lstlisting}[language=lean]
theorem T_is_conn : IsConnected T :=
  IsConnected.subset_closure (isConnected_Ioi.image sine_curve <|
    continuous_id.continuousOn.prodMk <|
    Real.continuous_sin.comp_continuousOn <|
    ContinuousOn.inv₀ continuous_id.continuousOn
    (fun _ hx => ne_of_gt hx)) (by tauto_set) T_sub_cls_S
\end{lstlisting}
Making these amendments is not only for the sake of shortening the proof.
Lean will, obviously, compile the proof faster by not
entering tactic mode or using multiple tactics.
Tactics internally hide many operations they automatically perform to close the goal.
Moreover, if we directly provide
a term for the proof, Lean will infer and unify everything by definitional
equality.
By providing explicit proof terms, we give Lean less work to do, making the
proof more transparent and efficient.
This practice of "golfing" is essential in a huge library such as Mathlib
community that needs to balance performance and maintainability.
From now on the rest of th code will be presented in it's reduced form.
Here is the link to the entire first part of the proof:
[\href{https://live.lean-lang.org//#codez=JYWwDg9gTgLgBAWQIYwBYBtgCMBQEwCmAdnACr4ToQDmAnnAGLDowFRwDKB8A8lACZscOQQDM4kAM4B9KASTo4ALgC8nbgDoAkhGBwAFAAZlcQLiEAShxEIRAMYRwAVxhIs6AnDFxJwIgWm2jlAAbh6qcKKOJAAecIBlhAbRADRwAEry6Bo+JPrRgN4EAJ3mll4cJlzw+qZwAOtm5spq+tn+gSEEDQDknRIQMnIKIgTiAFrl3AbVdRaNcADeBoYphg0AvkPipOOVU/WzZYBURHAjwu4gIEhk0pKOWAHoMmVKZHCAYURwtlQ3cpyzWLQ4OBwXwwKAQOCxVDRQEfJCSAiSOBQuAAd2AaBhAB9gURFFCyioAHxwAjRJC2eA3LDwmD3PpBDz4rE49hQsZEmFAoFQFFwADabIAupyue54OJnoBUQjggCTCMy1eXhSIkEhEgz6EgAKjSGQ0YGA5kKy0sXK5qCQoSRErIxH4khg4PEKHIYAMgFYNwC7O4tjQ1wv8Raa5KJfB49WD+AgANbSIioO1wAD80jgrCIdodARs9pjcckAa5pPJ8FT6Yg0l8wWkzvw0gAXmwIBp7OB86b9EwWGwNKRbfbG9WwNIQI50JmiPaeukFHrgNIpCnexmiCgAMJw2kDqswF0m01Ii2M0Q9Z6AACJAAW4cBIvjgA5S4hIgAgiX5+gF7rnBztQaSOMAopACRE+QIUIiBgRwFHQWhpGoDdt3wOBDEFVF0VQS8kSIVsSTJCk4EAC/INTgbUp0yfVDQKJJMKBCs5z6DRwHYfRh1Hed9AAORQJt1xoyQ6LAVkiAaYiZ248wKLfU1/m8UBXT5UQUhaAIglCFJqIrFIhOyGMUCHEc52AQVAEvyfNCxwkACBAOlvn8CBRGkEs+ytK1OmENACGgcyrmAGR7CIEhni0SQVxsPwKQIfgXlUGEAqC3yCFC/gsluGlLIZAwvJikLWH4aQdGADRQCQagPAUtpLQAHkxEUfJgXxHAgRwZGABLqtq+rJB4IgwwgCNIzgCqRSElqiDqhrrl8JsHEHIaRvakh+tNGKauGtqOvyohgkAAIIPhsJaZvLZqdtahqOpFfRlTgZNkTVPxpBs6D4ChYoDEklxnDLGkGlIa5bnuR4gA}{link to Lean live}]
\begin{note}
  The proof merged into the Mathlib library,
  takes $Z$ as $\{0\} \times [-1,1]$
  instead of the singleton $\{(0,0)\}$.
  This, together with the fact that $T$ equals the closure of $S$,
  yields a stronger and more general result.
  This stronger version shows that a closed set
  (specifically, the closure of $S$) can be connected but not path-connected.
  Showing that forming closure can destroy
  the property of path connectedness for subsets of a topological space.
\end{note}
\section{$T$ is not path-connected}
The main and most substantial part is showing that $T$ is not path-connected.
Showing this informally already requires constructing and
pointing out various steps in order to convince
an ideal reader.
One can argue informally by contradiction.
Suppose a path exists in the topologist's sine curve $T$
connecting a point in $S$ to a point in $Z$.
As the path approaches the $y$-axis (where $x \to 0$),
the $y$-coordinate must oscillate infinitely between $-1$ and $1$
due to the behavior of $\sin(1/x)$ as $x \to 0^+$.
This infinite oscillation contradicts the continuity of the path,
which is a fundamental requirement for path-connectedness.
To be more precise, we need to construct a sequence
that it eventually oscillates, establishing the contradiction.
We start by setting up the theorem:
\begin{lstlisting}[language=lean]
theorem T_is_not_path_conn : ¬ (IsPathConnected T) := 
  by sorry
\end{lstlisting}
In mathematics, we normally define a path-connected space as follows.
\begin{definition}
  A topological space $X$ is said to be path-connected if for every two points $a, b \in X$, there
  exists a path, i.e., a continuous map $p : [0, 1] \to X$ such that $p(0) = a$ and $p(1) = b$.
\end{definition}
The interval $[0, 1]$ is the standard choice for the domain of paths.
% In Mathlib, \lstinline[language=lean]|PathConnectedSpace X| is a type class that asserts the entire
% topological space $X$ is path-connected, while
\lstinline[language=lean]|IsPathConnected S| is a predicate used to infer that a subset $S$
of a topological space is path-connected.
\begin{lstlisting}[language=lean]
def IsPathConnected (F : Set X) : Prop :=
  ∃ x ∈ F, ∀ ⦃y⦄, y ∈ F → JoinedIn F x y
\end{lstlisting}
The auxiliary predicate \lstinline[language=lean]|JoinedIn| is defined as:
\begin{lstlisting}[language=lean]
def JoinedIn (S : Set X) (x y : X) : Prop :=
  ∃ γ : Path x y, ∀ t, γ t ∈ S
\end{lstlisting}
where \lstinline[language=lean]|Path x y| denotes a continuous map $\gamma : [0,1] \to X$ with
$\gamma(0) = x$ and $\gamma(1) = y$.
We can use the \lstinline[language=lean]|unitInterval| \textbf{subtype} of
\lstinline[language=lean]|ℝ| representing the interval
$[0,1]$.
Now let's start with the first part of the proof:
\begin{lstlisting}[language=lean]
theorem T_is_not_path_conn : ¬ (IsPathConnected T) := by
  -- Assume we have a path from z = (0, 0) to w = (1, sin(1))
  have hz : z ∈ T := Or.inr rfl
  have hw : w ∈ T := Or.inl ⟨1, ⟨zero_lt_one' ℝ, rfl⟩⟩
  intro p_conn
  apply IsPathConnected.joinedIn at p_conn
  specialize p_conn z hz w hw
  let p := JoinedIn.somePath p_conn
\end{lstlisting}
We introduce two points: $z = (0, 0)$ and $w = (1, \sin(1))$, and prove they are both in $T$, in
\lstinline[language=lean]|hz hw| and \lstinline[language=lean]|hw| (we use the introduction rule
for Or, seen it as a sum type).
Using \lstinline[language=lean]|intro p_conn|, we assume that $T$ is path-connected.
Notice that the goal is now \lstinline[language=lean]|False|, meaning we must find a contradiction.
The last three lines extract an explicit path \lstinline[language=lean]|p| connecting $z$ and $w$.
\lstinline[language=lean]|apply IsPathConnected.joinedIn at p_conn|
transforms the path-connectedness assumption into the statement
that any two points in $T$ are joined.
\lstinline[language=lean]|specialize p_conn z hz w hw| specializes this
to our specific points $z$ and $w$.
Then we extrat a path and store it using
\lstinline[language=lean]|let p := JoinedIn.somePath p_conn|.

Conrad's paper (\cite{Conrad_connnotpathconn}) defines a time $t_0 \in [0, 1]$
as the first time the path $p$ jumps from $(0,0)$ to the graph of $\sin(1/x)$, where
the x-coordinate map ($x: \mathbb{R}^2 \to \mathbb{R} $) of $p$ is positive.
$$
  t_0 = \inf\{t \in [0, 1] : x(p(t)) > 0\}
$$
The argument then uses the continuity of the $x$-coordinate map composed with the path $p$.
By continuity at $t_0$, we can find a neighborhood around $t_0$
where the path stays close to $(0,0)$. Specifically, with $\varepsilon = 1/2$,
there exists $\delta > 0$ such that for all $t$ with $|t - t_0| < \delta$,
we have $\|p(t) - p(t_0)\| < 1/2$.
We want to show the oscillating behavior around (0,0) indeed.
To simplify some steps, we instead define
$$
  t_0 = \sup\{t \in [0, 1] : x(p(t)) = 0\}
$$
to be the last time the path remains at $(0,0)$.
The same continuity argument applies with this definition.
\begin{lstlisting}[language=lean]
-- Consider the composition of the x-coordinate map with p, which is continuous
have xcoord_pathcont : Continuous fun t ↦ (p t).1 := continuous_fst.comp p.continuous
-- Let t₀ be the last time the path is on the y-axis
let t₀ : unitInterval := sSup {t | (p t).1 = 0}
let xcoord_path := fun t => (p t).1
-- The x-coordinate of the path at t₀ is 0
have hpt₀_x : (p t₀).1 = 0 :=
  (isClosed_singleton.preimage xcoord_pathcont).sSup_mem ⟨0, by aesop⟩
-- By continuity of the path, we can find a δ > 0 such that
-- for all t in [t₀ - δ, t₀ + δ], ||p(t) - p(t₀)|| < 1/2
-- Hence the path stays in a ball of radius 1/2 around (0, 0)
obtain ⟨δ, hδ, ht⟩ : ∃ δ > 0, ∀ t, dist t t₀ < δ →
  dist (p t) (p t₀) < 1/2 :=
  Metric.eventually_nhds_iff.mp <| Metric.tendsto_nhds.mp (p.continuousAt t₀) _ one_half_pos
\end{lstlisting}
The final statement uses the \lstinline[language=lean]|obtain| tactic to extract witnesses from an existential statement.
This tactic destructures the existential quantifier $\exists \delta > 0, \ldots$ into
\lstinline[language=lean]|δ| (the distance), \lstinline[language=lean]|hδ| (the proof that $\delta > 0$),
and \lstinline[language=lean]|ht| (the proof that the distance condition holds).
Since $\mathbb{R}^2$ is a metric space, we can work with the distance function \lstinline[language=lean]|dist : ℝ × ℝ → ℝ × ℝ → ℝ|,
which computes the Euclidean distance between two points.
The statement \lstinline[language=lean]|dist t t₀ < δ| expresses $|t - t_0| < \delta$ in the unit interval,
while \lstinline[language=lean]|dist (p t) (p t₀) < 1/2| expresses $\|p(t) - p(t_0)\| < 1/2$ in $\mathbb{R}^2$.
We assert that the path $p$ is
continuous at $t_0$ by \lstinline[language=lean]|p.continuousAt t₀|.
\lstinline[language=lean]|Metric.tendsto_nhds.mp| converts this
to the metric space characterization:
for any $\varepsilon > 0$, there exists $\delta > 0$ such that
points within $\delta$ of $t_0$ map to points within $\varepsilon$ of $p(t_0)$.
While, \lstinline[language=lean]|Metric.eventually_nhds_iff.mp| further unpacks
this into the $\forall t, dist\ t\ t_0 < \delta \to dist\ (p\ t)\ (p\ t_0) < \varepsilon$
form, requiring positivity of $\varepsilon = 1/2$ (\lstinline[language=lean]|one_half_pos|).

We can find a time $t_1$ greater than $t_0$ that remains in the neighborhood of $t_0$,
and obtain a point $a = x(p(t_1))) > 0$ which is positive.
\begin{lstlisting}[language=lean]
-- Let t₁ be a time when the path is not on the y-axis
-- t₁ is in (t₀, t₀ + δ], hence t₁ > t₀
obtain ⟨t₁, ht₁⟩ : ∃ t₁, t₁ > t₀ ∧ dist t₀ t₁ < δ := by
  let s₀ := (t₀ : ℝ) -- cast t₀ from unitInterval to ℝ for manipulation
  let s₁ := min (s₀ + δ/2) 1
  have hs₀_delta_pos : 0 ≤ s₀ + δ/2 := add_nonneg t₀.2.1 (by positivity)
  have hs₁ : 0 ≤ s₁ := le_min hs₀_delta_pos zero_le_one
  have hs₁': s₁ ≤ 1 := min_le_right ..
  sorry
-- Let a = xcoord_path t₁ > 0
-- This follows from the definition of t₀ and t₀ < t₁
-- so t₁ must be in S, which has positive x-coordinate
let a := (p t₁).1
have ha : a > 0 := by
  obtain ⟨x, hxI, hx_eq⟩ : p t₁ ∈ S := by
    cases p_conn.somePath_mem t₁ with
    | inl hS => exact hS
    | inr hZ =>
      -- If p t₁ ∈ Z, then (p t₁).1 = 0
      have : (p t₁).1 = 0 := by rw [hZ]
      -- So t₁ ≤ t₀, contradicting t₁ > t₀
      have hle : t₁ ≤ t₀ := le_sSup this
      have hle_real : (t₁ : ℝ) ≤ (t₀ : ℝ) := Subtype.coe_le_coe.mpr hle
      have hgt_real : (t₁ : ℝ) > (t₀ : ℝ) := Subtype.coe_lt_coe.mpr ht₁.1
      linarith
  simpa only [a, ← hx_eq] using hxI
\end{lstlisting}
The code is quite convoluted, and i will omit a
detailed explanation as well as some part of it.
However, it's worth mentioning a few key technical points.
The type \lstinline[language=lean]|unitInterval| is a \textbf{subtype}
of $\mathbb{R}$,
defined as $\{x : \mathbb{R} \mid 0 \leq x \leq 1\}$.
In Lean, a subtype \lstinline[language=lean]|{x : α // P x}|
bundles a value $x$ of type $\alpha$
together with a proof that $x$ satisfies the predicate $P$,
similar as for  \lstinline[language=lean]|Set|.
Anyway, rather than
being \lstinline[language=lean]|Set| (as subsets), they are type itself.
In particular, their terms do not share the type
of the underlying \lstinline[language=lean]|Set|.
Consequently, they lack of arithmetic properties
of the real numbers for instance.
We need to cast them to $\mathbb{R}$
(with \lstinline[language=lean]|let s₀ := (t₀ : ℝ)|)
first, then cast back to \lstinline[language=lean]|unitInterval|
by providing proofs that the bounds $[0, 1]$ are satisfied (\lstinline[language=lean]|hs₁|, \lstinline[language=lean]|hs₁'|).
In the second case of the inner statment of have \lstinline[language=lean]|ha : a > 0|,
if $p(t_1) \in Z$, then $(p\ t_1).1 = 0$ by definition of $Z = \{(0,0)\}$.
This implies $t_1 \leq t_0$ by the definition of $t_0$ as the supremum.
However, we also have $t_1 > t_0$ from our construction of $t_1$ (\lstinline[language=lean]|ht₁.1|).
The tactic \lstinline[language=lean]|linarith|, an automated solver for linear arithmetic,
recognizes this contradiction by observing both
\lstinline[language=lean]|hle_real : (t₁ : ℝ) ≤ (t₀ : ℝ)| and
\lstinline[language=lean]|hgt_real : (t₁ : ℝ) > (t₀ : ℝ)|.
Since these statements are contradictory, \lstinline[language=lean]|linarith|
proves \lstinline[language=lean]|False|.
Lemmas like \lstinline[language=lean]|Subtype.coe_lt_coe|
allow us to transfer inequalities between the subtype and its underlying type,
needed for \lstinline[language=lean]|linarith|.

Finally, \lstinline[language=lean]|simpa only [a, ← hx_eq] using hxI| completes the proof.
The tactic \lstinline[language=lean]|simpa| combines simplification (\lstinline[language=lean]|simp|)
with assumption matching. The directive \lstinline[language=lean]|only [a, ← hx_eq]|
unfolds the definition of $a = (p\ t_1).1$ and rewrites using \lstinline[language=lean]|hx_eq|
in the reverse direction, transforming the goal from \lstinline[language=lean]|(p t₁).1 > 0|
to \lstinline[language=lean]|(sine_curve x).1 > 0|.
Since \lstinline[language=lean]|sine_curve x = (x, sin(1/x))|, this simplifies to \lstinline[language=lean]|x > 0|,
which is exactly the hypothesis \lstinline[language=lean]|hxI|.
The \lstinline[language=lean]|using hxI| clause, applies this hypothesis to close the goal.

Next, the image $x(p([t_0, t_1]))$ is connected (as the continuous image of a connected set),
and it contains $0 = x(p(t_0))$ and $a = x(p(t_1))$.
Since every connected subset of $\mathbb{R}$ is an interval, we have
$$
  [0, a] \subseteq x(p([t_0, t_1]))
$$
This will be crucial for the next step, where we show that the path must oscillate.
\begin{lstlisting}[language=lean]
  -- The image x(p([t₀, t₁])) is connected and contains 0 and a
  -- Therefore [0, a] ⊆ x(p([t₀, t₁]))
  have Icc_of_a_b_sub_Icc_t₀_t₁: Set.Icc 0 a ⊆ xcoord_path '' Set.Icc t₀ t₁ :=
     IsConnected.Icc_subset
      ((isConnected_Icc (le_of_lt ht₁.1)).image _ xcoord_pathcont.continuousOn)
      (⟨t₀, left_mem_Icc.mpr (le_of_lt ht₁.1), hpt₀_x⟩)
      (⟨t₁, right_mem_Icc.mpr (le_of_lt ht₁.1), rfl⟩)
\end{lstlisting}
Now we construct a sequence that demonstrates the contradiction.
Recall that $\sin(\theta) = 1$ if and only if $\theta = \frac{(4k + 1)\pi}{2}$ for some $k \in \mathbb{Z}$.
Therefore, $(x, \sin(1/x)) = (x, 1)$ when
$$
  x = \frac{2}{(4k + 1)\pi}
$$
for $k \in \mathbb{N}$. As $k \to \infty$, these $x$-values approach 0,
so infinitely many of them lie in any interval $[0, a]$.
We define this sequence and establish its key properties:
\begin{lstlisting}[language=lean]
noncomputable def xs_pos_peak := fun (k : ℕ) => 2/((4 * k + 1) * Real.pi)
lemma xs_pos_peak_tendsto_zero : Tendsto xs_pos_peak atTop (𝓝 0) := sorry
lemma xs_pos_peak_nonneg : ∀ k : ℕ, 0 ≤ xs_pos_peak k := sorry
lemma sin_xs_pos_peak_eq_one (k : ℕ) : Real.sin ((xs_pos_peak k)⁻¹) = 1 := sorry
\end{lstlisting}
The crucial property is that this sequence eventually enters $[0, a]$:
\begin{lstlisting}[language=lean]
-- For any k ∈ ℕ, sin(1/xs_pos_peak(k)) = 1
-- Since xs_pos_peak converges to 0 as k → ∞,
-- there exist indices i ≥ 1 for which xs_pos_peak i ∈ [0, a]
have xpos_has_terms_in_Icc_of_a_b : ∃ i : ℕ, i ≥ 1 ∧ xs_pos_peak i ∈ Set.Icc 0 a := sorry
\end{lstlisting}
This gives us points on the topologist's sine curve with $y$-coordinate equal to $1$,
lying arbitrarily close to the $y$-axis.

Now we can establish the final contradiction.
Since $[0, a] \subseteq x(p([t_0, t_1]))$ by the previous argument,
and $\text{xs\_pos\_peak}(i) \in [0, a]$ for some $i$,
there must exist some $t' \in [t_0, t_1]$ such that $x(p(t')) = \text{xs\_pos\_peak}(i)$.
This means $p(t') = (\text{xs\_pos\_peak}(i), \sin(1/\text{xs\_pos\_peak}(i))) = (\text{xs\_pos\_peak}(i), 1)$,
so the $y$-coordinate of $p(t')$ equals $1$.
However, since $t' \in [t_0, t_1] \subseteq [t_0, t_0 + \delta)$,
we have $\text{dist}(t', t_0) < \delta$, which by our earlier continuity argument implies
$\|p(t') - p(t_0)\| < 1/2$.
But $\|p(t') - (0,0)\| \geq |(p(t')).2| = |1| = 1 > 1/2$,
yielding a contradiction.
\begin{lstlisting}[language=lean]
-- Show there exists time t' in [t₀, t₁] ⊆ [t₀, t₀ + δ) such that p(t') = (*, 1)
obtain ⟨t', ht', hpath_t'⟩ : ∃ t' ∈ Set.Icc t₀ t₁, (p t').2 = 1 := sorry
-- Derive the final contradiction using t', ht', hpath_t'
-- First show that p t₀ = (0, 0)
have hpt₀ : p t₀ = (0, 0) := sorry
-- t' is within δ of t₀ (since t' ∈ [t₀, t₁] and dist t₀ t₁ < δ)
have t'_close : dist t' t₀ < δ := by
  calc dist t' t₀
      ≤ dist t₁ t₀ := dist_right_le_of_mem_uIcc (Icc_subset_uIcc' ht')
    _ = dist t₀ t₁ := dist_comm _ _
    _ < δ := ht₁.2
-- By continuity, p(t') should be close to p(t₀)
have close : dist (p t') (p t₀) < 1/2 := ht t' t'_close
-- But p(t') has y-coordinate 1, so it's actually far from p(t₀) = (0, 0)
have far : 1 ≤ dist (p t') (p t₀) := by
  calc 1 = |(p t').2 - (p t₀).2| := by simp [hpath_t', hpt₀]
      _ ≤ ‖p t' - p t₀‖ := norm_ge_abs_snd
      _ = dist (p t') (p t₀) := by rw [dist_eq_norm]
-- This is a contradiction: 1 ≤ dist (p t') (p t₀) < 1/2
linarith
\end{lstlisting}
% The proof proceeds by deriving two contradictory bounds on $\text{dist}(p(t'), p(t_0))$:

% \begin{enumerate}
% \item \textbf{Extracting the critical time:} The first \lstinline[language=lean]|obtain| extracts a time $t' \in [t_0, t_1]$ where $(p\ t')_2 = 1$. This gives us three components:
% \begin{itemize}
%   \item \lstinline[language=lean]|t' : unitInterval| — the time value
%   \item \lstinline[language=lean]|ht' : t' ∈ Set.Icc t₀ t₁| — proof that $t' \in [t_0, t_1]$
%   \item \lstinline[language=lean]|hpath_t' : (p t').2 = 1| — proof that the $y$-coordinate is $1$
% \end{itemize}

% \item \textbf{Establishing the base point:} We prove \lstinline[language=lean]|hpt₀ : p t₀ = (0, 0)|, confirming that the path is at the origin at time $t_0$.

% \item \textbf{Showing proximity in time:} The statement \lstinline[language=lean]|t'_close| proves that $\text{dist}(t', t_0) < \delta$. The proof uses a \lstinline[language=lean]|calc| chain:
% \begin{itemize}
%   \item First, since $t' \in [t_0, t_1]$, we have $\text{dist}(t', t_0) \leq \text{dist}(t_1, t_0)$ (the distance from $t'$ to $t_0$ is at most the distance from $t_1$ to $t_0$)
%   \item By symmetry of distance, $\text{dist}(t_1, t_0) = \text{dist}(t_0, t_1)$
%   \item From our earlier work, $\text{dist}(t_0, t_1) < \delta$
% \end{itemize}

% \item \textbf{Upper bound from continuity:} The statement \lstinline[language=lean]|close| applies our earlier continuity result: since $\text{dist}(t', t_0) < \delta$, we have $\text{dist}(p(t'), p(t_0)) < 1/2$.

% \item \textbf{Lower bound from geometry:} The statement \lstinline[language=lean]|far| proves that $1 \leq \text{dist}(p(t'), p(t_0))$. The \lstinline[language=lean]|calc| chain shows:
% \begin{itemize}
%   \item $1 = |(p\ t')_2 - (p\ t_0)_2|$ by substituting $(p\ t')_2 = 1$ and $(p\ t_0)_2 = 0$
%   \item $|(p\ t')_2 - (p\ t_0)_2| \leq \|p(t') - p(t_0)\|$ by the fact that the norm dominates the absolute value of any component (\lstinline[language=lean]|norm_ge_abs_snd|)
%   \item $\|p(t') - p(t_0)\| = \text{dist}(p(t'), p(t_0))$ by the definition of distance in a normed space
% \end{itemize}
% This completes the proof by contradiction, showing that $T$ is not path-connected.
% \section{$T$ is connected not path-connected}
\section{Wrapping up}
Finally, we combine the two parts in the following concise and pleasant theorem:
\begin{lstlisting}[language=lean]
theorem T_is_conn_not_pathconn : IsConnected T ∧ ¬IsPathConnected T :=
  ⟨T_is_conn, T_is_not_path_conn⟩
\end{lstlisting}
And now, since this code compiles successfully, these two lines stand as verified witnesses
to the correctness of our entire proof.
This showcases the power of proof assistants and formal reasoning.
Mathematics becomes not only more rigorous but also automatically verifiable.
Furthermore, the formalization becomes a learning tool in its own right.
Future readers can inspect each part of the code.
Here the full proof: [\href{https://live.lean-lang.org/#codez=JYWwDg9gTgLgBAWQIYwBYBtgCMBQEwCmAdnAEoFLpwDKB8AYsOjAVHACr4ToQDmAnjhwATAgDM4kAM4B9KBSoAuALw06AOgCSEYHAAUABjiK4gXEIAlDiIQiAYwjgArjCRZ0BOKIlTgRAjNtHKAA3DxU4MUcSAA84QDLCfWiAGjIFdR8SPWjAbwIATvNLLxpjNXg9UzgAdbNzY1U9DP9AkIJagHI2yQhZeUoRcTgALRLaMorqizq4AG99AxSDWoBffol2Ebp9cZqp6jhAKiIhoQBaY444YCk4eyI/WxZhHHcQECQOGSlHLAD0WT2TdaAMKJrjxPvJiuEsII4BciDAoBA4LFUNEcDDbEgpAQrii4AB3YBoNFwAA+sKoKL2ygAfHACNEkPc4J8sFiYD9ukEPJTiWTfGwUcMacSYTCoHi4ABtQUAXRFovc8AkJkAqIRwQBJhGYqlrwpESCQafo9CQAFSpSjqMDAcx5BaWUWi1BIUJwVDKjjEYRSGCIiQoThgfSAVg3ALs7czttUh0IdovkYl8HktCOECAA1jIiKgvXAAPwyOAsIhen0BGzejNZqTyh30xnwQvFiAyXzBGT+/AyABerAg6ns4GrMb0jGYrHU7E93t77bAMhAjnQpaI3s65AtVpk0gLk5LRBQAGFMeyZ22YAH7TGnS63Z0TIAAIkABbhwEi+OAzlISEiACCIIaooYOYXjUcoBkRwwDxJAoGzSUCFCOFHEodB+BkXhjzPfA4AMGV8UJVBn1dIgALpBkmUAC/JjTgM013QS1rVtIiYRbTdunUcA2D0edFy3PQADkUD7I9mKkViwAFIhamo2ihPMJIGLgKFmVAQNJTEFJGgCIJQhSJiWxSSSMgzFA5wXTdgBlQBL8kHWsmRAAgQA5MF/AgMQZAbKdXQkG8cDQAhoDs95LiXEgTE0KR9xsO4HnOFRiVC8LbgIe4CGEdIvjZByuX0S54si5KZG0YB1FAJBeA8dTmhdAAeEl5RuGBfEcCBHFkYAUrqhqmqkAB5IgkwgFNUzgar5Uk9qiEa5qPl8PsHFnMaJu6khhodeL6vGzqeqKohgkAAIJrhsNaFubNqDo65qevlPQ9TgfNcUNPwZGclD4BRAp9AUlxnCbNlanYD4vh+P4TjOdZLjgHiIHgAAFFBUGOG5cseU44GcJhCX4OBnleJ47NeZ9oHs0q21ZD4ixmN4sBKUwlhKQA0Aj0JAUiwcxabgQBTIlJLAySjRSB2RrwEzfLofHql0sQAR0cYhbA8V89k+Ww8LQFB8VQYh9vAbpkpwtACzVxSiH3TTZauSgIP4K4AEYrBsfswGcVx3E8AZolkaRNwoQbdSifQvbgFVakNAAmAB6PQ9AAFkouBBoAajgS3aiotIrUsLG3ldoSPaQdM3JLbsERKCci3czP3cIHO33QwM9FDTDIz/aM40F4vGxm8B9F4KZrr9ioEiDuAQ6NKOzTjhOk/NGjU4vGE823EvdwPQSTxnYkfA7mwkKlYRgFbAhxeMxddOZVAIAlOujFUOu9AHs1DCpt6FPXsA5Rhaz4FbqcZuXdlOLgAfqoegXr2O289GzNm2qeAMXYexDRqg6T+Pp1Ank4kub0k8pJbkAXnJsK9hDCDQeyKAwBeCoHgPmS2cD5SIJAWWX+JkZwRCaiBLcODjpCHTkiN23Rs7pmsAlLu94Y4lBVAsOAgATIi4VnCug0vaqB7nAYUMId6tn4X4LuegFLWCgPZcaIBaiaIxtIQku90Zp1xm8AyZceEyJkPvR6fhfYiMjBgjIRprFu09jHG0+RFEJymP+GEV4wiJG4Z4yuqYfF+MklaQe/84Dx0GnfW+GDU4BOjDCZ+UoPG8JlAAbgiMAAg6ACHPwKcQogvA15KSlGgS4ek0gGSQPgwy9DFwwDxE2K0DSLQGQ3Co1ynS5SrDgJ2KmlQKjhEMHaG2dhZoOzcB4IoEoTCmAmVMcqxt9CJyED5PyIAAqyGsOyMAsMgolAADX6FCjDNAOVEpRXYLUdJxJkYAEEpCfFsvibkzoPBvFObrMQCIDmdnqPMeuBZER4mUHoS2alfBwoKMSYJroxkmDGT+dY4QupQC2mwKAYh0Aor+a6FZ+I4BYqmLiraVBSLwrgKRAuTZmAOIIJ0UwKRCXoHMpZRicJC5zQisSJAYAwBbxubDe5SUUoACsdB+GEJoEgKshW3DXoQWwwBKDAG7JIc5YzUBjIlKgPExJFSSCmAAKQVclZV6QHAEFuXhNVhEYTIxuD4UQbAfIa2MfVGwcBnJ6w8NEew0Ad57hYHAV4gYCS6zACkAkitGJXHmp1ElLow0QAjZuWGdUSirTOlca68AEh6EDDAcw6hKHhHTZNMQ3p26BjAN/Q6Gb3VnAtTAPavr0BHgLKADwvrAV4TBoG31/BjhIGiJcc1mwe0lCiISZVLAQiUA2dQMCMx4BkgrQWatlDVAGBWDCC12bc2ju7j7eAhp91Vpra8s4vqL1QV8CgDwwar0q0XWDAwmbuRgB7TIWIJh707UPX4owMUHR6GyqCPKGReCKhsEmAgxVSpInDVBPNaA6rVqkFu2ctkDmkQhQpJA2J8B8rgMjPQABmdQQdagWsAK3AfjLYhyDkzDG810ZBokCO2GSaPAYhIPGMmbxAAtwHAWkRgFZKydDAJ9TC2CIQLLCWpO0UiLvjlJmUKQSQkjAHoKttHJCmYg0ZoaCcuMqbVnYYd+sr1g1fBTdTwaoDNOAM1WzA9IJNTJtM+uxIIBYBcK+UiMmUioCkzFmA5kSiAGAiOAMm5MpDvAWFIO90H1j2pVVLGp5Q5bKJWgxlaIM2c4wPGDooEB0GIbYdQsFiAwAQugJCFYvTNjEGIEScDEANeAE1thmYvT9YrW24tby8u1HzDYfwTp0AuWkCp7tgBAgnkk5/5g7vlCd1mDY5QaSCTunbOqsnaCybdc5kHtOm9p6dfrRs4DmZZXdkwWHaoXwtIEizAdb8X1uJZMCl/7OnNu0kXXAQA5ESeEuHl97BWZNRnlBaqQe0pmLtWXAWoHqB2LuBQ4FGRAV1wlYMEDdPotRiGgDGpAJP7b9oDW6hUmwpCbfCCAV8DQHupa47Ua2DpUWoHRzIUQzAkBCRKEYSR6OEl85q6oZpBC1EEC7j2pjNb3pGO6CY4IZj5TC/Z9LiRzIOeqHcHOV8Iudpi+KS4KXzKZCW4W4b0lIv1ttBMMbyRtbVBc6IM7/wxDSHwHUOoeUzUPCkXZzF2Prp2dtBo+iOhUBHD3GgPKAA7bzQMm8MaSlQj1lyzA1IA/R0kQACYQ0C+DAfghAZr+FZfYAgT2HSivFRjVlAfBw5/fvoVlyvi9ZxD2Qm61a2L6CWyt7oropMz1FDnjvW9WVPRdy5Rxehl0wFXeTi0LvHE9oXw6HPvh4SIjIV9mMl5SV4hA7hvCYHAyJy16oP38kMZZMlJsloSa2/X9dFJWA3sVrU+w412Fr3rwIGa2iBeh7TklRWiDsQPmZRKH3RfyPUwheQAJjHFClGr2AIPktn/xwP71QCA1t1RBwLPXfWISJBP0AOvCdm9021l15yky4ymADyDyDzEHIRukHGF0w29z2ll3N0xn8AD210xloNwlqA1yDiY0HDwMlC3XCygPUBK2QOy3h1LBeB0PLH3hSFcFkCelVw0QUkwD3DoNQHMBIJkOsNwifS7U2DeFUFfQISvX+w+3/UuzqSuHnHQRpw6zPg/2dgkGDUXXp2EBhEXQK3+xUykERG8MCPgCwFlhID2GTTwidCuH9V3lDX2gjXfRYHnXgDeCmUrXW0PQA0AJKDeDk2wNFDCwixIFImSFdGiE0BiyQP3mB0tW8J/H+EbkHAxCxDyKCgdVsmdTnH8m8PjVQEHD5CIApCpFpDIOoCWNhAFCFGpEHGRk0AkCqMpSGB0wc30CqMg2PSIlRSfyuyuKwJ5hUNQEGHsMu2oGSNYM+xSDqi8x3nuF8HVwh0+xuPd2YPe0kSxwt38EI23X8NBKYODwUDQO8NWVqEkUswfggPUIbxbx4JbxEgFHcARO5CL16CUH0FRJ2FpExLROxLr1xIgCb3ZAJMnzIXW0fRwKsMgicIdGfjeHzylEZjgGrxRGQOwmakBM6M0BU3YH1gw1DQrT0ElDuyuxlDejBgRgeW1miP2jhF+2XCwN1KQFlLVjjD8ilAhSQGwmBGiCVJVO0zVORSCVJU0FsFsEehckl2+BZHyndNclt3+xMFGC0HdKNLgFtOw08NhjgA6FKFDNsDANROUGrDigim1JSjdI9JZDZEHHDmynTJlT9MTL0HX2dzgI5MTmrQVJuiwxzRw1HTqim3WnOnEjzNIlVPcD4NmPsizMJIHychLwrJrRkldAoJA3MmP30A7IBzgFH1/jsmLP7NLMHPLNdH+xHK5SJUnLW02ByRkSFglilkcwEyRGOAp3QCliuDxDNLKkRU42iEDgThU0KmgNUzfCIAxkGh/DVARSIDhRDn3M9j0EiSfMF2eykXLi8RuFCCgFKiuCpyMExGEU1EADwiFTNAXzHycESCMqR1WEf4opK4XQGnNgbIyCmxLxXQG7S04wp7RA92XI1yVgEAFqQPLMz0tsGQSmEHC4ERbSdmfxWHICyuXQIYjQLM8MlHB0Fog0xlHiGLHiYHeoereEYbdQUbSsfrES3OHcJsVAhmfnPMhSfI/XOvAoGA+HWQUixCTi1eB0KPOnWIShHieUZuRxUiS3V4JArs/gmQFIHSjMdMrufy3MGQZPRgkJErUJaRLxDiGdfxHiN6IwArN4IeRXV0HiWslcucGdOQEhMffMGQBfFQySLQow5kAGZlYwkmMw4KmKqCnOIKgRcfbCH9VAOdB0bkmwlTCGCUPEUTenE+UI5WesW84ieHXbYdToV8B08HG0rTe7eXefSqlNUaizGANoJ8vQE0JIHZGEWSv7NoeLY6scs5Ta/o0HTocSmABMpM2c+9LapjcA6S5on7SLYAGLEhAgL65SpERizEZinRNi4suy7it3F0EwHSvin8Dwh/WMzoEMySxdZMuADip6b0/6b4Di4Dbwjq+UQ6tozak6nskm6IP6+EhyrEAsU6shNoHs+UZGagU+CUSta64oY0SGTTQYP8t7dazpBcGI14VMUNMJXhPi49BfYXXwGQf4AY9m4Yj/UYzEbEfVBGKYp1M5EjGmnWRY6/ZY1YxRdYkiF6TY/W7Y10XYkk10exVAkwNmvxYLRYXYDQEjKaSpFDQPYAXrfrF462qGsWg83QY9JonAjEdARM6G4AOSB0eoNmyDHmfk7JKMh/CUpDdc+m6IdIfgF4GO2s8FO0LXJ4iUaUW2nsN4mMfMEO8IblIiJOlUjqqQNOqUqxQOz2ZAtlC4Rms4agXwN7B2oYkTSK+Wx2jojIACx82oUi5kfC2IOTAm96tojo0KtWcWP6jqwPM20UZGAAIWcCRD8Sjv0EJwORfXhnrMjQ/VHKSINgnrAohu5FAwPvcLbtEtDsdFJTuM2oeNA0blzzeG/wTA0haGbsqT1N4CgDeTgrgChmTHUEbRen3mztzuv3DsjsdvjqLtUH8OQZADkirrrMvRjM2qmFruoPzoovCUGl0HCDppAzcpLtFIqp/y0k6P/371btiqavsQWy7s7QABFWACiQ1Ck9wqBfjvMATA1JSwHiaM6YtR1XI2gVNGAoB0EpAWa9ZVUwCC6QsXTrwKCSgKtHaIVnbXr0QVaJj1akjpita5i9oFj5QDbXQ1jBxCbGUOixTpAejkD16t6hdP7n7HiRiw7KB0H6gWHQ0E7VAyHyGKHwNomwj67InQGu4bgIGoGu5YH+p4H0ExT9486CHoNVByDgMqCYwv93ZyS2qXokDVtr9ureTRRlidijbiI6x4y3akNPbi9fbBgMKZrrzcJXwZNIi9oGg+7pqTi5q1SPyYjoqUbNskcLxUVNqHIorJqSG4jCtzHrgwm4dctOh4CcDJEFnNsoSDmiECr2Qyy3bHBJK9AOKcy6BQIszOg6apyCGFne1xCtD+wDkirBx8wkcph2SlDLsd7eNTpxp0YUgTNv6Lg00EMoUNqINajbBkWTBorHrytPtagCtqtQX6wjn6aMWtYVM974B4WtrACrgp0oySiPAGUb7CQ2gS1IIIgQVUXtr5hFhaixAOWTBKFTnJqcWLi8X360H/FVASQcXnqzgEmmNuY/6v9yDzraaKCK78xJFaYHazgKtWZwhtFCZ/ATDSZHgcCvnRX47xWj90k5yS7yqD5jWntGmiRvI1Z9lDkgogqTl80IoSg0yEoZVzhYcLlJU7lCzHk6hiRSI/pAoEYkh42jlIYH9zlLIgA}{link to Lean live}]



\printbibliography
\end{document}