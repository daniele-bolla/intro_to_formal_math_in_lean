\documentclass[12pt,a4paper]{book}

\usepackage{thesis}
\usepackage{hyperref}
\usepackage{stmaryrd} 
\usepackage{listings}
\usepackage{amssymb}
\usepackage{amsmath}
\usepackage{tikz}
\usepackage{float}
\usepackage{mathtools}
\usepackage{bussproofs}
\usepackage{color}
\usepackage[backend=biber,style=alphabetic]{biblatex}
\addbibresource{bibliography.bib}
% Listings settings
\definecolor{codegreen}{rgb}{0,0.6,0}
\definecolor{codegray}{rgb}{0.5,0.5,0.5}
\definecolor{codepurple}{rgb}{0.58,0,0.82}
\definecolor{backcolour}{rgb}{0.95,0.95,0.92}
\definecolor{keywordcolor}{rgb}{0.7, 0.1, 0.1}   % red
\definecolor{tacticcolor}{rgb}{0.0, 0.1, 0.6}    % blue
\definecolor{commentcolor}{rgb}{0.4, 0.4, 0.4}   % grey
\definecolor{symbolcolor}{rgb}{0.0, 0.1, 0.6}    % blue
\definecolor{sortcolor}{rgb}{0.1, 0.5, 0.1}      % green
\definecolor{attributecolor}{rgb}{0.7, 0.1, 0.1} % red
\def\lstlanguagefiles{lstlean.tex, lsttypescript.tex, lstjavascript.tex}
% set default language
\lstset{language=lean}

% Thesis metadata
\title{Introduction to Formal Mathematics in Lean  with an example in Topology}
\author{Daniele Bolla}
\matrikelnr{21-694-187}
\supervisor{Prof. David Loeffler}
\submitdate{December 15, 2025}

\newtheorem{theorem}{Theorem}[section]
\newtheorem{proposition}[theorem]{Proposition}
\newtheorem{definition}[theorem]{Definition}  
\newtheorem{remark}[theorem]{Remark}
\newtheorem{example}[theorem]{Example}
\newtheorem{notation}[theorem]{Notation}
\newtheorem{note}[theorem]{Note}
\numberwithin{figure}{theorem}
\begin{document}

% Title page
\makethesistitle

% Declaration
\begin{declaration}
    I hereby declare that this thesis is my own work, and that no other
    sources have been used
    except those indicated in the bibliography.
    I confirm that generative AI tools were not used in developing any
    part of this thesis, though they may have been used
    for preliminary research purposes only, and flow refinement.
    \vspace{3cm}
    \noindent
    Place, Date \hfill Signature
    \vspace{1cm}
    \noindent
    \rule{4cm}{0.5pt} \hfill \rule{6cm}{0.5pt}
\end{declaration}

% Abstract
\begin{thesisabstract}
    This thesis serves as an introduction to formal mathematics using the
    Lean proof assistant. After a brief overview of the Curry--Howard
    correspondence, we explore how mathematical structures and
    properties are defined in Lean.
    Finally, we present a formalization of the topologist's sine curve,
    which has been merged into the Mathlib library as a modest contribution
    to the broader Lean community.

    During the work on this thesis, I have explored new concepts
    in both mathematics and computer science, starting from logic and
    type theory within the Curry-Howard correspondence.
    The practice of formal and constructive mathematics has influenced
    my approach to mathematical reasoning. In building the rational
    numbers, I explored their algebraic construction using quotients,
    which is reflected in Lean through quotient types. I further examined
    how algebraic structures such as groups and rings are defined using
    structures and type classes. For more technical depth, I studied
    filters to generalize the notion of convergence in topology.
    The formalization of the topologist's sine curve has deepened
    my understanding of connectedness, closures, and continuity.

    The full code accompany this project is available
        [\href{https://github.com/daniele-bolla/intro_to_formal_math_in_lean_code}{here}].
\end{thesisabstract}

% Acknowledgments
\begin{acknowledgments}
    I would like to express my sincere gratitude to Prof. David Loeffler for
    the consistent support.
    His mentorship has been particualary beneficial in enlightening my path through
    formal mathematics, shaping the thesis and giving me a chance to
    actually contribute to Lean community.

    Moreover, I deeply thank FernUni for the opportunity to pursue my
    studies. I feel honored to have been able to study
    mathematics at this stage of my life, through a flexible and
    high-quality program.

    I hope to continue my studies in the future.
\end{acknowledgments}

% Table of contents
\tableofcontents

\section{Introduction}
This serves as a brief starting point for understanding how the Curry-Howard correspondence appears in Lean, 
as well as being an introduction to the language itslef. 
Lean is both a \textbf{functional programming language} and a \textbf{theorem prover}.
We'll focus primarily on its role as a theorem prover. 
But what does this mean, and how can that be achieved?

A programming language defines a \textbf{set of rules, semantics, and syntax} for writing programs. 
To achieve a goal, a programmer must write a program that meets given specifications. 
There are two primary approaches: \textbf{program derivation} and \textbf{program verification} 
( \cite{nordstrom1990programming} Section 1.1).
In \textbf{program verification}, the programmer first writes a program and then proves it meets 
the specifications. This approach checks for errors at \textbf{runtime} when the code executes.
In \textbf{program derivation}, the programmer writes a proof that a program with certain properties exists, 
then extracts a program from that proof. This approach enables specification 
checking during \textbf{compilation}, catching errors before execution.
This distinction corresponds to \textbf{dynamic} versus \textbf{static} \textbf{type systems}. 
Most programming languages employ both approaches. For example, C checks operations on \texttt{int} 
at compile time but requires the programmer to ensure correctness with \texttt{void*} at run time.
Lean emphasizes program derivation.
Its type system is highly advanced and flexible, allowing the expression and verification 
of a wide range of mathematical statements, and this is what makes Lean a powerful \textbf{theorem prover}. 
Lean's type system is based on \textbf{Type Theory}, a branch of marthematics and logic tahta aims to provide a 
foundation for all mathematics and wich is a programming language itself.

It's important to note that Type Theory is not a single, unified theory, but rather a family of 
related theories with various extensions and ongoing developments and rich hiostorical ramifications. 
Creating a language like Lean require
careful consideration of which rules and features to include.
We shall give a brief overview of the historical development of type theory, and an a introduction on 
what comes next.

(From \cite{carneiro2019typetheorylean})
Type theory emerged as a fundamental response to Bertrand Russell's paradox. 
Considers the set $S = \{x \mid x \notin x\}$ 
(the set of all sets that do not contain themselves). This is a paradoxical construction, 
leading to the contradiction $S \in S \iff S \notin S$. 
Ernst Zermelo and Fraenkel addressed the contradiction by introducing Zermelo-Fraenkel set theory (ZFC), 
which became the standard axiomatization in modern mathematics. 
ZFC provides an untyped but stratified view of the mathematical universe, 
maintaining classical logical principles while avoiding paradoxes through careful axiomatization.
Russell chose a fundamentally different path. He recognized expressions 
like $A(A)$ or $x \in x$ as ill-typed, introducing his theory of types.
Hs's first systematic response was \textbf{Ramified Type Theory}, wich turned out to be problematic.
In the 1930s, Alonzo Church developed \textbf{Lambda Calculus} as a foundation for mathematics, 
initially pursuing a type-free approach. However, Church's original untyped system suffered from 
inconsistencies. To address these issues, 
Church introduced the \textbf{Simply Typed Lambda Calculus} in 1940 (\cite{church1940formulation}).  
This system is a version of \textbf{Simple Type Theory}, a framework able to replace 
set-theory and propositional logic.
Lambda calculus influenced the development of many programming languages as being a foundation for
functional programming.
Per Martin-L\"{o}f revolutionized type theory in the 1970s by introducing 
\textbf{dependent types} that can depend on values of other types.
Think for instance of a of vectors of length $n$ or a sequnce of $n$ elements.
\textbf{Dependent Type Theory} extends the expressive power of type systems 
by allowing the representation of quantifiers, 
providing a framework capable of replacing set theory and predicate logic.
Dependent Type Theory is a derivation of \textbf{Martin-L\"{o}f Type Theory} 
(also known as \textbf{Intuitionistic Type Theory}).
Martin-L\"{o}f's system embraced constructive principles, requiring that the existence of mathematical 
objects be demonstrated through explicit construction rather than classical proof by contradiction. 
Martin-L\"{o}f Type Theory also introduced \textbf{identity types} to represent equality.
In the 1980s, Thierry Coquand and G\'{e}rard Huet introduced the \textbf{Calculus of Constructions} (CoC), 
synthesizing insights from Martin-L\"{o}f's dependent type theory with higher-order \textbf{polymorphism}. 
The Calculus of Constructions served as the theoretical foundation for the Coq proof assistant, 
one of the most influential interactive theorem provers.
The original CoC was later extended with \textbf{inductive types} to form the 
\textbf{Calculus of Inductive Constructions} (CIC). Inductive types allow for the definition 
of data structures like natural numbers, lists, and trees. 
The Lean theorem prover, developed by Leonardo de Moura and others, is also based on CIC 
but incorporates several important refinements and differences from Coq's implementation. 

A central insight in type theory is the \textbf{Curry-Howard correspondence}, 
which establishes a profound connection between logic and computation. 
Also known as the \textbf{propositions-as-types} principle, this correspondence 
represents one of the most elegant discoveries in the foundations 
of mathematics and computer science.
It serves also well as a good introduction to type theory, and will be used in this discussion.
Nontheless it continuelsy shows new applications and interpretations in modern type theories.
The Curry-Howard correspondence was independently discovered by multiple researchers.
\textbf{Haskell Curry} (1934) first observed the connection between combinatory logic and 
Hilbert-style proof systems.
\textbf{William Alvin Howard} (1969) significantly extended the correspondence to natural deduction 
and the simply typed lambda calculus in his seminal work ``The Formulae-as-Types Notion of Construction.'' 
The correspondence was further developed through \textbf{N.G. de Bruijn's AUTOMATH system} (1967), 
which was the first working proof checker and demonstrated the practical viability of mechanical 
proof verification. Amongst its technical innovations are a discussion of the
irrelevance of proofs when working in a classical context, which is one
of the reasons advanced by de Bruijn for the separation between the notions of 
type and prop in the system \cite{thompson1999types}. Lean also adopts this separation .
\textbf{Per Martin-L\"{o}f's type theory} extended the correspondence to
dependent types, allowing for the representation of quantifiers and identity types.
Modern proof assistants like Coq, Lean, Agda, and Isabelle/HOL all leverage 
variants of the Curry-Howard correspondence to enable formal verification of mathematical theorems 
and software correctness properties.

% Throughout this discussion, we will introduce how to read Type Theory notation and explore key concepts 
% including:
% \begin{itemize}
% \item Impredicativity
% \item Decidability
% \item Computability
% \item Derivability
% \item Soundness
% \item Completeness
% \item Intyensionality 
% \item Extensionality
% \end{itemize}


\chapter{Logic and Proposition as Types}

\section{First Order Logic}
Logic is the study of reasoning, branching into various systems.
We refer to \textbf{classical logic} as the one that underpins much
of traditional mathematics.
It's the logic of truthtables.
We first introduce \textbf{propositional logic}, which is the simplest
form of classical logic.
Later we will extend this to \textbf{predicate (or first-order) logic}, which includes
\textbf{predicates} and \textbf{quantifiers}.
In this setting, a \textbf{proposition} is a statement that is either true or false,
and a \textbf{proof} is a logical argument that establishes the truth of a
proposition.
Propositions can be combined with logical \textbf{connectives} such as ``and'' ($\wedge$),
``or'' ($\vee$), ``not'' ($\neg$),``false'' ($\bot$), ,``true'' ($\top$)
``implies'' ($\Rightarrow$),  and ``if and only if'' ($\Leftrightarrow$).
These connectives allow the creation of complex or compound propositions.
Here how connectives are defined in Lean:
\begin{example}[LogicaL connectives in Lean]\mbox{}
  \begin{lstlisting}[language=lean]
    #check And (a b : Prop) : Prop
    #check Or (a b : Prop) : Prop
  \end{lstlisting}
  \lstinline[language=lean]|Prop| is the proposition type mentioned before.
\end{example}
Logic is often formalized through a framework known as the \textbf{natural deduction system},
developed by Gentzen in the 1930s (\cite{wadler2015propositions}).
This approach brings logic closer to a computable, algorithmic system.
It specifies rules for deriving
\textbf{conclusions} from \textbf{premises} (assumptions from other propositions),
called \textbf{inference rules}.
\begin{example}[Deductive style rule]
  Here is an hypothetical example of inference rule (\cite{nordstrom1990programming}, page 35).
  \begin{prooftree}
    \AxiomC{$P_1$}
    \AxiomC{$P_2$}
    \AxiomC{$\cdots$}
    \AxiomC{$P_n$}
    \QuaternaryInfC{$C$}
  \end{prooftree}
  Where the $P_1, P_2, \ldots, P_n$, above the line, are hypothetical
  premises and, the hypothetical conclusion $C$ is below the line.
\end{example}
The inference rules needed are, \textbf{introduction rules} wich specify
how to form compound propositions from simpler ones, and
\textbf{elimination rules} needed to derive information about them.
Let's look at how we can define some connectives, first using
natural deduction (from \cite{thompson1999types}, Section 1.1).
\paragraph{Conjunction ($\land$)}
\paragraph{Introduction Rule}
\begin{prooftree}
  \AxiomC{$A$}
  \AxiomC{$B$}
  \RightLabel{$\land$-Intro}
  \BinaryInfC{$A \land B$}
\end{prooftree}
\paragraph{Elimination Rule}
\mbox{}\\[0.5em]
\noindent
\begin{minipage}[t]{0.48\textwidth}
  \vspace{0pt}
  \begin{prooftree}
    \AxiomC{$A \land B$}
    \RightLabel{$\land$-Elim$_1$}
    \UnaryInfC{$A$}
  \end{prooftree}
\end{minipage}\hfill
\begin{minipage}[t]{0.48\textwidth}
  \vspace{0pt}
  \begin{prooftree}
    \AxiomC{$A \land B$}
    \RightLabel{$\land$-Elim$_2$}
    \UnaryInfC{$B$}
  \end{prooftree}
\end{minipage}
\paragraph{Disjunction ($\lor$)}
\paragraph{Introduction Rule}
\mbox{}\\[0.5em]
\noindent
\begin{minipage}[t]{0.48\textwidth}
  \vspace{0pt}
  \begin{prooftree}
    \AxiomC{$A$}
    \RightLabel{$\lor$-Intro$_1$}
    \UnaryInfC{$A \lor B$}
  \end{prooftree}
\end{minipage}\hfill
\begin{minipage}[t]{0.48\textwidth}
  \vspace{0pt}
  \begin{prooftree}
    \AxiomC{$B$}
    \RightLabel{$\lor$-Intro$_2$}
    \UnaryInfC{$A \lor B$}
  \end{prooftree}
\end{minipage}
\paragraph{ Elimination (Proof by cases)}
\begin{prooftree}
  \AxiomC{$A \lor B$}
  \AxiomC{$[A] \vdash C$}
  \AxiomC{$[B] \vdash C$}
  \RightLabel{$\lor$-Elim}
  \TrinaryInfC{$C$}
\end{prooftree}
\paragraph{Implication ($\to$)}
\paragraph{Introduction Rule}
\begin{prooftree}
  \AxiomC{$[A] \vdash B$}
  \RightLabel{$\to$-Intro}
  \UnaryInfC{$A \to B$}
\end{prooftree}
\paragraph{ Elimination (Modus Ponens)}
\begin{prooftree}
  \AxiomC{$A \to B$}
  \AxiomC{$A$}
  \RightLabel{$\to$-Elim}
  \BinaryInfC{$B$}
\end{prooftree}
\begin{notation}
  We use $A \vdash B$ (called turnstile) to designate a
  deduction of $B$ from $A$.
  It is used in judgments and type theory with
  the meaning of ``entails that''.
  The square brackets around a premise $[A]$ mean that the premise $A$ is meant to
  be \textbf{discharged} at the conclusion. The classical example is the
  introduction rule for the implication connective.
  To prove an implication $A \to B$, we assume $A$
  (shown as $[A]$), derive $B$ under this assumption, and then discharge the
  assumption $A$ to conclude that $A \to B$ holds without the assumption.
\end{notation}
\section{Primitive Types}
Type theory employs this porocedure too,
by referring to deduction
rules as \textbf{judments}.
A type judgment has the form $\Gamma \vdash t : T$,
meaning: under \textbf{context} $\Gamma$ (a list of typed variables),
the term $t$ has type $T$.
Using formal inference rules in the type judgment
system, such as \textbf{introduction} and \textbf{elimination} rules,
we can construct new compound types from existing ones.
\begin{example}[Judgment style rule]
  \mbox{}
  \begin{prooftree}
    \AxiomC{$\Gamma \vdash$}
    \AxiomC{$p_1:P_1$}
    \AxiomC{$p_2:P_2$}
    \AxiomC{$\cdots$}
    \AxiomC{$P_n$}
    \QuinaryInfC{$C$}
  \end{prooftree}
\end{example}
Technically, there are two more inference rules that we will not consider in this setting:
\textbf{formation rules}, used to declare that a type is well-defined, and
\textbf{computation rules}, which specify how a term will be evaluated.
Moreover, without going too deep into the jargon,
one specific judgment is
$\Gamma \vdash A \equiv B\ \text{type}$, which means ``types $A$ and $B$ are
\textbf{judgmentally (or definitionally) equal} in context $\Gamma$.''
Similarly for terms, $\Gamma \vdash t_1 \equiv t_2 : A$ means ``terms $t_1$ and $t_2$ are
judgmentally equal of type $A$ in context $\Gamma$.''
\paragraph{Brief explanation of equality in type theory}

In Lean, the operator \lstinline[language=lean]|:=|
stands for definitional equality and is used by the kernel to verify proof equality.
A mathematician proving a theorem applies a series of \textbf{reduction rules} to
simplify the proof.
Similarly, one can think of this
computational reduction process in formal verification.
However, a computer cannot simply employ the same informal
approach to equality that a mathematician might use intuitively.
A rigorous explanation of definitional equality
goes beyond the scope of this thesis.
To state it simply: \textbf{two terms are definitionally equal when they
  reduce to the same normal for}. A \textbf{normal for} represents the most
reduced state of a term, obtained by systematically applying a
sequence of reduction rules until no further reductions are possible.
In contrast, \textbf{propositional equality} requires additional logical
bridges and propositions to establish equivalence.
Because it is grounded in logical propositions rather
than pure computation, propositional equality is not
directly computable by the type checker and must be explicitly proved.

Let's now construct new types from given types $A$ and $B$.
\paragraph{Product Type}
As a fundamental example, $A \times B$
denotes the type of pairs $(a, b)$ where $a : A$ and $b : B$,
called the \textbf{product type}.
\paragraph{Introduction Rule (pairing)}
\begin{prooftree}
  \AxiomC{$a : A$}
  \AxiomC{$b : B$}
  \BinaryInfC{$(a, b) : A \times B$}
\end{prooftree}
In Lean:
\begin{lstlisting}[language=lean]
Prod.mk a b : Prod A B   -- or A × B
(a, b) : A × B           
⟨a, b⟩ : A × B           
\end{lstlisting}
\paragraph{Elimination Rules (projections)}\mbox{}\\[0.5em]
\noindent
\begin{minipage}[t]{0.48\textwidth}
  \vspace{0pt}
  \begin{prooftree}
    \AxiomC{$p : A \times B$}
    \UnaryInfC{$\mathsf{fst}(p) : A$}
  \end{prooftree}
\end{minipage}\hfill
\begin{minipage}[t]{0.48\textwidth}
  \vspace{0pt}
  \begin{prooftree}
    \AxiomC{$p : A \times B$}
    \UnaryInfC{$\mathsf{snd}(p) : B$}
  \end{prooftree}
\end{minipage}

\noindent In Lean:
\begin{lstlisting}[language=lean]
  p.1 : A       -- or Prod.fst p
  p.2 : B       -- or Prod.snd p
\end{lstlisting}
\paragraph{Sum Type}
The \textbf{sum type} $A + B$ (also called a coproduct or disjoint union) consists of values that are
either of type $A$ (tagged with $\mathsf{inl}$) or
of type $B$ (tagged with $\mathsf{inr}$).
\paragraph{Introduction Rules (injections)}
\mbox{}\\[0.5em]
\noindent
\begin{minipage}[t]{0.48\textwidth}
  \vspace{0pt}
  \begin{prooftree}
    \AxiomC{$a : A$}
    \UnaryInfC{$\mathsf{inl}(a) : A + B$}
  \end{prooftree}
\end{minipage}\hfill
\begin{minipage}[t]{0.48\textwidth}
  \vspace{0pt}
  \begin{prooftree}
    \AxiomC{$b : B$}
    \UnaryInfC{$\mathsf{inr}(b) : A + B$}
  \end{prooftree}
\end{minipage}

\noindent In Lean:
\begin{lstlisting}[language=lean]
Sum.inl a : Sum A B   -- or A ⊕ B
Sum.inr b : Sum A B
\end{lstlisting}
\paragraph{Elimination Rule (case analysis)}
\begin{prooftree}
  \AxiomC{$p : A + B$}
  \AxiomC{$\begin{array}{c}  f : (A \implies C) \end{array}$}
  \AxiomC{$\begin{array}{c}  g : (B \implies C) \end{array}$}
  \TrinaryInfC{$\mathsf{cases}(p, f, g) : C$}
\end{prooftree}
\newpage
In Lean:
\begin{lstlisting}[language=lean]
example (p : Sum A B) (f : A → C) (g : B → C) : C := by
  cases p with
  | inl x => f x
  | inr y => g y
\end{lstlisting}
\paragraph{Function Types}
The type of the form $A \to B$, used in the sum elimination rule
represents functions from $A$ to $B$.
\paragraph{Introduction Rule (function application or lambda abstraction)}
\begin{prooftree}
  \AxiomC{$\begin{array}{c} x : A  \vdash  \Phi : B \end{array}$}
  % \UnaryInfC{$\lambda x.\Phi : A \to B$}
  \UnaryInfC{$f  : A \to B$}
\end{prooftree}
Where $f$ is a function that maps any element $x : A$ to an element $\Phi : B$.
In Lean, lambda abstraction is written using \lstinline[language=lean]|fun| or \lstinline[language=lean]|λ|:
\begin{lstlisting}[language=lean]
def identityFun (A : Type) : A → A := fun x => x
\end{lstlisting}
\paragraph{Elimination Rule (application)}
\begin{prooftree}
  \AxiomC{$f : A \to B$}
  \AxiomC{$a : A$}
  \BinaryInfC{$f(a) : B$}
\end{prooftree}
In Lean, function application is written using juxtaposition:
\begin{lstlisting}[language=lean]
example (f : A → B) (a : A) : B := f a
\end{lstlisting}
Functions are a primitive concept in type theory. We can \textbf{apply} a function
$f : A \to B$ to an element $a : A$ to obtain an element of $B$, denoted $f(a)$.
In type theory, it is common to omit the parentheses and write the application
simply as $f \, a$.

\section{The Curry Howard Isomorphism}
We have been preparing for this argument, and the reader will have surely
noticed a strong similarity when defining logical connectives
using deduction rules; they are remarkably similar to types
constructed using type judgments. For instance, function
types can be seen as implications.
This is not a coincidence, but rather a fundamental theorem
first proven by Haskell Curry and William Howard.
It forms the core of type theory and establishes
a deep connection between logic, computation, and mathematics.
\textbf{Implication} ($P \Rightarrow Q$) corresponds to the \textbf{function type} ($P \to Q$).
A proof of an implication is a function that transforms any proof
of the premise into a proof of the conclusion.
\noindent\textbf{Conjunction} ($P \land Q$) corresponds
to the \textbf{product type} ($P \times Q$).
A proof of a conjunction consists of a pair containing proofs of both conjuncts.
\noindent\textbf{Disjunction} ($P \lor Q$) corresponds
to the \textbf{sum type} ($P + Q$).
A proof of a disjunction is either a proof of the
first disjunct or a proof of the second disjunct.
Same goes for the rest of the connectives.
Lean uses inference rules and type
judgments as well as computing connectives using each related type.
For instance, $A \land B$ can be represented as \lstinline[language=lean]|And(A, B)| or \lstinline[language=lean]|A ∧ B|.
Its introduction rule is constructed by
\lstinline[language=lean]|And.intro _ _| or simply
\lstinline[language=lean]|⟨_, _⟩| (underscores are placeholders).
The pair $A \land B$ can then be consumed using elimination
rules \lstinline[language=lean]|And.left| and \lstinline[language=lean]|And.right|.

\begin{example}\label{ex:conj_intro_2}
  Let's look at a simple Lean example:
  \begin{lstlisting}[language=lean]
    example {a b : Prop} (ha : a) (hb : b) : (a ∧ b) := And.intro ha hb
  \end{lstlisting}
  Using brackets \lstinline[language=lean]|{ }|, we let Lean infer
  that $a$ and $b$ are propositions (\lstinline[language=lean]|Prop|).
  The example means that given a proof of $a$ (\lstinline[language=lean]|ha|)
  and a proof of $b$  (\lstinline[language=lean]|hb|) ,
  we can form a proof of $(a \land b)$.
  \lstinline[language=lean]|And.intro| is implemented as:
  \begin{lstlisting}[language=lean]
    And.intro : p -> q -> (p ∧ q)
  \end{lstlisting}
  It says: if you give me a proof of $p$ and a proof of $q$,
  then I return a proof of $p \land q$.
  We therefore conclude the proof by directly giving
  \lstinline[language=lean]|And.intro ha hb|.
  Here is another way of writing the same statement:
  \begin{lstlisting}[language=lean]
    example (ha : a) (hb : b) : And(a, b) := ⟨ha, hb⟩
  \end{lstlisting}
\end{example}
\noindent
For a more concrete example, let's look at how
proof normalization using a system of inference rules
corresponds to computation in Lean.
To reduce complexity of a \textbf{proof tree} in natural deduction,
one, tipically, follows a
\textbf{top-down} approach,
unfolding each component to be proved step by step.
\begin{example}[Associativity of Conjunction]
  We prove that $(A \land B) \land C$ implies $A \land (B \land C)$.
  First, from the assumption $(A \land B) \land C$, we can derive $A$:
  \begin{prooftree}
    \AxiomC{$(A \land B) \land C$}
    \RightLabel{$\land E_1$}
    \UnaryInfC{$A \land B$}
    \RightLabel{$\land E_1$}
    \UnaryInfC{$A$}
  \end{prooftree}
  Second, we can derive $B \land C$:
  \begin{prooftree}
    \AxiomC{$(A \land B) \land C$}
    \RightLabel{$\land E_1$}
    \UnaryInfC{$A \land B$}
    \RightLabel{$\land E_2$}
    \UnaryInfC{$B$}
    \AxiomC{$(A \land B) \land C$}
    \RightLabel{$\land E_2$}
    \UnaryInfC{$C$}
    \RightLabel{$\land I$}
    \BinaryInfC{$B \land C$}
  \end{prooftree}
  Finally, combining these derivations we obtain $A \land (B \land C)$:
  \begin{prooftree}
    \AxiomC{$(A \land B) \land C \vdash A$}
    \AxiomC{$(A \land B) \land C \vdash B \land C$}
    \RightLabel{$\land I$}
    \BinaryInfC{$A \land (B \land C)$}
  \end{prooftree}
\end{example}
\newpage
\begin{example}[Lean Implementation]
  Let us now implement the same proof in Lean.
  \begin{lstlisting}[language=lean]
theorem and_associative {a b c : Prop} : (a ∧ b) ∧ c → a ∧ (b ∧ c) :=
  fun h : (a ∧ b) ∧ c →
  -- First, from the assumption (a ∧ b) ∧ c, we can derive a:
  have hab : a ∧ b := h.left
  have ha : a := hab.left 
  -- Second, we can derive b ∧ c (here we only extract b and c and combine them in the next step)
  have hc : c := h.right
  have hb : b := hab.right
  -- Finally, combining these derivations we obtain a ∧ (b ∧ c)
  show a ∧ (b ∧ c) from ⟨ha, ⟨hb, hc⟩⟩
\end{lstlisting}
  We introduce the \lstinline[language=lean]|theorem| with the name
  \lstinline[language=lean]|and_associative|.
  The type signature \lstinline[language=lean]|(a ∧ b) ∧ c → a ∧ (b ∧ c)|
  represents our logical implication.
  Here, we construct the implication proof using a
  function (following the Curry Howard isomorphism) with the \lstinline[language=lean]|fun| keyword.
  The \lstinline[language=lean]|have| keyword introduces local
  lemmas within our proof scope, allowing us to break down complex
  reasoning into manageable intermediate steps,
  mirroring our natural deduction proof from before.
  Just before the keyword \lstinline[language=lean]|show|,
  the info view displays the following
  context and goal:
  \begin{lstlisting}[language=lean]
  a b c : Prop
  h : (a ∧ b) ∧ c
  hab : a ∧ b
  ha : a
  hc : c
  hb : b
  ⊢ a ∧ b ∧ c
\end{lstlisting}
  Resembling type judgments, the goal is juxtaposed after the turnstile ($\vdash$).
  What comes before it is the current context.
  Finally, \lstinline[language=lean]|show a ∧ (b ∧ c) from ⟨ha, ⟨hb, hc⟩⟩|
  asserts that we are constructing a proof of \lstinline[language=lean]|a ∧ (b ∧ c)|
  using the term \lstinline[language=lean]|⟨ha, ⟨hb, hc⟩⟩|.
  The \lstinline[language=lean]|show| keyword makes the proof
  more readable
  and ensures that the provided
  proof term matches the stated
  goal up to definitional equality.
  As mentioned already, two types (or terms) are definitionally equal in Lean when they
  are identical after computation
  and unfolding of definitions; in other words, when Lean's type checker
  can mechanically verify they are the same without requiring additional proof steps.
  Here, the goal \lstinline[language=lean]|⊢ a ∧ b ∧ c| is definitionally
  equal to \lstinline[language=lean]|a ∧ (b ∧ c)| due to how conjunction
  associates, so \lstinline[language=lean]|show| accepts this statement.
  If we had tried to use \lstinline[language=lean]|show| with a type that
  was only propositionally equal
  but not definitionally equal, Lean would reject it.
\end{example}
\section{Predicate Logic and Dependency}
To capture more complex mathematical ideas, we extend our system from
propositional logic to \textbf{predicate logic}.
A \textbf{predicate} is a statement or proposition that depends on a variable.
In propositional logic we represent a proposition simply by $P$.
In predicate logic, this is generalized.
A predicate is written as $P(a)$,
where $a$ is a variable. Notice that a predicate is just a function.
This extension allows us to introduce \textbf{quantifiers}:
$\forall$ (``for all'') and $\exists$ (``there exists'').
These quantifiers express that a given formula holds either for every object
or for at least one object, respectively.
In Lean if \lstinline[language=lean]|α| is any type, we can represent a
predicate \lstinline[language=lean]|P| on \lstinline[language=lean]|α| as
an object of type \lstinline[language=lean]|α → Prop|.
Thus given an \lstinline[language=lean]|x : α| (an element
with type \lstinline[language=lean]|α| )
\lstinline[language=lean]|P(x) : Prop| would be representative of a proposition
holding for \lstinline[language=lean]|x|.
We can give an informal reading of the quantifiers as infinite logical operations:
\begin{align*}
  \forall x.\,P(x) & \equiv P(a) \land P(b) \land P(c) \land \ldots \\
  \exists x.\,P(x) & \equiv P(a) \lor P(b) \lor P(c) \lor \ldots
\end{align*}
The dot symbol following the quantifier, as in $\forall x.,$, binds
every occurrence of the variable $x$ in the expression $P(x)$.
The expression $\forall x.\, P(x)$ can be understood as a generalized form of conjunction.
It expresses that $P$ holds for all possible values of $x$.
Similarly, $\exists x.\, P(x)$ is a generalized disjunction, expressing that $P$ holds
for at least one value of $x$.
Under the Curry-Howard isomorphism, universal quantifiers correspond to
\textbf{dependent function types} (also called Pi types, written $\Pi$),
while existential quantifiers correspond to
\textbf{dependent pair types} (also called Sigma types, written $\Sigma$).
These are constructs from dependent type theory, which provides a way to interpret
predicates or, more generally, types depending on some data or variable.
Technically the correspondence is not that immediate and actually Lean implements,
\lstinline[language=lean]|Exists| and \lstinline[language=lean]|Forall|
using as inductive types (this follows also for the previously defined connectives).
This time we are not going to involve deduction rules or type judgments.
Instead, we will extend the isomorphism
to quantifiers directly
by presenting the Lean syntax.
\begin{example}[Quantifiers in Lean]
  Lean expresses quantifiers as follows:
  \begin{lstlisting}[language=lean]
variable (X : Type) (P : X → Prop)
 (∀ (x : X), P x) -- ∀ corresponds to Pi type Π
 (∃ (x : X), P x) -- ∃ corresponds to Sigma type Σ
  \end{lstlisting}
\end{example}
\newpage
\begin{example}[Universal introduction in Lean]
  The \textbf{universal introduction rule} allows us to prove $\forall x, P(x)$
  by proving $P(x)$ for an \textbf{arbitrary} $x$.
  In Lean, this corresponds to constructing a function:
  \begin{lstlisting}[language=lean]
  example : ∀ n : Nat, n ≥ 0 :=
    fun n => Nat.zero_le n 
  \end{lstlisting}
  From the \lstinline[language=lean]|Nat| module in Lean, we use
  \lstinline[language=lean]|zero_le|, a built-in theorem that already
  proves the statement.
\end{example}
\begin{example}[Universal elimination in Lean]
  The \textbf{universal elimination rule} allows us to instantiate
  a universally quantified statement with a specific value.
  In Lean, this is simply function application:
  \begin{lstlisting}[language=lean]
  example (h : ∀ n : Nat, n ≥ 0) : 5 ≥ 0 :=
    h 5
  \end{lstlisting}
\end{example}
\begin{example}[Existential introduction in Lean]
  When introducing an \textbf{existential} proof,
  we need a \textbf{pair} consisting
  of a witness and a proof that this witness
  satisfies the statement.
  \begin{lstlisting}[language=lean]
  example (x : Nat) (h : x > 0) : ∃ y, y < x :=
    ⟨0, h⟩
  \end{lstlisting}
  Notice that \lstinline[language=lean]|⟨0, h⟩| is a product type holding
  data (the witness) and a proof that it satisfies the property.
\end{example}
\begin{example}[Existential elimination in Lean]
  The \textbf{existential elimination rule}
  (\lstinline[language=lean]|Exists.elim|) allows us to prove a proposition $Q$
  from $\exists x, P(x)$ by showing that $Q$ follows from $P(w)$
  for an \textbf{arbitrary} value $w$.
  The existential quantifier can be interpreted as an infinite disjunction,
  so existential elimination naturally corresponds to a \textbf{proof by cases}
  (with a single case).
  In Lean, this is done using \textbf{pattern matching}
  with \lstinline[language=lean]|cases|:
  \begin{lstlisting}[language=lean]
  example (h : ∃ n : Nat, n > 0) : ∃ n : Nat, n > 0 := by
    cases h with
    | intro witness proof => ⟨witness, proof⟩
  \end{lstlisting}
\end{example}

\section{Constructive Mathematics}

Mathematicians have traditionally worked within \textbf{classical logic},
using \textbf{sets} as the primary means of structuring mathematical objects.
In contrast, \textbf{type theory} does not take sets as its primitive notion,
nor is it built by first applying logic and then adding structure.
Instead, logic is internal to type theory and is based on \textbf{constructive}
(or \textbf{intuitionistic}) logic, introduced by Brouwer and formalized by
Heyting (see, e.g., \cite{girard1989proofs}, Ch 1, page 6).
A major point of departure from classical logic is that, in constructive logic,
statements cannot simply be classified as true or false;
their truth depends on whether a proof exists.
There are many conjectures, such as the Riemann Hypothesis,
for which we do not yet know whether a proof or disproof exists,
so we cannot say whether they are true or false.
Consequently, constructive logic does not universally accept principles such
as the \textbf{axiom of choice} or the \textbf{law of excluded middle}
(every proposition is either true or false) as axioms.
As a consequence, proof by contradiction does not work in this setting
without additional justification.
Constructive logic emphasizes that a statement is only
considered true if we can explicitly provide a \textbf{witness} for it.
This is what makes constructive mathematics inherently \textbf{computable}.
% We already touched on this concept in the previous section.
% In particular, we presented the logical connectives via the
% Brouwer--Heyting--Kolmogorov (BHK) interpretation.
We also emphasized that, constructively,
a proof of existence consists of a pair:
a witness together with a proof that the stated property holds for that witness.
\begin{example}[Constructive existence proof]
  We give a \textbf{constructive proof} in Lean that there exist natural numbers
  $a$ and $b$ such that $a + b = 7$:
  \begin{lstlisting}[language=lean]
example : ∃ a b : Nat, a + b = 7 := ⟨3, 4, rfl⟩
\end{lstlisting}
  To prove an existential statement, we provide \textbf{witnesses}
  (concrete values $a = 3$ and $b = 4$) and a \textbf{proof}
  that the predicate holds ($3 + 4 = 7$).
\end{example}
In classical mathematics, one might attempt a proof by contradiction.
However, this approach is not directly accepted in constructive mathematics,
as it doesn't provide explicit witnesses for the claimed objects.
Nonetheless, while constructive at its core, Lean allows users to
invoke classical principles, such as contraposition or proof by contradiction,
through \textbf{tactics} ((to be explained later))
like \lstinline[language=lean]|exfalso|.
\begin{example}[Reasoning from false]
  Here is an example of deriving any proposition from a contradiction:
  \begin{lstlisting}[language=lean]
  example (p : Prop) (h : False) : p := by
    exfalso
    exact h
  \end{lstlisting}
  This example takes a proposition $p$ to prove and a false hypothesis $h$.
  The \lstinline[language=lean]|exfalso| tactic transforms the goal into
  $\vdash \mathsf{False}$, meaning we now need to derive a contradiction.
  Since we already have a false hypothesis $h$,
  we can provide it using the \lstinline[language=lean]|exact| tactic.
\end{example}
\input{modeling_mathematical_objects}
\chapter{Formalizing the topologist's sine curve}

As part of my thesis work, with the help and revision from Prof David Loeffler,
I have formalized a well-known counterexample in topology:
the \textbf{topologist’s sine curve}.
This classic example illustrates a space that is \textbf{connected}
but not \textbf{path-connected}.
My original proof follows Conrad's paper (\cite{Conrad_connnotpathconn}),
with a few modifications
and some differences from the final
formalization \href{https://leanprover-community.github.io/mathlib4_docs/Counterexamples/TopologistsSineCurve.html}{\textbf{Counterexamples – Topologist's Sine Curve}}.
The topologist's sine curve is defined as the graph of $y = \sin(1/x)$
for $x \in (0, \infty)$,
together with the origin $(0, 0)$.
We define three sets in $\mathbb{R}^2$:
\begin{itemize}
  \item $S$: the oscillating curve $\{(x, \sin(1/x)) : x > 0\}$
  \item $Z$: the singleton set $\{(0, 0)\}$
  \item $T$: their union $S \cup Z$
\end{itemize}
In Lean, this is expressed as follows:
\begin{lstlisting}[language=lean]
  open Real Set
  def pos_real := Ioi (0 : ℝ)
  noncomputable def sine_curve := fun x ↦ (x, sin (x⁻¹))
  def S : Set (ℝ × ℝ) := sine_curve '' pos_real
  def Z : Set (ℝ × ℝ) := { (0, 0) }
  def T : Set (ℝ × ℝ) := S ∪ Z
\end{lstlisting}
We open the \lstinline[language=lean]|Real| and \lstinline[language=lean]|Set| namespaces
to avoid prefixing real number and set operations with \lstinline[language=lean]|Real.|
and \lstinline[language=lean]|Set.|, respectively.
We define the interval $(0, \infty)$ as \lstinline[language=lean]|pos_real|,
using the predefined notation \lstinline[language=lean]|Ioi 0|, from \lstinline[language=lean]|Set|.
The function \lstinline[language=lean]|sine_curve| maps a positive real number
to a point on the topologist's sine curve in $\mathbb{R}^2$.
Here, \lstinline[language=lean]|''| denotes the image of a set under a function.
It's noncomputable because it involves the sine function,
which is not computable in Lean's core logic.
The sets \lstinline[language=lean]|S|, \lstinline[language=lean]|Z|,
and \lstinline[language=lean]|T|
are defined using set operations,
and \lstinline[language=lean]|{ (0, 0) }| denotes the singleton
set containing the point $(0, 0)$.
The sets are subsets of the product space $\mathbb{R}^2$,
represented as \lstinline[language=lean]|ℝ × ℝ|.
The sin function \lstinline[language=lean]|sin| is defined in the
\lstinline[language=lean]|Real|.

The goal is to prove that $T$ is connected but not path-connected.
Let's start with connectedness.
\section{$T$ is connected}
First of all one can directlly see that  $S$ is connected, since it is the
image of the set ($(0, \infty)$) under the continuous map
$x \mapsto (x, \sin(1/x))$ and a interval in $\mathbb{R}$ is connected.
Moreover, the closure of $S$ is connected, and every set in between a connected
set and its closure are connected.
Since $T$ is contained in the closure of  $S$, $T$ is connected.
This is how a mathematician would argue informally, using known facts.
However, in a formal proof, one must justify each step.
For instance, justifying that $S$ is connected
requires proving that the map
$x \mapsto (x, \sin(1/x))$ is continuous on $(0, \infty)$
and that $(0, \infty)$ is connected.
As we have seen, even showing that a rational number is non-negative
requires several steps and the use of various lemmas from Mathlib.
Similarly, proving that a set is connected can involve multiple steps
for the
newer programmer.
We can use the structure \lstinline[language=lean]|IsConnected|,
to set up the statement and see if we can argue similarly in Lean.
\begin{lstlisting}[language=lean]
lemma S_is_conn : IsConnected S := by sorry
\end{lstlisting}
In the file where \lstinline[language=lean]|IsConnected| is defined,
\texttt{Topology/Connected/Basic.lean}, we see that it requires $S$ to be nonempty and preconnected.
One can verify this by unfolding \lstinline[language=lean]|IsConnected| in the goal.
\begin{lstlisting}[language=lean]
lemma S_is_conn : IsConnected S := by
  unfold IsConnected 
  ⊢ S.Nonempty ∧ IsPreconnected S
  sorry
\end{lstlisting}
Following the definition of \lstinline[language=lean]|IsPreconnected|, we see that it captures
the usual definition; $S$ cannot be
partitioned into two nonempty disjoint open sets.
This trivially requires
nonemptiness to make sense.
The \lstinline[language=lean]|unfold| tactic helps to expand definitions; one can use it to expand the definition of $S$ or
\lstinline[language=lean]|pos_real| defined before, as well as other Mathlib expressions.
Reflecting our argument, we can check if Mathlib includes the fact
that every interval
is connected and that connectedness is preserved
under continuous maps.
Indeed, in \texttt{Topology/Connected/Interval.lean}, we find the theorem
\lstinline[language=lean]|isConnected_Ioi.image|, stating that the image of an
interval of the form $(a, \infty)$
under a continuous map is connected.

\begin{lstlisting}[language=lean]
lemma S_is_conn : IsConnected S := by
  apply isConnected_Ioi.image
  -- ⊢ ContinuousOn sine_curve (Ioi 0) 
  sorry
\end{lstlisting}
The theorem \lstinline[language=lean]|isConnected_Ioi.image| requires proving the continuity of the map
on the interval $(0, \infty)$, which is expressed as
\lstinline[language=lean]|ContinuousOn sine_curve (Ioi 0)|.
The predicate \lstinline[language=lean]|ContinuousOn f S|
expresses that a function $f$ is continuous on a set $S$, which is what we need to prove now.
The function $x \mapsto (x, \sin(1/x))$ is continuous on $(0, \infty)$ as the
product of two functions continuous on the given domain; the identity map $x \mapsto x$
and the map $x \mapsto \sin(1/x)$.
Here is the full proof in Lean:
\begin{lstlisting}[language=lean]
lemma inv_is_continuous_on_pos_real : ContinuousOn (fun x : ℝ => x⁻¹) (pos_real) := by
  apply ContinuousOn.inv₀
  · exact continuous_id.continuousOn
  · intro x hx; exact ne_of_gt hx

lemma sin_comp_inv_is_continuous_on_pos_real : ContinuousOn
 (sine_curve) (pos_real) := by
  apply ContinuousOn.prodMk continuous_id.continuousOn
  apply continuous_sin.comp_continuousOn
  exact inv_is_continuous_on_pos_real
\end{lstlisting}
Starting from the bottom lemma, \lstinline[language=lean]|ContinuousOn.prodMk| states that the product of two functions continuous on a set is continuous on that set,
requiring a proof of the continuity of each component.
The first component is the identity map, which is continuous on any set. Mathlib provides
\lstinline[language=lean]|continuous_id.continuousOn| for this purpose.
The second component is the composition of the sine function with the inverse function.
The sine function is continuous everywhere, and for this we can use
\lstinline[language=lean]|continuous_sin|.
The method \lstinline[language=lean]|comp_continuousOn| is accessible from the
fact that \lstinline[language=lean]|continuous_sin| gives
an instance of a continuous map and is generalized
in the \lstinline[language=lean]|ContinuousOn| module.
The theorem \lstinline[language=lean]|Continuous.comp_continuousOn|
states that the composition of a continuous function with a function
that is continuous on a set is continuous on that set,
and requires proof of the continuity
on the set of the inner function.
We separate the proof that the inverse function is continuous on the positive reals
into the auxiliary lemma \lstinline[language=lean]|inv_is_continuous_on_pos_real|.
The theorem \lstinline[language=lean]|continuousOn_inv₀| states that, if a function
is continuous and non-zero on a set, then its inverse is continuous on that set.
The continuity of the identity map is proved as before.
The second argument requires proving that $x \neq 0$ for all $x$ in $(0, \infty)$.
\begin{lstlisting}[language=lean]
  · intro x hx
    exact ne_of_gt hx
\end{lstlisting}
The hypothesis \lstinline[language=lean]|hx| states that $x$ is in $(0, \infty)$,
which implies that $x > 0$.
The theorem \lstinline[language=lean]|ne_of_gt| states that if a
real number is greater than zero,
then it is non-zero, which completes the proof.
Thus the final proof goes as follows:
\begin{lstlisting}[language=lean]
lemma S_is_conn : IsConnected S := by
  apply isConnected_Ioi.image 
  · exact sin_comp_inv_is_continuous_on_pos_real
\end{lstlisting}

When writing a proof, one starts by working out the informal argument on paper.
Then one tries to translate it into Lean, step by step, looking for theorems in Mathlib.
Afterwards, one can try to optimize the proof by removing unnecessary steps or refactoring it.
Proving properties like continuity and connectedness is very common,
and there are obviously ways to achieve this with less work.
Let's showcase a refactoring of the entire proof.
First, the auxiliary lemmas
can be reduced to one-liners.
\begin{lstlisting}[language=lean]
lemma inv_is_continuous_on_pos_real : ContinuousOn (fun x : ℝ => x⁻¹) (pos_real) :=
  ContinuousOn.inv₀ (continuous_id.continuousOn) (fun _ hx =>  ne_of_gt hx)
  
lemma sin_comp_inv_is_continuous_on_pos_real : ContinuousOn
 (sine_curve) (pos_real) :=
 ContinuousOn.prodMk continuous_id.continuousOn <|
  Real.continuous_sin.comp_continuousOn <| (inv_is_continuous_on_pos_real)
\end{lstlisting}
We removed the \lstinline[language=lean]|by| keyword since we can provide a \textbf{term}
that directly proves the statement.
In \lstinline[language=lean]|inv_is_continuous_on_pos_real|, we directly apply
\lstinline[language=lean]|ContinuousOn.inv₀| with the two required arguments.
Notice that we can use a lambda function \lstinline[language=lean]|fun _ hx =>  ne_of_gt hx|
to prove that $x \neq 0$ for all $x$ in $(0, \infty)$
(recall the propositions-as-types correspondence).
In the next lemma, we use
the \lstinline[language=lean]|<\|| reverse application operator,
which allows us to avoid parentheses by changing the order of application.
This means that \lstinline[language=lean]|f <| g <| h| is
equivalent to \lstinline[language=lean]|f (g h)|.
% In our case,
% \lstinline[language=lean]|Real.continuous_sin.comp_continuousOn <\||
% \lstinline[language=lean]|(inv_is_continuous_on_pos_real)|
% is equivalent to
% \lstinline[language=lean]|Real.continuous_sin.comp_continuousOn (inv_is_continuous_on_pos_real)|.
We can inline these two lemmas into the main proof to get a final one-liner:
\begin{lstlisting}[language=lean]
lemma S_is_conn : IsConnected S :=
  isConnected_Ioi.image sine_curve <| continuous_id.continuousOn.prodMk <|
    continuous_sin.comp_continuousOn <|
    ContinuousOn.inv₀ continuous_id.continuousOn (fun _ hx => ne_of_gt hx)
\end{lstlisting}
Notice again the use of the \textbf{pipe} operator.
Reading from left to right, we are building up the proof by successive applications:
\begin{itemize}
  \item We start with \lstinline[language=lean]|isConnected_Ioi.image sine_curve|, which states that the image of $(0, \infty)$ under \lstinline[language=lean]|sine_curve| is connected if we can prove the function is continuous.
  \item We then apply \lstinline[language=lean]|continuous_id.continuousOn.prodMk|, which constructs the product of two continuous functions.
  \item Next, \lstinline[language=lean]|continuous_sin.comp_continuousOn| provides the continuity of the sine composition.
  \item Finally, \lstinline[language=lean]|ContinuousOn.inv₀ continuous_id.continuousOn (fun _ hx => ne_of_gt hx)| proves the continuity of the inverse function on positive reals.
\end{itemize}
The entire chain can be read as building the continuity proof from the innermost function (the inverse) outward to the complete sine curve function, which is then used to prove that $S$ is connected.
% The expression \lstinline[language=lean]|continuousOn_inv₀.mono fun _ hx ↦ hx.ne'|
% applies the theorem and provides the required arguments.
% The \lstinline[language=lean]|mono| method allows us to weaken the domain of
% continuity from \lstinline[language=lean]|x : x = 0 |
% to \lstinline[language=lean]|pos_real|,
% which is a subset of the former.
% The lambda function \lstinline[language=lean]|fun _ hx ↦ hx.ne'| proves that $x \neq 0$ for all $x$ in $(0, \infty)$.
% This is a common pattern in Lean, where we often need to prove that certain
% conditions hold for all elements of a set.


% \begin{lstlisting}[language=lean]
% lemma sine_curve_is_continuous_on_pos_real_one_liner : ContinuousOn (sine_curve) (pos_real) :=
%  continuous_id.continuousOn.prodMk <| Real.continuous_sin.comp_continuousOn
%    <| continuousOn_inv₀.mono fun _ hx ↦ hx.ne'
% \end{lstlisting}
Since the intersection of $Z$ and $S$ is empty, we cannot
directly conclude that $T$ is connected from the connectedness of its components alone.
However, we can use the fact that every subset between a connected set and its closure is connected.
\begin{theorem}
  Let $C$ be a connected topological space, and denote $\overline{C}$ as its closure.
  It follows that every subset $C \subseteq S \subseteq \overline{C}$ is connected.
\end{theorem}
In Mathlib, this theorem is available as
\lstinline[language=lean]|IsConnected.subset_closure|.
We can set up the statement and progress from there.
\begin{lstlisting}[language=lean]
theorem T_is_conn : IsConnected T := by
  apply IsConnected.subset_closure
  · exact S_is_conn -- ⊢ IsConnected ?s
  · tauto_set -- ⊢ S ⊆ T
  · sorry -- ⊢ T ⊆ closure S
\end{lstlisting}
The theorem requires three goals:
\begin{itemize}
  \item That $S$ is connected, which was already proved in \lstinline[language=lean]|S_is_conn|.
  \item That $S \subseteq T$, which is a trivial set operation.
        The tactic \lstinline[language=lean]|tauto_set| handles this kind of set tautologies.
  \item That $T \subseteq \overline{S}$ (the closure of $S$), which requires proof.
\end{itemize}
Let's continue with the final point.
\begin{lstlisting}[language=lean]
lemma T_sub_cls_S : T ⊆ closure S := by
  intro x hx
  cases hx with
  | inl hxS => exact subset_closure hxS
  | inr hxZ =>
      sorry
\end{lstlisting}
Proving that one set is contained in another can be done naively in a pointwise manner.
We introduce an element $x \in \mathbb{R}^2$ together with the proof that $x \in T$.
Since $T$ is a union, we use \lstinline[language=lean]|cases| to separate the two cases.
When $x \in S$, the goal is trivially solved by \lstinline[language=lean]|exact subset_closure hxS|.
The case where $x \in Z$, requires more work.

Now a trick. Looking for existing theorems using mathlib documentation is
quiet challenging, while you are still learning
the sintax and adapt to the naming convention.
One can use several ways to look for the exact theorems.
A useful tool is Loogle (similar to Haskell's Hoogle),
which helps you find theorems by their type signature or name patterns.
You can access it at \url{https://loogle.lean-lang.org/} or
use it directly in VS Code.
Depending on the previous work in the file,
Lean can already unify the goal with available
theorems and suggest the next step.
For some tactics, adding a question mark causes Lean to automatically search for
the next step involving the use of that tactic.
For instance, one can type
\lstinline[language=lean]|apply?| or
\lstinline[language=lean]|exact?|, which search
for applicable lemmas or definitions
to close the goal.
The tactics \lstinline[language=lean]|rw?|
and \lstinline[language=lean]|simp?|
work similarly but for rewriting and simplification.

At this point, \lstinline[language=lean]|apply?| suggests several ways to proceed,
some involving filters:
\begin{lstlisting}
Try this: refine Frequently.mem_closure ?_
Remaining subgoals:
  ⊢ ∃ᶠ (x : ℝ × ℝ) in 𝓝 x, x ∈ S
\end{lstlisting}
and others following the more familiar metric space approach:
\begin{lstlisting}
Try this: refine Metric.mem_closure_iff.mpr ?_
Remaining subgoals:
  ⊢ ∀ ε > 0, ∃ b ∈ S, dist x b < ε
\end{lstlisting}

The best approach, however, is to think first about how you would tackle the problem
on paper, as mentioned earlier. Since we are working with a metrizable topology on $\mathbb{R}$,
we know that the closure of a set contains all its limit points.
To show that the point $(0, 0)$ is contained in the closure of $S$,
we need to show that it is a limit point of $S$.
Thus, one can define a sequence in $S$ tending to $(0, 0)$,
and the result follows. Instead of using properties of a metric space,
we use filters, as explained before to work with limits.
We can prove that $T \subseteq \overline{S}$, by
showing that the origin is a limit point of $S$.
We construct a sequence $f : \mathbb{N} \to \mathbb{R}^2$ in $S$
converging to $(0,0)$ using
\lstinline[language=lean]|Tendsto|:
\newpage
\begin{lstlisting}[language=lean]
lemma T_sub_cls_S : T ⊆ closure S := by
  intro x hx
  cases hx with
  | inl hxS => exact subset_closure hxS
  | inr hxZ =>
      rw [hxZ]
      -- Define sequence: f(n) = (1/(nπ), 0)
      let f : ℕ → ℝ × ℝ := fun n => ((n * Real.pi)⁻¹, 0)
      -- Show f converges to (0, 0)
      have hf : Tendsto f atTop (𝓝 (0, 0)) := by
        refine Tendsto.prodMk_nhds ?_ tendsto_const_nhds
        exact tendsto_inv_atTop_zero.comp
          (Tendsto.atTop_mul_const' Real.pi_pos tendsto_natCast_atTop_atTop)
      -- Show f eventually takes values in S
      have hf' : ∀ᶠ n in atTop, f n ∈ S := by
        filter_upwards [eventually_gt_atTop 0] with n hn
        exact ⟨(n * Real.pi)⁻¹,
          inv_pos.mpr (mul_pos (Nat.cast_pos.mpr hn) Real.pi_pos),
          by simp [f, sine_curve, inv_inv, Real.sin_nat_mul_pi]⟩
      -- Apply sequential characterization of closure
      exact mem_closure_of_tendsto hf hf'
\end{lstlisting}
The proof is already reduced as much as possible.
Let's break down what's happening in without getting into details.
Using \lstinline[language=lean]|let|, we define
$f(n) = \left(\frac{1}{n\pi}, 0\right)$,
which we will show converges to $(0,0)$
and stays in $S$.

\begin{itemize}

  \item \textbf{Convergence proof} (\lstinline[language=lean]|hf|):
        We show \lstinline[language=lean]|Tendsto f atTop (𝓝 (0, 0))|.
        With \lstinline[language=lean]|Tendsto.prodMk_nhds|,
        we need to show that both coordinates tend to $0$, separately.
        For the first coordinate, we compose
        \lstinline[language=lean]|tendsto_inv_atTop_zero|
        (which states $\frac{1}{x} \to 0$ as $x \to \infty$) with the
        fact that $n\pi \to \infty$.
        The second constant coordinate is handled by
        \lstinline[language=lean]|tendsto_const_nhds|
  \item \textbf{Membership proof} (\lstinline[language=lean]|hf'|):
        We show \lstinline[language=lean]|∀ᶠ n in atTop, f n ∈ S|, meaning $f(n) \in S$
        for all sufficiently large $n$.
        We use \lstinline[language=lean]|filter_upwards|,
        which allows us to combine
        hypotheses about properties that hold eventually to prove another property holds eventually.
        Here, we combine it with \lstinline[language=lean]|eventually_gt_atTop 0|,
        which states that eventually $n > 0$.
        For such $n$, we show $f(n) = \left(\frac{1}{n\pi}, 0\right)$ is in $S$ by noting that
        the second term is:
        $$
          \sin\left(\frac{1}{\left(\frac{1}{n\pi}\right)}\right) = \sin(n\pi) = 0.
        $$

\end{itemize}
Finally, \lstinline[language=lean]|mem_closure_of_tendsto|
combines these facts:
if a sequence eventually stays in $S$ and converges to $x$,
then $x$ is in the closure of $S$.
% (ALTERNATIVE PROOF)

% \begin{proof}
%   To show that $S$ lies in the closure of $S^+$, we have to express each $p \in S$ as a limit of a
%   sequence of points in $S^+$. If $p \in S^+$, we use the constant sequence $\{p, p, \ldots\}$. If $p = (0, y)$ with
%   $|y| \leq 1$, we argue as follows. Certainly $y = \sin(\theta)$ for some $\theta \in [-\pi, \pi]$, whence $y = \sin(\theta + 2n\pi)$
%   for all positive integers $n$. Thus, for $x_n = 1/(\theta + 2n\pi) > 0$ we have $\sin(1/x_n) = y$ for all $n$. Since
%   $x_n \to 0$ as $n \to \infty$, we have $(x_n, \sin(1/x_n)) = (x_n, y) \to (0, y)$.
% \end{proof}
\subsubsection{Finalising the first part of the proof}
If you are a one-liner enthusiast like me, you don't mind trying to combine
bits and pieces to get a clean final result.
We can simplify the final theorem as follows initially:
\begin{lstlisting}[language=lean]
theorem T_is_conn : IsConnected T := 
  IsConnected.subset_closure S_is_conn (by tauto_set) T_sub_cls_S
\end{lstlisting}
The second argument is still in tactic mode with \lstinline[language=lean]|by tauto_set|,
but it looks clean and we can keep it as is.
With a bit of courage, we can also inline the proof of \lstinline[language=lean]|S_is_conn|
(while \lstinline[language=lean]|T_sub_cls_S| is way too long to inline)
to get a more self-contained one-liner:
\begin{lstlisting}[language=lean]
theorem T_is_conn : IsConnected T :=
  IsConnected.subset_closure (isConnected_Ioi.image sine_curve <|
    continuous_id.continuousOn.prodMk <|
    Real.continuous_sin.comp_continuousOn <|
    ContinuousOn.inv₀ continuous_id.continuousOn
    (fun _ hx => ne_of_gt hx)) (by tauto_set) T_sub_cls_S
\end{lstlisting}
Making these amendments is not only for the sake of shortening the proof.
Lean will, obviously, compile the proof faster by not
entering tactic mode or using multiple tactics.
Tactics internally hide many operations they automatically perform to close the goal.
Moreover, if we directly provide
a term for the proof, Lean will infer and unify everything by definitional
equality.
By providing explicit proof terms, we give Lean less work to do, making the
proof more transparent and efficient.
This practice of "golfing" is essential in a huge library such as Mathlib
community that needs to balance performance and maintainability.
From now on the rest of th code will be presented in it's reduced form.
Here is the link to the entire first part of the proof:
[\href{https://live.lean-lang.org//#codez=JYWwDg9gTgLgBAWQIYwBYBtgCMBQEwCmAdnACr4ToQDmAnnAGLDowFRwDKB8A8lACZscOQQDM4kAM4B9KASTo4ALgC8nbgDoAkhGBwAFAAZlcQLiEAShxEIRAMYRwAVxhIs6AnDFxJwIgWm2jlAAbh6qcKKOJAAecIBlhAbRADRwAEry6Bo+JPrRgN4EAJ3mll4cJlzw+qZwAOtm5spq+tn+gSEEDQDknRIQMnIKIgTiAFrl3AbVdRaNcADeBoYphg0AvkPipOOVU/WzZYBURHAjwu4gIEhk0pKOWAHoMmVKZHCAYURwtlQ3cpyzWLQ4OBwXwwKAQOCxVDRQEfJCSAiSOBQuAAd2AaBhAB9gURFFCyioAHxwAjRJC2eA3LDwmD3PpBDz4rE49hQsZEmFAoFQFFwADabIAupyue54OJnoBUQjggCTCMy1eXhSIkEhEgz6EgAKjSGQ0YGA5kKy0sXK5qCQoSRErIxH4khg4PEKHIYAMgFYNwC7O4tjQ1wv8Raa5KJfB49WD+AgANbSIioO1wAD80jgrCIdodARs9pjcckAa5pPJ8FT6Yg0l8wWkzvw0gAXmwIBp7OB86b9EwWGwNKRbfbG9WwNIQI50JmiPaeukFHrgNIpCnexmiCgAMJw2kDqswF0m01Ii2M0Q9Z6AACJAAW4cBIvjgA5S4hIgAgiX5+gF7rnBztQaSOMAopACRE+QIUIiBgRwFHQWhpGoDdt3wOBDEFVF0VQS8kSIVsSTJCk4EAC/INTgbUp0yfVDQKJJMKBCs5z6DRwHYfRh1Hed9AAORQJt1xoyQ6LAVkiAaYiZ248wKLfU1/m8UBXT5UQUhaAIglCFJqIrFIhOyGMUCHEc52AQVAEvyfNCxwkACBAOlvn8CBRGkEs+ytK1OmENACGgcyrmAGR7CIEhni0SQVxsPwKQIfgXlUGEAqC3yCFC/gsluGlLIZAwvJikLWH4aQdGADRQCQagPAUtpLQAHkxEUfJgXxHAgRwZGABLqtq+rJB4IgwwgCNIzgCqRSElqiDqhrrl8JsHEHIaRvakh+tNGKauGtqOvyohgkAAIIPhsJaZvLZqdtahqOpFfRlTgZNkTVPxpBs6D4ChYoDEklxnDLGkGlIa5bnuR4gA}{link to Lean live}]
\begin{note}
  The proof merged into the Mathlib library,
  takes $Z$ as $\{0\} \times [-1,1]$
  instead of the singleton $\{(0,0)\}$.
  This, together with the fact that $T$ equals the closure of $S$,
  yields a stronger and more general result.
  This stronger version shows that a closed set
  (specifically, the closure of $S$) can be connected but not path-connected.
  Showing that forming closure can destroy
  the property of path connectedness for subsets of a topological space.
\end{note}
\section{$T$ is not path-connected}
The main and most substantial part is showing that $T$ is not path-connected.
Showing this informally already requires constructing and
pointing out various steps in order to convince
an ideal reader.
One can argue informally by contradiction.
Suppose a path exists in the topologist's sine curve $T$
connecting a point in $S$ to a point in $Z$.
As the path approaches the $y$-axis (where $x \to 0$),
the $y$-coordinate must oscillate infinitely between $-1$ and $1$
due to the behavior of $\sin(1/x)$ as $x \to 0^+$.
This infinite oscillation contradicts the continuity of the path,
which is a fundamental requirement for path-connectedness.
To be more precise, we need to construct a sequence
that it eventually oscillates, establishing the contradiction.
We start by setting up the theorem:
\begin{lstlisting}[language=lean]
theorem T_is_not_path_conn : ¬ (IsPathConnected T) := 
  by sorry
\end{lstlisting}
In mathematics, we normally define a path-connected space as follows.
\begin{definition}
  A topological space $X$ is said to be path-connected if for every two points $a, b \in X$, there
  exists a path, i.e., a continuous map $p : [0, 1] \to X$ such that $p(0) = a$ and $p(1) = b$.
\end{definition}
The interval $[0, 1]$ is the standard choice for the domain of paths.
% In Mathlib, \lstinline[language=lean]|PathConnectedSpace X| is a type class that asserts the entire
% topological space $X$ is path-connected, while
\lstinline[language=lean]|IsPathConnected S| is a predicate used to infer that a subset $S$
of a topological space is path-connected.
\begin{lstlisting}[language=lean]
def IsPathConnected (F : Set X) : Prop :=
  ∃ x ∈ F, ∀ ⦃y⦄, y ∈ F → JoinedIn F x y
\end{lstlisting}
The auxiliary predicate \lstinline[language=lean]|JoinedIn| is defined as:
\begin{lstlisting}[language=lean]
def JoinedIn (S : Set X) (x y : X) : Prop :=
  ∃ γ : Path x y, ∀ t, γ t ∈ S
\end{lstlisting}
where \lstinline[language=lean]|Path x y| denotes a continuous map $\gamma : [0,1] \to X$ with
$\gamma(0) = x$ and $\gamma(1) = y$.
We can use the \lstinline[language=lean]|unitInterval| \textbf{subtype} of
\lstinline[language=lean]|ℝ| representing the interval
$[0,1]$.
Now let's start with the first part of the proof:
\begin{lstlisting}[language=lean]
theorem T_is_not_path_conn : ¬ (IsPathConnected T) := by
  -- Assume we have a path from z = (0, 0) to w = (1, sin(1))
  have hz : z ∈ T := Or.inr rfl
  have hw : w ∈ T := Or.inl ⟨1, ⟨zero_lt_one' ℝ, rfl⟩⟩
  intro p_conn
  apply IsPathConnected.joinedIn at p_conn
  specialize p_conn z hz w hw
  let p := JoinedIn.somePath p_conn
\end{lstlisting}
We introduce two points: $z = (0, 0)$ and $w = (1, \sin(1))$, and prove they are both in $T$, in
\lstinline[language=lean]|hz hw| and \lstinline[language=lean]|hw| (we use the introduction rule
for Or, seen it as a sum type).
Using \lstinline[language=lean]|intro p_conn|, we assume that $T$ is path-connected.
Notice that the goal is now \lstinline[language=lean]|False|, meaning we must find a contradiction.
The last three lines extract an explicit path \lstinline[language=lean]|p| connecting $z$ and $w$.
\lstinline[language=lean]|apply IsPathConnected.joinedIn at p_conn|
transforms the path-connectedness assumption into the statement
that any two points in $T$ are joined.
\lstinline[language=lean]|specialize p_conn z hz w hw| specializes this
to our specific points $z$ and $w$.
Then we extrat a path and store it using
\lstinline[language=lean]|let p := JoinedIn.somePath p_conn|.

Conrad's paper (\cite{Conrad_connnotpathconn}) defines a time $t_0 \in [0, 1]$
as the first time the path $p$ jumps from $(0,0)$ to the graph of $\sin(1/x)$, where
the x-coordinate map ($x: \mathbb{R}^2 \to \mathbb{R} $) of $p$ is positive.
$$
  t_0 = \inf\{t \in [0, 1] : x(p(t)) > 0\}
$$
The argument then uses the continuity of the $x$-coordinate map composed with the path $p$.
By continuity at $t_0$, we can find a neighborhood around $t_0$
where the path stays close to $(0,0)$. Specifically, with $\varepsilon = 1/2$,
there exists $\delta > 0$ such that for all $t$ with $|t - t_0| < \delta$,
we have $\|p(t) - p(t_0)\| < 1/2$.
We want to show the oscillating behavior around (0,0) indeed.
To simplify some steps, we instead define
$$
  t_0 = \sup\{t \in [0, 1] : x(p(t)) = 0\}
$$
to be the last time the path remains at $(0,0)$.
The same continuity argument applies with this definition.
\begin{lstlisting}[language=lean]
-- Consider the composition of the x-coordinate map with p, which is continuous
have xcoord_pathcont : Continuous fun t ↦ (p t).1 := continuous_fst.comp p.continuous
-- Let t₀ be the last time the path is on the y-axis
let t₀ : unitInterval := sSup {t | (p t).1 = 0}
let xcoord_path := fun t => (p t).1
-- The x-coordinate of the path at t₀ is 0
have hpt₀_x : (p t₀).1 = 0 :=
  (isClosed_singleton.preimage xcoord_pathcont).sSup_mem ⟨0, by aesop⟩
-- By continuity of the path, we can find a δ > 0 such that
-- for all t in [t₀ - δ, t₀ + δ], ||p(t) - p(t₀)|| < 1/2
-- Hence the path stays in a ball of radius 1/2 around (0, 0)
obtain ⟨δ, hδ, ht⟩ : ∃ δ > 0, ∀ t, dist t t₀ < δ →
  dist (p t) (p t₀) < 1/2 :=
  Metric.eventually_nhds_iff.mp <| Metric.tendsto_nhds.mp (p.continuousAt t₀) _ one_half_pos
\end{lstlisting}
The final statement uses the \lstinline[language=lean]|obtain| tactic to extract witnesses from an existential statement.
This tactic destructures the existential quantifier $\exists \delta > 0, \ldots$ into
\lstinline[language=lean]|δ| (the distance), \lstinline[language=lean]|hδ| (the proof that $\delta > 0$),
and \lstinline[language=lean]|ht| (the proof that the distance condition holds).
Since $\mathbb{R}^2$ is a metric space, we can work with the distance function \lstinline[language=lean]|dist : ℝ × ℝ → ℝ × ℝ → ℝ|,
which computes the Euclidean distance between two points.
The statement \lstinline[language=lean]|dist t t₀ < δ| expresses $|t - t_0| < \delta$ in the unit interval,
while \lstinline[language=lean]|dist (p t) (p t₀) < 1/2| expresses $\|p(t) - p(t_0)\| < 1/2$ in $\mathbb{R}^2$.
We assert that the path $p$ is
continuous at $t_0$ by \lstinline[language=lean]|p.continuousAt t₀|.
\lstinline[language=lean]|Metric.tendsto_nhds.mp| converts this
to the metric space characterization:
for any $\varepsilon > 0$, there exists $\delta > 0$ such that
points within $\delta$ of $t_0$ map to points within $\varepsilon$ of $p(t_0)$.
While, \lstinline[language=lean]|Metric.eventually_nhds_iff.mp| further unpacks
this into the $\forall t, dist\ t\ t_0 < \delta \to dist\ (p\ t)\ (p\ t_0) < \varepsilon$
form, requiring positivity of $\varepsilon = 1/2$ (\lstinline[language=lean]|one_half_pos|).

We can find a time $t_1$ greater than $t_0$ that remains in the neighborhood of $t_0$,
and obtain a point $a = x(p(t_1))) > 0$ which is positive.
\begin{lstlisting}[language=lean]
-- Let t₁ be a time when the path is not on the y-axis
-- t₁ is in (t₀, t₀ + δ], hence t₁ > t₀
obtain ⟨t₁, ht₁⟩ : ∃ t₁, t₁ > t₀ ∧ dist t₀ t₁ < δ := by
  let s₀ := (t₀ : ℝ) -- cast t₀ from unitInterval to ℝ for manipulation
  let s₁ := min (s₀ + δ/2) 1
  have hs₀_delta_pos : 0 ≤ s₀ + δ/2 := add_nonneg t₀.2.1 (by positivity)
  have hs₁ : 0 ≤ s₁ := le_min hs₀_delta_pos zero_le_one
  have hs₁': s₁ ≤ 1 := min_le_right ..
  sorry
-- Let a = xcoord_path t₁ > 0
-- This follows from the definition of t₀ and t₀ < t₁
-- so t₁ must be in S, which has positive x-coordinate
let a := (p t₁).1
have ha : a > 0 := by
  obtain ⟨x, hxI, hx_eq⟩ : p t₁ ∈ S := by
    cases p_conn.somePath_mem t₁ with
    | inl hS => exact hS
    | inr hZ =>
      -- If p t₁ ∈ Z, then (p t₁).1 = 0
      have : (p t₁).1 = 0 := by rw [hZ]
      -- So t₁ ≤ t₀, contradicting t₁ > t₀
      have hle : t₁ ≤ t₀ := le_sSup this
      have hle_real : (t₁ : ℝ) ≤ (t₀ : ℝ) := Subtype.coe_le_coe.mpr hle
      have hgt_real : (t₁ : ℝ) > (t₀ : ℝ) := Subtype.coe_lt_coe.mpr ht₁.1
      linarith
  simpa only [a, ← hx_eq] using hxI
\end{lstlisting}
The code is quite convoluted, and i will omit a
detailed explanation as well as some part of it.
However, it's worth mentioning a few key technical points.
The type \lstinline[language=lean]|unitInterval| is a \textbf{subtype}
of $\mathbb{R}$,
defined as $\{x : \mathbb{R} \mid 0 \leq x \leq 1\}$.
In Lean, a subtype \lstinline[language=lean]|{x : α // P x}|
bundles a value $x$ of type $\alpha$
together with a proof that $x$ satisfies the predicate $P$,
similar as for  \lstinline[language=lean]|Set|.
Anyway, rather than
being \lstinline[language=lean]|Set| (as subsets), they are type itself.
In particular, their terms do not share the type
of the underlying \lstinline[language=lean]|Set|.
Consequently, they lack of arithmetic properties
of the real numbers for instance.
We need to cast them to $\mathbb{R}$
(with \lstinline[language=lean]|let s₀ := (t₀ : ℝ)|)
first, then cast back to \lstinline[language=lean]|unitInterval|
by providing proofs that the bounds $[0, 1]$ are satisfied (\lstinline[language=lean]|hs₁|, \lstinline[language=lean]|hs₁'|).
In the second case of the inner statment of have \lstinline[language=lean]|ha : a > 0|,
if $p(t_1) \in Z$, then $(p\ t_1).1 = 0$ by definition of $Z = \{(0,0)\}$.
This implies $t_1 \leq t_0$ by the definition of $t_0$ as the supremum.
However, we also have $t_1 > t_0$ from our construction of $t_1$ (\lstinline[language=lean]|ht₁.1|).
The tactic \lstinline[language=lean]|linarith|, an automated solver for linear arithmetic,
recognizes this contradiction by observing both
\lstinline[language=lean]|hle_real : (t₁ : ℝ) ≤ (t₀ : ℝ)| and
\lstinline[language=lean]|hgt_real : (t₁ : ℝ) > (t₀ : ℝ)|.
Since these statements are contradictory, \lstinline[language=lean]|linarith|
proves \lstinline[language=lean]|False|.
Lemmas like \lstinline[language=lean]|Subtype.coe_lt_coe|
allow us to transfer inequalities between the subtype and its underlying type,
needed for \lstinline[language=lean]|linarith|.

Finally, \lstinline[language=lean]|simpa only [a, ← hx_eq] using hxI| completes the proof.
The tactic \lstinline[language=lean]|simpa| combines simplification (\lstinline[language=lean]|simp|)
with assumption matching. The directive \lstinline[language=lean]|only [a, ← hx_eq]|
unfolds the definition of $a = (p\ t_1).1$ and rewrites using \lstinline[language=lean]|hx_eq|
in the reverse direction, transforming the goal from \lstinline[language=lean]|(p t₁).1 > 0|
to \lstinline[language=lean]|(sine_curve x).1 > 0|.
Since \lstinline[language=lean]|sine_curve x = (x, sin(1/x))|, this simplifies to \lstinline[language=lean]|x > 0|,
which is exactly the hypothesis \lstinline[language=lean]|hxI|.
The \lstinline[language=lean]|using hxI| clause, applies this hypothesis to close the goal.

Next, the image $x(p([t_0, t_1]))$ is connected (as the continuous image of a connected set),
and it contains $0 = x(p(t_0))$ and $a = x(p(t_1))$.
Since every connected subset of $\mathbb{R}$ is an interval, we have
$$
  [0, a] \subseteq x(p([t_0, t_1]))
$$
This will be crucial for the next step, where we show that the path must oscillate.
\begin{lstlisting}[language=lean]
  -- The image x(p([t₀, t₁])) is connected and contains 0 and a
  -- Therefore [0, a] ⊆ x(p([t₀, t₁]))
  have Icc_of_a_b_sub_Icc_t₀_t₁: Set.Icc 0 a ⊆ xcoord_path '' Set.Icc t₀ t₁ :=
     IsConnected.Icc_subset
      ((isConnected_Icc (le_of_lt ht₁.1)).image _ xcoord_pathcont.continuousOn)
      (⟨t₀, left_mem_Icc.mpr (le_of_lt ht₁.1), hpt₀_x⟩)
      (⟨t₁, right_mem_Icc.mpr (le_of_lt ht₁.1), rfl⟩)
\end{lstlisting}
Now we construct a sequence that demonstrates the contradiction.
Recall that $\sin(\theta) = 1$ if and only if $\theta = \frac{(4k + 1)\pi}{2}$ for some $k \in \mathbb{Z}$.
Therefore, $(x, \sin(1/x)) = (x, 1)$ when
$$
  x = \frac{2}{(4k + 1)\pi}
$$
for $k \in \mathbb{N}$. As $k \to \infty$, these $x$-values approach 0,
so infinitely many of them lie in any interval $[0, a]$.
We define this sequence and establish its key properties:
\begin{lstlisting}[language=lean]
noncomputable def xs_pos_peak := fun (k : ℕ) => 2/((4 * k + 1) * Real.pi)
lemma xs_pos_peak_tendsto_zero : Tendsto xs_pos_peak atTop (𝓝 0) := sorry
lemma xs_pos_peak_nonneg : ∀ k : ℕ, 0 ≤ xs_pos_peak k := sorry
lemma sin_xs_pos_peak_eq_one (k : ℕ) : Real.sin ((xs_pos_peak k)⁻¹) = 1 := sorry
\end{lstlisting}
The crucial property is that this sequence eventually enters $[0, a]$:
\begin{lstlisting}[language=lean]
-- For any k ∈ ℕ, sin(1/xs_pos_peak(k)) = 1
-- Since xs_pos_peak converges to 0 as k → ∞,
-- there exist indices i ≥ 1 for which xs_pos_peak i ∈ [0, a]
have xpos_has_terms_in_Icc_of_a_b : ∃ i : ℕ, i ≥ 1 ∧ xs_pos_peak i ∈ Set.Icc 0 a := sorry
\end{lstlisting}
This gives us points on the topologist's sine curve with $y$-coordinate equal to $1$,
lying arbitrarily close to the $y$-axis.

Now we can establish the final contradiction.
Since $[0, a] \subseteq x(p([t_0, t_1]))$ by the previous argument,
and $\text{xs\_pos\_peak}(i) \in [0, a]$ for some $i$,
there must exist some $t' \in [t_0, t_1]$ such that $x(p(t')) = \text{xs\_pos\_peak}(i)$.
This means $p(t') = (\text{xs\_pos\_peak}(i), \sin(1/\text{xs\_pos\_peak}(i))) = (\text{xs\_pos\_peak}(i), 1)$,
so the $y$-coordinate of $p(t')$ equals $1$.
However, since $t' \in [t_0, t_1] \subseteq [t_0, t_0 + \delta)$,
we have $\text{dist}(t', t_0) < \delta$, which by our earlier continuity argument implies
$\|p(t') - p(t_0)\| < 1/2$.
But $\|p(t') - (0,0)\| \geq |(p(t')).2| = |1| = 1 > 1/2$,
yielding a contradiction.
\begin{lstlisting}[language=lean]
-- Show there exists time t' in [t₀, t₁] ⊆ [t₀, t₀ + δ) such that p(t') = (*, 1)
obtain ⟨t', ht', hpath_t'⟩ : ∃ t' ∈ Set.Icc t₀ t₁, (p t').2 = 1 := sorry
-- Derive the final contradiction using t', ht', hpath_t'
-- First show that p t₀ = (0, 0)
have hpt₀ : p t₀ = (0, 0) := sorry
-- t' is within δ of t₀ (since t' ∈ [t₀, t₁] and dist t₀ t₁ < δ)
have t'_close : dist t' t₀ < δ := by
  calc dist t' t₀
      ≤ dist t₁ t₀ := dist_right_le_of_mem_uIcc (Icc_subset_uIcc' ht')
    _ = dist t₀ t₁ := dist_comm _ _
    _ < δ := ht₁.2
-- By continuity, p(t') should be close to p(t₀)
have close : dist (p t') (p t₀) < 1/2 := ht t' t'_close
-- But p(t') has y-coordinate 1, so it's actually far from p(t₀) = (0, 0)
have far : 1 ≤ dist (p t') (p t₀) := by
  calc 1 = |(p t').2 - (p t₀).2| := by simp [hpath_t', hpt₀]
      _ ≤ ‖p t' - p t₀‖ := norm_ge_abs_snd
      _ = dist (p t') (p t₀) := by rw [dist_eq_norm]
-- This is a contradiction: 1 ≤ dist (p t') (p t₀) < 1/2
linarith
\end{lstlisting}
% The proof proceeds by deriving two contradictory bounds on $\text{dist}(p(t'), p(t_0))$:

% \begin{enumerate}
% \item \textbf{Extracting the critical time:} The first \lstinline[language=lean]|obtain| extracts a time $t' \in [t_0, t_1]$ where $(p\ t')_2 = 1$. This gives us three components:
% \begin{itemize}
%   \item \lstinline[language=lean]|t' : unitInterval| — the time value
%   \item \lstinline[language=lean]|ht' : t' ∈ Set.Icc t₀ t₁| — proof that $t' \in [t_0, t_1]$
%   \item \lstinline[language=lean]|hpath_t' : (p t').2 = 1| — proof that the $y$-coordinate is $1$
% \end{itemize}

% \item \textbf{Establishing the base point:} We prove \lstinline[language=lean]|hpt₀ : p t₀ = (0, 0)|, confirming that the path is at the origin at time $t_0$.

% \item \textbf{Showing proximity in time:} The statement \lstinline[language=lean]|t'_close| proves that $\text{dist}(t', t_0) < \delta$. The proof uses a \lstinline[language=lean]|calc| chain:
% \begin{itemize}
%   \item First, since $t' \in [t_0, t_1]$, we have $\text{dist}(t', t_0) \leq \text{dist}(t_1, t_0)$ (the distance from $t'$ to $t_0$ is at most the distance from $t_1$ to $t_0$)
%   \item By symmetry of distance, $\text{dist}(t_1, t_0) = \text{dist}(t_0, t_1)$
%   \item From our earlier work, $\text{dist}(t_0, t_1) < \delta$
% \end{itemize}

% \item \textbf{Upper bound from continuity:} The statement \lstinline[language=lean]|close| applies our earlier continuity result: since $\text{dist}(t', t_0) < \delta$, we have $\text{dist}(p(t'), p(t_0)) < 1/2$.

% \item \textbf{Lower bound from geometry:} The statement \lstinline[language=lean]|far| proves that $1 \leq \text{dist}(p(t'), p(t_0))$. The \lstinline[language=lean]|calc| chain shows:
% \begin{itemize}
%   \item $1 = |(p\ t')_2 - (p\ t_0)_2|$ by substituting $(p\ t')_2 = 1$ and $(p\ t_0)_2 = 0$
%   \item $|(p\ t')_2 - (p\ t_0)_2| \leq \|p(t') - p(t_0)\|$ by the fact that the norm dominates the absolute value of any component (\lstinline[language=lean]|norm_ge_abs_snd|)
%   \item $\|p(t') - p(t_0)\| = \text{dist}(p(t'), p(t_0))$ by the definition of distance in a normed space
% \end{itemize}
% This completes the proof by contradiction, showing that $T$ is not path-connected.
% \section{$T$ is connected not path-connected}
\section{Wrapping up}
Finally, we combine the two parts in the following concise and pleasant theorem:
\begin{lstlisting}[language=lean]
theorem T_is_conn_not_pathconn : IsConnected T ∧ ¬IsPathConnected T :=
  ⟨T_is_conn, T_is_not_path_conn⟩
\end{lstlisting}
And now, since this code compiles successfully, these two lines stand as verified witnesses
to the correctness of our entire proof.
This showcases the power of proof assistants and formal reasoning.
Mathematics becomes not only more rigorous but also automatically verifiable.
Furthermore, the formalization becomes a learning tool in its own right.
Future readers can inspect each part of the code.
Here the full proof: [\href{https://live.lean-lang.org/#codez=JYWwDg9gTgLgBAWQIYwBYBtgCMBQEwCmAdnAEoFLpwDKB8AYsOjAVHACr4ToQDmAnjhwATAgDM4kAM4B9KBSoAuALw06AOgCSEYHAAUABjiK4gXEIAlDiIQiAYwjgArjCRZ0BOKIlTgRAjNtHKAA3DxU4MUcSAA84QDLCfWiAGjIFdR8SPWjAbwIATvNLLxpjNXg9UzgAdbNzY1U9DP9AkIJagHI2yQhZeUoRcTgALRLaMorqizq4AG99AxSDWoBffol2Ebp9cZqp6jhAKiIhoQBaY444YCk4eyI/WxZhHHcQECQOGSlHLAD0WT2TdaAMKJrjxPvJiuEsII4BciDAoBA4LFUNEcDDbEgpAQrii4AB3YBoNFwAA+sKoKL2ygAfHACNEkPc4J8sFiYD9ukEPJTiWTfGwUcMacSYTCoHi4ABtQUAXRFovc8AkJkAqIRwQBJhGYqlrwpESCQafo9CQAFSpSjqMDAcx5BaWUWi1BIUJwVDKjjEYRSGCIiQoThgfSAVg3ALs7czttUh0IdovkYl8HktCOECAA1jIiKgvXAAPwyOAsIhen0BGzejNZqTyh30xnwQvFiAyXzBGT+/AyABerAg6ns4GrMb0jGYrHU7E93t77bAMhAjnQpaI3s65AtVpk0gLk5LRBQAGFMeyZ22YAH7TGnS63Z0TIAAIkABbhwEi+OAzlISEiACCIIaooYOYXjUcoBkRwwDxJAoGzSUCFCOFHEodB+BkXhjzPfA4AMGV8UJVBn1dIgALpBkmUAC/JjTgM013QS1rVtIiYRbTdunUcA2D0edFy3PQADkUD7I9mKkViwAFIhamo2ihPMJIGLgKFmVAQNJTEFJGgCIJQhSJiWxSSSMgzFA5wXTdgBlQBL8kHWsmRAAgQA5MF/AgMQZAbKdXQkG8cDQAhoDs95LiXEgTE0KR9xsO4HnOFRiVC8LbgIe4CGEdIvjZByuX0S54si5KZG0YB1FAJBeA8dTmhdAAeEl5RuGBfEcCBHFkYAUrqhqmqkAB5IgkwgFNUzgar5Uk9qiEa5qPl8PsHFnMaJu6khhodeL6vGzqeqKohgkAAIJrhsNaFubNqDo65qevlPQ9TgfNcUNPwZGclD4BRAp9AUlxnCbNlanYD4vh+P4TjOdZLjgHiIHgAAFFBUGOG5cseU44GcJhCX4OBnleJ47NeZ9oHs0q21ZD4ixmN4sBKUwlhKQA0Aj0JAUiwcxabgQBTIlJLAySjRSB2RrwEzfLofHql0sQAR0cYhbA8V89k+Ww8LQFB8VQYh9vAbpkpwtACzVxSiH3TTZauSgIP4K4AEYrBsfswGcVx3E8AZolkaRNwoQbdSifQvbgFVakNAAmAB6PQ9AAFkouBBoAajgS3aiotIrUsLG3ldoSPaQdM3JLbsERKCci3czP3cIHO33QwM9FDTDIz/aM40F4vGxm8B9F4KZrr9ioEiDuAQ6NKOzTjhOk/NGjU4vGE823EvdwPQSTxnYkfA7mwkKlYRgFbAhxeMxddOZVAIAlOujFUOu9AHs1DCpt6FPXsA5Rhaz4FbqcZuXdlOLgAfqoegXr2O289GzNm2qeAMXYexDRqg6T+Pp1Ank4kub0k8pJbkAXnJsK9hDCDQeyKAwBeCoHgPmS2cD5SIJAWWX+JkZwRCaiBLcODjpCHTkiN23Rs7pmsAlLu94Y4lBVAsOAgATIi4VnCug0vaqB7nAYUMId6tn4X4LuegFLWCgPZcaIBaiaIxtIQku90Zp1xm8AyZceEyJkPvR6fhfYiMjBgjIRprFu09jHG0+RFEJymP+GEV4wiJG4Z4yuqYfF+MklaQe/84Dx0GnfW+GDU4BOjDCZ+UoPG8JlAAbgiMAAg6ACHPwKcQogvA15KSlGgS4ek0gGSQPgwy9DFwwDxE2K0DSLQGQ3Co1ynS5SrDgJ2KmlQKjhEMHaG2dhZoOzcB4IoEoTCmAmVMcqxt9CJyED5PyIAAqyGsOyMAsMgolAADX6FCjDNAOVEpRXYLUdJxJkYAEEpCfFsvibkzoPBvFObrMQCIDmdnqPMeuBZER4mUHoS2alfBwoKMSYJroxkmDGT+dY4QupQC2mwKAYh0Aor+a6FZ+I4BYqmLiraVBSLwrgKRAuTZmAOIIJ0UwKRCXoHMpZRicJC5zQisSJAYAwBbxubDe5SUUoACsdB+GEJoEgKshW3DXoQWwwBKDAG7JIc5YzUBjIlKgPExJFSSCmAAKQVclZV6QHAEFuXhNVhEYTIxuD4UQbAfIa2MfVGwcBnJ6w8NEew0Ad57hYHAV4gYCS6zACkAkitGJXHmp1ElLow0QAjZuWGdUSirTOlca68AEh6EDDAcw6hKHhHTZNMQ3p26BjAN/Q6Gb3VnAtTAPavr0BHgLKADwvrAV4TBoG31/BjhIGiJcc1mwe0lCiISZVLAQiUA2dQMCMx4BkgrQWatlDVAGBWDCC12bc2ju7j7eAhp91Vpra8s4vqL1QV8CgDwwar0q0XWDAwmbuRgB7TIWIJh707UPX4owMUHR6GyqCPKGReCKhsEmAgxVSpInDVBPNaA6rVqkFu2ctkDmkQhQpJA2J8B8rgMjPQABmdQQdagWsAK3AfjLYhyDkzDG810ZBokCO2GSaPAYhIPGMmbxAAtwHAWkRgFZKydDAJ9TC2CIQLLCWpO0UiLvjlJmUKQSQkjAHoKttHJCmYg0ZoaCcuMqbVnYYd+sr1g1fBTdTwaoDNOAM1WzA9IJNTJtM+uxIIBYBcK+UiMmUioCkzFmA5kSiAGAiOAMm5MpDvAWFIO90H1j2pVVLGp5Q5bKJWgxlaIM2c4wPGDooEB0GIbYdQsFiAwAQugJCFYvTNjEGIEScDEANeAE1thmYvT9YrW24tby8u1HzDYfwTp0AuWkCp7tgBAgnkk5/5g7vlCd1mDY5QaSCTunbOqsnaCybdc5kHtOm9p6dfrRs4DmZZXdkwWHaoXwtIEizAdb8X1uJZMCl/7OnNu0kXXAQA5ESeEuHl97BWZNRnlBaqQe0pmLtWXAWoHqB2LuBQ4FGRAV1wlYMEDdPotRiGgDGpAJP7b9oDW6hUmwpCbfCCAV8DQHupa47Ua2DpUWoHRzIUQzAkBCRKEYSR6OEl85q6oZpBC1EEC7j2pjNb3pGO6CY4IZj5TC/Z9LiRzIOeqHcHOV8Iudpi+KS4KXzKZCW4W4b0lIv1ttBMMbyRtbVBc6IM7/wxDSHwHUOoeUzUPCkXZzF2Prp2dtBo+iOhUBHD3GgPKAA7bzQMm8MaSlQj1lyzA1IA/R0kQACYQ0C+DAfghAZr+FZfYAgT2HSivFRjVlAfBw5/fvoVlyvi9ZxD2Qm61a2L6CWyt7oropMz1FDnjvW9WVPRdy5Rxehl0wFXeTi0LvHE9oXw6HPvh4SIjIV9mMl5SV4hA7hvCYHAyJy16oP38kMZZMlJsloSa2/X9dFJWA3sVrU+w412Fr3rwIGa2iBeh7TklRWiDsQPmZRKH3RfyPUwheQAJjHFClGr2AIPktn/xwP71QCA1t1RBwLPXfWISJBP0AOvCdm9021l15yky4ymADyDyDzEHIRukHGF0w29z2ll3N0xn8AD210xloNwlqA1yDiY0HDwMlC3XCygPUBK2QOy3h1LBeB0PLH3hSFcFkCelVw0QUkwD3DoNQHMBIJkOsNwifS7U2DeFUFfQISvX+w+3/UuzqSuHnHQRpw6zPg/2dgkGDUXXp2EBhEXQK3+xUykERG8MCPgCwFlhID2GTTwidCuH9V3lDX2gjXfRYHnXgDeCmUrXW0PQA0AJKDeDk2wNFDCwixIFImSFdGiE0BiyQP3mB0tW8J/H+EbkHAxCxDyKCgdVsmdTnH8m8PjVQEHD5CIApCpFpDIOoCWNhAFCFGpEHGRk0AkCqMpSGB0wc30CqMg2PSIlRSfyuyuKwJ5hUNQEGHsMu2oGSNYM+xSDqi8x3nuF8HVwh0+xuPd2YPe0kSxwt38EI23X8NBKYODwUDQO8NWVqEkUswfggPUIbxbx4JbxEgFHcARO5CL16CUH0FRJ2FpExLROxLr1xIgCb3ZAJMnzIXW0fRwKsMgicIdGfjeHzylEZjgGrxRGQOwmakBM6M0BU3YH1gw1DQrT0ElDuyuxlDejBgRgeW1miP2jhF+2XCwN1KQFlLVjjD8ilAhSQGwmBGiCVJVO0zVORSCVJU0FsFsEehckl2+BZHyndNclt3+xMFGC0HdKNLgFtOw08NhjgA6FKFDNsDANROUGrDigim1JSjdI9JZDZEHHDmynTJlT9MTL0HX2dzgI5MTmrQVJuiwxzRw1HTqim3WnOnEjzNIlVPcD4NmPsizMJIHychLwrJrRkldAoJA3MmP30A7IBzgFH1/jsmLP7NLMHPLNdH+xHK5SJUnLW02ByRkSFglilkcwEyRGOAp3QCliuDxDNLKkRU42iEDgThU0KmgNUzfCIAxkGh/DVARSIDhRDn3M9j0EiSfMF2eykXLi8RuFCCgFKiuCpyMExGEU1EADwiFTNAXzHycESCMqR1WEf4opK4XQGnNgbIyCmxLxXQG7S04wp7RA92XI1yVgEAFqQPLMz0tsGQSmEHC4ERbSdmfxWHICyuXQIYjQLM8MlHB0Fog0xlHiGLHiYHeoereEYbdQUbSsfrES3OHcJsVAhmfnPMhSfI/XOvAoGA+HWQUixCTi1eB0KPOnWIShHieUZuRxUiS3V4JArs/gmQFIHSjMdMrufy3MGQZPRgkJErUJaRLxDiGdfxHiN6IwArN4IeRXV0HiWslcucGdOQEhMffMGQBfFQySLQow5kAGZlYwkmMw4KmKqCnOIKgRcfbCH9VAOdB0bkmwlTCGCUPEUTenE+UI5WesW84ieHXbYdToV8B08HG0rTe7eXefSqlNUaizGANoJ8vQE0JIHZGEWSv7NoeLY6scs5Ta/o0HTocSmABMpM2c+9LapjcA6S5on7SLYAGLEhAgL65SpERizEZinRNi4suy7it3F0EwHSvin8Dwh/WMzoEMySxdZMuADip6b0/6b4Di4Dbwjq+UQ6tozak6nskm6IP6+EhyrEAsU6shNoHs+UZGagU+CUSta64oY0SGTTQYP8t7dazpBcGI14VMUNMJXhPi49BfYXXwGQf4AY9m4Yj/UYzEbEfVBGKYp1M5EjGmnWRY6/ZY1YxRdYkiF6TY/W7Y10XYkk10exVAkwNmvxYLRYXYDQEjKaSpFDQPYAXrfrF462qGsWg83QY9JonAjEdARM6G4AOSB0eoNmyDHmfk7JKMh/CUpDdc+m6IdIfgF4GO2s8FO0LXJ4iUaUW2nsN4mMfMEO8IblIiJOlUjqqQNOqUqxQOz2ZAtlC4Rms4agXwN7B2oYkTSK+Wx2jojIACx82oUi5kfC2IOTAm96tojo0KtWcWP6jqwPM20UZGAAIWcCRD8Sjv0EJwORfXhnrMjQ/VHKSINgnrAohu5FAwPvcLbtEtDsdFJTuM2oeNA0blzzeG/wTA0haGbsqT1N4CgDeTgrgChmTHUEbRen3mztzuv3DsjsdvjqLtUH8OQZADkirrrMvRjM2qmFruoPzoovCUGl0HCDppAzcpLtFIqp/y0k6P/371btiqavsQWy7s7QABFWACiQ1Ck9wqBfjvMATA1JSwHiaM6YtR1XI2gVNGAoB0EpAWa9ZVUwCC6QsXTrwKCSgKtHaIVnbXr0QVaJj1akjpita5i9oFj5QDbXQ1jBxCbGUOixTpAejkD16t6hdP7n7HiRiw7KB0H6gWHQ0E7VAyHyGKHwNomwj67InQGu4bgIGoGu5YH+p4H0ExT9486CHoNVByDgMqCYwv93ZyS2qXokDVtr9ureTRRlidijbiI6x4y3akNPbi9fbBgMKZrrzcJXwZNIi9oGg+7pqTi5q1SPyYjoqUbNskcLxUVNqHIorJqSG4jCtzHrgwm4dctOh4CcDJEFnNsoSDmiECr2Qyy3bHBJK9AOKcy6BQIszOg6apyCGFne1xCtD+wDkirBx8wkcph2SlDLsd7eNTpxp0YUgTNv6Lg00EMoUNqINajbBkWTBorHrytPtagCtqtQX6wjn6aMWtYVM974B4WtrACrgp0oySiPAGUb7CQ2gS1IIIgQVUXtr5hFhaixAOWTBKFTnJqcWLi8X360H/FVASQcXnqzgEmmNuY/6v9yDzraaKCK78xJFaYHazgKtWZwhtFCZ/ATDSZHgcCvnRX47xWj90k5yS7yqD5jWntGmiRvI1Z9lDkgogqTl80IoSg0yEoZVzhYcLlJU7lCzHk6hiRSI/pAoEYkh42jlIYH9zlLIgA}{link to Lean live}]



\printbibliography
\end{document}