\documentclass{article}

\usepackage[T1]{fontenc}
\usepackage[utf8]{inputenc}
\usepackage{listings}
\usepackage{amssymb}
\usepackage{amsmath}
\usepackage{mathtools}
\usepackage{bussproofs} % for natural deduction rules
\usepackage{color}

\definecolor{codegreen}{rgb}{0,0.6,0}
\definecolor{codegray}{rgb}{0.5,0.5,0.5}
\definecolor{codepurple}{rgb}{0.58,0,0.82}
\definecolor{backcolour}{rgb}{0.95,0.95,0.92}

\usepackage[backend=biber,style=alphabetic]{biblatex}
\addbibresource{bibliography.bib}

\definecolor{keywordcolor}{rgb}{0.7, 0.1, 0.1}   % red
\definecolor{tacticcolor}{rgb}{0.0, 0.1, 0.6}    % blue
\definecolor{commentcolor}{rgb}{0.4, 0.4, 0.4}   % grey
\definecolor{symbolcolor}{rgb}{0.0, 0.1, 0.6}    % blue
\definecolor{sortcolor}{rgb}{0.1, 0.5, 0.1}      % green
\definecolor{attributecolor}{rgb}{0.7, 0.1, 0.1} % red
\def\lstlanguagefiles{lstlean.tex}
% set default language
\lstset{language=lean}
\lstset{literate={\#}{{\#}}1}
\newtheorem{theorem}{Theorem}[section]
\newtheorem{proposition}[theorem]{Proposition}
\newtheorem{definition}[theorem]{Definition}  
\newtheorem{remark}[theorem]{Remark}
\newtheorem{example}[theorem]{Example}
\newtheorem{notation}[theorem]{Notation}
\numberwithin{figure}{theorem}
\title{A Gentle Introduction to Dependent Type Theory and Lean}
\author{Daniele Bolla}
\begin{document}


\tableofcontents

% \section{Introduction}
This serves as a brief starting point for understanding how the Curry-Howard correspondence appears in Lean, 
as well as being an introduction to the language itslef. 
Lean is both a \textbf{functional programming language} and a \textbf{theorem prover}.
We'll focus primarily on its role as a theorem prover. 
But what does this mean, and how can that be achieved?

A programming language defines a \textbf{set of rules, semantics, and syntax} for writing programs. 
To achieve a goal, a programmer must write a program that meets given specifications. 
There are two primary approaches: \textbf{program derivation} and \textbf{program verification} 
( \cite{nordstrom1990programming} Section 1.1).
In \textbf{program verification}, the programmer first writes a program and then proves it meets 
the specifications. This approach checks for errors at \textbf{run-time} when the code executes.
In \textbf{program derivation}, the programmer writes a proof that a program with certain properties exists, 
then extracts a program from that proof. This approach enables specification 
checking at \textbf{compilation-time}, catching errors before execution.
This distinction corresponds to \textbf{dynamic} versus \textbf{static} \textbf{type systems}. 
Most programming languages combine both approaches; providing basic types for annotation and compile-time checking, 
while leaving the remaining checks to be performed at runtime.
\begin{example}
    In a dynamically typed language, like JavaScript, variables can change type after 
    they are created. For example, a variable defined as a number can later 
    be reassigned to a string:
    \begin{lstlisting}[language=JavaScript]
    let price = 100;
    price = "100";   // valid in JavaScript
    \end{lstlisting}
    TypeScript is a statically typed superset of JavaScript.  
    Unlike JavaScript, it performs type checking at compile time.  
    This means we can prevent the previous behavior simply by adding 
    a type annotation:
    \begin{lstlisting}[language=TypeScript]
    let price: number = 100;
    price = "100";  // Error: Type 'string' is not assignable to type 'number'
    \end{lstlisting} 
\end{example}
Nonetheless, TypeScript, even though it has a quite sophisticated type system, cannot fully 
capture complex data structures. In many cases, program specifications can 
only be enforced at runtime.  
Lean, by contrast, is fully grounded in program derivation and static type checking, 
which is precisely what makes it a powerful \textbf{theorem prover}.  
At its core lies an advanced and flexible type system, capable of expressing 
and verifying a wide range of mathematical statements.  
This system is built on on \textbf{Type Theory}, a branch of marthematics and logic 
that aims to provide a foundation for all mathematics and wich is a (abstract) 
programming language itself.

It's important to note that Type Theory is not a single, unified theory, but rather a family of 
related theories with various extensions, ongoing developments and rich hiostorical ramifications. 
Creating a language like Lean requirev careful consideration of which rules and features to include.
We shall give a brief overview of the historical development of type theory, and an a introduction on 
what comes next.

(The following historical intro is insbyured by \cite{carneiro2019typetheorylean} intro)
Type theory emerged as a fundamental response to Bertrand Russell's paradox. 
Considers the set $S = \{x \mid x \notin x\}$ 
(the set of all sets that do not contain themselves). This is a paradoxical construction, 
leading to the contradiction $S \in S \iff S \notin S$. 
Ernst Zermelo and Fraenkel addressed the contradiction by introducing Zermelo-Fraenkel set theory (ZFC), 
which became the standard in modern mathematics. 
ZFC provides an untyped but stratified view of the mathematical universe, 
maintaining classical logical principles while avoiding paradoxes through careful axiomatization.
Russell chose a fundamentally different path. He recognized expressions 
like $A(A)$ or $x \in x$ as ill-typed, introducing his theory of types.
Hs's first systematic response was \textbf{Ramified Type Theory}, wich turned out to be problematic.
In the 1930s, Alonzo Church developed \textbf{Lambda Calculus} as a foundation for mathematics, 
initially pursuing a type-free approach. However, Church's original untyped system suffered from 
inconsistencies similar to Russell's paradox (\cite{wadler2015propositions}). 
To address these issues, 
Church introduced the \textbf{Simply Typed Lambda Calculus} in 1940 (\cite{church1940formulation}).  
This system is a version of \textbf{Simple Type Theory}, a framework able to replace 
set-theory and propositional logic.
Lambda calculus influenced the development of many programming languages as being a foundation for
functional programming.
Per Martin-L\"{o}f revolutionized type theory in the 1970s by introducing 
\textbf{dependent types} that can depend on values of other types.
Think for instance of a of vectors of length $n$ or a sequnce of $n$ elements.
\textbf{Dependent Type Theory} extends the expressive power of type systems by allowing the direct representation
 of predicates and quantifiers (in the sense of Frege),  powerful enough to replace set theory and predicate logic."
Dependent Type Theory is a derivation of \textbf{Martin-L\"{o}f Type Theory} 
(also known as \textbf{Intuitionistic Type Theory}).
Martin-L\"{o}f's system embraced constructive (intuitionistic) principles, requiring that the existence of mathematical 
objects be demonstrated through explicit construction rather than classical proof by contradiction. 
Martin-L\"{o}f Type Theory also introduced \textbf{identity types} to represent equality.
In the 1980s, Thierry Coquand and G\'{e}rard Huet introduced the \textbf{Calculus of Constructions} (CoC), 
synthesizing insights from Martin-L\"{o}f's dependent type theory with higher-order \textbf{polymorphism}. 
The Calculus of Constructions served as the theoretical foundation for the Coq proof assistant, 
one of the most influential interactive theorem provers.
The original CoC was later extended with \textbf{inductive types} to form the 
\textbf{Calculus of Inductive Constructions} (CIC). Inductive types allow for the definition 
of data structures like natural numbers, lists, and trees. 
The Lean theorem prover, developed by Leonardo de Moura and others, is also based on CIC 
but incorporates several important refinements and differences from Coq's implementation. 

A central insight in type theory is the \textbf{Curry-Howard correspondence}, 
which establishes a profound connection between logic and computation. 
Also known as the \textbf{propositions-as-types} principle, this correspondence 
represents one of the most elegant discoveries in the foundations 
of mathematics and computer science.
It serves also well as a good introduction to type theory, and will be used in this discussion.
Nontheless it continuelsy shows new applications and interpretations in modern type theories.
The Curry-Howard correspondence was independently discovered by multiple researchers.
\textbf{Haskell Curry} (1934) first observed the connection between combinatory logic and 
Hilbert-style proof systems.
\textbf{William Alvin Howard} (1969) significantly extended the correspondence to natural deduction 
and the simply typed lambda calculus in his seminal work ``The Formulae-as-Types Notion of Construction.'' 
The correspondence was further developed through \textbf{N.G. de Bruijn's AUTOMATH system} (1967), 
which was the first working proof checker and demonstrated the practical viability of mechanical 
proof verification. Amongst its technical innovations are a discussion of the
irrelevance of proofs when working in a classical context, which is one
of the reasons advanced by de Bruijn for the separation between the notions of 
type and prop in the system \cite{thompson1999types}. Lean also adopts this separation .
\textbf{Per Martin-L\"{o}f's type theory} extended the correspondence to
dependent types, allowing for the representation of quantifiers and identity types.
Modern proof assistants like Coq, Lean, Agda, and Idris all leverage 
variants of the Curry-Howard correspondence to enable formal verification of mathematical theorems 
and software correctness properties.

% Throughout this discussion, we will introduce how to read Type Theory notation and explore key concepts 
% including:
% \begin{itemize}
% \item Impredicativity
% \item Decidability
% \item Computability
% \item Derivability
% \item Soundness
% \item Completeness
% \item Intyensionality 
% \item Extensionality
% \end{itemize}


\section{Logic and Proposition as Types}

Logic is the study of reasoning, branching into various systems.
We refer to \textbf{classical logic} as the one that underpins much
of traditional mathematics.
It's the logic of the ancient Greeks (not fair) and truthtables, and it remains
used nowadays for pedagogical reasons.
We first introduce \textbf{propositional logic}, which is the simplest
form of classical logic.
Later we will extend this to \textbf{predicate (or first-order) logic}, which includes
\textbf{predicates} and \textbf{quantifiers}.
In this setting, a \textbf{proposition} is a statement that is either true or false,
and a \textbf{proof} is a logical argument that establishes the truth of a
proposition.
Propositions can be combined with logical \textbf{connectives} such as ``and'' ($\wedge$),
``or'' ($\vee$), ``not'' ($\neg$),``false'' ($\bot$), ,``true'' ($\top$) ``implies'' ($\Rightarrow$),  and ``if and only if'' ($\Leftrightarrow$).
These connectives allow the creation of complex or compound propositions.
\newpage
Here how connectives are defined in Lean:
\begin{example}[LogicaL connectives in Lean]\mbox{}
  \begin{lstlisting}[language=lean]
    #check And (a b : Prop) : Prop
    #check Or (a b : Prop) : Prop
    #check True : Prop
    #check False : Prop
    #check Not (a : Prop) : Prop
    #check Iff (a b : Prop) : Prop
  \end{lstlisting}
  \lstinline[language=lean]|Prop| stands for proposition, and it is an
  essential component of Lean’s type system.
  For now, we can think of it as a special type whose
  inhabitants are proofs; somewhat
  paradoxically, a type of types.
\end{example}

% \begin{definition}[Propositional Formula](\cite{thompson1999types})
% A \textbf{propositional formula} is either:
% \begin{itemize}
%     \item A \textbf{propositional variable}: $X_0, X_1, X_2, \ldots$, or
%     \item A \textbf{compound formula} formed by combining formulas using connectives:
%     [[
%         (A \wedge B), \quad (A \Rightarrow B), \quad (A \vee B), \quad \bot, \quad (A \Leftrightarrow B), \quad (\neg A)
%     ]]
%     where $A$ and $B$ are formulas themselves.
% \end{itemize}
% \end{definition}

Logic is often formalized through a framework known as the \textbf{natural deduction system},
developed by Gentzen in the 1930s (\cite{wadler2015propositions}).
This approach brings logic closer to a computable, algorithmic system.
It specifies rules for deriving
\textbf{conclusions} from \textbf{premises} (assumptions from other propositions),
called \textbf{inference rules}.
\begin{example}[Deductive style rule]
  Here is an hypothetical example of inference rule.
  \begin{prooftree}
    \AxiomC{$P_1$}
    \AxiomC{$P_2$}
    \AxiomC{$\cdots$}
    \AxiomC{$P_n$}
    \QuaternaryInfC{$C$}
  \end{prooftree}
  Where the $P_1, P_2, \ldots, P_n$, above the line, are hypothetical premises and, the hypothetical conclusion $C$ is below the line.
\end{example}
The inference rules needed are:
\begin{itemize}
  \item \textbf{Introduction rules} specify how to form compound propositions from simpler ones, and
  \item \textbf{Elimination rules} specify how to use compound propositions to derive information about their components.
\end{itemize}

Type theory employs this porocedure too, by referring to deduction
rules as judments, and each elemtns of a with terms ans types.
here an examople:
\begin{example}
  \begin{prooftree}
    \AxiomC{$\Gamma, $}
    \AxiomC{$p_1:P_1$}
    \AxiomC{$p_2:P_2$}
    \AxiomC{$\cdots$}
    \AxiomC{$P_n$}
    \QuinaryInfC{$C$}
  \end{prooftree}
\end{example}
\begin{notation}
  We use $A \vdash B$ (called turnstile) to designate a deduction of $B$ from $A$.
  It is employed in Gentzen’s \textbf{sequent calculus} (\cite{girard1989proofs})
  and moslty used in type theory.
  The square brackets around a premise $[A]$ mean that the premise $A$ is meant to
  be \textbf{discharged} at the conclusion. The classical example is the
  introduction rule for the implication connective.
  To prove an implication $A \to B$, we assume $A$
  (shown as $[A]$), derive $B$ under this assumption, and then discharge the
  assumption $A$ to conclude that $A \to B$ holds without the assumption.
  The turnstile is predominantly used in judgments and type theory with
  the meaning of ``entails that''.
\end{notation}

\begin{example}[Type Theory Judgments and Contexts]
  In type theory, \textbf{judgments} are formal statements about the well-formedness of types and terms.
  There are typically four fundamental kinds of judgments \cite{web:1}:

  \begin{enumerate}
    \item $\Gamma \vdash A \text{ type}$ -- ``$A$ is a well-formed type in context $\Gamma$''
    \item $\Gamma \vdash t : A$ -- ``$t$ is a term of type $A$ in context $\Gamma$''
    \item $\Gamma \vdash A \equiv B \text{ type}$ -- ``types $A$ and $B$ are judgmentally equal in context $\Gamma$''
    \item $\Gamma \vdash t_1 \equiv t_2 : A$ -- ``terms $t_1$ and $t_2$ are judgmentally equal of type $A$ in context $\Gamma$''
  \end{enumerate}

  The \textbf{context} $\Gamma$ (Greek capital gamma) represents a finite list of type declarations for variables,
  formally written as $\Gamma = x_1:A_1, x_2:A_2, \ldots, x_n:A_n$ (cite).
  The context encodes the \textbf{assumptions} or \textbf{hypotheses} under which a judgment is made.

  For example, consider these concrete judgments:
  \begin{align}
     & \vdash \mathbb{N} \text{ type} \quad \text{(natural numbers form a type)}                                 \\
     & x:\mathbb{N} \vdash x : \mathbb{N} \quad \text{(variable $x$ has type $\mathbb{N}$ when declared)}        \\
     & x:\mathbb{N}, y:\mathbb{N} \vdash x + y : \mathbb{N} \quad \text{(addition of naturals yields a natural)} \\
     & \vdash \lambda x:\mathbb{N}.\ x + 1 : \mathbb{N} \to \mathbb{N} \quad \text{(successor function)}
  \end{align}
\end{example}

% \begin{example}[Context Formation Rules]
%   The context itself must be well-formed according to specific rules \cite{web:41}:

%   \begin{prooftree}
%     \AxiomC{}
%     \UnaryInfC{$\diamond \text{ ctx}$}
%   \end{prooftree}
%   \textbf{Empty context rule}: The empty context $\diamond$ (or $\emptyset$) is always well-formed.

%   \begin{prooftree}
%     \AxiomC{$\Gamma \text{ ctx}$}
%     \AxiomC{$\Gamma \vdash A \text{ type}$}
%     \BinaryInfC{$\Gamma, x:A \text{ ctx}$}
%   \end{prooftree}
%   \textbf{Context extension rule}: If $\Gamma$ is a well-formed context and $A$ is a well-formed type in $\Gamma$,
%   then extending $\Gamma$ with a fresh variable $x$ of type $A$ yields a well-formed context.
% \end{example}

% \begin{example}[Typing Rules with Contexts]
%   Here are fundamental typing rules that show how contexts work in practice:

%   \begin{prooftree}
%     \AxiomC{$\Gamma, x:A, \Delta \text{ ctx}$}
%     \UnaryInfC{$\Gamma, x:A, \Delta \vdash x : A$}
%   \end{prooftree}
%   \textbf{Variable rule}: A variable $x$ has its declared type $A$ in any context where it appears.

%   \begin{prooftree}
%     \AxiomC{$\Gamma \vdash f : A \to B$}
%     \AxiomC{$\Gamma \vdash a : A$}
%     \BinaryInfC{$\Gamma \vdash f(a) : B$}
%   \end{prooftree}
%   \textbf{Application rule}: Function application preserves typing under the same context assumptions.

%   \begin{prooftree}
%     \AxiomC{$\Gamma, x:A \vdash b : B$}
%     \UnaryInfC{$\Gamma \vdash \lambda x:A.\ b : A \to B$}
%   \end{prooftree}
%   \textbf{Lambda abstraction rule}: To form a function of type $A \to B$, assume $x:A$ and derive $b:B$.
% \end{example}


Let's look at how we can define some connectives.
The inference rules needed are:
\begin{itemize}
  \item \textbf{Introduction rules} specify how to form compound propositions from simpler ones, and
  \item \textbf{Elimination rules} specify how to use compound propositions to derive information about their components.
\end{itemize}

\paragraph{Conjunction ($\land$)}
\begin{itemize}
  \item Introduction
        \begin{prooftree}
          \AxiomC{$A$}
          \AxiomC{$B$}
          \RightLabel{$\land$-Intro}
          \BinaryInfC{$A \land B$}
        \end{prooftree}
  \item Elimination
        \noindent
        \begin{minipage}[t]{0.5\textwidth}
          \begin{prooftree}
            \AxiomC{$A \land B$}
            \RightLabel{$\land$-Elim$_1$}
            \UnaryInfC{$A$}
          \end{prooftree}
        \end{minipage}\hfill
        \begin{minipage}[t]{0.5\textwidth}
          \begin{prooftree}
            \AxiomC{$A \land B$}
            \RightLabel{$\land$-Elim$_2$}
            \UnaryInfC{$B$}
          \end{prooftree}
        \end{minipage}
\end{itemize}
\paragraph{Disjunction ($\lor$)}
\begin{itemize}
  \item Introduction
        \begin{minipage}[t]{0.5\textwidth}
          \begin{prooftree}
            \AxiomC{$A$}
            \RightLabel{$\lor$-Intro$_1$}
            \UnaryInfC{$A \lor B$}
          \end{prooftree}
        \end{minipage}\hfill
        \begin{minipage}[t]{0.5\textwidth}
          \begin{prooftree}
            \AxiomC{$B$}
            \RightLabel{$\lor$-Intro$_2$}
            \UnaryInfC{$A \lor B$}
          \end{prooftree}
        \end{minipage}
  \item Elimination (Proof by cases)
        \begin{prooftree}
          \AxiomC{$A \lor B$}
          \AxiomC{$[A] \vdash C$}
          \AxiomC{$[B] \vdash C$}
          \RightLabel{$\lor$-Elim}
          \TrinaryInfC{$C$}
        \end{prooftree}
\end{itemize}
\paragraph{Implication ($\to$)}
\begin{itemize}
  \item Introduction
        \begin{prooftree}
          \AxiomC{$[A] \vdash B$}
          \RightLabel{$\to$-Intro}
          \UnaryInfC{$A \to B$}
        \end{prooftree}
  \item Elimination (Modus Ponens)
        \begin{prooftree}
          \AxiomC{$A \to B$}
          \AxiomC{$A$}
          \RightLabel{$\to$-Elim}
          \BinaryInfC{$B$}
        \end{prooftree}
\end{itemize}


% \section{Judgments and Propositions}

% Logic is often formalized through a framework that distinguishes clearly between
% \textbf{judgments} and \textbf{propositions}, following Martin-Löf's foundational approach
% (\cite{martin-lof-1983}, \cite{plato:intuitionistic-type-theory}).
% A \textbf{judgment} represents something we may know — an object of knowledge that becomes
% evident once we have a proof of it.

% The most fundamental form of judgment in logic is ``\textit{A is true}'', where \( A \) is a
% proposition. This is formally written as \( A\ \text{true} \).  
% When we derive such judgments under assumptions, we write
% $$
% \Gamma \vdash A\ \text{true},
% $$
% where \( \Gamma \) represents our hypothetical assumptions
% (\cite{pfenning-natded}).

% \begin{example}[Judgment-Based Inference Rules]
% All logical reasoning can be expressed through \textbf{introduction} and
% \textbf{elimination} rules for judgments.
% For instance, conjunction is characterized as follows.

% \textbf{Introduction:} how to establish the judgment \( A \land B\ \text{true} \):

% \begin{prooftree}
%   \AxiomC{\(A\ \text{true}\)}
%   \AxiomC{\(B\ \text{true}\)}
%   \RightLabel{\(\land\text{-I}\)}
%   \BinaryInfC{\(A \land B\ \text{true}\)}
% \end{prooftree}

% \textbf{Elimination:} how to use the judgment \( A \land B\ \text{true} \):

% \begin{minipage}[t]{0.45\textwidth}
% \begin{prooftree}
%   \AxiomC{\(A \land B\ \text{true}\)}
%   \RightLabel{\(\land\text{-E}_1\)}
%   \UnaryInfC{\(A\ \text{true}\)}
% \end{prooftree}
% \end{minipage}\hfill
% \begin{minipage}[t]{0.45\textwidth}
% \begin{prooftree}
%   \AxiomC{\(A \land B\ \text{true}\)}
%   \RightLabel{\(\land\text{-E}_2\)}
%   \UnaryInfC{\(B\ \text{true}\)}
% \end{prooftree}
% \end{minipage}
% \end{example}

% \begin{notation}[The Turnstile Symbol]
% The symbol \( \vdash \) (the \textit{turnstile}) separates assumptions from conclusions
% in judgments.  
% $$
% \Gamma \vdash J
% $$
% means that the judgment \( J \) follows from the assumptions \( \Gamma \), or equivalently,
% that \( J \) is evident given evidence for \( \Gamma \).

% When assumptions are \textit{discharged} during reasoning, we indicate this with square
% brackets \([A]\).
% For example, to establish an implication \( A \rightarrow B \), we assume \( A \)
% (written \([A]\)), derive \( B \) under this assumption, then discharge \( A \):

% \begin{prooftree}
%   \AxiomC{\([A\ \text{true}]^{u}\)}
%   \AxiomC{\(\vdots\)}
%   \AxiomC{\(B\ \text{true}\)}
%   \RightLabel{\(\rightarrow\text{-I}^{u}\)}
%   \TrinaryInfC{\(A \rightarrow B\ \text{true}\)}
% \end{prooftree}
% \end{notation}

% This judgment-based framework extends naturally to \textbf{type theory}, where additional
% judgment forms are introduced.  
% While logic focuses on the judgment \( A\ \text{true} \),
% type theory introduces several fundamental forms 
% (\cite{plato:intuitionistic-type-theory}):

% \begin{align}
%   &\Gamma \vdash A\ \text{type}
%     && \text{(\(A\) is a well-formed type)} \notag \\
%   &\Gamma \vdash t : A
%     && \text{(\(t\) is a term of type \(A\))} \notag \\
%   &\Gamma \vdash A \equiv B\ \text{type}
%     && \text{(types \(A\) and \(B\) are judgmentally equal)} \notag \\
%   &\Gamma \vdash t_1 \equiv t_2 : A
%     && \text{(terms \(t_1\) and \(t_2\) are judgmentally equal)} \notag \\
% \end{align}

% The \textbf{context} \( \Gamma \) represents a list of assumptions about variables and
% their types:
% $$
% \Gamma = x_1 : A_1,\, x_2 : A_2,\, \ldots,\, x_n : A_n.
% $$

% \begin{example}[Unified Introduction/Elimination Pattern]
% Both logical connectives and type constructors follow the same
% \textit{introduction/elimination} pattern.
% For function types (corresponding to implication):

% \textbf{Formation:}
% \begin{prooftree}
%   \AxiomC{\(\Gamma \vdash A\ \text{type}\)}
%   \AxiomC{\(\Gamma \vdash B\ \text{type}\)}
%   \BinaryInfC{\(\Gamma \vdash A \rightarrow B\ \text{type}\)}
% \end{prooftree}

% \textbf{Introduction:}
% \begin{prooftree}
%   \AxiomC{\(\Gamma, x : A \vdash b : B\)}
%   \RightLabel{\(\rightarrow\text{-I}\)}
%   \UnaryInfC{\(\Gamma \vdash \lambda x : A.\, b : A \rightarrow B\)}
% \end{prooftree}

% \textbf{Elimination:}
% \begin{prooftree}
%   \AxiomC{\(\Gamma \vdash f : A \rightarrow B\)}
%   \AxiomC{\(\Gamma \vdash a : A\)}
%   \RightLabel{\(\rightarrow\text{-E}\)}
%   \BinaryInfC{\(\Gamma \vdash f(a) : B\)}
% \end{prooftree}

% This correspondence reveals a deep connection:
% logical implication and function types share the same structural rules,
% differing only in focus — whether on truth (\(A \rightarrow B\ \text{true}\))
% or on typing (\(\lambda x.\,b : A \rightarrow B\)).
% \end{example}

% \subsection*{References}
% Per Martin-Löf, \textit{Intuitionistic Type Theory}, 1983–1984. \\
% P. Dybjer, \textit{Intuitionistic Type Theory}, \textit{Stanford Encyclopedia of Philosophy}, 2016. \\
% Frank Pfenning, \textit{Logical Frameworks and Natural Deduction}, lecture notes. \\
% Additional formal presentations and lecture notes on type theory and natural deduction, CMU (various).


% %  \lstinline[language=lean]|And.intro _ _| or shortly
% %  \lstinline[language=lean]|⟨_, _⟩| (underscore are placeholder for assumptions or "propositional functions"). 
% %  The pair $A \land B$ can be then consumed using elimination 
% rules \lstinline[language=lean]|And.left| and \lstinline[language=lean]|And.right|.
% \begin{example}\label{ex:conj_intro}
% Let's look at our first Lean example:
% \begin{lstlisting}[language=lean]
%   example (a b : Prop) (ha : a) (hb : b) : (a \land b) := And.intro ha hb
% \end{lstlisting}
% This illustrates the \textbf{introduction rule} for conjunction in Lean.  
% The declaration
% \begin{lstlisting}[language=lean]
%   (a b : Prop) (ha : a) (hb : b) : (a \land b)
% \end{lstlisting}
% means that, given proofs \lstinline[language=lean]|ha| and \lstinline[language=lean]|hb|
% of the propositions \lstinline[language=lean]|a| and \lstinline[language=lean]|b|,
% we can construct a proof of \lstinline[language=lean]|(a \land b)|.
% Try placing the cursor just before \lstinline[language=lean]|And.intro|.
% Observe how the \textit{infoview} updates to display:
% \begin{lstlisting}[language=lean]
%   a b : Prop
%   ha : a
%   hb : b
%   \vdash a \land b
% \end{lstlisting}
% Here we see again the turnstile symbol (\(\vdash\)), introduced earlier.
% Everything before it represents the \textbf{context} (\(\Gamma\))—our variables and hypotheses—
% while everything after it denotes the \textbf{goal} to be proved.
% Lean implements the constructor \lstinline[language=lean]|And.intro| as:
% \begin{lstlisting}[language=lean]
%   And.intro : p \to q \to (p \land q)
% \end{lstlisting}
% This means: given a proof of \lstinline[language=lean]|p| and a proof of
% \lstinline[language=lean]|q|, Lean can produce a proof of
% \lstinline[language=lean]|(p \land q)|.
% We therefore complete the proof by directly providing
% \lstinline[language=lean]|And.intro ha hb|.
% Alternatively, we can express the same statement more compactly:
% \begin{lstlisting}[language=lean]
%   example (a b : Prop) (ha : a) (hb : b) : (a \land b) := ⟨ha, hb⟩
% \end{lstlisting}
% Here, \lstinline[language=lean]|⟨ha, hb⟩| is a \textbf{pair} (or product) term—
% you can think of it as a Cartesian pair combining the two proofs into one.
% \end{example}
% This system of inference rules allows us to construct proofs in an 
% algorithmic and systematical way, organized in what is called a \textbf{proof tree}. 
% To reduce complexity, we follow a 
% \textbf{top-down} approach (see \cite{thompson1999types} and 
% \cite{nordstrom1990programming}).


Let's look at how we can define some connectives.
\paragraph{Conjunction ($\land$)}
\begin{itemize}
  \item Introduction
        \begin{prooftree}
          \AxiomC{$A$}
          \AxiomC{$B$}
          \RightLabel{$\land$-Intro}
          \BinaryInfC{$A \land B$}
        \end{prooftree}
  \item Elimination
        \noindent
        \begin{minipage}[t]{0.5\textwidth}
          \begin{prooftree}
            \AxiomC{$A \land B$}
            \RightLabel{$\land$-Elim$_1$}
            \UnaryInfC{$A$}
          \end{prooftree}
        \end{minipage}\hfill
        \begin{minipage}[t]{0.5\textwidth}
          \begin{prooftree}
            \AxiomC{$A \land B$}
            \RightLabel{$\land$-Elim$_2$}
            \UnaryInfC{$B$}
          \end{prooftree}
        \end{minipage}
\end{itemize}
\paragraph{Disjunction ($\lor$)}
\begin{itemize}
  \item Introduction
        \begin{minipage}[t]{0.5\textwidth}
          \begin{prooftree}
            \AxiomC{$A$}
            \RightLabel{$\lor$-Intro$_1$}
            \UnaryInfC{$A \lor B$}
          \end{prooftree}
        \end{minipage}\hfill
        \begin{minipage}[t]{0.5\textwidth}
          \begin{prooftree}
            \AxiomC{$B$}
            \RightLabel{$\lor$-Intro$_2$}
            \UnaryInfC{$A \lor B$}
          \end{prooftree}
        \end{minipage}
  \item Elimination (Proof by cases)
        \begin{prooftree}
          \AxiomC{$A \lor B$}
          \AxiomC{$[A] \vdash C$}
          \AxiomC{$[B] \vdash C$}
          \RightLabel{$\lor$-Elim}
          \TrinaryInfC{$C$}
        \end{prooftree}
\end{itemize}
\paragraph{Implication ($\to$)}
\begin{itemize}
  \item Introduction
        \begin{prooftree}
          \AxiomC{$[A] \vdash B$}
          \RightLabel{$\to$-Intro}
          \UnaryInfC{$A \to B$}
        \end{prooftree}
  \item Elimination (Modus Ponens)
        \begin{prooftree}
          \AxiomC{$A \to B$}
          \AxiomC{$A$}
          \RightLabel{$\to$-Elim}
          \BinaryInfC{$B$}
        \end{prooftree}
\end{itemize}

\begin{notation}
  We use $A \vdash B$ (called turnstile) to designate a deduction of $B$ from $A$.
  It is employed in Gentzen’s \textbf{sequent calculus} (\cite{girard1989proofs}),
  whereas in natural deduction the corresponding symbol is
  $$
    \begin{array}{c}
      A      \\
      \vdots \\
      B
    \end{array}
  $$
  There are some minor differences, in fact, which I don't fully understand.
  The square brackets around a premise $[A]$ mean that the premise $A$ is meant to
  be \textbf{discharged} at the conclusion. The classical example is the
  introduction rule for the implication connective.
  To prove an implication $A \to B$, we assume $A$
  (shown as $[A]$), derive $B$ under this assumption, and then discharge the
  assumption $A$ to conclude that $A \to B$ holds without the assumption.
  The turnstile is predominantly used in judgments and type theory with
  the meaning of ``entails that''.
\end{notation}

Lean has its own syntax for connectives and their relative inference rules.
For instance $A \land B$ can be presented as \lstinline[language=lean]|And(A, B)| or \lstinline[language=lean]|A \land B|,
. Its untroduction rule is constructed by
\lstinline[language=lean]|And.intro _ _| or shortly
\lstinline[language=lean]|⟨_, _⟩| (underscore are placeholder for assumptions or "propositional functions").
The pair $A \land B$ can be then consumed using elimination
rules \lstinline[language=lean]|And.left| and \lstinline[language=lean]|And.right|.
\begin{example}\label{ex:conj_intro_2}
  Let's look at our first Lean example
  \begin{lstlisting}[language=lean]
    example (H_A : A) (H_B : B) : (A \land B) := And.intro H_A H_B
  \end{lstlisting}
  Lean aims to resemble the language used in mathematics.
  For instance, when defining a function or expression, one can use keywords such as
  \lstinline[language=lean]|theorem| or \lstinline[language=lean]|def|.
  Here, I used \lstinline[language=lean]|example|, which is handy for defining anonymous expressions
  for demonstration purposes.
  After that comes the statement to be proved:
  \begin{lstlisting}[language=lean]
  (H_A : A) (H_B : B) : (A \land B) 
\end{lstlisting}
  Meaning given a proof of $A$ and a proof of $B$ we can form a proof of $(A \land B)$.
  The operator \lstinline[language=lean]|:=| assigns a value (or return an expression) for the statement which
  "has to be a proof of it".
  \lstinline[language=lean]|And.intro| is implemented as:
  \begin{lstlisting}[language=lean]
  And.intro: p \to q \to (p \land q).
\end{lstlisting}
  It says: if you give me a proof of $p$ and a proof of $q$,
  then i return a proof of $p \land q$.
  We therefore conclude the proof by directly giving
  \lstinline[language=lean]|And.intro H_A H_B|.
  Here another way of writing the same statment.
  \begin{lstlisting}[language=lean]
  example (H_p : p) (H_B : B) : And(A, B) := ⟨H_p, H_B⟩
\end{lstlisting}
\end{example}

This system of inference rules allows us to construct proofs in an
algorithmic and systematical way, organized in what is called a \textbf{proof tree}.
To reduce complexity, we follow a
\textbf{top-down} 
% approach (see \cite{thompson1999types} and
% \cite{nordstrom1990programming})
.
This methodology forms the basis of \textbf{proof assistants} like Lean,
Coq, and Agda, which help
verify the correctness of mathematical proofs by checking each step
against these rules.
We will see later that Lean, in fact, provides an info view of the proof tree
which helps us
understand and visualize
the proof structure.


Let's examine a concrete example of a proof.
\begin{example}[Associativity of Conjunction]
  We prove that $(A \land B) \land C$ implies $A \land (B \land C)$.
  First, from the assumption $(A \land B) \land C$, we can derive $A$:
  \begin{prooftree}
    \AxiomC{$(A \land B) \land C$}
    \RightLabel{$\land E_1$}
    \UnaryInfC{$A \land B$}
    \RightLabel{$\land E_1$}
    \UnaryInfC{$A$}
  \end{prooftree}
  Second, we can derive $B \land C$:
  \begin{prooftree}
    \AxiomC{$(A \land B) \land C$}
    \RightLabel{$\land E_1$}
    \UnaryInfC{$A \land B$}
    \RightLabel{$\land E_2$}
    \UnaryInfC{$B$}
    \AxiomC{$(A \land B) \land C$}
    \RightLabel{$\land E_2$}
    \UnaryInfC{$C$}
    \RightLabel{$\land I$}
    \BinaryInfC{$B \land C$}
  \end{prooftree}
  Finally, combining these derivations we obtain $A \land (B \land C)$:
  \begin{prooftree}
    \AxiomC{$(A \land B) \land C \vdash A$}
    \AxiomC{$(A \land B) \land C \vdash B \land C$}
    \RightLabel{$\land I$}
    \BinaryInfC{$A \land (B \land C)$}
  \end{prooftree}
\end{example}
\begin{example}[Lean Implementation]
  Let us now implement the same proof in Lean.
  \begin{lstlisting}[language=lean]
theorem and_associative (a b c : Prop) : (a \land b) \land c \to a \land (b \land c) :=
fun h : (a \land b) \land c =>
-- First, from the assumption (a ∧ b) ∧ c, we can derive a:
have hab : a \land b := h.left -- extracts (derive) a proof of (a ∧ b) from the assumption
have ha : a := hab.left -- extracts a from (a ∧ b)
-- Second, we can derive b ∧ c (here we only extract b and c and combine them in the next step)
have hc : c := h.right
have hb : b := hab.right
-- Finally, combining these derivations we obtain a ∧ (b ∧ c)
show a \land (b \land c) from ⟨ha, ⟨hb, hc⟩⟩
\end{lstlisting}
  We introduce the \lstinline[language=lean]|theorem| with the name
  \lstinline[language=lean]|and_associative|.
  The type signature \lstinline[language=lean]|(a ∧ b) ∧ c → a ∧ (b ∧ c)|
  represents our logical implication.
  Here, we construct tehe proof term using a function with the \lstinline[language=lean]|fun| keyword.
  Why a function? We have already encountered the Curry-Howard correspondence in Lean
  previously, though without explicitly stating it.
  According to this correspondence, a proof of an implication can be
  understood as a function that takes a hypothesis as input and produces
  the desired conclusion as output. We will revisit this concept in more
  detail later.
  The \lstinline[language=lean]|have| keyword introduces local
  lemmas within our proof scope, allowing us to break down complex
  reasoning into manageable intermediate steps, mirroring our natural deduction proof from before.
  Just before the keyword \lstinline[language=lean]|show|, the info view displays the following
  context and goal:
  \begin{lstlisting}[language=lean]
a b c : Prop
h : (a \land b) \land c
hab : a \land b
ha : a
hc : c
hb : b
\vdash a \land b \land c
\end{lstlisting}
  Finally, the \lstinline[language=lean]|show| keyword explicitly states what
  we are proving and verifies that our provided term has the correct type.
  In this case, \lstinline[language=lean]|show a ∧ (b ∧ c) from ⟨ha, ⟨hb, hc⟩⟩|
  asserts that we are constructing a proof of \lstinline[language=lean]|a ∧ (b ∧ c)|
  using the term \lstinline[language=lean]|⟨ha, ⟨hb, hc⟩⟩|.
  The \lstinline[language=lean]|show| keyword serves two purposes:
  it makes the proof more readable by explicitly documenting what is being proved at this step,
  and it performs a type check to ensure the provided proof term matches the stated
  goal up to \emph{definitional equality}.
  Two types are definitionally equal in Lean when they are identical after computation
  and unfolding of definitions—in other words, when Lean's type checker
  can mechanically verify they are the same without requiring additional proof steps.
  Here, the goal \lstinline[language=lean]|⊢ a ∧ b ∧ c| is definitionally
  equal to \lstinline[language=lean]|a ∧ (b ∧ c)| due to how conjunction
  associates, so \lstinline[language=lean]|show| accepts this statement.
  If we had tried to use \lstinline[language=lean]|show| with a type that
  was only \emph{propositionally} equal (requiring a proof to establish equality)
  but not definitionally equal, Lean would reject it.
\end{example}
\subsection{Predicate logic and dependency}
To capture more complex mathematical ideas, we extend our system from
propositional logic to \textbf{predicate logic}.
A \textbf{predicate} is a statement or proposition that depends on a variable.
In propositional logic we represent a proposition simply by $P$.
In predicate logic, this is generalized: a predicate is written as $P(a)$,
where $a$ is a variable. Notice that a predicate is just a function.
This extension allows us to introduce \textbf{quantifiers}:
$\forall$ (``for all'') and $\exists$ (``there exists'').
These quantifiers express that a given formula holds either for every object
or for at least one object, respectively.
In Lean if \lstinline[language=lean]|α| is any type, we can represent a
predicate \lstinline[language=lean]|P| on \lstinline[language=lean]|α| as
an object of type \lstinline[language=lean]|α → Prop|.
Thus given an \lstinline[language=lean]|x : α| (an element
with type \lstinline[language=lean]|α| )
\lstinline[language=lean]|P(x) : Prop| would be representative of a proposition
holding for \lstinline[language=lean]|x|.

% When introducing variables into a formal language we must keep in mind that the specific choice 
of a variable name can be substituted without
changing the meaning of the predicate or statement. This should feel familiar from mathematics,
where the meaning of an expression does not depend on the names we assign to variables.
% Some variables are \textbf{bound} (constrained), while others remain \textbf{free} 
% (arbitrary, in programming often called "dummy" variables). 
% When substituting variables, it is important to ensure that this distinction is preserved.
% This phenomenon, called \textbf{variable capture}, parallels familiar mathematical practice: 
% if $f(x) \coloneqq \int_1^2 \frac{dt}{x-t}$, then $f(t)$ equals $\int_1^2 \frac{ds}{t-s}$, 
% not the ill-defined $\int_1^2 \frac{dt}{t-t}$. The same principle applies to predicate logic. For example, consider
% [[
% $\exists y.\,(y > x)$.
% ]]
% This states that for a given $x$ there exists a $y$ such that $y > x$. 
% If we naively substitute $y+1$ for $x$, we would obtain
% [[
% $\exists y.\,(y > y+1)$,
% ]]
% where the $y$ in $y+1$ has been \textbf{captured} by the quantifier $\exists y$. 
% This transforms the original statement from "there exists some $y$ greater than the free variable $x$" into the always-false statement 
% "there exists some $y$ greater than itself plus one."
% To avoid the probelm, in the above example, we would first rename the bound variable to something fresh 
% say $z$, obtaining $\exists z.\,(z > x)$, and then safely substitute to get $\exists z.\,(z > y+1)$.
% \begin{notation}
%   We use the notation $\phi[t/x]$ for \textbf{substitution}, meaning all occurrences of the free 
%   variable $x$ in formula (or expression) $\phi$ are replaced by term $t$.
% \end{notation}
% We can now present the inference rules for quantifiers.
% \paragraph{Universal Quantification ($\forall$)}
% \begin{itemize}
%     \item Introduction
%     \begin{prooftree}
%     \AxiomC{$A$}
%     \RightLabel{$\forall I$}
%     \UnaryInfC{$\forall x.\,A$}
%     \end{prooftree}
%     The variable $x$ must be arbitrary in the derivation of $A$. 
%     This rule captures statements like 
%     $\forall x \in \mathbb{N}$, $x$ has a successor, 
%     but would not apply to $\forall x \in \mathbb{N}$, $x$ is prime 
%     (since we cannot derive this for an arbitrary natural number).
%     \item Elimination
%     \begin{prooftree}
%     \AxiomC{$\forall x.\,A$}
%     \RightLabel{$\forall E$}
%     \UnaryInfC{$A[t/x]$}
%     \end{prooftree}
%     The conclusion $A[t/x]$ represents the substitution of term $t$ for variable $x$ in formula $A$. 
%     From a proof of $\forall x.\,A(x)$ we can infer $A(t)$ for any term $t$.
% \end{itemize}
% \paragraph{Existential Quantification ($\exists$)}
% \begin{itemize}
%     \item Introduction
%     \begin{prooftree}
%     \AxiomC{$A[t/x]$}
%     \RightLabel{$\exists I$}
%     \UnaryInfC{$\exists x.\,A$}
%     \end{prooftree}
%     The substitution premise means that if we can find a specific term $t$ for which $A(t)$ holds, 
%     then we can introduce the existential quantifier. 
%     The introduction rule requires a witness $t$ for which the predicate holds.
%     \item Elimination
%     \begin{prooftree}
%     \AxiomC{$\exists x.\,A$}
%     \AxiomC{$[A] \vdash B$}
%     \RightLabel{$\exists E$}
%     \BinaryInfC{$B$}
%     \end{prooftree}
%     To eliminate an existential quantifier, we assume $A$ holds for some witness 
%     and derive $B$ without making any assumptions about the specific witness.
% \end{itemize}
We can give an informal reading of the quantifiers as infinite logical operations:
\begin{align*}
  \forall x.\,A(x) &
  \equiv A(a) \land A(b) \land A(c) \land \ldots \\
  \exists x.\,A(x) &
  \equiv A(a) \lor A(b) \lor A(c) \lor \ldots
\end{align*}
The expression $\forall x.\, P(x)$ can be understood as  generalized form of implication.
If $P$ is any proposition, then $\forall x.\, P$ expresses that $P$ holds
regardless of the choice of $x$. When $P$ is a predicate, depending on $x$, this captures the
idea that we can derive $P$ from any assumption about $x$.
% Morover, there is a duality between universal and existential quantification.
% We shall develop all this dicussions further after 
% exploring their computational (type theoretical) meaning.
\begin{example}
  Lean expresses quantifiers as follow.
  \begin{lstlisting}[language=lean, caption=For All]
  ∀ (x : X), P x
  forall (x : X), P x -- another notation
  \end{lstlisting}
  \begin{lstlisting}[language=lean, caption=Exists]
  ∃ (x : X), P x
  exist (x : X), P x -- another notation
  \end{lstlisting}
  Where \lstinline[language=lean]|x| is a varible with a type \lstinline[language=lean]|X|,
  and \lstinline[language=lean]|P x| is a proposition, or predicate, holding for \lstinline[language=lean]|x|.
\end{example}

\begin{example}[Existential introduction in Lean]
  When introducing an \textbf{existential} proof,
  we need a \textbf{pair} consisting
  of a witness and a proof that this witness
  satisfies the statement.
  \begin{lstlisting}[language=lean]
  example (x : Nat) (h : x > 0) : ∃ y, y < x :=
    Exists.intro 0 h -- or shortly ⟨0, h⟩
\end{lstlisting}
\end{example}
Notice that \lstinline[language=lean]|⟨0, h⟩| is product type holding a data (⟨witness)
and a proof of it.
The \textbf{existential elimination rule}
( \lstinline[language=lean]|Exists.elim|) performs the opposite operation.
It allows us to prove a proposition $Q$
from $\exists x, P(x)$ by showing
that $Q$  follows from $P(w)$  for an \textbf{arbitrary}
value $w$.
\begin{example}[Existential elimination in Lean]
  The existential rules can be interpreted as an infinite
  disjunction,
  so that existential elimination naturally corresponds to a \textbf{proof by cases}
  (with only one single case).
  In Lean, this reasoning is carried out using \textbf{pattern matching},
  a known mechanism in functional programming for dealing with cases,
  with \lstinline[language=lean]|let| or \lstinline[language=lean]|match|,
  as well as by using \lstinline[language=lean]|cases| or
  \lstinline[language=lean]|rcases| construct.
  \begin{lstlisting}[language=lean]
  example (h : ∃ n : Nat, n > 0) : ∃ n : Nat, n > 0 :=
    match h with
    | ⟨witness, proof⟩ => ⟨witness, proof⟩
  \end{lstlisting}
\end{example}
\begin{example}
  The \textbf{universal quantifier} may be regarded as a generalized function.
  Accordingly, In Lean, universal elimination is simply function application.
  \begin{lstlisting}[language=lean]
example : ∀ n : Nat, n ≥ 0 :=
  fun n => Nat.zero_le n
\end{lstlisting}
\end{example}
\section{Describing and use properties}
Functions are primitive objects in type theory.
For example, it is interesting to note that a relation can be expressed as a function:
\lstinline[language=lean]|R : α → α → Prop|.
Similarly, when defining a predicate (\lstinline[language=lean]|P : α → Prop|) we must first declare
\lstinline[language=lean]|α : Type| to be some arbitrary type.
This is what is called \textbf{polymorphism}, more specifically \textbf{parametrical polymorphism}.
A canonical example is the identity function, written as
\lstinline[language=lean]|α → α|, where
\lstinline[language=lean]|α| is a type variable.
It has the same type for
both its domain and codomain, this means it can be
applied to booleans (returning a boolean), numbers (returning a number),
functions (returning a function), and so on.
In the same spirit, we can define a transitivity property of a relation as follows:
\begin{lstlisting}[language=lean]
def Transitive (α : Type) (R : α → α → Prop) : Prop :=
  ∀ x y z, R x y → R y z → R x z
\end{lstlisting}
To use \lstinline[language=lean]|Transitive|, we must provide both the type
\lstinline[language=lean]|α| and the relation itself.
For example, here is a proof of transitivity for the less-than relation on
$\mathbb{N}$ ( in Lean \lstinline[language=lean]|Nat| or \lstinline[language=lean]|ℕ|):
\begin{lstlisting}[language=lean]
theorem le_trans_proof : Transitive Nat (· ≤ · : Nat → Nat → Prop) :=
  fun x y z h1 h2 => Nat.le_trans h1 h2 -- this lemma is provided by Lean 
\end{lstlisting}
Looking at this code, we immediately notice that explicitly
passing the type argument \lstinline[language=lean]|Nat| is somewhat repetitive.
Lean allows us to omit it by letting the type inference mechanism fill it in automatically.
This is achieved by using \textbf{implicit arguments} with curly brackets:
\begin{lstlisting}[language=lean]
def Transitive {α : Type} (R : α → α → Prop) : Prop :=
  ∀ x y z, R x y → R y z → R x z
theorem le_trans_proof : Transitive (· ≤ · : Nat → Nat → Prop) :=
  fun x y z h1 h2 => Nat.le_trans h1 h2 
\end{lstlisting}
Lean's type inference system is quite powerful: in many cases, types can be completely
inferred without explicit annotations. For instance, (NEED TO EXPLAIN TYPE INFERENECE).
Let us now revisit the transitivity proof, but this time for the less-than-equal relation on
the rational numbers (\lstinline[language=lean]|Rat| or \lstinline[language=lean]|ℚ|) instead.
\newpage
\begin{lstlisting}[language=lean]
import Mathlib

theorem rat_le_trans : Transitive (· ≤ · :   Rat → Rat → Prop) :=
  fun _ _ _ h1 h2 => Rat.le_trans h1 h2
\end{lstlisting}
Here, \lstinline[language=lean]|Rat| denotes the rational numbers in Lean,
and \lstinline[language=lean]|Rat.le_trans| is the transitivity lemma
for \lstinline[language=lean]|≤| on rational numbers, provided by Mathlib.
We import Mathlib to access \lstinline[language=lean]|Rat|
and \lstinline[language=lean]|le_trans|.
Mathlib is the community‑driven mathematical
library for Lean, containing a large body of formalized mathematics
and ongoing development.
It is the defacto standard library for both programming and proving
in Lean \cite{mathlib2020}, we will dig into it as we go along.
Notice that we used a function to discharge the universal
quantifiers required by transitivity. The underscores indicate
unnamed variables that we do not use later. If we had named
them, say \lstinline|x y z|, then:
\lstinline[language=lean]|h1| would be a proof of \lstinline[language=lean]|x ≤ y|,
\lstinline[language=lean]|h2| would be a proof of \lstinline[language=lean]|y ≤ z|,
and \lstinline[language=lean]|Rat.le_trans h1 h2| produces a proof of \lstinline[language=lean]|x ≤ z|.
The \lstinline[language=lean]|Transitive| definition is imported from Mathlib and similarly
defined as before.
\begin{example}
  The code can be made more readable using tactic mode.
  In this mode, you use tactics—commands provided by Lean or defined by users—to
  carry out proof steps succinctly, avoid code repetition,
  and automate common patterns.
  This often yields shorter, clearer proofs than writing the full term by hand.
  \begin{lstlisting}[language=lean]
import Mathlib

theorem rat_le_trans : Transitive (· ≤ · : Rat → Rat → Prop) := by
intro x y z hxy hyz
exact Rat.le_trans hxy hyz
\end{lstlisting}
  This proof performs the same steps but is much easier to read.
  Using \lstinline[language=lean]|by| we enter Lean's tactic mode,
  which (together with the info view)
  shows the current goal and context.
  Move your cursor just before \lstinline[language=lean]|by|
  and observe how the info view changes.
  The goal is initially displayed as \lstinline[language=lean]|⊢ Transitive fun x1 x2 ↦ x1 ≤ x2|.
  The tactic \lstinline[language=lean]|intro| is mainly used to introduce
  variables and hypotheses corresponding to universal quantifiers
  and assumptions into the context (essentially deconstructing universal quantifiers and implications).
  Now position your cursor just before \lstinline[language=lean]|exact|
  and observe the info view again.
  The goal is now \lstinline[language=lean]|⊢ x ≤ z|, with the context
  showing the variables and hypotheses introduced by the previous tactic.
  The \lstinline[language=lean]|exact| tactic closes the goal
  by supplying the term \lstinline[language=lean]|Rat.le_trans hxy hyz| that exactly matches the goal
  (the specification of \lstinline[language=lean]|Transitive|).
  You can hover over each tactic to see its definition and documentation.
\end{example}
\subsection{Exploring Mathlib (The Rat structure)}
In these examples we cheated and have used predefined lemmas such as
\lstinline[language=lean]|Nat.le_trans| and
\lstinline[language=lean]|Rat.le_trans|, just to simplify the presentation.
We can now dig into the implementation of these lemmas.
Let's look at the source code of \lstinline[language=lean]|Rat.le_trans|.
The Mathlib 4 documentation website is at
\url{https://leanprover-community.github.io/mathlib4_docs}, and
the documentation for
\lstinline[language=lean]|Rat.le_trans| is at
\url{https://leanprover-community.github.io/mathlib4_docs/Mathlib/Algebra/Order/Ring/Unbundled/Rat.html#Rat.le_trans}.
Click the "source" link there to jump to the implementation in the Mathlib repository. In editors like
VS Code you can also jump directly to the definition (Ctrl+click; Cmd+click on macOS).
Another way to check source code is by using \lstinline[language=lean]|#print Rat.le_trans|.
\newpage
\begin{lstlisting}[language=lean]
variable (a b c : Rat)
protected lemma le_trans (hab : a ≤ b) (hbc : b ≤ c) : a ≤ c := by
  rw [Rat.le_iff_sub_nonneg] at hab hbc
  have := Rat.add_nonneg hab hbc
  simp_rw [sub_eq_add_neg, add_left_comm (b + -a) c (-b), add_comm (b + -a) (-b), add_left_comm (-b) b (-a), add_comm (-b) (-a), add_neg_cancel_comm_assoc, ← sub_eq_add_neg] at this
  rwa [Rat.le_iff_sub_nonneg]
\end{lstlisting}
The proof uses several tactics and lemmas from Mathlib.
The \lstinline[language=lean]|rw| or \lstinline[language=lean]|rewrite| tactic
is very common and sintactically similar to
the mathematical practice of rewriting an expression using an equality.
In this case, with \lstinline[language=lean]|at|, we use it to rewrite the
hypotheses \lstinline[language=lean]|hab|
and \lstinline[language=lean]|hbc|
using the another Mathlib's lemma \lstinline[language=lean]|Rat.le_iff_sub_nonneg|,
which states that for any two rational numbers \lstinline[language=lean]|x| and
\lstinline[language=lean]|y|, \lstinline[language=lean]|x ≤ y|
is equivalent to \lstinline[language=lean]|0 ≤ y - x|.
Thus we now have the hypotheses tranformerd to :
\begin{lstlisting}[language=lean]
  hab : 0 ≤ b - a
  hbc : 0 ≤ c - b
\end{lstlisting}
The \lstinline[language=lean]|have| tactic introduces an intermediate result.
If you omit a name, Lean assigns it the default name \lstinline[language=lean]|this|.
In our situation, from \lstinline[language=lean]|hab : a ≤ b| and \lstinline[language=lean]|hbc : b ≤ c|
we can derive that \lstinline[language=lean]|b - a| and \lstinline[language=lean]|c - b|
are nonnegative, hence their sum is nonnegative:
\begin{lstlisting}[language=lean]
  this : 0 ≤ b - a + (c - b)
\end{lstlisting}
The most involved step uses \lstinline[language=lean]|simp_rw| to
simplify the expression via a sequence of other existing Mathlib's lemmas.
The tactic \lstinline[language=lean]|simp_rw| is a variant of \lstinline[language=lean]|simp|:
it performs rewriting using the simp set (and any lemmas you provide), applying the rules
in order and in the given direction. Lemmas that \lstinline[language=lean]|simp| can use
are typically marked with the \lstinline[language=lean]|@[simp]| attribute.
This is particularly useful for simplifying algebraic expressions and equations.
After these simplifications we obtain:
\begin{lstlisting}[language=lean]
  this : 0 ≤ c - a
\end{lstlisting}
Clearly, the proof relies mostly on \lstinline[language=lean]|Rat.add_nonneg|.
Its source code is fairly involved and uses advanced features
that are beyond our current scope. Nevertheless, it highlights
an important aspect of formal mathematics in Mathlib.
Mathlib defines \lstinline[language=lean]|Rat| as an instance of
a linear ordered field, implemented via a normalized fraction
representation: a pair of integers (numerator and denominator)
with positive denominator and coprime numerator and denominator \cite{mathlibdoc}.
To achieve this, it uses a \textbf{structure}. In Lean, a structure is a dependent record
(or product type) type  used to group together related fields or properties as a single data type.
Unlike ordinary records, the type of later fields may depend on the values of earlier ones.
Defining a structure automatically introduces a constructor (usually mk) and projection
functions that retrieve (deconstruct) the values of its fields.
Structures may also include proofs expressing properties that the fields must satisfy.
\newpage
\begin{lstlisting}[language=lean]
  structure Rat where
    /-- Constructs a rational number from components.
    We rename the constructor to `mk'` to avoid a clash with the smart constructor. -/
    mk' ::
    /-- The numerator of the rational number is an integer. -/
    num : Int
    /-- The denominator of the rational number is a natural number. -/
    den : Nat := 1
    /-- The denominator is nonzero. -/
    den_nz : den ≠ 0 := by decide
    /-- The numerator and denominator are coprime: it is in "reduced form". -/
    reduced : num.natAbs.Coprime den := by decide
\end{lstlisting}
In order to work with rational numbers in Mathlib, we use the
\lstinline[language=lean]|Rat.mk'| constructor to create a rational number from
its numerator and denominator, if omitted the default would be \lstinline[language=lean]|Rat.mk|.
The fields \lstinline[language=lean]|den_nz| and \lstinline[language=lean]|reduced| are proofs that
the denominator is nonzero and that the numerator and denominator are coprime, respectively.
These proofs are automatically generated by Lean's \lstinline[language=lean]|decide| tactic, which can
solve certain decidable propositions (to be discussed in the next section).
\begin{example}
  Here is how we can define and manipulate rational numbers in Lean.
  \begin{lstlisting}[language=lean]
    def half : Rat := Rat.mk' 1 2
    def third : Rat := Rat.mk' 1 3
  \end{lstlisting}
\end{example}
When working with rational numbers, or more generally with structures, we must provide the
required proofs as arguments to the constructor (or Lean must be able to ensure them).
For instance \lstinline[language=lean]|Rat.mk' 1 0| or \lstinline[language=lean]|Rat.mk' 2 6|
would be rejected.
In the case of rationals, Mathlib unfolds the definition through
\lstinline[language=lean]|Rat.numDenCasesOn|. This principle states that, to prove a property of an
arbitrary rational number, it suffices to consider numbers of the form \lstinline[language=lean]|n /. d|
in canonical (normalized) form, with \lstinline[language=lean]|d > 0| and \lstinline[language=lean]|gcd n d = 1|.
This reduction allows mathlib to transform proofs about \lstinline[language=lean]|ℚ|
into proofs about \lstinline[language=lean]|ℤ| and \lstinline[language=lean]|ℕ|,
and then lift the result back to rationals.
\begin{example}
  We present a simplified implementation of addition
  non-negativity for rationals (\lstinline[language=lean]|Rat.add_nonneg| ),
  maintaining a similar approach: projecting everything to the natural numbers and
  integers first. To illustrate the proof technique clearly, we avoid using existing lemmas
  from the Rat module in Mathlib.
  Mathlib is indeed organized into modules by mathematical
  domain (e.g., Nat, Int, Rat).
  % Lemmas are typically namespaced (e.g., Rataddnonneg) 
  % and often marked protected to prevent namespace pollution. 

  We start by defining helper lemmas needed in the main proof.
  Given a natural number
  (which in this case represents the denominator of a rational number) that is not
  equal to zero, we prove it must be positive. This follows directly by
  applying the Mathlib lemma \lstinline[language=lean]|Nat.pos_of_ne_zero|:
  \newpage
  \begin{lstlisting}[language=lean]
  import Mathlib

  lemma nat_ne_zero_pos (den : ℕ) (h_den_nz : den ≠ 0) : 0 < den :=
    Nat.pos_of_ne_zero h_den_nz
\end{lstlisting}
  The naming convention follows Mathlib
  best practices aiming to be descriptive by indicating
  types and properties involved.

  The following lemma is slightly more involved.
  It states that if a rational number (num / den)
  is non-negative, then its numerator must also be non-negative:
  \begin{lstlisting}[language=lean]
lemma rat_num_nonneg {num : ℤ} {den : ℕ} (hden_pos : 0 < den)
    (h : (0 : ℚ) ≤ num / den) : 0 ≤ num := by
  contrapose! h
  have hden_pos_to_rat : (0 : ℚ) < den := Nat.cast_pos.mpr hden_pos
  have hnum_neg_to_rat : num < (0 : ℚ) := Int.cast_lt.mpr h
  exact div_neg_of_neg_of_pos hnum_neg_to_rat hden_pos_to_rat
\end{lstlisting}
  The lemma requires the denominator to be positive as well as the non-negativity of the rational number,
  expressed as \lstinline[language=lean]|num / den| where the types of
  \lstinline[language=lean]|num| and \lstinline[language=lean]|den| are inferred.
  First, notice the type annotation \lstinline[language=lean]|(0 : ℚ)|.
  This explicit type annotation on zero forces the entire equation to be casted
  into rational numbers.
  Without this annotation, Lean would infer \lstinline[language=lean]|0|
  as a natural number by default. However, since the main theorem we are proving concerns rational numbers,
  we must ensure all comparisons occur in \lstinline[language=lean]|ℚ|.
  The tactic \lstinline[language=lean]|contrapose!| does what you might expect: it proves a statement by contraposition. According to the documentation:
  \begin{itemize}
    \item \lstinline[language=lean]|contrapose| turns a goal \lstinline[language=lean]|P → Q| into \lstinline[language=lean]|¬ Q → ¬ P|
    \item \lstinline[language=lean]|contrapose!| turns a goal \lstinline[language=lean]|P → Q| into \lstinline[language=lean]|¬ Q → ¬ P| and pushes negations inside \lstinline[language=lean]|P| and \lstinline[language=lean]|Q| using \lstinline[language=lean]|push_neg|
    \item \lstinline[language=lean]|contrapose h| first reverts the local assumption \lstinline[language=lean]|h|, then uses \lstinline[language=lean]|contrapose| and \lstinline[language=lean]|intro h|
    \item \lstinline[language=lean]|contrapose! h| first reverts the local assumption \lstinline[language=lean]|h|, then uses \lstinline[language=lean]|contrapose!| and \lstinline[language=lean]|intro h|
  \end{itemize}
  In our case, \lstinline[language=lean]|contrapose! h| transforms the goal from
  proving \lstinline[language=lean]|0 ≤ num| to assuming \lstinline[language=lean]|num < 0|
  and proving \lstinline[language=lean]|num / den < 0|.
  We then introduce two local hypotheses. The first, \lstinline[language=lean]|hden_pos_to_rat|,
  proves that the denominator
  is positive when cast to rationals, using \lstinline[language=lean]|Nat.cast_pos|.
  The suffix \lstinline[language=lean]|.mpr| selects the ``modus ponens reverse''
  direction of the biconditional (the \lstinline[language=lean]|←|
  direction of the \lstinline[language=lean]|↔|).
  Next, we introduce \lstinline[language=lean]|hnum_neg_to_rat|,
  which expresses that the numerator is negative when cast to rationals, using \lstinline[language=lean]|Int.cast_lt| with \lstinline[language=lean]|.mpr| again.
  Finally, we apply \lstinline[language=lean]|div_neg_of_neg_of_pos|,
  which states that dividing a negative number by a positive number yields a
  negative result, thus completing the proof by contraposition.
  Note that we are allowing ourselves to use existing lemmas from Mathlib,
  such as \lstinline[language=lean]|div_neg_of_neg_of_pos| from the \lstinline[language=lean]|Field| module,
  but not from the \lstinline[language=lean]|Rat| module,
  to keep the presentation clear and focused on the main proof techniques.
  \newpage
  Now we can prove the main result:
  \begin{lstlisting}[language=lean]
lemma rat_add_nonneg (a b : Rat) : 0 ≤ a → 0 ≤ b → 0 ≤ a + b := by

  intro ha hb
  cases a with | div a_num a_den a_den_nz a_cop =>
  cases b with | div b_num b_den b_den_nz b_cop =>
  -- Goal: ⊢ 0 ≤ ↑a_num / ↑a_den + ↑b_num / ↑b_den
  rw[div_add_div] -- applies the addition formula requiring two new goals 
  · sorry 
  · sorry 
  · sorry
\end{lstlisting}
  \newpage
  We first introduce the two hypotheses \lstinline[language=lean]|ha| and
  \lstinline[language=lean]|hb| into the context using \lstinline[language=lean]|intro|.
  As mentioned earlier, a structure can be viewed as a product type or a record type with
  a single constructor. The tactic \lstinline[language=lean]|cases a with|
  exposes the fields of \lstinline[language=lean]|Rat|: the
  numerator ()\lstinline[language=lean]|a_num|), denominator
  (\lstinline[language=lean]|a_den|), the proof that the denominator is non-zero
  (\lstinline[language=lean]|a_den_nz|), and the coprimality condition
  (\lstinline[language=lean]|a_cop|). Notice how the goal transforms
  the rationals \lstinline[language=lean]|a| and \lstinline[language=lean]|b| into:
  \begin{lstlisting}[language=lean]
⊢ 0 ≤ ↑a_num / ↑a_den + ↑b_num / ↑b_den
\end{lstlisting}
  where \lstinline[language=lean]|↑| denotes type coercion from
  \lstinline[language=lean]|ℤ| or \lstinline[language=lean]|ℕ| to
  \lstinline[language=lean]|ℚ|.
  Now we rewrite the goal using \lstinline[language=lean]|rw [div_add_div]|,
  a theoprem from the \lstinline[language=lean]|Field| module,
  which applies the addition formula for division.
  Let us briefly examine the source code of this theorem:
  \begin{lstlisting}[language=lean]
variable [Semifield K] {a b d : K}

theorem div_add_div (a : K) (c : K) (hb : b ≠ 0) (hd : d ≠ 0) :
    a / b + c / d = (a * d + b * c) / (b * d) := ...
\end{lstlisting}
  The type \lstinline[language=lean]|K| here is assumed to be a
  \lstinline[language=lean]|Semifield|. The \lstinline[language=lean]|variable|
  keyword is a way to declare parameters that are potentially used across
  multiple theorems or definitions. We will explore Lean's powerful algebraic
  hierarchy and the meaning of the square brackets \lstinline[language=lean]|[ ]|
  in a later section.
  Using this rewrite is particularly time-saving, since otherwise one would have to
  establish the well-definedness of rational addition in terms of the underlying
  structure (a non-trivial task).
  This theorem requires proofs \lstinline[language=lean]|(hb : b ≠ 0)| and
  \lstinline[language=lean]|(hd : d ≠ 0)|, generating two additional side goals.
  We handle each goal separately using the focusing bullet \lstinline[language=lean]|·|.
  The first bullet addresses the main goal (proving the sum is non-negative),
  while the subsequent bullets discharge the non-zero denominator conditions.
  I have omitted the actual proofs, here, using \lstinline[language=lean]|sorry|,
  which we haven't mentioned before. \lstinline[language=lean]|sorry| is a useful
  feature of Lean that tells the system to accept an incomplete proof for the time being,
  allowing you to continue development without proving every detail immediately.
  We can now tackle the remaining goals:
  \newpage
  \begin{lstlisting}[language=lean]
· -- Goal: ⊢ 0 ≤ (↑a_num * ↑b_den + ↑a_den * ↑b_num) / (↑a_den * ↑b_den)
  have hnum_nonneg : (0 : ℚ) ≤ a_num * b_den + a_den * b_num := by
    have ha_num_nonneg := by
      have ha_den_pos := nat_ne_zero_pos a_den a_den_nz
      exact rat_num_nonneg ha_den_pos ha
    have hb_num_nonneg := by
      have hb_den_pos := nat_ne_zero_pos b_den b_den_nz
      exact rat_num_nonneg hb_den_pos hb
    apply add_nonneg -- works for any OrderedAddCommMonoid
    · apply mul_nonneg -- works for any OrderedSemiring
      · exact Int.cast_nonneg.mpr ha_num_nonneg
      · exact Nat.cast_nonneg b_den
    · apply mul_nonneg
      · exact Nat.cast_nonneg a_den
      · exact Int.cast_nonneg.mpr hb_num_nonneg

  have hden_nonneg : (0 : ℚ) ≤ a_den * b_den := by
    rw [← Nat.cast_mul]
    exact Nat.cast_nonneg (a_den * b_den)
  exact div_nonneg hnum_nonneg hden_nonneg

· exact Nat.cast_ne_zero.mpr a_den_nz -- Goal: ⊢ ↑a_den ≠ 0
· exact Nat.cast_ne_zero.mpr b_den_nz -- Goal: ⊢ ↑b_den ≠ 0
\end{lstlisting}

  We introduce two key hypotheses, \lstinline[language=lean]|hnum_nonneg|
  and \lstinline[language=lean]|hden_nonneg|, which will be required by
  \lstinline[language=lean]|div_nonneg| from the
  \lstinline[language=lean]|GroupWithZero| module.
  This lemma provides us with a term that directly validates our statement.
  Note that \lstinline[language=lean]|div_nonneg| is a generalized lemma that applies
  not only to rational numbers but to all ordered groups with zero that are
  also partially ordered.
  The hypothesis \lstinline[language=lean]|hnum_nonneg| proves that the numerator
  is non-negative by working with the coerced expressions in
  \lstinline[language=lean]|ℚ|. It uses \lstinline[language=lean]|add_nonneg| and
  \lstinline[language=lean]|mul_nonneg|, which are general theorems that work for
  any ordered additive commutative monoid and ordered semiring, respectively.
  The actual reasoning is done using integer-related theorems
  (via \lstinline[language=lean]|Int.cast_nonneg|) for the numerators and natural
  number theorems (via \lstinline[language=lean]|Nat.cast_nonneg|) for the denominators.
  The hypothesis \lstinline[language=lean]|hden_nonneg| proves that the
  denominator is non-negative by working entirely with natural numbers.
  We use the rewrite \lstinline[language=lean]|rw [← Nat.cast_mul]|,
  which moves the coercion (in this case from \lstinline[language=lean]|ℕ|
  to \lstinline[language=lean]|ℚ|) inside the multiplication:
  \lstinline[language=lean]|↑(m * n) = ↑m * ↑n|. The \lstinline[language=lean]|←|
  symbol means that we want the transformation from right to
  left (i.e., we apply the equality in reverse to move the cast inward).
  Type casts and coercions require these kinds of rewrite rules, not only
  for multiplication but also for addition and other operations, and
  similarly for \lstinline[language=lean]|ℤ| or other numerical types.
  These lemmas, such as \lstinline[language=lean]|Nat.cast_mul|,
  \lstinline[language=lean]|Int.cast_add|, etc., ensure that algebraic operations
  commute with type coercions.
\end{example}
\subsection{Coercions and Type Casting}
We extensively used type casting and coercions in this proof, which requires some
explanation \cite{lewis_madelaine_simplifying_casts_coercions_2020}.
Lean's type system lacks subtyping, means that types
like \lstinline[language=lean]|ℕ|, \lstinline[language=lean]|ℤ|, and \lstinline[language=lean]|ℚ|
are distinct and do not have a subtype relationship. In order to translate between these types,
we need to use explicit type casts or rely on automatic coercions. For example,
natural numbers (\lstinline[language=lean]|ℕ|) can be coerced to integers (\lstinline[language=lean]|ℤ|),
and integers can be coerced to rational numbers (\lstinline[language=lean]|ℚ|).
Casting and coercion are related but distinct concepts:
\begin{itemize}
  \item \textbf{Casting} refers to the explicit conversion of a value from one type to another,
        typically using functions like \lstinline[language=lean]|Int.cast| or \lstinline[language=lean]|Nat.cast|.
        These functions have accompanying lemmas that preserve properties across type conversions,
        such as \lstinline[language=lean]|Int.cast_lt| and \lstinline[language=lean]|Nat.cast_pos|.
  \item \textbf{Coercion}, on the other hand, is a more general mechanism that allows
        Lean to automatically convert between types when needed.
        More generally, in expressions like \lstinline[language=lean]|x + y| where \lstinline[language=lean]|x|
        and \lstinline[language=lean]|y| are of different types,
        Lean will automatically coerce them to a common type. For example, if \lstinline[language=lean]|x : ℕ|
        and \lstinline[language=lean]|y : ℤ|, then \lstinline[language=lean]|x|
        will be coerced to \lstinline[language=lean]|ℤ|.
\end{itemize}
The notation \lstinline[language=lean]|↑| denotes an explicit coercion
(in between cast and coercion).
To illustrate the expected behavior of coercion simplification, consider
the expression \lstinline[language=lean]|↑m + ↑n < (10 : ℤ)|,
where \lstinline[language=lean]|m, n : ℕ| are cast to \lstinline[language=lean]|ℤ|.
The expected normal form is \lstinline[language=lean]|m + n < (10 : ℕ)|,
since \lstinline[language=lean]|+|,
<,
and the numeral \lstinline[language=lean]|10| are polymorphic
(i.e., they can work with any numerical type such as \lstinline[language=lean]|ℤ|
or \lstinline[language=lean]|ℕ|). The simplification should proceed as follows:
\begin{enumerate}
  \item Replace the numeral on the right with the cast of a natural number:
        \lstinline[language=lean]|↑m + ↑n < ↑(10 : ℕ)|
  \item Factor \lstinline[language=lean]|↑| to the outside on the left:
        \lstinline[language=lean]|↑(m + n) < ↑(10 : ℕ)|
  \item Eliminate both casts to obtain an inequality over \lstinline[language=lean]|ℕ|:
        \lstinline[language=lean]|m + n < (10 : ℕ)|
\end{enumerate}
Lean provides tactics like \lstinline[language=lean]|norm_cast|
to simplify expressions involving such coercions.
The \lstinline[language=lean]|norm_cast| tactic normalizes casts
by pushing them outward and eliminating redundant coercions, often simplifying
proofs significantly by reducing goals to their ``native'' types.

\subsection{Type Classes and Algebraic Hierarchy in Lean}

In our proof of \lstinline[language=lean]|rat_add_nonneg|,
we used many generalized lemmas from Mathlib, such as \lstinline[language=lean]|add_nonneg|,
\lstinline[language=lean]|mul_nonneg|, and \lstinline[language=lean]|div_nonneg|,
which apply to a wide range of types beyond just rational numbers.
Similarly, in our earlier work with natural numbers, we used \lstinline[language=lean]|Nat.le_trans|,
a theorem specifically for natural numbers that is part of Lean's core library
(\lstinline[language=lean]|lean/Init/Prelude.lean|). Mathlib is built on top of this base library.
However, the transitivity property holds not only for naturals but also for integers,
reals, and, in fact, for any partially ordered set.
Rather than duplicating this theorem for each type, Mathlib provides a general
lemma \lstinline[language=lean]|le_trans| that works for any type \lstinline[language=lean]|α|
endowed with a partial ordering.
This is achieved through \textbf{type classes}, Lean's mechanism for defining and working with
abstract algebraic structures in an ad hoc polymorphic manner.
Type classes provide a powerful and flexible way to specify properties and operations that can be
shared across different types, thereby enabling polymorphism and code reuse. Ad hoc polymorphism
arises when a function or operator is defined over several distinct types, with behavior that varies
depending on the type.
A standard example \cite{wadler_blott_ad_hoc_polymorphism_1988} is overloaded multiplication:
the same symbol \lstinline[language=lean]|*| denotes multiplication of integers
(e.g., \lstinline[language=lean]|3 * 3|) and of floating-point numbers
(e.g., \lstinline[language=lean]|3.14 * 3.14|).
By contrast, parametric polymorphism occurs when a function is defined over a range of types
but acts uniformly on each of them. For instance, the \lstinline[language=lean]|List.length|
function applies in the same way to a list of integers and to a list of floating-point numbers.

Under the hood, a type class is a structure.
An important aspect of structures, and hence type classes, is that they support hierarchy
and composition through inheritance. For example, mathematically, a monoid is a semigroup
with an identity element, and a group is a monoid with inverses. In Lean, we can express this
by defining a \lstinline[language=lean]|Monoid| structure that extends the
\lstinline[language=lean]|Semigroup| structure, and a
\lstinline[language=lean]|Group| structure that extends the
\lstinline[language=lean]|Monoid| structure using the
\lstinline[language=lean]|extends| keyword:
\begin{lstlisting}[language=lean]
-- A semigroup has an associative binary operation
structure Semigroup (α : Type*) where
  mul : α → α → α
  mul_assoc : ∀ a b c : α, mul (mul a b) c = mul a (mul b c)

-- A monoid extends semigroup with an identity element  
structure Monoid (α : Type*) extends Semigroup α where
  one : α
  one_mul : ∀ a : α, mul one a = a
  mul_one : ∀ a : α, mul a one = a

-- A group extends monoid with inverses
structure Group (α : Type*) extends Monoid α where
  inv : α → α
  mul_left_inv : ∀ a : α, mul (inv a) a = one
\end{lstlisting}
The symbol \lstinline[language=lean]|*| in \lstinline[language=lean]|(α : Type*)|
indicates a universe variable (we will discuss universes later). Sometimes,
to avoid inconsistencies between types (such as Girard's paradox),
universes must be specified explicitly. This is an example of universe polymorphism.
Thus we have now seen all the polymorphism flavors in Lean: parametric, ad hoc, and universe polymorphism.

Type classes are defined using the \lstinline[language=lean]|class| keyword,
which is syntactic sugar for defining a structure. Thus, the previous example
can be rewritten using type classes:
\newpage
\begin{lstlisting}[language=lean]
-- A semigroup has an associative binary operation
class Semigroup (α : Type*) where
  mul : α → α → α
  mul_assoc : ∀ a b c : α, mul (mul a b) c = mul a (mul b c)

-- A monoid extends semigroup with an identity element  
class Monoid (α : Type*) extends Semigroup α where
  one : α
  one_mul : ∀ a : α, mul one a = a
  mul_one : ∀ a : α, mul a one = a

-- A group extends monoid with inverses
class Group (α : Type*) extends Monoid α where
  inv : α → α
  mul_left_inv : ∀ a : α, mul (inv a) a = one
\end{lstlisting}
The main difference is that type classes support \textbf{instance resolution}.
We use the keyword \lstinline[language=lean]|instance| to declare that a particular type is an
instance of a type class, which inherits the properties and operations defined in the type class.
Instances can be automatically inferred by Lean's type inference system,
allowing for concise and expressive code.
For example, we can declare that \lstinline[language=lean]|ℤ| is a group under addition:
\begin{lstlisting}[language=lean]
instance : Group ℤ where
  mul := Int.add
  one := 0
  inv := Int.neg
  mul_assoc := Int.add_assoc
  one_mul := Int.zero_add
  mul_one := Int.add_zero
  mul_left_inv := Int.neg_add_cancel
\end{lstlisting}
Now, any theorem proven for an arbitrary \lstinline[language=lean]|Group α|
automatically applies to \lstinline[language=lean]|ℤ| without any additional work.
This mechanism is particularly useful for defining and working with order structures
like preorders and partial orders. Mathematically, a preorder consists
of a set \lstinline[language=lean]|P| and a binary relation \lstinline[language=lean]|≤|
on \lstinline[language=lean]|P| that is reflexive and transitive \cite{mathinlean}.
\newpage
\begin{lstlisting}[language=lean, caption=Preorder Type Class in Lean]
-- A preorder is a reflexive, transitive relation `≤` with `<` defined in terms of `≤`
class Preorder (α : Type*) extends LE α, LT α where
  le_refl : ∀ a : α, a ≤ a
  le_trans : ∀ a b c : α, a ≤ b → b ≤ c → a ≤ c
  lt := fun a b => a ≤ b ∧ ¬ b ≤ a
  lt_iff_le_not_ge : ∀ a b : α, a < b ↔ a ≤ b ∧ ¬ b ≤ a := by intros; rfl

instance [Preorder α] : Lean.Grind.Preorder α where
  le_refl := Preorder.le_refl
  le_trans := Preorder.le_trans _ _ _
  lt_iff_le_not_le := Preorder.lt_iff_le_not_ge _ _
\end{lstlisting}

The \lstinline[language=lean]|class Preorder| declares a type class over a type
\lstinline[language=lean]|α|, bundling the \lstinline[language=lean]|≤|
and \lstinline[language=lean]|<| relations
(inherited via \lstinline[language=lean]|extends LE α, LT α|)
with the preorder axioms: reflexivity (\lstinline[language=lean]|le_refl|)
and transitivity (\lstinline[language=lean]|le_trans|).
The field \lstinline[language=lean]|lt| provides a default definition of strict
inequality in terms of \lstinline[language=lean]|≤|,
and the theorem \lstinline[language=lean]|lt_iff_le_not_ge| characterizes this relationship,
proved automatically via reflexivity (\lstinline[language=lean]|by intros; rfl|).
The \lstinline[language=lean]|instance| declaration connects the \lstinline[language=lean]|Preorder|
class to Lean's \lstinline[language=lean]|Grind| tactic automation, which allows automatic
reasoning with preorder properties during proof search.
Returning to our rational number proof, this explains why lemmas
like \lstinline[language=lean]|add_nonneg| and \lstinline[language=lean]|mul_nonneg|
work seamlessly: \lstinline[language=lean]|ℚ| is an instance of
\lstinline[language=lean]|OrderedSemiring|, which extends
\lstinline[language=lean]|Preorder| and other algebraic structures,
automatically providing all their theorems.

\section{Example in Topology: A Connected but not Path-Connected Space}

As part of my thesis work, with the help and revision from Prof David Loeffler, 
I have formalized a well-known counterexample in topology: the \textbf{topologist’s sine curve}.
This classic example illustrates a space that is \emph{connected} but not \emph{path-connected}.
It demonstrates the use of type classes, structures, and several tactics in Lean that were discussed earlier.
In my presentation, I will provide a high-level overview along with some specific examples, 
of my original code, wich is way longer than the final result.
The proof follows Conrad’s paper (\cite{Conrad_connnotpathconn}), with a few modifications, and the formalization is available in the Mathlib documentation under \href{https://leanprover-community.github.io/mathlib4_docs/Counterexamples/TopologistsSineCurve.html}{\emph{Counterexamples – Topologist’s Sine Curve}}.

The topologist's sine curve is defined as the graph of $y = \sin(1/x)$ for $x \in (0, 1]$,
together with the vertical line segment $\{0\} \times [-1, 1]$.
We define three sets in $\mathbb{R}^2$:
\begin{itemize}
  \item $S$ the oscillating curve $\{(x, \sin(1/x)) : x > 0\}$
  \item $Z$ the vertical segment $\{(0, y) : y \in [-1, 1]\}$   
  \item $T$ their union $S \cup Z$
\end{itemize}
In Lean goes as follows:

\begin{lstlisting}[language=lean]
  def S : Set (ℝ × ℝ) := (fun x ↦ (x, sin x⁻¹)) '' Ioi 0
  def Z : Set (ℝ × ℝ) := (fun y ↦ (0, y)) '' Icc (-1) 1
  def T : Set (ℝ × ℝ) := S ∪ Z
\end{lstlisting}

\lstinline[language=lean]|Set| is the type of sets, defined as predicates
(i.e., functions from a type to \lstinline[language=lean]|Prop|).
The notation \lstinline[language=lean]|''| denotes the image of a set under a function,
and \lstinline[language=lean]|Ioi 0| and \lstinline[language=lean]|Icc (-1) 1|
denote the intervals $(0, \infty)$ and $[-1, 1]$, respectively.
The sets are subsets of the product space $\mathbb{R}^2$,
 represented as \lstinline[language=lean]|ℝ × ℝ|.
The sin function \lstinline[language=lean]|sin| is defined in the 
\lstinline[language=lean]|Analysis.SpecialFunctions.Trigonometric|
The goal is to prove that $T$ is connected but not path-connected.

First of all one can directlly see that  $S$ is path-connected, since it is the
image of the path-connected set ($(0, \infty)$) under the continuous map
$x \mapsto (x, \sin(1/x))$. 
A path connected space is always connected, so $S$ is connected.
Morover the closure of $S$ is $T$, and the closure of a connected set is connected,
so $T$ is connected.
This is how a matehamtician would argue informally, using facts known. 
Anyway in a formal proof one has to justify each step.

For instance justiyging that $S$ is path-connected requires to prove that the map
$x \mapsto (x, \sin(1/x))$ is continuous on $(0, \infty)$.
This is done in two steps: first proving that $x \mapsto \sin(1/x)$ is continuous on $(0, \infty)$,
then using the fact that the product of two continuous functions is continuous.
Here is the relevant part of the formalization:
\begin{lstlisting}[language=lean]
lemma sine_curve_is_continuous_on_pos_real : ContinuousOn (sine_curve) (Ioi 0) :=
  ContinuousOn.prodMk continuous_id.continuousOn sin_comp_inv_is_continuous_on_pos_real

lemma pos_real_is_path_conn : IsPathConnected (Ioi 0) := by
  apply Convex.isPathConnected
  · exact convex_Ioi 0
  · use 1
    unfold pos_real
    simp

lemma S_is_path_conn : IsPathConnected S := by
  apply IsPathConnected.image'
  · exact pos_real_is_path_conn
  · exact sine_curve_is_continuous_on_pos_real
\end{lstlisting}



\begin{lstlisting}[language=lean]
lemma sine_curve_is_continuous_on_pos_real :
  ContinuousOn (sine_curve) (Ioi 0) :=
  ContinuousOn.prodMk continuous_id.continuousOn
    sin_comp_inv_is_continuous_on_pos_real
\end{lstlisting}
The lemma \lstinline[language=lean]|ContinuousOn.prodMk| states that the product of two continuous functions
is continuous, allowing us to combine the identity map $x \mapsto x$ and the function
$x \mapsto \sin(1/x)$.
The type class system in Lean supplies the required \lstinline[language=lean]|TopologicalSpace| instances
for $\mathbb{R}$ and $\mathbb{R} \times \mathbb{R}$, allowing \lstinline[language=lean]|ContinuousOn| to be used
without explicitly mentioning the topologies.

Let's digress briefly to examine the definition of \lstinline[language=lean]|ContinuousOn| in Mathlib:
\begin{lstlisting}[language=lean]
def ContinuousOn (f : X → Y) (s : Set X) : Prop :=
  ∀ x ∈ s, ContinuousAt f x
\end{lstlisting}
This definition states that a function \lstinline[language=lean]|f : X → Y| is continuous on a set
\lstinline[language=lean]|s : Set X| if it is continuous at every point in \lstinline[language=lean]|s|.
The predicate \lstinline[language=lean]|ContinuousAt f x| 
means that \lstinline[language=lean]|f| is continuous at the point \lstinline[language=lean]|x : X|.

% \subsection{Path-connectedness of the positive reals}

\begin{lstlisting}[language=lean]
lemma pos_real_is_path_conn : IsPathConnected (Ioi 0) := by
  apply Convex.isPathConnected
  · exact convex_Ioi 0
  · use 1
    unfold pos_real
    simp
\end{lstlisting}

This lemma proves that $(0, \infty)$ is path-connected. The theorem
\lstinline[language=lean]|Convex.isPathConnected| in \lstinline[language=lean]|mathlib| 
states that every nonempty convex
subset of a real vector space is path-connected. We provide two arguments:
\begin{enumerate}
  \item \lstinline[language=lean]|convex_Ioi 0| proves that $(0, \infty)$ is convex;
  \item \lstinline[language=lean]|use 1| provides an explicit point showing nonemptiness.
\end{enumerate}
The strucutre\lstinline[language=lean]|IsPathConnected| depends on a \lstinline[language=lean]|TopologicalSpace| 
instance,
which is also provided by Lean’s type class inference.

% \subsection{Path-connectedness of the sine curve}

\begin{lstlisting}[language=lean]
lemma S_is_path_conn : IsPathConnected S := by
  apply IsPathConnected.image'
  · exact pos_real_is_path_conn
  · exact sine_curve_is_continuous_on_pos_real
\end{lstlisting}

The theorem \lstinline[language=lean]|IsPathConnected.image'| states that if a set $A$ is path-connected
and $f$ is continuous on $A$, then $f''A$ (the image of $A$ under $f$) is path-connected.
Combining the previous two lemmas proves that the sine curve $S$ is path-connected.

\begin{lstlisting}[language=lean]
def IsPathConnected (F : Set X) : Prop :=
  ∃ x ∈ F, ∀ ⦃y⦄, y ∈ F → JoinedIn F x y
\end{lstlisting}

% This structure expresses that a subset $S$ of a topological space $X$ is path-connected.
% It has two fields:
% \begin{itemize}
%   \item \lstinline[language=lean]|nonempty : S.Nonempty|, asserting that $S$ is not empty;
%   \item \lstinline[language=lean]|joined : ∀ x y ∈ S, JoinedIn S x y|, meaning that every pair of points
%   in $S$ can be joined by a continuous path inside $S$.
% \end{itemize}

% The auxiliary predicate \lstinline[language=lean]|JoinedIn| is defined as:
% \begin{lstlisting}[language=lean]
% def JoinedIn (S : Set X) (x y : X) : Prop :=
%   ∃ γ : Path x y, ∀ t, γ t ∈ S
% \end{lstlisting}
% where \lstinline[language=lean]|Path x y| denotes a continuous map $\gamma : [0,1] \to X$ with
% $\gamma(0) = x$ and $\gamma(1) = y$.

% \subsection{Global version: \lstinline[language=lean]|PathConnectedSpace}|

There is also a global type class representing that an entire space is path-connected:

\begin{lstlisting}[language=lean]
class PathConnectedSpace (X : Type*) [TopologicalSpace X] : Prop :=
(to_isPathConnected_univ : IsPathConnected (Set.univ))
\end{lstlisting}

% Hence, a \lstinline[language=lean]|PathConnectedSpace| is a space for which
% \lstinline[language=lean]|IsPathConnected (Set.univ)| holds.

% \subsection{Summary}

% \begin{center}
% \begin{tikzpicture}[node distance=1.2cm, auto]
% \node (A) [draw, rounded corners, fill=gray!10, text width=6cm]
%   {TopologicalSpace $X$};
% \node (B) [below of=A, draw, rounded corners, fill=gray!10, text width=6.5cm]
%   {IsPathConnected $S$\\
%    $\quad\begin{cases}
%        \text{nonempty} : S.\text{Nonempty} \\
%        \text{joined} : \forall x,y\in S,\ \exists \gamma:[0,1]\to S
%      \end{cases}$};
% \node (C) [below of=B, draw, rounded corners, fill=gray!10, text width=6cm]
%   {PathConnectedSpace $X$\\
%    $\qquad\Longleftrightarrow\ \text{IsPathConnected }(\mathrm{univ})$};
% \draw[->] (A) -- (B);
% \draw[->] (B) -- (C);
% \end{tikzpicture}
% \end{center}

% Lean’s combination of \emph{structures} and \emph{type classes} allows for a highly
% modular and reusable formalization:
% \begin{itemize}
%   \item \lstinline[language=lean]|TopologicalSpace X| provides the ambient topology;
%   \item \lstinline[language=lean]|ContinuousOn f s| depends on that topology to define continuity;
%   \item \lstinline[language=lean]|Convex.isPathConnected| relies on algebraic and topological structure;
%   \item \lstinline[language=lean]|IsPathConnected.image'| uses continuity and topology to preserve
%   path-connectedness under images.
% \end{itemize}

% This design makes Lean’s topology library both flexible and highly compositional.





% Before presenting the formalization, we introduce the relevant topological concepts as they
% are implemented in Mathlib. Our development relies on several key modules:
% \begin{itemize}
%   \item \lstinline[language=lean]|Topology.Basic|
%         --- defines topological spaces and open sets
%   \item \lstinline[language=lean]|Topology.Connected.PathConnected|
%         --- defines connected and path-connected spaces
%   \item \lstinline[language=lean]|Topology.Instances.Real|
%         --- provides the standard topology on $\mathbb{R}$
%   \item \lstinline[language=lean]|Data.Set.Basic|
%         --- provides set-theoretic operations
%   \item \lstinline[language=lean]|Analysis.SpecialFunctions.Trigonometric|
%         --- defines sine and related functions
% \end{itemize}

% Topology in Lean is built upon the concept of open sets, defined
% using the \lstinline[language=lean]|IsOpen| predicate. A topological space is a type
% equipped with a collection of open sets satisfying the standard axioms.
% In Lean, this is represented using the \lstinline[language=lean]|TopologicalSpace| type class:
% \newpage
% \begin{lstlisting}[language=lean]
% class TopologicalSpace (α : Type*) where
%   IsOpen : Set α → Prop
%   isOpen_univ : IsOpen univ
%   isOpen_inter : ∀ s t, IsOpen s → IsOpen t → IsOpen (s ∩ t)
%   isOpen_sUnion : ∀ s, (∀ t ∈ s, IsOpen t) → IsOpen (sUnion s)
% \end{lstlisting}

% Key topological concepts used in our formalization include:

% \begin{description}
%   \item[Connectedness:] A space $X$ is connected if it cannot be written as the union of
%         two disjoint non-empty open sets. Equivalently, every continuous function from $X$
%         to a discrete space is constant. In Lean, this is formalized
%         as \lstinline[language=lean]|IsConnected : Set α → Prop|.
%   \item[Path-connectedness:] A space $X$ is path-connected if for any two points
%         $x, y \in X$, there exists a continuous path $\gamma : [0,1] \to X$
%         with $\gamma(0) = x$ and $\gamma(1) = y$. In Lean, the unit interval $[0,1]$
%         is represented as the type \lstinline[language=lean]|unitInterval|,
%         and path-connectedness is formalized as \lstinline[language=lean]|IsPathConnected : Set α → Prop|.
%   \item[Continuous functions:] A function $f : X \to Y$ between topological spaces
%         is continuous if the preimage of every open set is open. Lean provides
%         \lstinline[language=lean]|Continuous f : Prop|, as well as \lstinline[language=lean]|ContinuousAt|
%         and \lstinline[language=lean]|ContinuousOn| for local versions.
%   \item[Closure:] The closure of a set $S$ is the smallest closed set containing $S$,
%         equivalently the set of all limit points of $S$. In Lean, this is
%         \lstinline[language=lean]|closure : Set α → Set α|.
% \end{description}

% We note that path-connectedness implies connectedness (since $[0,1]$ is connected),
% but the converse is false, as our example demonstrates.

% \subsection{Structure of the Formalization}

% Our formalization proceeds in four main steps:

% \subsubsection{Step 1: Defining the Space}

% We define three sets in $\mathbb{R}^2$:
% \begin{itemize}
%   \item \lstinline[language=lean]|S| --- the oscillating curve $\{(x, \sin(1/x)) : x > 0\}$
%   \item \lstinline[language=lean]|Z| --- the vertical segment $\{(0, y) : y \in [-1, 1]\}$
%   \item \lstinline[language=lean]|T| --- their union $S \cup Z$
% \end{itemize}

% A key technical tool is a sequence \lstinline[language=lean]|x_seq : ℝ → ℕ → ℝ| that,
% for each $y \in [-1, 1]$, produces a sequence of $x$-values tending to $0$
% % x_{\text{seq}}(y, k) = \frac{1}{\arcsin(y) + (k+1) \cdot 2\pi}
% This sequence is crucial for proving that every point in $Z$ is a limit point of $S$.
% \subsubsection{Step 2: Proving T is Closed}
% We prove that \lstinline[language=lean]|closure S = T| by characterizing when
% a point $(x, y)$ is in the closure of $S$:
% \begin{itemize}
%   \item If $x > 0$, then $(x, y) \in \overline{S}$ iff $y = \sin(1/x)$, by continuity of sine.
%   \item If $x = 0$, then $(x, y) \in \overline{S}$ iff $y \in [-1, 1]$, using \lstinline[language=lean]|x_seq| to construct explicit sequences.
% \end{itemize}
% The proof uses Lean's sequential characterization of closure and demonstrates working
% with limits and continuous functions.
% \subsubsection{Step 3: Proving $T$ is Connected}

% We show $T$ is connected by observing that:
% \begin{enumerate}
%   \item $S$ is the continuous image of $(0, \infty)$ under the function $x \mapsto (x, \sin(1/x))$.
%   \item $(0, \infty)$ is connected (as a connected subset of $\mathbb{R}$).
%   \item The continuous image of a connected space is connected.
%   \item The closure of a connected set is connected.
% \end{enumerate}

% This proof is relatively short, leveraging Mathlib's general theorems about connected spaces.

% \subsubsection{Step 4: Proving $T$ is Not Path-Connected (Main Result)}

% This is the most technically involved part. We prove by contradiction that there is no continuous path from $(0, 0) \in Z$ to $(1, \sin(1)) \in S$. The proof strategy is:

% \begin{enumerate}
%   \item \textbf{Assume a path exists:} Suppose $p : [0, 1] \to T$ is continuous with $p(0) = (0, 0)$ and $p(1) = (1, \sin(1))$.

%   \item \textbf{Find the last time on the $y$-axis:} Let $t_0 = \sup\{t : p(t)_x = 0\}$ be the supremum of times when the path is on the $y$-axis. By closedness, $p(t_0)_x = 0$.

%   \item \textbf{Use local continuity:} By continuity at $t_0$, there exists $\delta > 0$ such that $\|p(t) - p(t_0)\| < 1$ for all $t$ with $|t - t_0| < \delta$.

%   \item \textbf{Find a nearby point off the axis:} Choose $t_1 \in (t_0, t_0 + \delta)$, and let $a = p(t_1)_x > 0$. By connectedness of $[t_0, t_1]$, the $x$-coordinates of $p([t_0, t_1])$ cover the interval $[0, a]$.

%   \item \textbf{Exploit the oscillation:} For any $y \in [-1, 1]$, we can find $x \in (0, a]$ with $\sin(1/x) = y$ (using the oscillation property). Since $p$ covers $[0, a]$ in $x$-coordinates and $p(t) \in T$ for all $t$, there must be times $t_+$ and $t_-$ near $t_0$ with $p(t_+)_y = 1$ and $p(t_-)_y = -1$.

%   \item \textbf{Derive the contradiction:} Both $p(t_+)$ and $p(t_-)$ are within distance $1$ of $p(t_0)$, so they are within distance $< 2$ of each other by the triangle inequality. But $|1 - (-1)| = 2$, giving the required contradiction.
% \end{enumerate}

% This proof demonstrates several advanced techniques:
% \begin{itemize}
%   \item Working with suprema and using closedness to establish membership
%   \item Using the $\varepsilon$-$\delta$ definition of continuity via Mathlib's \lstinline[language=lean]|Metric.tendsto_nhds|
%   \item Applying the intermediate value theorem for connected sets (preconnectedness)
%   \item Careful manipulation of distances and inequalities
%   \item Case analysis on set membership using Lean's pattern matching
% \end{itemize}

% In the following sections, we present each part of the formalization in detail, explaining the key lemmas, tactics, and design decisions.


%

% \section{Types and Terms and Judgments}
We now delve into the study of type theory. We write $a : A$ to mean that $a$ is a term of type $A$, where the 
colon signifies a kind of membership relation.
Types and terms can be understood through a set-theoretic interpretation. In this view, a type is analogous to 
a set, it classifies a collection of mathematical objects (terms) that belong to it. For example, the 
type $\mathbb{N}$ classifies terms like $0, 1, 2, \ldots$.

From a logical perspective, inspired by the Curry–Howard correspondence, a type can also be seen as a proposition, 
and the terms inhabiting the type are proofs of that proposition. For instance, the type $P \to Q$ (implication) 
is inhabited by functions that transform proofs of $P$ into proofs of $Q$.

Finally, a type specifies the shape of data structures or functions, enforcing which operations are valid. 
You might have worked with programming languages with type systems such where in order to use some 
primitive data structure such as string or int or float ecc ecc.
The type system is used by the compiler to be bale to infer the tyope of the varibale and allow ceratin operations 
and throwing errors in case this operation are not allowes on that specific tyope.
For instance in C Typescripot...
yuou have to specify it's type.


Lean has an extremely powerfull type system indeed and derived by type theory
\begin{example}
    \begin{lstlisting}[language=Lean]
    
    def m : Nat := 1       -- m is a natural number
    def n : Nat := 0
    
    #check m            -- output: Nat
    #check n
    
    \end{lstlisting}
\end{example}

A type judgment has the form $\Gamma \vdash t : T$,
 meaning: under \textbf{context} $\Gamma$ (a list of typed variables), the term $t$ has type $T$.
% \begin{example}
% We write
% $$\Gamma \vdash e : \tau$$
% to denote ``$e$ has type $\tau$ in context $\Gamma$''.
% \end{example}
Analogous to deduction rules, \textbf{type judgments} are formal assertions about the validity of types and terms.
\begin{example}
    \[
\begin{array}{c}
\Gamma \vdash t : A \qquad \Gamma \vdash A = B \\ [1ex]
\hline
\Gamma \vdash t : B \quad \text{(Type Equality)}
\end{array}
\]
\end{example}
\begin{example}
The Lean kernel uses type judgments.
    \begin{lstlisting}[language=Lean]
#check 2 + 2  -- ⊢ 2 + 2 : Nat (Judgment: This term has type Nat)
#eval 2 + 2   -- ⊢ 4 : Nat      (Judgment: Computes to 4, which is a Nat)

\end{lstlisting}
\end{example}


Using formal inference rules in the type judgment system, such as \textbf{introduction} and \textbf{elimination} 
rules, we can construct new types from existing ones. In addition to these, 
we also require \textbf{formation rules}, which specify when and how a type is well-formed, 
and \textbf{computation rules}, which describe how terms can be \textbf{normalized} (to be described later). 
(INTRODUCE SIMPLY TYPED LAMBDA CALCULUS HERE WITH REDUCTION STRONG NORMLIZATION ADN FIRST LOOK AT SIMPLE TYPE THEORY)

% \section{Function Types}

Given types $A$ and $B$, we can construct the function type $A \to B$, representing functions from $A$ to $B$. Functions are a primitive concept in type theory, and we provide a brief introduction here.

We can \textbf{apply} a function $f : A \to B$ to an element $a : A$ to obtain an element of $B$, denoted $f(a)$. In type theory, it is common to omit the parentheses and write the application simply as $f\, a$.

There are two equivalent ways to construct function types: either by direct definition or by using $\lambda$-abstraction. Introducing a function by definition means that we introduce a function by giving it a name (let's say, $f$) and saying we define $f : A \to B$ by giving an equation
\begin{equation} \label{lambda_definition}
f(x) \coloneqq \Phi 
\end{equation}
where $x$ is a variable and $\Phi$ is an expression which may use $x$. In order for this to be valid, we have to check that $\Phi : B$ assuming $x : A$.

Now we can compute $f(a)$ by replacing the variable $x$ in $\Phi$ with $a$. As an example, consider the function $f : \mathbb{N} \to \mathbb{N}$ which is defined by $f(x) \coloneqq x + x$. Then $f(2)$ is \emph{judgmentally equal} to $2 + 2$.

If we don't want to introduce a name for the function, we can use \textbf{$\lambda$-abstraction}. 
Given an expression $\Phi$ of type $B$ which may use $x : A$, as above, we write $\lambda(x : A). \Phi$ to 
indicate the same function defined by (\ref{lambda_definition}). Thus, we have
$$ (\lambda(x : A). \Phi) : A \to B. $$

\begin{example}
The previously defined function  has the typing judgment
$$ (\lambda(x : \mathbb{N}). x + x) : \mathbb{N} \to \mathbb{N}. $$

As another example, for any types $A$ and $B$ and any element $y : B$, we have a \textbf{constant function}
$$ (\lambda(x : A). y) : A \to B. $$

The \textbf{identity function} on any type $A$ is given by
$$ (\lambda(x : A). x) : A \to A. $$
\end{example}

By convention, the ``scope'' of the variable binding ``$\lambda x.$'' is the entire rest of the expression, unless delimited with parentheses. Thus, for instance, $\lambda x. x + x$ should be parsed as $\lambda x.(x + x)$, not as $(\lambda x. x) + x$ .


Now a $\lambda$-abstraction is a function, so we can apply it to an argument $a : A$. We then have the following computation rule ($\beta$-reduction), which is a \textbf{definitional equality}:
$$ (\lambda x. \Phi)(a) \equiv \Phi' $$ 
where $\Phi'$ is the expression $\Phi$ in which all occurrences of $x$ have been replaced by $a$.
\begin{example}
    Continuing the above example, we have
    $$ (\lambda x. x + x)(2) \equiv 2 + 2. $$
\end{example}

Note that from any function $f : A \to B$, we can construct a lambda abstraction function $\lambda x. f(x)$. 
Since this is by definition ``the function that applies $f$ to its argument'' we consider it to be definitionally
 equal to $f$ (\textbf{$\eta$-conversion}):
$$ f \equiv (\lambda x. f(x)). $$

This equality is the uniqueness principle for function types, because it shows that $f$ is uniquely determined by its values.

The introduction of functions by definitions with explicit parameters can be reduced to simple definitions by using $\lambda$-abstraction: i.e., we can read a definition of $f : A \to B$ by
$$ f(x) \coloneqq \Phi 
$$ 
as
$$ f \coloneqq \lambda x. \Phi. $$

% When doing calculations involving variables, we have to be careful when replacing a variable with an expression that also involves variables, because we want to preserve the binding structure of expressions. As an example, consider $f : \mathbb{N} \to (\mathbb{N} \to \mathbb{N})$ defined as
% $$  f(x) \coloneqq \lambda y. x + y. $$

% Now if we have assumed somewhere that $y : \mathbb{N}$, then what is $f(y)$? It would be wrong to just naively replace $x$ by $y$ everywhere in the expression ``$\lambda y. x + y$'' defining $f(x)$, obtaining $\lambda y. y + y$, because this means that $y$ gets captured. Previously, the substituted $y$ was referring to our assumption, but now it is referring to the argument of the $\lambda$-abstraction. Hence, this naive substitution would destroy the binding structure, allowing us to perform calculations which are semantically unsound.

% But what is $f(y)$ in this example? Note that bound variables such as $y$ in the expression $\lambda y. x + y$ have only a local meaning, and can be consistently replaced by any other variable, preserving the binding structure. Indeed, $\lambda y. x + y$ is declared to be judgmentally equal (\textbf{\alpha-conversion}) to $\lambda z. x + z$. It follows that $f(y)$ is judgmentally equal to $\lambda z. y + z$, and that answers our question.

% Of course, this should all be familiar to any mathematician: it is the same phenomenon as the fact that if $f(x) \coloneqq \int_1^2 \frac{dt}{x-t}$, then $f(t)$ is not $\int_1^2 \frac{dt}{t-t}$ but rather $\int_1^2 \frac{ds}{t-s}$. A $\lambda$-abstraction binds a dummy variable in exactly the same way that an integral does.

% We have seen how to define functions in one variable. One way to define functions in several variables is called currying (after the mathematician Haskell Curry).

% The idea of currying is to represent a function of two inputs $a : A$ and $b : B$ as a function which takes one input $a : A$ and returns another function, which then takes a second input $b : B$ and returns the result. That is, we consider two-variable functions to belong to an iterated function type, $f : A \to (B \to C)$. We may also write this without the parentheses, as $f : A \to B \to C$, with associativity to the right as the default convention. Then given $a : A$ and $b : B$, we can apply $f$ to $a$ and then apply the result to $b$, obtaining $f(a)(b) : C$. To avoid the proliferation of parentheses, we allow ourselves to write $f(a)(b)$ as $f(a, b)$ even though there are no products involved. When omitting parentheses around function arguments entirely, we write $f \, a \, b$ for $(f \, a) \, b$, with the default associativity now being to the left so that $f$ is applied to its arguments in the correct order.

% Our notation for definitions with explicit parameters extends to this situation: we can define a named function $f : A \to B \to C$ by giving an equation
% $$  f(x, y) \coloneqq \Phi $$
% where $\Phi : C$ assuming $x : A$ and $y : B$. Using $\lambda$-abstraction this corresponds to
% $$ f \coloneqq \lambda x. \lambda y. \Phi, $$
% which may also be written as
% $$ f \coloneqq x \mapsto y \mapsto \Phi. $$


When performing calculations involving variables, we must carefully preserve the \textbf{binding structure} of expressions during substitution. Consider the function $f : \mathbb{N} \to (\mathbb{N} \to \mathbb{N})$ defined as:
$$ f(x) \coloneqq \lambda y. x + y $$

Suppose we have assumed $y : \mathbb{N}$ somewhere in our context. What is $f(y)$? 

A naive approach would replace $x$ with $y$ directly in the expression $\lambda y. x + y$, yielding $\lambda y. y + y$. However, this substitution is \textbf{semantically incorrect} because it causes \textbf{variable capture}: the free variable $y$ (referring to our assumption) becomes bound by the $\lambda$-abstraction, fundamentally altering the expression's meaning.

The correct approach uses \textbf{$\alpha$-conversion} (variable renaming). 
Since bound variables have only local scope, we can consistently rename them while preserving binding structure. 
The expression $\lambda y. x + y$ is judgmentally equal to $\lambda z. x + z$ for any fresh variable $z$. Therefore:
$$ f(y) \equiv \lambda z. y + z $$

This phenomenon parallels familiar mathematical practice: if $f(x) \coloneqq \int_1^2 \frac{dt}{x-t}$, 
then $f(t)$ equals $\int_1^2 \frac{ds}{t-s}$, not the ill-defined $\int_1^2 \frac{dt}{t-t}$. 
Lambda abstractions bind dummy variables exactly as integrals do.

For functions of multiple variables, we employ \textbf{currying} (named after mathematician Haskell Curry). Instead of using product types, we represent a two-argument function as a function returning another function.

A function taking inputs $a : A$ and $b : B$ to produce output in $C$ has type:
$$ f : A \to (B \to C) \equiv A \to B \to C $$
where the arrow associates to the right by convention.

Given $a : A$ and $b : B$, we apply $f$ sequentially: first to $a$, then the result to $b$, obtaining $f(a)(b) : C$.

To simplify notation and avoid excessive parentheses, we adopt several conventions. We write $f(a)(b)$ as $f(a, b)$ for abbreviated application. Without parentheses entirely, $f \, a \, b$ means $(f \, a) \, b$ following left-associative application. For multi-parameter definitions, we write $f(x, y) \coloneqq \Phi$ where $\Phi : C$ under assumptions $x : A$ and $y : B$.

Using $\lambda$-abstraction, such definitions correspond to:
$$ f \coloneqq \lambda x. \lambda y. \Phi $$ 

Alternative notation using map symbols:
$$ f \coloneqq x \mapsto y \mapsto \Phi $$

This currying approach extends naturally to functions of three or more arguments, allowing us to represent any multi-argument function as a sequence of single-argument functions.
\begin{example}[Functions in Lean]
    \begin{lstlisting}[language=Lean]
-- A simple function
def double : Nat -> Nat := x + x

#eval double 3     -- Output: 6

\end{lstlisting}
in steps
\begin{align}
  double 3 \\
  &\equiv (x + x) [x:= 3] \text{substitution}\\ 
  &\equiv 3 + 3 \\
  &\equiv 6
\end{align}
\end{example}
\begin{example}[Functions in Lean using lmabda ]
    \begin{lstlisting}[language=Lean]
-- Lambda abstraction form
def double' : Nat -> Nat := fun x => x + x
-- or Equivalently
def double' := (\lambda x : Nat => x + x)

#eval double' 3     -- Output: 6

\end{lstlisting}
in steps
\begin{align}
  double 3 \\
  &\equiv (\lambda x : Nat, x + x)(3) \text{Beta reduction} \\
  &\equiv (x + x) [x:= 3] \text{substitution}\\ 
  &\equiv 3 + 3 \\
  &\equiv 6
\end{align}
\end{example}
\begin{example}[Currying in Lean]
    \begin{lstlisting}[language=Lean]
def add : Nat -> (Nat -> Nat) := fun x => (fun y => x + y)
-- or equivalently (rightassociative notation)
def add : Nat -> Nat -> Nat := fun x => fun y => x + y
#eval add 3 4   -- Output: 7

\end{lstlisting}
in steps
\begin{align}
  add 3 4  \\
  &\equiv (add 3) 4  \\
  &\equiv (\lambda x : Nat, x + x)(3) \text{Beta reduction} \\
  &\equiv (x + x) [x:= 3] \text{substitution}\\ 
  &\equiv 3 + 3 \\
  &\equiv 6
\end{align}
\end{example}
\begin{example}[Higher order Function in Lean]
    \begin{lstlisting}[language=Lean]
def HAdd : (Nat -> Nat) -> Nat := h => h 3 + h 4

#eval HAdd add -- Output: 14

\end{lstlisting}
in steps
\begin{align}
  add 3 4  \\
  &\equiv (add 3) 4  \\
  &\equiv (\lambda x : Nat, x + x)(3) \text{Beta reduction} \\
  &\equiv (x + x) [x:= 3] \text{substitution}\\ 
  &\equiv 3 + 3 \\
  &\equiv 6
\end{align}
\end{example}
\begin{example}
  [Copmposing Functions in Lean]
    \begin{lstlisting}[language=Lean]
def compose : (B -> C) -> (A -> B) -> (A -> C) :=
  fun f g => fun x => g (f x) --  left associative 
#eval compose add 3 -- Output: 12
\end{lstlisting}
\end{example}
\begin{example}
  [Combinators in Lean]
    \begin{lstlisting}[language=Lean]
def K : A -> (B -> A) := fun x => fun _ => x  
def S : (A -> B -> C) -> (A -> B) -> (A -> C) :=
  fun f g => fun x => f x (g x)

def Y : (A -> A) -> A := 
  fun f => (fix Y x => f (Y f x)) -- Y combinator Not strongly normalizing
#eval K 5 true -- Output: 5 
#eval S add 3 4 -- Output: 7
\end{lstlisting}
\end{example}
We now examine the inference rules for function types.
\paragraph{Formation Rule}
\[
  \frac{A : \mathsf{Type} \quad B : \mathsf{Type}}{A \to B : \mathsf{Type}}
\]

\paragraph{Introduction Rule (lambda abstraction)}
\[
  \frac{x : A \vdash \Phi : B}{\lambda x.\Phi : A \to B}
\]

\paragraph{Elimination Rule (application)}
\[
  \frac{f : A \to B \quad a : A}{f(a) : B}
\]

\paragraph{Computation Rule (beta reduction)}
\[
  (\lambda x. \Phi)(a) \equiv \Phi[x := a]
\]

\paragraph{Uniqueness Principle (eta conversion) or equality}
\[
  f \equiv \lambda x. f(x)
\]
\begin{notation}
    $[x := a]$ stands for substitution.
    % or as explained before $\alpha$-conversion.
\end{notation}
Strong normalization Church Rossen theorem states that every term in the simply typed lambda calculus 
reduces to a normal form, meaning that it cannot be reduced further. 
This is a crucial property of the type system, ensuring that every well-typed term has a unique normal form.

POLYMORPHISM HERE?
\begin{example}[Polmorhic Function in Lean]
    \begin{lstlisting}[language=Lean] 
def id : {A : Type} -> A -> A := fun {A} x => x 
#eval id 5         -- Output: 5
#eval id true      -- Output: true  
#eval id "Hello"   -- Output: "Hello"
\end{lstlisting}
\end{example}

% \section{Product Types}

As another fundamental example $A \times B$ denotes the type of pairs $(A,B)$, called \textbf{product type}.

\paragraph{Formation Rule}
\[
  \frac{A : \mathsf{Type} \quad B : \mathsf{Type}}{A \times B : \mathsf{Type}}
\]

\paragraph{Introduction Rule (pairing)}
\[
  \frac{a : A \quad b : B}{(a, b) : A \times B}
\]

\paragraph{Elimination Rules (projections)}
\[
  \frac{p : A \times B}{\mathsf{fst}(p) : A}
  \qquad
  \frac{p : A \times B}{\mathsf{snd}(p) : B}
\]

\paragraph{Computation Rules}
\[
  \mathsf{fst}(a, b) \equiv a
  \qquad
  \mathsf{snd}(a, b) \equiv b
\]
\begin{notation}
    $\mathsf{fst}$ and $\mathsf{snd}$ represent the projections to the first and second components of a pair $(a, b)$, respectively.  
    In Lean, these can be accessed using \textbf{(a, b).1} and \textbf{(a, b).2}.
\end{notation}
\begin{example}
    
\end{example}
\begin{lstlisting}[language=Lean]
def myPair : Nat \times Bool := (3, true)

#eval myPair.1     -- Output: 3
#eval myPair.2     -- Output: true

\end{lstlisting}\
constructior in lean as copatternmatching 
https://brandonrozek.com/blog/lean4-tutorial/

diffretn way to definiing pairs using curry: A -> B -> C samne as A x B -> C
We can also define pairs using a function that takes two arguments:
\begin{lstlisting}[language=Lean]
def pair : Nat -> Bool -> Nat \times Bool := fun a b => (a, b)
#eval pair 3 true -- Output: (3, true)
\end{lstlisting}

we can also uncurry the pair using a function that takes two arguments:
\begin{lstlisting}[language=Lean]
def uncurry : (A \times B) -> C := fun p => (p.1, p.2)
#eval uncurry myPair -- Output: (3, true)
\end{lstlisting}


The \emph{sum type} $A + B$ consists of values that are either of type $A$ (tagged with \textbf{inl}) or of type $B$ (tagged with \textbf{inr}).

\paragraph{Formation Rule}
\[
  \frac{A : \mathsf{Type} \quad B : \mathsf{Type}}{A + B : \mathsf{Type}}
\]

\paragraph{Introduction Rules}
\[
  \frac{a : A}{\mathsf{inl}(a) : A + B}
  \qquad
  \frac{b : B}{\mathsf{inr}(b) : A + B}
\]

\paragraph{Elimination Rule (case analysis)}
\[
  \frac{ 
    e : A + B
    \quad
    x : A \vdash f(x) : C
    \quad
    y : B \vdash g(y) : C
  }{
    \mathsf{match}\ e\ \mathsf{with} \ \mathsf{inl}(x) \mapsto f(x) \mid \mathsf{inr}(y) \mapsto g(y) : C
  }
\]

\paragraph{Computation Rules}
\[
  \mathsf{match}\ \mathsf{inl}(a) \ \mathsf{with} \ \dots \equiv f(a)
  \qquad
  \mathsf{match}\ \mathsf{inr}(b) \ \mathsf{with} \ \dots \equiv g(b)
\]

\begin{example}[Sum Type in Leans]
    \begin{lstlisting}[language=Lean]

\end{lstlisting}
\end{example}

Why do we not have unions and intersections in type theory?
THey depend opn elements and not on types, moreover they are not closed under the type system.
The reason is that we can construct unions and intersections using sum types and product types.
b

case: (A -> C) -> (B -> C) -> (A + B) -> C
uncurrycase: (A + B) -> (A -> C) x (B -> C) -> C


Unit type emptyt product type one element
Empty type is empty coproduct

caseFalse: false -> C
impossibnle pattern
fale quo libit


more connectives wich are derived and not primiitvives
negatrion: P -> False
iff P -> Q and Q -> P

prove DeMorgan's law: in Lean
\begin{lstlisting}[language=Lean]   
theorem deMorgan (P Q : Prop) : ¬(P ∨ Q) ↔ ¬P ∧ ¬Q :=
  Iff.intro
    (assume h : ¬(P ∨ Q),
      And.intro
        (assume p : P, h (Or.inl p))
        (assume q : Q, h (Or.inr q)))
    (assume h : ¬P ∧ ¬Q,
      assume pq : P ∨ Q,
      match pq with
      | Or.inl p => h.left p
      | Or.inr q => h.right q
      end)
\end{lstlisting}  
% \section{Sum Types}
The \emph{sum type} $A + B$ consists of values that are either of type $A$ (tagged with \textbf{inl}) or of type $B$ (tagged with \textbf{inr}).  
% \section{Curry–Howard Correspondence}
We have been preparing for this argument, and the reader will have surely 
noticed a strong similarity when defining logical connectives 
using deduction rules; they are remarkably similar to types constructed 
using type judgments. For instance, function types can be seen as implications.
This is not a coincidence, but rather a fundamental theorem first proven by 
Haskell Curry and William Howard. It forms the core of modern type theory and 
establishes a deep connection between logic, computation, and mathematics.
The following table represents the Curry-Howard correspondence 
(also known as the Curry-Howard isomorphism)
\begin{table}[htbp]
\centering
\caption{Curry-Howard Correspondence: Logic and Type Theory}


\begin{align}
\text{Propositions} &\leftrightarrow \text{Types} \\
\text{Proofs} &\leftrightarrow \text{Programs} \\
\text{Proof Normalization} &\leftrightarrow \text{Program Evaluation}
\end{align}

\textbf{Implication} ($P \Rightarrow Q$) corresponds to the \textbf{function type} ($P \to Q$). 
A proof of an implication is a function that transforms any proof of the premise into a proof of the conclusion.

\textbf{Conjunction} ($P \land Q$) corresponds to the \textbf{product type} ($P \times Q$). 
A proof of a conjunction consists of a pair containing proofs of both conjuncts.

\textbf{Disjunction} ($P \lor Q$) corresponds to the \textbf{sum type} ($P + Q$). 
A proof of a disjunction is either a proof of the first disjunct or a proof of the second disjunct.

\textbf{Universal quantification} ($\forall x:A.P(x)$) corresponds to the \textbf{dependent product type} 
($\prod x:A.P(x)$). 
A proof of a universal statement is a function that, given any element of the domain, 
produces a proof of the property for that element.

\textbf{Existential quantification} ($\exists x:A.P(x)$) corresponds to the \textbf{dependent sum type} 
($\sum x:A.P(x)$). A proof of an existential statement consists of a witness element and a proof 
that this element satisfies the required property.
\textbf{Negation} ($\neg P$) corresponds to the \textbf{function type to the empty type} ($P \to \bot$).

\begin{tabular}{|l|c|l|l|}
\hline
\textbf{Logic} & \textbf{Symbol} & \textbf{Type Theory} & \textbf{Lean Notation} \\
\hline
\hline
\multicolumn{4}{|c|}{\textbf{Basic Correspondence}} \\
\hline
Proposition & $P$ & Type & \textbf{Type} \\
Proof & $\pi$ & Term/Program & \textbf{t : T} \\
\hline
\hline
\multicolumn{4}{|c|}{\textbf{Logical Connectives}} \\
\hline
Conjunction & $P \land Q$ & Product Type & \textbf{P} $\times$ \textbf{Q} \\
Disjunction & $P \lor Q$ & Sum Type & \textbf{P + Q} \\
Implication & $P \to Q$ & Function Type & \textbf{P $\to$ Q} \\
Biconditional & $P \leftrightarrow Q$ & Isomorphism & \textbf{P $\leftrightarrow$ Q} \\
Negation & $\neg P$ & Function to Empty & \textbf{P $\to$ False} \\
\hline
\hline
\multicolumn{4}{|c|}{\textbf{Constants}} \\
\hline
True & $\top$ & Unit Type & \textbf{True} or \textbf{Unit} \\
False & $\bot$ & Empty Type & \textbf{False} or \textbf{Empty} \\
\hline
\hline
\multicolumn{4}{|c|}{\textbf{Quantifiers}} \\
\hline
Universal & $\forall x. P(x)$ & Dependent Product & $\textbf{(x : A)} \to \textbf{P x}$ \\
Existential & $\exists x. P(x)$ & Dependent Sum & \textbf{Sigma x : A, P x} \\
\multicolumn{4}{|c|}{\textbf{Proof Operations}} \\
\hline
Proof by cases & Case analysis & Pattern matching & \textbf{match ... with} \\
Modus ponens & $P, P \to Q \vdash Q$ & Function application & \textbf{f a} \\
$\land$-introduction & $P, Q \vdash P \land Q$ & Pair construction & \textbf{(a, b)} \\
$\land$-elimination & $P \land Q \vdash P$ & Projection & \textbf{p.1, p.2} \\
$\to$-introduction & $P \vdash Q \Rightarrow \vdash P \to Q$ & Lambda abstraction & \textbf{$\lambda$ x => t} \\
\hline
\end{tabular}
\end{table}

As a strategy for implementing the Curry-Howard isomorphism, Lean introduces a special type \textbf{Prop} to represent propositions, and provides constructors to build new propositions from existing ones.

\begin{example}[from Theorem Proving in Lean 4]
In Lean, logical connectives are implemented as functions that take propositions and return new propositions:

\begin{lstlisting}[language=Lean]
-- Define implication using function types
def Implies (p q : Prop) : Prop := p \to q

-- Check the types of built-in logical connectives
#check And -- Prop \to Prop \to Prop
#check Or -- Prop \to Prop \to Prop
#check Implies -- Prop \to Prop \to Prop

-- Declare some proposition variables
variable (p q r : Prop)

-- Build complex propositions using these constructors
#check And p q -- Prop
\end{lstlisting}

This demonstrates how Lean treats propositions as first-class types. The logical connectives And, Or, and Implies are type constructors that take propositions as arguments and produce new propositions. Notice how:

\begin{itemize}
\item And p q constructs the conjunction of propositions p and q
\item Or (And p q) r constructs a disjunction between the conjunction $(p \land q)$ and proposition r
\item Implies (And p q) (And q p) expresses that conjunction is commutative
\end{itemize}

\end{example}
\begin{example}[from Theorem Proving in Lean 4]
    The \textbf{\#print} command reveals the underlying proof term of a theorem,
 
    demonstrating how proofs are simply terms in Lean's type system:

\begin{lstlisting}[language=Lean]
theorem t1 : p \to q \to p := fun hp : p => fun hq : q => hp

#print t1
-- Output: theorem t1 : p \to q \to p := fun hp hq => hp
\end{lstlisting}

 The theorem \textbf{t1} proves the logical statement $p \to q \to p$, which reads as "if $p$ is true, then regardless of $q$, $p$ remains true." The proof term \textbf{fun hp : p => fun hq : q => hp} is a nested lambda function that:

\begin{enumerate}
\item Takes a proof \textbf{hp} that $p$ holds
\item Takes a proof \textbf{hq} that $q$ holds (though we don't use it)
\item Returns the original proof \textbf{hp} of $p$
\end{enumerate}

When we use \textbf{\#print t1}, Lean shows the simplified (or \textbf{normalized}) form of the proof term, revealing how the theorem is actually implemented as a function. This transparency demonstrates the Curry-Howard correspondence in action: the logical implication $p \to q \to p$ corresponds exactly to the function type that takes two arguments and returns the first one.
\end{example}


Lean evaluates proofs with type checking and simplifications (wich we technically call \textbf{normalization}).
In the following section, we’ll explore this concept in detail.

% \section{Dependent Types}
pi type are dfependent function type gejneralizes function type
sigma type are dependent pair (product) type generalizes product and sum types

but we also have a relation between pi types and product types (Thorsten Altenkirch video)
(TO USE LATER mopre precisely a dependent type csan be seen as function where its codomain depends on its domain. 
In thius light a function is a 
 just a weaker form of a dependent type also dependen type avopid run time errors)

talk abnout family types

Connect to predicate logic and extend Curry-Howard.

a prediacate is a function from a type to Prop
P : A -> Prop
actually a dependent function 
and a term of type
pi x : A, P x
is a proof that P holds for all x : A in A
which is a proof of the universal quantification
forall x : A, P x
family of types indexed by A

a term of type
sigma x : A, P x
is a pair (a, p) where a : A and p : P a
which is a proof of the existential quantification
exists x : A, P x
family of types indexed by A  

Tautology top prove 

for (all x : A, P x -> Q x) iff (there exist x : A, P x )-> Q x (Thorsten Altenkirch  video)
tyhe proof is currying and uncurrying

% \section{Inductive Types}
% \section{Universes and Prop}
% Propositions is a type in that should have only one elemnt or none. (Thorsten / Vladimir Voevodsky)

% proposaitional truncation. if a type is inhabited then it's true otherwise false.
% with proposaitional truncation we hide the witness adn we can rwerite propostion as types (thorsten)
% Diaconescu's theorem: the axiom of choice implies the law of the excluded middle.

% \section{Proof Evaluation}

% In type theory, proof evaluation is achieved through \textbf{normalization}, a process that simplifies terms (proofs) into irreducible forms while preserving their semantic meaning. This computational foundation underlies how logical reasoning translates into algorithmic verification.

% If $p$ is a proof of $A$ and $q$ is a proof of $B$, then the pair $(p, q)$ constitutes a proof of $A \land B$. Conversely, given such a proof, we can extract the individual proofs of $A$ and $B$ using the projection functions $\mathsf{fst}$ and $\mathsf{snd}$, which extract the first and second components of the pair, respectively. The computational behavior of these projections is governed by the reduction rules $\mathsf{fst}(p, q) \rightsquigarrow p$ and $\mathsf{snd}(p, q) \rightsquigarrow q$, where the symbol $\rightsquigarrow$ denotes \emph{reduces to}, indicating a single computational step. We use $\twoheadrightarrow$ to denote the reflexive-transitive closure of this relation, representing multi-step reduction.

% Normalization applies two fundamental reduction rules. \textbf{Beta-reduction} substitutes arguments into function bodies, as exemplified by $(\lambda x. x + 1)(2) \rightsquigarrow 2 + 1 \rightsquigarrow 3$. \textbf{Eta-reduction} eliminates redundant lambda abstractions, such as $\lambda x. f(x) \rightsquigarrow f$.

% Lean's kernel validates proofs by normalizing proof terms through type checking, which verifies that terms conform to their assigned types (propositions), and reduction, which simplifies proof terms using beta and eta rules to confirm equivalence. 
% \begin{example}
%     For example, when proving $2 + 2 = 4$, the system reduces the left-hand side to its normal form and verifies definitional equality with the right-hand side.

%     \begin{lstlisting}[language=Lean]
%     example : 2 + 2 = 4 := by
%       simp [Nat.add_assoc, Nat.add_comm, Nat.add_left_comm]
%     \end{lstlisting}
% \end{example}

% Here, the \textbf{simp} tactic applies computational rules to simplify the terms. In this simple case, one could have directly use \textbf{rfl} (reflexivity) to check for the equality.


% Profof by inducirton lic associativer for Nat  addition
% is recurisonj and pattern matching  

% You can prove induction by recursion thus induction and recursion are the same  in Type Theory      
% \begin{lstlisting}[language=Lean]
% theorem t_add_assoc (a b c : Nat) : (a + b) + c = a + (b + c) := by
%   induction a with
%   | zero => rw[Nat.zero_add, Nat.add_zero] -- Base case: normalizes ‘0 + b + c‘ to ‘b + c‘ and proves ‘b + c = 0 + (b + c)‘
%   | succ a ih => -- Inductive step: uses hypothesis ‘ih : (a + b) + c = a + (b + c)‘
%     rw [Nat.succ_add, Nat.add_succ, ih] -- Normalizes ‘(a + b) + c + 1‘ to ‘(a + b + 1)‘ and proves ‘(a + b) + c + 1 = a + (b + c + 1)‘
% \end{lstlisting}
% \section{Equality}

% \begin{quote}
% "Syntactic equality is they look identical, definitional equality is they are the same, propositional equality is they turn out to be the same." – Bhavik Mehta
% \end{quote}

% Lean distinguishes between three different kinds of equality, and understanding their differences is crucial for effective proof development.

% \textbf{Syntactic equality} is the strongest form, where two expressions are literally identical in their textual representation. This is primarily what tactics like \textbf{rw} and \textbf{simp} work with when performing substitutions (eta conversion).

% \textbf{Definitional equality} ($\equiv$) occurs when two expressions represent the same mathematical object after applying computational reductions and definitional unfolding. For example, $2 + 2 \equiv 4$ because the left side reduces to the right side through computation. This is the equality that tactics like \textbf{exact}, \textbf{intro}, and \textbf{refl} recognize automatically without requiring explicit proof.

% \textbf{Propositional equality} ($=$) corresponds to the "usual" mathematical notion of equality and requires an explicit proof term to establish. For instance, proving $a + b = b + a$ requires demonstrating commutativity through mathematical reasoning rather than mere computation.

% The relationship between these equalities follows a hierarchy: syntactic equality implies definitional equality, which in turn can be used to establish propositional equality, but not vice versa.

% Identity type J combinator
% % \textbf{Subject Reduction and Kernel Validation:} Lean's type system ensures that if $t : T$ and $t$ reduces through normalization to $t'$ (written $t \rightsquigarrow t'$), then $t'$ maintains the same type: $t' : T$. This property guarantees that well-typed terms remain well-typed throughout evaluation.

% % Lean's kernel validates proofs through a systematic process. It first normalizes proof terms by reducing complex expressions to their canonical forms through $\beta$-reduction and definitional unfolding. Then it checks syntactic equality by verifying that normalized proof terms are syntactically identical to what the target type expects.
% \begin{example}
% Consider this proof of commutativity for natural number addition:

% \begin{lstlisting}[language=Lean]
% theorem t_add_comm (a b : Nat) : a + b = b + a := by
%   induction a with
%   | zero => rw[Nat.zero_add, Nat.add_zero] -- Base case: normalizes ‘0 + b‘ to ‘b‘ and proves ‘b = b + 0‘
%   | succ a ih => -- Inductive step: uses hypothesis ‘ih : a + b = b + a‘
%     rw [Nat.succ_add, Nat.add_succ, ih] -- Normalizes ‘(a + b) + 1‘ to ‘b + (a + 1)‘ and proves ‘b + a + 1 = b + (a + 1)‘

% \end{lstlisting}
    
% \end{example}

% During type checking, Lean normalizes the proof terms at each step, ensuring that the inductive reasoning is sound and that the final proof term indeed has type \textbf{a + b = b + a}. This demonstrates how all three forms of equality work together: the kernel uses definitional equality to normalize terms, checks syntactic equality for verification, and ultimately establishes propositional equality through the completed proof.
% Definitional equality ($\equiv$) is equality verified by computational reduction, determined by the type checker automatically (e.g., $2 + 2 \equiv 4$), as seen in precedence. Propositional equality ($=$) is equality that requires an explicit proof term to establish (e.g., $a + b = b + a$).

% If $t : T$ and $t$ reduces (by normalizaion) to $t'$ (written $t \rightsquigarrow t'$), then $t'$ maintains the same type: $t' : T$. This guarantees that well-typed terms remain well-typed throughout evaluation.

% From a computational perspective, any two proofs are essentially the same. If $P : \text{Prop}$ is any proposition, Lean's kernel treats any two elements $t_1, t_2 : P$ as being definitionally equal, much the same way as it treats $(\lambda x \mapsto t) \, s$ and $t[s/x]$ as definitionally equal. This principle is known as \textbf{proof irrelevance}.

% % \begin{definition}[Proof Irrelevance] 
% % For any proposition $P : \text{Prop}$ and any two proof terms $t_1, t_2 : P$, we have:
% % $$t_1 \equiv t_2$$
% % where $\equiv$ denotes definitional equality in Lean's type theory.
% % \end{definition}

% This means that even though we can treat proofs $t : P$ as ordinary objects in the language of type theory, they carry no computational information beyond the fact that $P$ is true. 

% \begin{example}
% Consider two different proofs of the same proposition:

% \begin{lstlisting}[language=Lean]
% theorem proof1 (p q : Prop) (hp : p) (hq : q) : p \land q := 
%   And.intro hp hq

% theorem proof2 (p q : Prop) (hp : p) (hq : q) : p \land q := 
%   ⟨hp, hq⟩

% -- These are definitionally equal despite different constructions
% example (p q : Prop) (hp : p) (hq : q) : 
%   proof1 p q hp hq = proof2 p q hp hq := rfl
% \end{lstlisting}
% \end{example}

% As a consequence, proofs can be erased during compilation without affecting program behavior, allowing computational efficiency.


% \section{Summary}
% It's a quiet informal and naive introduction. I intend to add more refinements such as proper definitions and examples (i need to work on the lean formatting as well), but the overall style and approach will stay the same.
% Anyway o don't want to give a full explanation of Lean tactic or syntax, just briefly touching some of those concepts.

% I mostly used Type Theory and Functional
% Programming by Simon Thompson \cite{thompson1999type} and the Theorem Prover Book. 
% I've borrowed the lambda calculus part from the Hott Book and conveniently 
% added when discussing function types.
% Later, i may add a dedicated (short) section on lambda calculus, although i am still deciding.

% I'd like to add more context throughout, especially some Lean examples. 
% For instance the first part need bit of more attention.

% I'm planning to continue with these topics next, though i want to revisit what i've written 
% so far since i'm not entirely happy with it yet:


% From Programming in Martin-Löf's Type Theory:

% \begin{itemize}
% \item Quantifiers and Dependent Types
% \item Universes
% \item Inductive Types


% \end{itemize}

% "Intuitively, a type is a collection of objects together with an equivalence relation. 
% Examples of types are the type of sets, the type of elements in a set, the type of propositions, 
% ...the type of predicates over a given set."

% % nice larticle: https://functional.works-hub.com/learn/dependent-types-explained-2e233
% \section{Lean's specific flavor of DTT has:}


% \begin{itemize}
%   \item an impredicative universe \texttt{Prop} of propositions
%   \item definitional proof irrelevance
%   \item no universe cumulativity
%   \item indexed, mutual and nested inductive types
%   \item an $\eta$ reduction rule for lambdas and structures
% \end{itemize}

% \section{Actual axioms}

% \begin{itemize}
%   \item Propositional extensionality
%   \[
%     \mathsf{propext} : \forall p, q : \mathbb{P},\ (p \leftrightarrow q) \to p = q
%   \]

%   \item Quotient types
%   \[
%     \mathsf{quot} : \forall \alpha : U_n,\ (\alpha \to \alpha \to \mathbb{P}) \to U_n
%   \]
%   \[
%     \mathsf{mk}_{\alpha, r} : \alpha \to \alpha / r
%   \]
%   \[
%     \mathsf{lift}_{\alpha, r} : \forall \beta.\ \forall f : \alpha \to \beta.\ 
%     (\forall x\, y.\ r\ x\ y \to f\ x = f\ y) \to \alpha / r \to \beta
%   \]
%   \[
%     \mathsf{sound}_{\alpha, r} : \forall x\, y.\ r\ x\ y \to \mathsf{mk}\ x = \mathsf{mk}\ y
%   \]
%   \[
%     \mathsf{lift}\ \beta\ f\ H\ (\mathsf{mk}\ x) \equiv f\ x
%   \]

%   \item The axiom of choice
%   \[
%     \mathsf{nonempty}\ \alpha := \mu T : U_0.\ (\mathsf{intro} : \alpha \to T)
%   \]
%   \[
%     \mathsf{choice} : \forall \alpha : U_n,\ \mathsf{nonempty}\ \alpha \to \alpha
%   \]
% \end{itemize}
\printbibliography
\end{document}